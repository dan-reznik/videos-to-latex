Consider a family of triangles T=ABC (blue) rigidly rotating about a point P in the plane. Now consider a family T' (red) with vertices at the inversions of A,B,C with respect to an inversion circle (dashed green) centered at an arbitrary point O. The video illustrates a few observations. Let X(k) [resp. X'(k)] denote the kth triangle center of T [resp. T'].

1) Trivial: for any choice of O, if P=X(3) of T, then (i) O,X(3),X'(3) are collinear, and (ii) X'(3) is stationary.
2) For any choice of P, the locus of X'(3) is always a conic with major axis along OP. The *type* (ellipse, parabola, hyperbola) of conic depends on the choice of O.
3) For any choice of O, if P=X(6) of T, then (i) O,X(6),X'(6) are collinear, and (ii) X'(6) is stationary
.4) For any choice of P,O, the locus of X'(6), i.e., the X(6) of T', is always an ellipse with minor axis along OP.

Related phenomena happen with X'(k), k=15,16,61,62,187, to be described in a new video.