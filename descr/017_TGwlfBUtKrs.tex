This is a result privately communicated to us by A. Akopyan, which we simulate here. A first video of this phemomenon appeared in [1].

The so-called "bicentric Poncelet" family is 1d family of polygons inscribed in an outer circle while simultaneously circumscribing a second inner circle. Recently we proved that  this family conserves the sum of its internal angle cosines [3].

Let the vertices of polygons in the family be labeled P1, P2, ..., PN. We call "japanese-style" triangulation (in reference to the much famed "japanese theorem" [2]) a subdivision in of an N-gon into N-2 triangles obtained as P(1)P(i)P(i+1), where i in [2,N-1] The original japanese theorem accepts any triangulation, but here we will stick to the just described "fan" style triangulation (all triangles share P1).

The video shows two phenomena. Over the bicentric family:

a) the sum of inradii is constant.
b) the locus of sum incenters (all in the N=4,5 cases, and some in the N=6,7 cases) are circles (solid reds). The remaining loci are non-conics (dashed red).

[1] https://youtu.be/BEvdUUolUXI
[2] https://mathworld.wolfram.com/JapaneseTheorem.html
[3] https://arxiv.org/abs/2103.11260 -- to appear, Arnold Math. J. 2021