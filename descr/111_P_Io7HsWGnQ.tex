An a/b=1.5 Elliptic Billiard (black) is shown. The family of triangular orbits (blue) have a stationary Mittenpunkt X(9) at the EB center [1]. For each orbit we draw E1 and E2, the Incenter X(1)- and Barycenter X(2)-centered Circumellipses. The latter is known as the Steiner Circumellipse, least-area amongst all Circumellipses. It is known that X(100) (resp. X(190)) lie on the X(9)-centered Circumellipse (our Billiard). Additionally, we have detected / proven [2] the following invariants over the N=3 family:

- E1's axes are parallel to the EB
- E1's axes ratio is constant
- E2's axes are only parallel when orbit triangle is isosceles (a vertex is on one of the EB vertices).
- P12, the fourth intersection of E1 with E2 is colinear with X(75), X(77) (not shown). Viewer proofs are welcome! 

[1] Reznik et al., "Can the Elliptic Billiard Still Surprise Us?", The Mathematical Intelligencer, 2020. ArXiv: https://arxiv.org/abs/1911.01515
[2] Reznik et al., "New Properties of Triangular Orbits in Elliptic Billiards", 2020, in preparation.