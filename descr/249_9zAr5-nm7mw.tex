A light ray is injected into the interior of a reflective ellipse, at fixed angle (30 deg w vertical) and varying x position. 

A 100-reflection trajectory is shown, revealing certain well-known patterns, chiefly that all rays are tangent to a virtual confocal conic, known as the "caustic": an ellipse if the initial ray does not pass between the foci, and a hyperbola if it does [1].

Notice that for most x positions, the trajectory is space-filling, however, for a dense set, it becomes N-periodic, i.e., polygonal with N sides. This happens when a certain quantity τ (the translation in the billiard map) divided by 2π is a rational number [1].

See [2] for further experimental work and 100s of videos on this channel on youtube.

[1] S. Tabachnikov, "Geometry and Billiards", 1991. http://www.personal.psu.edu/sot2/books/billiardsgeometry.pdf

[2] D. Reznik, R. Garcia, and J. Koiller, "Invariants of N-Perioics in the Elliptic Billiard", 2019. https://dan-reznik.github.io/Elliptical-Billiards-Triangular-Orbits/