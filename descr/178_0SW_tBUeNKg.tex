Start from a base curve (black) defined as :

C(t) = [x(t),y(t)] = [ 2 cos(t) - sin(2t)/4 + (3/10) cos(2t), sin(t) + (3/4) cos(2t)]

and compute its Steiner curvature centroid K, i.e., the average of curve points (taken as a finely sampled polygon), each weighed by the sin(2 theta). Numerically, for C(t) this yields:

K= [0.185793, 0.0197731]

Now choose a point M in the plane. In this case it is the point on a circle of radius 1.25 (could be anything), centered on K, counterclockwise 220 degrees from horizontal, i.e.:

M= [ -0.771762, -0.783711 ].

Shown are the pedal (red) and contrapedal (green) curves of C(t) with respect to the above M.

Also shown is the linearly interpolated curve (blue) of pedal and contrapedal, i.e., at every t compute:

interpolated(t) = (1- mu) pedal(t) + mu contrapedal(t)

Letting mu vary from -.25 to 1.25.

Note: a dashed-gray circle of radius 1.25 is shown around K, since, as shown in another video [1], the area of the pedal, contrapedal, and interpolated pedal, is constant for all points on that circle.

[1] D. Reznik, "Invariant Area of Interpolated Pedal", 2021. https://youtu.be/gR8Axe823_M