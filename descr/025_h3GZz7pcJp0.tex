Consider Poncelet 3-periodics (blue) in the so-called Brocard Porism (external circumcircle and internal Brocard Inellipse -- the caustic). This remarkable triangle family has stationary Brocard points at the caustic foci, and is equibrocardal, i.e., it conserves Brocard angle ω. X3, X6, X39, X182, and several other triangle centers are also stationary (X15 and X16 to name a few).

Over the family, the animation shows:

a) the circular locus of the Barycenter X2 (brown), centered at "C" = X(11171) and of
radius = R*(2*Cos(2ω)-1)/3 (Peter Moses, 12-Sept-2020).
b) the Steiner ellipse (red) and inellipse (dashed red) to the 3-periodics. A key  property is that though its axes have variable length, their ratio is conserved. Notice its axes pass through the left and right extremes of the circle (a).

The corollary to (b) is that the Brocard porism can be regarded as an image of 3-periodics in the Homothetic Poncelet pair under a variable similarity transform (rigid translation and rotation + isotropic dilation), i.e., it is angle preserving. This is consistent with the fact that both systems conserve sum of cotangents.

Recall the homothetic pair actually conserves both L2 = sum of squared sidelenghts and area A, and therefore Brocard angle ω. The Brocard porism conserves the ratio of L2 and A, though neither is constant.