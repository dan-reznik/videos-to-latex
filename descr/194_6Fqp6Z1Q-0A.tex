Shown is a "tricentric" Poncelet family of triangles P1P2P3 inscribed in an outer circle C with diameter R. Sides P1P3 and P1P3 (blue) are kept tangent to an internal circle C' (the first "caustic", brown), with radius r. In this case, and according to Poncelet's General Closure Thm, side P2P3 (red) will envelop a circle (dashed red) in the pencil of C, C', i.e., a second "caustic".

Shown are the loopy loci of X2, X4, X5 over variable "r". Since these lie at proportional positions with respect to the (fixed) circumcenter X3 on the Euler line (dashed black), their traces are homothetic.

Notice how the curve can at times be simple, self-intersecting, contain cusps. Also notice that when "r" is such that the system is poristic (all sides are tangent to C'), the 3 loci collapse to circles, as predicted in [1].

[1] B. Odehnal, "Poristic Loci of Triangle Centers", J. Geom. Graphics 15/1 (2011), 45-67.