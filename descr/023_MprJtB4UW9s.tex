The Brocard porism is a 1d Poncelet family of triangles T=ABC (blue) inscribed in a circle Γ (black) and circumscribed about an inellipse E (black) known as the Brocard inellipse [1]. The Brocard points Ω1,Ω2 are stationary at the foci of E and the Brocard angle ω is invariant. Also stationary are X15, and X16, the first and second Isodynamic points [4], though only the first one is shown. The Brocard circle K (green) contains Ω1,Ω2,X3,X6 and is stationary.

The 2nd Brocard Triangle T' (gold) has vertices A'B'C' at the intersections of symmedians (cevians thru X6) with the Brocard circle [2]. It is therefore inscribed in it.

One key observation borne out by the video is that the Brocard points  Ω1',Ω2' of T' are *also* stationary over the porism. Another known fact is that the isodynamic points of T' are congruent with those of T and are therefore also stationary.

But the main implication is that the family of 2nd Brocard triangles is a new Brocard porism inscribed in K and circumscribed about a 2nd, smaller Brocard inellipse E' (dashed black) of lower eccentricity than E. Interestingly, it can be shown the Brocard circle K' (dashed green) of this family is properly contained in K. 

If second Brocard triangles are computed recursively,  one obtains an infinite sequence of ever-smaller Brocard porisms which converge to the isodynamic point X15 common to all of them. Furthermore,  the sequence of Brocard circles K, K', K'', ..., form a Russian-doll (matryoshka) nesting which also shrinks to X15.

Note: also shown are the points of contact DEF of E to T: since X6 is the Brianchon point (perspector) of E [3], said points occur at the intersection of the symmedians with the sidelengths. Note also that the foci on an inellipse are isogonal conjugates, which is consistent with the fact that Ω1,Ω2 are such a pair. 

за здоро́вье!

[1] E. Weisstein, "Brocard Inellipse", Mathworld, 2020. 
[2] E. Weisstein, "Second Brocard Triangle", Mathworld, 2020. 
[3] E. Weisstein, "Brianchon Point", Mathworld, 2020. 
[4] E. Weisstein, "Isodynamic Points", Mathworld, 2020.