In honor of eminent mathematician John H. Conway, lover of triangle geometry [1], deceased today, Sunday April 12, 2020.

The poristic triangle family (blue) [2] has a fixed incircle (green) and circumcircle (purple). These are centered on X(1) and X(3). Also shown is the excentral triangles (green). Their locus (orange) is a circle of radius 2R centered on X(40) [3].

The circumbilliard (black) E9 to the poristic triangles is a circumellipse which renders the latter a 3-periodic billiard orbit. It is dynamically centered on X(9) [4]. Also shown (light blue) is the circumconic E6' to the Excentrals centered on their Symmedian X(6) [X(9) of poristics]. The axes of E9 and E6 are congruent, and both circumconics maintain constant aspect ratio, though their axes are variable.

Note: the locus of the foci of E9 is a perfect circle [5]. However, the locus of the foci of E6' is non-elliptic (probably a quartic, to be determined). 

Sountrack: Pachelbel Canon in D.

References:

[1] J. H. Conway, "Conway Triangle Notation", https://en.wikipedia.org/wiki/Conway_triangle_notation
[2] W. Gallatly, "Modern Geometry of the Triangle", F. Hodgson, 1914.
[3] B. Odehnal, "Poristic Loci of Triangle Centers", 2011. http://www.heldermann-verlag.de/jgg/jgg15/j15h1odeh.pdf
[4] D. Reznik and R. Garcia, "The Circumbilliard: any triangle can have its own Elliptic Billiard", 2020. In preparation.
[5] D. Reznik and R. Garcia, "Poristic Family: center and foci of the Circumbilliard have circular loci.", YouTube, 2020. https://youtu.be/LGgh11LMGGY