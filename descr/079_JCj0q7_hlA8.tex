Consider an affine image of regular polygons rotating in a circle, identical to Poncelet N-periodics in the homothetic ellipse pair (black and brown ellipses). Let Pi denote its vertices, i=1,...N.

This family conserves (i) sum sidelengths squared, (ii) area, and (iii) sum of cotangents, see [1,2].

Here we show a new, curious phenomenon.

Define the Generalized Evolute Polygon (GEP, pink) with vertices Pi' along the ellipse normals at Pi, at a distance s r from the Pi, where s is a real number and r is the radius of curvature at Pi. 

Pi' = Pi + s r n

Note: when s=0 (resp. s=1), the Pi' sweep the ellipse (resp. the ellipse evolute [3]).

It turns out for any choice of s, the signed area of the GEPs is invariant over the family.

The video shows the N=5 family for four values of s={1/4, 1/2, 3/4, 1}. Also shown is a circle (dashed pink) centered on P1' and passing through P1. When s=1 (lower right), this circle is the classic osculating circle at P1.

Postscript: the GEPs also conserve sum of sidelengths squared except when N=4 or 6.

[1] https://youtu.be/2PdsC3CcqaE
[2] https://youtu.be/30cuWWaZv7A
[3] E. Weisstein, "Ellipse Evolute", MathWorld 2021. https://mathworld.wolfram.com/EllipseEvolute.html