We consider families of triangles T=P1P2P3 inscribed in an outer ellipse E (black) and with two sides P1P2 and P1P3 tangent to an inner, confocal ellipse E' (brown). Let Pi' be the excenter opposite to Pi, i=1,2,3.

a) standard confocal pair, all sides are tangent, uniquely for this configuration, the locus of X1 is an ellipse (dark green) interior to E. The 3 excenters P1', P2', P3'  sweep the same elliptic path (dark green) exterior to E. These two elliptic loci have reciprocal aspect ratios.

b) A system with a confocal caustic slightly larger than in (a). The locus of X1 is no longer a conic, and neither is that of P1' (light green). Interestingly, the locus of both P2' and P3' are the same ellipse (dark green).

c) The same as (b) but now E' is slightly smaller than in (a). Again, X1 and P1' are non-conics, while P2' and P3' (dark green) sweep the same ellipse.