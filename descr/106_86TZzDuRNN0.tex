The simulation shows the family of triangular orbits (black rotating triangle) in an a/b=1.5 elliptic billiard E. Notice the orbits' Mittenpunkt X(9) is stationary at the billiard's center [1,3].

I) Moses' Points

Also shown are 29 points discovered by Peter J.C. Moses which lie on the boundary of E, reported on ETC [2] on July, 1st 2019, to be sure:

X(i), for i = 190, 651, 655, 658, 660, 662, 673, 771, 799, 823, 897, 1156, 1492, 1821, 2349, 2580, 2581, 3257, 4598, 4599, 4604, 4606, 4607, 8052, 20332, 23707, 24624, 27834, 32680.

X(100) and X(88), also shown, have been known to lie on E both from ETC [2] and experimentation by the authors [3]. ETC also reports X(100), X(1), and X(88) lie on a single line (shown dashed green).

II) Main Result

Also shown is E's anticomplementary triangle T' (in blue) as well as its incircle, centered on X(8), the Nagel Point of T, the anticomplement of X(1), the incenter of T. The contact triangle T'' of T', shown green, is perspective with T' via X(144), which we call the "Darboux Point".

The amazing property is that for any T, the vertices of T'' will be on E. Notice none of the other triangular points drawn on the boundary track the vertices of T''. To interact with this geometry please go to [4].

III) The Darboux Axis

Also shown is the fact that points Xi = 7,142,2,9,144 are collinear (dashed purple line). We call this axis the "Darboux Axis". Each point has a very interesting property in connection with the geometry shown, namely:

a) X(9), the Mittenpunkt, is stationary for all orbits and is congruent with the center of the billiard
b) X(1), the Barycenter is the point about which the orbit is reflected (and scaled by 2) to generate the anticomplementary triangle.
c) X(144), the "Darboux" point, is the perspector of the orbit's anticomplementary triangle T' and its contact triangle T'', as well as the intouch triangle (not shown).
d) X(7), the Gergonne Point, is the anticomplement of X(9), and therefore the (moving) Mittenpunkt of T', i.e., it will be center of its circumbilliard.
e) X(142), is the complement (reflect on X(2) and divide by 2) of X(9) and will therefore be the (moving) Mittenpunkt of the orbit's medial triangle (not shown), i.e., the center of its circumbilliard [5].

The following facts are true about the above points:

- X(9) is the midpoint of X(7) X(144)
- X(142) is the midpoint between X(7) and X(9)
- X(9) to X(2) is 1/3 of X(9) to X(7).
- X(142) is the mdpoint of X(9) and X(7)

[1] https://www.youtube.com/watch?v=tMrBqfRBYik
[2] https://faculty.evansville.edu/ck6/encyclopedia/ETC.html
[3] https://dan-reznik.github.io/Elliptical-Billiards-Triangular-Orbits/
[4] https://www.wolframcloud.com/objects/user-abf31092-d7c1-4e49-8701-dc65d547b021/peter%20moses%20points%20on%20X9-centered%20circumellipse
[5] https://www.youtube.com/watch?v=gwfx6LDJnsE

More Info: https://dan-reznik.github.io/Elliptical-Billiards-Triangular-Orbits/