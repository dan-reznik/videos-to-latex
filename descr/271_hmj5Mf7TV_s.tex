Take a random triangle T and consider inversive images T' of it wrt to circles in an elliptic pencil [1]. Namely, a T' has vertices at the inversions of T wrt to a circle in the pencil. Given a triangle center X(k) the video explores the locus of X(k) of T' over the family of inversions. We find that:

a) The X(k) of T' are conics (mostly ellipses) for k = 3, 6, 15*,16*, 61,62, 371, 372, 1151, 1152, 3311, 3312. (*) means they are circles.

Note: at 15:30 I claim the circular loci of inversive X(15) and X(16) are of the same radius, though the ellipse detector clearly shows they are not...

b) Said conics contain X(k) of T.

[1] E. Weisstein, "Coaxal Circles", https://mathworld.wolfram.com/CoaxalCircles.html