Are there any new observations possible in the world of regular polygons?

Take a regular N-gon w center at O, N greater than 3. Consider a circle C1 centered on O with radius R1 = |Q-O|, where Q is the intersection of side i with side i+2 of the N-gon.

1) For P on any circle centered on O, the pedal polygon [3,*] of the N-gon wrt to P has constant area. This result was proven by J. Steiner in 1825 [1]. This builds on an 1823 proof by Sturm applied to triangles [2].

2) New observation: the signed area of the antipedal polygon [4,**] of the N-gon with respect to P is constant and equal to ZERO, for all P on C1.

2.1) For N=7 (and 8) you can find a 2nd circle C2 centered on O and with radius R2=|Q'-O|, where Q' is the intersection of side i with side i+3. Observation (2) also holds for P on C2.

2.2) For higher N there are more such circles.

Notes:

(*) The pedal polygon are the feet of perpendiculars dropped from P onto the sides.
(**) The antipedal polygon has sides through the vertices of the N-gon and perpendicular to lines drawn from P onto the vertices of the N-gon
References: 

[1] Steiner's Theorem (1825) for constant area pedal polygons -- http://users.math.uoc.gr/~pamfilos/eGallery/problems/PedalPolygons.html
[2] Ostermann et al, "Geometry and Its History", Sturm's Theorem, https://books.google.com.br/books?id=eOSqPHwWJX8C&pg=PA221&lpg=PA221&dq=sturm%27s+theorem+area+pedal&source=bl&ots=nj4PosDX0w&sig=ACfU3U0iSH7An02LbgdCdYMwkPN290P0rQ&hl=en&sa=X&ved=2ahUKEwit-9bVnajqAhUFIrkGHTZFAnwQ6AEwCHoECAoQAQ#v=onepage&q=sturm's%20theorem%20area%20pedal&f=false

[3] Pedal Triangle, https://mathworld.wolfram.com/PedalTriangle.html 
[4] Antipedal Triangle, https://mathworld.wolfram.com/AntipedalTriangle.html