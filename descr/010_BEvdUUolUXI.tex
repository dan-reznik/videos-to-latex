The video shows (for N=5,6) a conjecture by A. Akopyan [2]. Over the Bicentric family, i.e., Poncelet N-gons interscribed between two non-concentric circles, the sum of the inradii of the triangles in a tesselation of said N-gons is invariant.  

Recall the so-called "Japanese Theorem" [1] to which this is related: given a convex, circle-inscribed polygon P with N vertices, subdivide it in subtriangles. The Japanese Theorem states the sum of the inradii of said subtriangles is invariant over the particular subdivision picked.

[1] E. Weisstein, "Japanese Theorem", Mathworld. https://mathworld.wolfram.com/JapaneseTheorem.html
[2] A. Akopyan, "Japanese Thm Invariant Inradius Sum for Non-Concentric Circular Poncelet", Private Comm., Oct 11, 2020.