Let [a,b] and [a',b'] denote the semi-axes of an outer E and inner E' concentric ellipses. E' will be interior and homothetic to a rotated copy of E by choosing a k in the interval (0,b/a) and setting:

b'=k a
a'=k b

The video shows families of Poncelet N-periodics between  E and E', for N=5 and 7 and their alternative "star" (self-intersected). For each of them the appropriate value of k has been numerically computed.

For all N, this family of rotated-homothetic ellipse pairs has the remarkable property that all altitudes meet at the common center. For N greater than 3 these are defined as the segments between a vertex and the opposite side and the opposite side (N odd) or vertex (N even). 

This is because E is "self-dual" with respect to a conic sending E' to E (+ symmetry). In this case said conic is a circle, so duality goes to perpendicularity of the radius-vector and the corresponding side [1].

Apart from that, no other metric nor angular invariants have been found so far for this ellipse pair family.

[1] A. Akopyan, Private Communication, Aug 11, 2020.