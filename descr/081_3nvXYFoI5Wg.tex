It can be shown the Poncelet family in the homothetic pair conserves (i) sum sidelengths squared, (ii) area, and (iii) sum of cotangents, see [1,2]. Let Pi, i=1,...,N denote its vertices.

Define the " generalized evolute polygon" (GEP, pink) as having vertices Pi' such that:

Pi' = Pi + s ri ni,    i=1,...N

where: s is a scalar, ni is the inward unit normal at Pi, and ri is the radius of curvature at Pi, respectively.

Note: when s=0 (resp. s=1), the Pi' sweep the ellipse (resp. the ellipse evolute). 

It turns out that for any choice of "s", the GEP conserves area (exception: N=4, see below). In fact, one can always choose two "s" such that the signed area is dynamically zeto.

The video shows one such zero-signed-area GEP for N=3,5,6,8 cases.  A few observations:

a) the zero-area GEPs for N=3 are segments (see [1])
b) the "s" required for N=3 and N=6 are the same
c) for N=4 (not shown), the area of the GEP variable.

[1] https://youtu.be/OFA_j25R8ks