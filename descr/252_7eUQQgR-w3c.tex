This is an improvement upon a previous video: https://youtu.be/sViggJv4xyQ

Let E be an ellipse (black) with a,b semiaxes, and P be a point on its boundary. Consider a parallelogram Q (red) inscribed in E [note 1]. Drop perpendiculars from P onto the sides of Q, let their feet define a "pedal polygon"  Q' [note 2], shown black and pink-filled in the video.

Observation 0: for all P and Q, the product Area(Q).Area(Q') is invariant!

Observation 2: consider the vertex centroid C0, area centroid C2 of the pedal polygon Q'. claim: C0,C2,P are collinear!

Observation 3: take a fixed E-inscribed parallelogram Q. the area of Q', its pedal polygon, is invariant over all P on the boundary of E. The following proof has been contributed: The area of quadrilateral is the product of diagonals by sin of angle between them. So for any fixed parallelogram and any point p it is constant [1].

Isn't this cool?

note 1: this is a 1-d poncelet family w a confocal caustic, i.e., the parallelograms are 4-periodic billiard trajectories within E
note 2: Alternate pairs of feet are collinear with P (since alternate sides of Q are parallel). Let these feet defined a "pedal polygon"  (black, pink-filled).

References:

[1] A. Alopyan, Private Communication, June 7, 2020.