The video shows the family of 5-periodics (blue) in the elliptic billiard (black), tangent to a confocal caustic (brown). Let Pi denote its vertices, i=1,...5.

A classic invariant of billiard N-periodics is perimeter. A more recently discovered one is the sum of its internal angle cosines [1].

Also shown is the "focus-inversive" polygon family (pink) whose vertices lie at inversions of the Pi with respect to a circle centered on a focus, e.g., the left one. Since the inversion of an ellipse wrt to a focus is a loopless Pascal's Limaçon (green), focus-inversives will be inscribed in said curve.

The surprising phenomenon here is that the focus-inversive family *also* has constant perimeter and *also* conserves the sum of its internal angle cosines. This has been recently proven in [2].

[1] A. Akopyan, R.Schwartz, and S. Tabachikov, "Billiards in Ellipses Revisited", Eur. J. Math, Sept. 2020, doi:10.1007/s40879-020-00426-9
[2] P. Roitman, R. Garcia, and D. Reznik, "New Invariants of Poncelet-Jacobi Bicentric Polygons", Mar 2020, arXiv:2103.11260.