Six copies of the same a/b=1.5 elliptic billiard (EB, black) and its confocal caustic (brown) are shown.  Let the N-periodics (blue) be denoted as P and the tangent polygon (green) P'.

Let A, A' denote the areas of P and P'. It has been shown A'/A (resp. A A') is constant for odd (resp. even) N [1,2].

Here we are interested in the areas of pedal polygons defined by the feet of perpendiculars (dashed blue, green) dropped from a point "m" onto either P or P'. Call the former P_m (transparent blue) and latter P'_m (transparent green). Let A_m and A'_m denote their areas, respectively. 

More specifically, the video illustrates the invariance of ratio q=A_m/A'_m when "m" is:

left: at the center of the EB
middle:  at the right-side focus
right: at a fixed location on the first quadrant. 

1) Top row (N=5). q is conserved only when "m " is at the center of the EB or at a focus. This extends to all odd N.
 
2) Bottom row (N=6): q is conserved for *any* position of "m". This is true for all even N.

Soundtrack: Villa-Lobos

References:

[1] Arseniy Akopyan, Richard Schwartz, Serge Tabachnikov, "Billiards in ellipses revisited", 2020. https://arxiv.org/abs/2001.02934

[2] Ana C. Chavez-Caliz, "More about areas and centers of Poncelet polygons", 2020. https://arxiv.org/abs/2004.05404

[3] D. Reznik, R. Garcia, J. Koiller, "Can the Elliptic Billiard Still Surprise Us?", 2020, Math. Intelligencer, https://rdcu.be/b2cg1