Left: 3-periodic Poncelet family (blue) between an external ellipse (black) and an intrerior, concentric circle (purple). By definition, this family has constant inradius. Surprisingly, it also conserves circumradius R, sum of cosines, product of half sines, and X3 moves along a circle centered on the common center O.

Right: external ellipse (black) and concentric internal circle (purple) suitable for 5-periodic Poncelet (blue) family. Suprisingly, this family *also* conserves the sum of cosines, and the product of half sines. Since we no longer have X3, let's use its polygonal "proxy", Jakob Steiner's Curvature Centroid K (Krümmungs Schwerpunkt): like X3, it is the aveage of the vertices where each is weighted by the sine of the double angle. Surprise: K also moves along a circle centered at O.