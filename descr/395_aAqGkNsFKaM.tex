In [1],  a curious phenomenon, akin to "Archimedes' Principle", is described:

given an N-gon and some triangulation of its interior, the "circumcenter of mass" (X3*) is defined as the weighted average of circumcenters X3 of individual triangles in the triangulation, weighed by their areas.

It turns out X3* is independent of the particular triangulation picked. For an interactive visualization of this phenomenon, see the video by E. Tsukerman [2].

The video demonstrates a result in Section 4 of [1], namely: the X(k)-of-mass will be independent of triangulation if X(k) is on the Euler line of a triangle and is a fixed linear combination of X2 and X3. Amongst the first 1000 on ETC [3] the following triangle centers have this property:

2 (barycenter), 3 (circumcenter), 4 (orthocenter), 5 (nine-point center), 20 (de Longchamps), 140, 376, 381, 382, 546, 547, 548, 549, 550, 631, 632.

[1] S. Tabachnikov and E. Tsukerman, "Remarks on the Circumcenter of
Mass", Arnold Mathematical Journal volume 1, pp. 101–112, 2015.

[2] E. Tsukerman, "Circumcenter of Mass", YouTube, 2015. https://youtu.be/V8aZbMdxwIU

[3] C. Kimberling, "Encyclopedia of Triangle Centers", 2021. https://faculty.evansville.edu/ck6/encyclopedia/ETC.html