In 1871 Cayley published a paper about a very curious property of octagons inscribed in conics, known as "Octagrammum Mysticum", also explained here: https://arxiv.org/pdf/1209.4795.pdf

Namely, if an octagon is inscribed in an ellipse, the intersection of lines passing thru alternate sides will all land on the same conic, e.g., an outer ellipse. 

The video shows what happens when the octagon is the orbit of an elliptic billiard (a/b=1.5). The outer conic (locus of intersections of alternate sides) will be an ellipse *confocal* with the billiard. The video shows that in fact N=5,6,7 will also produce elliptic foci which are confocal.

https://dan-reznik.github.io/Elliptical-Billiards-Triangular-Orbits/