The observations below were co-developed with P. Roitman, A. Akopyan, S. Tabachnikov, H. Stachel, R. Garcia, and J. Koiller.

Consider the family of N-periodics (blue) in an elliptic billiard (black). Consider the so-called "focal spokes", i.e., segments connecting each vertex to one of the foci (dashed blue). Let r1(i) denote their lenghts. It turns out the sum of 1/r1(i), i=1,...,N is invariant over the family! 

The inversion of an ellipse with respect to a focus is Pascal's Limaçon [1]. So the inverted focal spokes can be visualized inscribed in said Limaçon (orange).

By lateral symmetry, the sum of inverse focal distances to the other focus will also be invariant and of the same value. Invariance of both sums 1/r1(i) and 1/r2(i) implies that the sum of [r1(i)+r2(i)]/(r1(i) r2(i)] is invariant. Since r1(i)+r2(i) is constant (property of the ellipse), the sum of 1/[r1(i) r2(i)] is also invariant. 

A known relation is that the curvature k(i) of the ellipse at point P(i) is equal to a b (r1 r2)^(-3/2) [2], where a,b are the semiaxes. Therefore the sum of curvatures k(i)^(2/3) is also invariant. Prof. Hellmuth Stachel derived the following expression for the sum of k(i)^2/3:

sum of k(i)^(2/3) = L/(2 J (a b)^(4/3))

where L,J are the invariant perimeter and Joachimsthal's constant.

Note: the video shows an N=5 family in an a/b=golden ratio billiard, but the above invariants work for all N and aspect ratios.

Audio track: Valse Musette in homage to Blaise Pascal (1623–1662)

[1] R. Ferréol, "Limaçon de Pascal", Mathcurve, 2020. https://bit.ly/2GO9EF1
[2] J. Calvert, "Ellipse", Math. Index, 2005. http://mysite.du.edu/~jcalvert/math/ellipse.htm