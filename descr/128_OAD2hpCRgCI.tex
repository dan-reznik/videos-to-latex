Shown is the family of 3-periodics (blue) in the elliptic billiard E (black). This family has invariant perimeter, sum of cosines, and the Mittenpunkt X(9), not shown, is stationary at the center of E.

Also shown is the derived family of "focus-inversive" triangles (pink), whose vertices are inversions of the 3-periodic ones wrt a unit circle (dashed black) centered on a focus of E (left in video).

That the focus-inversive family is inscribed in Pascal's Limaçon (dashed pink) is easy to show (inversion of ellipse wrt to its focus). Unexpectedly (and unproven) is the fact that both the perimeter and sum of cosines are invariant.

To boot, the Gergonne point X(7) is stationary.

The main theme of the animation is the following surprising observation:

the loci of X(k), k=1,2,3,4,5,8,9,10,11 of the inversives are all circles whose centers lie on the x axis (E's major axis).

In fact, between X(1) and X(100) the following centers of the focus-inversive family are circles [1]:

X(k), k=1, 2, 3, 4, 5, 8, 9, 10, 11, 12, 20, 21, 35, 36, 40, 46, 55, 56, 57, 63, 65, 72, 78, 79, 80, 84, 90, 100.

[1] https://youtu.be/srjm23nQbMc