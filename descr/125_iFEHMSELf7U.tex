Consider a family of triangles T=ABC (blue) rigidly rotating about their first isodynamic point X(15). Consider a family T' (red) with vertices at the inversions of A,B,C with respect to an inversion circle (dashed green) centered at an arbitrary point O. The video illustrates phenomena pertaining to the locus of the two isodynamic point X'(15) of T', namely:

1) When the moving circumcircle C of T contains (resp. does not contain) O, X'(15) moves along an arc of a circle (resp. is stationary). Let R be the circumradius of T. There are 3 possibilities:
1.1) If |O-X(15)|  greater than R, O always exterior to C, X'(15) is a stationary point over the entire motions
1.2) If |O-X(15)|  less than R: (i) while O is exterior to C: X'(15) is stationary, (ii) while O is interior, X'(16) moves along a disjoint circular arc.
1.3) O is sufficiently close to X(15) such that O is always interior to C, X'(15) moves along a circle.

2) Similar phenomena happen when T is pivoting about its 2nd isodynamic point X(16) and one is tracking the X'(16) of T'.

3) If T is pivoting about a point distinct from X(15), the locus of X'(15) will be (i) a single circle (if O is always interior or always exterior to C), or (ii) two circular arcs (corresponding to when O is interior and/or exterior to C). In this case X'(15) never has a stationary phase. Similar things happen when X'(16) is tracked.

Related phenomena happen with X'(k), k=61,62,187, to be described in upcoming videos.