Feuerbach's theorem states the 9-point circle touches the incircle on one point known as the Feuerbach point X(11), as well as the excircles on one point each. The latter form the Feuerbach Triangle. Each excircle touches a side of the triangle on a separate location (extouch points), and these form the external contact or "extouch" triangle.

Shown is the family of triangular orbits in an a/b=1.5 elliptic billiard, along with its incircle, 9-point circle, and 3 excircles. Three phenomena are illustrated:

1) The locus of the Feuerbach point X(11), brown "F"
2) The locus of the vertices of the extouch triangle (t12, t23, t31) sweep the confocal caustic (shown brown).
2) Any one vertex of the Feuerbach triangle (marked "xFeu") will sweep a self intersecting curve (show brown around the billiard).