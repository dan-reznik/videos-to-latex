An a/b=1.5 Elliptic Billiard is shown (blue) as well as the family of 3-periodic orbits (blue).

Shown (green) is the Yff Parabola [1] with focus on X101. This parabola is an inconic since it touches the reference triangle (the orbit) one each side (or an extension thereof). The touch points define de Yff Contact Triangle [2] (red), whose area is miraculously twice that of the reference triangle!

X190 is the perspector (or Brianchon Point) of the inconic, i.e., lines drawn from each orbit vertex to the contact points meet at X190.

Also shown is the parabola's directrix (dashed black) passing through the Mittenpunkt X9 and the Orthocenter X4. Recall the former is stationary at the Billiard center.

An interesting observation (proofs accepted!) is that when X4 is on the Billiard (this happens when the orbit is a right-triangle), X101 will be on one side of the Yff contact triangle! At this position, the axis of the Yff parabola (dashed black) passes through the Mittenpunkt X9.

The parabola vertex V is at X(3234) [3]. Shown is its astroid-like locus (pink) over the 3-periodic family. What curve is that?

[1] http://mathworld.wolfram.com/YffParabola.html

[2] http://mathworld.wolfram.com/YffContactTriangle.html

[3] Peter Moses (via Clark Kimberling), private communication, Feb 2020.