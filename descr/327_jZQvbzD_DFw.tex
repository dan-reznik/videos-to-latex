The video shows a circle-inscribed N=6 Poncelet transverse which alternates between two concentric, elliptic caustics (identical  modulo 90 degree rotation). These have been chosen so as to perfectly close and N=6 Poncelet polygon for some starting P1 on the outer circle. Surprisingly, as P1 is slid, closure is nearly maintained, i.e., the distance between first and last vertex is variable but remains within a very small fraction (10^-4 to 10^-5) of the circumradius.