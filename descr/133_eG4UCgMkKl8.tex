Video shows the 1d family of N-periodics (blue) in the elliptic billiard inscribed in an outer ellipse E (black) w semi-axes (a,b), and circumscribed about a confocal caustic (brown). Let Pi denote its vertices. Without loss of generality, the video depicts the case of N=5, a/b=1.5.

The inversive polygon I (pink) has vertices Pi* which are inversions of the Pi with respect to a unit circle centered on the left focus (f1). We have identified a few suprising invariants for this family, valid for all N: (1) the sum of distances from f1 to its vertices, (2) the sum of its cosines (except for N=4), and (3) perimeter. Thelocus of its vertices is Pascal's Limaçon (not shown), depicted in previous videos.

The "outer" (or tangential) polygon (green) has sides are tangent to E at the Pi. Let Pi' denote its vertices. The locus of the Pi' is also an ellipse (dashed green), concentric though non-confocal with E. Its foci are shown as green dots.

The outer-inversive polygon I' (light green) has vertices Pi'* whose vertices are inversions of the Pi' wrt to f1. Note that since the locus of the Pi' is an ellipse non-confocal with E, the locus of the Pi'* is not a Limaçon, since it will correspond to the inversion of an ellipse with a point on its major axis which is not a focus.

The new invariant depicted in this video is that for all N, the ratio of areas between I and I' is invariant. Suprisingly, when N=4, A(I)/A(I') is always 2, regardless of a,b.