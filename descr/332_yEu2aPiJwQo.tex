The Poristic Triangle Family [1,2] is a set of triangles (blue) with a fixed Incircle (green) and Circumcircle (purple). Let the Circumcenter X3 be on the origin and the Incenter X1 be "d" units above it on the y axis.  Weaver [3] proved their Antiorthic Axis [6] is stationar0 (blue). Its (stationary) intersection with the X1X3 axis is X1155.

Odehnal showed [4] the vertices of the Excentral Triangle (green) sweep a circular locus (orange), centered on X40 and with radius 2R, where R is the circumradius (R=1 on the video). He also showed [4, pp 17] the locus of the Mittenpunkt X9 is a circle (red) centered on M9 = [0, d(2R−r)/(4R+r)] and with radius ρ9 = 2 R(R−2r) /(4R+r). 

This video shows two interesting new phenomena:

a) the Circumbilliard [8] of the poristic family has fixed aspect ratio. This stems from the fact that the family has fixed r/R [7].

b) Also, the Caustic to the Excentral Triangles (dashed green) is centered on X3 and has foci X40 and X1, i.e., it is the MacBeath Inconic of the Excentral Triangle [5].

Sountrack: Liszt - Liebestraum No. 3 (Love Dream)

References:

[1] William Chapple, Surveyor, https://en.wikipedia.org/wiki/William_Chapple_(surveyor)
[2] W. Gallatly, "The Modern Geometry of the Triangle", F. Hodgson, 1914.
[3] J. H. Weaver, "Invariants of a poristic system of triangles", Bull. Amer. Math. Soc., 33:2, 1927. https://projecteuclid.org/download/pdf_1/euclid.bams/1183492031
[4] Boris Odehnal, "Poristic Loci of Triangle Centers, Journal of Geometry and Graphics", 15(1), 2011. https://www.geometrie.tuwien.ac.at/odehnal/pltc.pdf
[5] MacBeath Inconic, https://mathworld.wolfram.com/MacBeathInconic.html
[6] Antiorthic Axis, https://mathworld.wolfram.com/AntiorthicAxis.html
[7] Dan Reznik, Ronaldo Garcia, and Jair Koiller, "Can the Elliptic Billiard Still Surprise Us?", Math, Intelligencer, 42, 2019, https://rdcu.be/b2cg1
[8] Dan Reznik and Ronaldo Garcia, "Circuminvariants of 3-periodics in the Elliptic Billiard, 2020. arXiv: https://arxiv.org/abs/2004.02680