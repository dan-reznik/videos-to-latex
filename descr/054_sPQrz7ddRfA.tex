Consider the triangle family T(t)=V1V2P(t) inscribed in an ellipse E (black) of semiaxes a,b. Let V1 and V2 be fixed on E while P(t) executes one revolution on it.

1) Let X be a triangle center. Over the T(t), the locus of X will be an ellipse if X is a fixed linear combination of the barycenter X2 and orthocenter X4. (X will necessarily lie on the Euler Line).

2) Consider the case where X=X4. In this case, it can be shown that for any choice of V1,V2, the locus is an upright ellipse passing thru said points, which is axis-aligned with E and has aspect ratio of b/a.

3) Now pick a V1. Over all possible (fixed) placements of V2, one obtains a family of elliptic loci w the features in (2). It can be shown their centers will lie on an ellipse (red) whose aspect ratio is also a/b.  The *envelope* of said family will be a (generally) non-convex shape (pink), which is the affine image of Pascal's Limaçon [1].

The video shows said locus family as one varies V1, as well as the locus of their centers (an red ellipse) and their varying envelope (pink). Surpringly, the envelope is area-invariant over all V1, and this is valid for any triangle center X.

Note: the origin of the term "Limaçon" is the latin for snail, "limax".

[1] E. Weisstein, "Pascal's Limaçon", Mathworld, 2020. https://mathworld.wolfram.com/Limacon.html