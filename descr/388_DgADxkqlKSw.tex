We demonstrate further properties of Steiner's Hat Δ: this the negative pedal curve NPC [1] of the ellipse E with respect to a fixed point M on its boundary. We've seen before that the area A of Δ is invariant over M.

This time we consider a "rotated" NPC: the envelope of lines passing through a point P(t) on the boundary of E, rotated θ degrees about P(t) with respect to the perpendicular to [P(t)-M].

1) Over all M, these are also constant-area deltoids Δ*(θ) with three cusps. 

2) Δ*(0) is the "original" Stainer's Hat. Let A(θ) denote the area of Δ*(θ). It turns out A(θ)/A = cos^2(θ). Δ*(90)=M, a point, and A(90)=0.

3) The pre-images of the cusp of Δ*(θ) coincide with Pi, the pre-images of the cusps of Δ. Let Pi' denote the cusps of Δ. Recall PiPi' concurred at C2, the center of area of Δ.

4) Let Pi* denote the cusps of Δ*(θ). Lines PiPi* concur at C2*, Δ*(θ)'s center of area.

5) We saw before M,C2,Pi were concyclic. It turns out C2* lies on this same circle. As one varies θ from 0 to 90, C2* moves along a circular arc on this circle from C2 to M.

Note: like Δ=Δ*(90), Δ*(θ) is likely an affine image of the Steiner Hypocyclid.

[1] Mathworld, "Ellipse negative pedal curve", https://mathworld.wolfram.com/EllipseNegativePedalCurve.html 
[2] Mathworld, "Steiner Deltoid", https://mathworld.wolfram.com/SteinerDeltoid.html