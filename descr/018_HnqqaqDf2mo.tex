The video shows an N=4 (left) and N=5 (right) family of polygons (blue) interscribed between two circles C (black) and C' (brown), also known as bicentric polygons. Let Pi=1,...,N denote the vertices of each family. Let Ti = P(i)P(i+1)P(i+2) denote N "peripheral" subtriangles (dashed blue), where indices are computed (mod N). The video shows three properties:

a) the sum of internal angle cosines of a bicentric family is conserved (proved in [1]).
b) the sum of the inradii r(i) of the T(i) is conserved -- unproved for N greater than 4
c) the locus of the incenters of the T(i) is a circle (dashed red).
d) (Sept 1, 2021): the sum of 1/r(i) is also conserved!

[1] P. Roitman, R. Garcia, and D. Reznik, "New Invariants of Poncelet-Jacobi Bicentric Polygons",  Arnold Mathematical Journal, 2021 (to appear).