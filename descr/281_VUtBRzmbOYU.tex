Six a/b=1.5 elliptic billiards (EB, black) are shown, as well as their N-Periodic families (blue). Top row: N=3,4,5, bottom row: N=6,7,8. For each N-periodic the tangential polygon (green) is shown, call it P', whose sides are tangent to the EB at the N-periodic vertices.

From each focus f1 (or f2), drop N perpedinculars to the sides of P'.

Property 1: their feet all lie on a circle with radius a, the 
EB major semi-axis.

Property 2: Let the lengths of perpendiculars from f1 and f2 be denoted by e_1(i) and e_2(i), i=1 to N. It turns out that for all N greater than 2 (odd or even):

\sum_{i=1}^N{ e_1(i)^2 } = \sum_{i=1}^N{ e_2(i)^2 } is invariant over the family of N-periodics.

Soundtrack: Ana Vidovic, "Asturias" (Isaac Albeñiz)