Shown are four elliptic billiards with a/b of 1.25, 1.395, 1.485, and 2. For each of these the family of N=3 (triangular) orbits is shown. For each billiard the orbit's incenter X1, the anticomplement of the Feuerbach X100, and X88 (the isogonal complement of X44 [1]) are shown. A few known properties of X88 for any triangle [1]:

- X88 and X100 lie on X9-centered circumconic (the fixed billiard in this case, as X9 is stationary for the 3-periodic family).
- X88 is collinear with X100 and X1.

New results [2]:

- At a/b less than 1.485 the movement of X88 is monotonically opposed to that of the orbit's vertices. At a/b greater than that threshold, the movement becomes non-monotonic. The velocity changes sign twice when X88 is in the vicinity of either billiard horizontal vertex.

Let T=ABC be the orbit triangle with sides a ≤  b ≤ c

- X88 will coincide with B iff b = (a+c)/2. This is in general a scalene triangle. In this case, X1 is the midpoint between X100 and X88. the letter ρ above each billiard denotes the |X1-X100|/|X1-X88| ratio. So when X88 is on an orbit vertex, ρ=1.
- The only right-triangle for which X88 coincides with B is a:b:c = 3:4:5
- A 3:4:5 triangle has an X9-centered circumconic with semiaxis ratio of 1.3924. Equivalently, this is the aspect ratio of the only billiard which admits a 3:4:5 triangle.

[1] Clark Kimberling, "Encyclopedia of Triangle Centers", https://faculty.evansville.edu/ck6/encyclopedia/ETC.html

[2] Dan Reznik, Ronaldo Garcia, and Jair Koiller, "Loci of Triangular Centers in an Elliptic Billiard", 2020. https://arxiv.org/abs/2001.08041