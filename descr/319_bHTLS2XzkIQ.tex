A pair of circles C1,C2 is associated with two limiting points [1] such that if the inversive images of C1,C2 with respect to a circle centered on either la or lb is a  pair of *concentric* circles. 

The video demos interesting dynamic phenomena of two pairs of circles:

pair 1: (circumcircle, bevan circle)
pair 2: (incircle, circumcircle)

derived from Poncelet 3-periodics in 2 distinct families:

fam. 1: confocal pair (elliptic billiard)
fam. 2: pair with concentric incircle

We uncover the following properties.

1) pair 1, for fam. 1 and fam. 2: the locus of la,lb are two concentric circles. 
2) pair 2, fam. 1: the loci of la and lb are concentric, axis aligned ellipses. 

Let Ra/ra and Rb/rb be the ratio of outer-to-inner radii in the first and second concentric pairs (after inversion of C1,C2 wrt la,lb respectively). A general property is that 

A. Akopyan [2] contributed the following explanation: You can consider any circle passing through the limits points. Let say it intersects the circle at points XY. Then we can look on the cross-ration of points (L1, X, Y, L2) which does not change under inversions with the center at points on the circle. So if we do an inversion at L1 or L2 the cross-ration goes to usual ratio, therefore they have to coincide.

In our case, these ratios are not only equal but *invariant* over the family.

P.S. -- it turns out the ratio between the outer-to-inner radius of either pair of concentric circles is known as "the inversive distance" between two circles [3]. Given a triangle, the inversive distance between circumcircle and incircle only depends on the ratio r/R (see Eqn. 3 in [3]), and this ratio is invariant over 3-periodics in the confocal and incircle pairs.

Note that experiments in the video show that for the same elliptic billiard the, inversive distance of (incircle,circumcircle) is equal to the one between (circumcircle, bevan circle), suggesting Eqn. 3 in [3] also applies to the latter circle pair.

[1] E. Weisstein, "Limiting Points", MathWorld 2021. https://mathworld.wolfram.com/LimitingPoint.html
[2] A. Akopyan, Private Communication, Feb 2021.
[3] E. Weisstein, "Inversive Distance", MathWorld, 2021. https://mathworld.wolfram.com/InversiveDistance.html