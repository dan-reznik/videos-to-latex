The video allows us to observe the family of 3-periodics in five Poncelet ellipse pairs, four of them which are concentric and axis-aligned, to be sure:

0. Confocal. Here 3-periodics are billiard trajectories (the ellipse normal is the bisector at every vertex). The Mittenpunkt X9 is stationary as the system's center. and the ratio of inradius to circumradius is conserved, therefore the sum of cosines is as well.

I. Incircle. The family has stationary incircle and X1-centered circumellipse. Suprisingly, this system conserves circumradius (not shown). Since inradius is conserved by definition, so is its ratio to circumradius and therefor it shares with system 0 invariant sum of cosines.

II. Inellipse. This family has stationary circumcircle and X3-centered inellipse. Its invariants include the sum of squared sidelengths and the product of cosines.

III. Homothetic. The inner ellipse is a half-sized copy of the outer one, i.e., the family's Steiner Circum- and Inellipse are stationary, with their center X2 (Barycenter) stationary. This system conserves area, sum of squared sidelengths, and as a corollary, the Brocard angle ω (equiv. to the sum of cotangents).

IV. Dual. The inner ellipse is homothetic to the outer ellipse rotated 90 degrees. The family's orthocenter is stationary at the common center. No other invariants have yet been identified for this system.

V. Poristic: the only non-concentric system in the set, this system is in fact an image of system I under a rigid (variable) rotation about X1, and therefore shares all metric invariants with it (r/R, sum of cosines, etc.)