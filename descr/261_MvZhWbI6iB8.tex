Consider an a/b=1.5 elliptic billiard (black) and its family of N-periodics (blue), tangent to a confocal caustic (brown). Also shown are the foci f1 and f2, and the polygon P' (green) tangent to the N-periodics.

1) Top row: N=3

left: the product of sums of altitudes (dashed blue) from each focus to the sides of the N-periodic is constant, for odd N only. the feet of said altitudes lie on a first circle (red) whose radius is equal to the major axis of the caustic.

middle: the sum of squared altitudes (dashed green) from either focus to the sides of P' is invariant. the feet of said altitudes lie on a second circle (red) whose radius is equal to the billiard's major axis.

right: the left and right pictures superimposed.

2) Bottom row: N=4

left (product of sums not conserved, N even), middle, right: the same as above.

Soundtrack: Francisco Tárrega, "Capricho Árabe"

Note: Companion video for N=5,6: https://youtu.be/ZMHLmWXeKrM