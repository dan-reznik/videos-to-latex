Note: the invariants described in the video are a corollary of theorem 3 in [1] since the circle and polar polygon are invariant under the affine group.

The video shows several families of Poncelet N-periodics interscribed in concentric ellipse pairs which are not axis-aligned.

It also defines a "polar polygon" whose sides are bounded by the polars of the N-periodic vertices wrt to a fixed-radius circle concentric with the ellipses.

Let A  (resp. A') denote the area of the N-periodic (resp. polar polygon).

The main (experimental conjecture) is that over an N-periodic family, A/A' is invariant for all odd N, and A.A' is invariant for all N even.

[1] Arseniy Akopyan, Richard Schwartz & Serge Tabachnikov, "Billiards in Ellipses Revisited', European Journal of Mathematics, 2020. 

40 Accesses

0 Altmetric

Metrics

https://link.springer.com/article/10.1007/s40879-020-00426-9