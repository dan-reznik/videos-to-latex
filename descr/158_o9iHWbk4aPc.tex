This is a continuation of the work by A. Skutin. In [1] he proves an observation by Akopyan: given a circle-inscribed Poncelet family of triangles, the locus of the isogonal conjugate of a fixed point P wrt triangles in the family is a circle. In a previous video [2] we extended this to consider Poncelet triangles interscribed between two nested ellipses E,E' in general position.

video shows:

(1) the frontiers of P such that the locus of the isogonal of P over Poncelet is an ellipse, parabola, or hyperbola. In particular

(2) the locus of P such that the isogonal locus is a rectangular hyperbola is a circle! Finally, we show that if P  is on the outer conic

(3) the straigh-line isogonal loci envelop an ellipse which is confocal with E'.

[1] A. Skutin, "On Rotation of a Isogonal Point", Journal of Classical Geometry, vol. 2, 2013. https://jcgeometry.org/Articles/Volume2/JCG2013V2pp66-67.pdf


[2]  D. Reznik,  "Extending Sktutin's Result, Part I", https://youtu.be/x0953ASLkuk