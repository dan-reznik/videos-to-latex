An elliptic billiard (a/b=1.5) is shown (black). Also shown (blue) is the family of N=4 non-intersecting (quadrangular) orbits. These orbits are known to have (i) constant perimeter, (ii) be parallelograms, and (iii) have vertices of the tangential polygon (green) which move along a circle (dashed green). The confocal caustic to the orbits is shown as the green ellipse.

The feet of perpendiculars dropped from each vertex of the tangential polygon to the corresponding side on the orbit coincide with the points of orbit-caustic tangency (green points). Interconnecting these produces another parallelogram (red).

For each vertex of the tangential polygon, draw a segment passing through the aforementioned foot. These meet at four points (shown red), which connected also form a parallelogram (shown red), and whose locus is the 4-petal upright rose.

https://dan-reznik.github.io/Elliptical-Billiards-Triangular-Orbits/