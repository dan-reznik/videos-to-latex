This is a better resolution version of a recently posted video [1].

Consider the 3-periodic Poncelet family inscribed in an outer circle of radius R and circumscribed about an inner concentric ellipse of semiaxes a,b. For Such a family to exist, R = (a+b). Invariants associated with this family include the product of cosines and the sum of sidelengths squared.

The video shows the following phenomena:

a) by definition, the circumcenter X3 is stationary at the common center O.
b) the locus of both the orthocenter X4 and the 9-point center are circles centered on O.
c) the MacBeath inconic to the 3-periodics (whose center is X5 and foci are X3 and X4) is rotating ridigly about X3=O.
d) consider the "spokes" connecting X5 to each of the vertices. let their lengths be s_i, i=1,2,3. It turns out the sum of s_i^2 is invariant. This stems from Equation 4 in [1] which states that the sum of the squared distances from X5 to the vertices is 3R^2-|X3-X5|^2: recall both R and the radius of the circular locus of X5 around X3 are invariant. 

[1] E. Weisstein, "Nine Point Center", MathWorld, https://mathworld.wolfram.com/Nine-PointCenter.html

---
[1] https://youtu.be/x646JiYZSYI