The Poristic Triangle Family [1,2,3] is a set of triangles (blue) with a fixed Incircle (green) and Circumcircle (purple). Let the Circumcenter X3 be on the origin and the Incenter X1 be "d" units above it on the y axis. 

Odehnal showed [4] the vertices of the Excentral Triangle (green) sweep a circular locus (orange), centered on X40 and with radius 2R, where R is the circumradius (R=1 on the video).

This video shows two interesting phenomena:

a) the Circumbilliard [8] of the poristic family has fixed aspect ratio. This stems from the fact that the family has fixed r/R [7].

b) Amazingly. the inconic (red) to the Excentral Triangle centered on its circumcenter (X40 of the original triangle) has axes parallel to the CB, and intersects it at X100, which is also on the Circumcircle. Amazingly, the length of its semi-axes is invariant and given by R+d and R-d, where d=Sqrt[R(R-2r)]=|X1X3|. invariant, i.e., it is a fixed ellipse rigidly pivoting about X40. The sum of its semi-axes is 2R, which is the radius of the circular locus of the Excenters with which the invariant ellipse is concentric.

Sountrack: Beethoven, Moonlight Sonata.

References:

[1] William Chapple, Surveyor, https://en.wikipedia.org/wiki/William_Chapple_(surveyor)
[2] W. Gallatly, "The Modern Geometry of the Triangle", F. Hodgson, 1914.
[3] J. H. Weaver, "Invariants of a poristic system of triangles", Bull. Amer. Math. Soc., 33:2, 1927. https://projecteuclid.org/download/pdf_1/euclid.bams/1183492031
[4] Boris Odehnal, "Poristic Loci of Triangle Centers, Journal of Geometry and Graphics", 15(1), 2011. https://www.geometrie.tuwien.ac.at/odehnal/pltc.pdf
[5] MacBeath Inconic, https://mathworld.wolfram.com/MacBeathInconic.html
[6] Antiorthic Axis, https://mathworld.wolfram.com/AntiorthicAxis.html
[7] Dan Reznik, Ronaldo Garcia, and Jair Koiller, "Can the Elliptic Billiard Still Surprise Us?", Math, Intelligencer, 42, 2019, https://rdcu.be/b2cg1
[8] Dan Reznik and Ronaldo Garcia, "Circuminvariants of 3-periodics in the Elliptic Billiard, 2020. arXiv: https://arxiv.org/abs/2004.02680