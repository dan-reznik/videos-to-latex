Let E be an ellipse (black) with a,b semiaxes, and P be a point on its boundary. Consider a parallelogram Q (red) inscribed in E [note 1]. Drop perpendiculars from P onto the sides of Q, let their feet define a "pedal polygon"  Q' [note 2], shown black and pink-filled in the video.

Observation: for a given Q, the area of Q' is invariant over all P on the boundary of E. This is cool! Could be obvious by the way. An elegant one-line proof has been kindly offered by A. Akopyan:

The area of quadrilateral is the product of diagonals by sin of angle between them. So for any fixed parallelogram and any point P it is constant

note 1: this is a 1-d poncelet family w a confocal caustic, i.e., the parallelograms are 4-periodic billiard trajectories within E
note 2: Alternate pairs of feet are collinear with P (since alternate sides of Q are parallel). Let these feet defined a "pedal polygon"  (black, pink-filled).