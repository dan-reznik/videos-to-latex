A parameter t in 0 to pi/3 is used to parametrize a continous family of Brocard porisms, whereby the Brocard angles Ω1,Ω2 lie on two circular arcs (red) centered on the (fixed) Beltrami points P(2) and U(2).

Left: samples of the family of inellipses (gray) are shown having an elliptic envelope (thick black), with foci on the isodynamic points X15 (and X16, above the page, not shown). Picking a t amounts to selecting one inellipse and circumcircle (thick blue). The Brocard circle (dashed blue) is centered on X183, and contains both Brocard points Ω1,Ω2, X3 and X6. Both the latter and the circumcircle go start as an infinite circle (t=0), converging to X15 at t=pi/3. Notice the Brocard circle cuts the inellipse exactly where it touches the envelope. Notice the (fixed) Brocard angle ω of each porism is equal to t/2.

Right: inifnite nesting of the set of all circumcircles (gray) of the family. Notice these are perpendicular to the two Beltrami circles (red).