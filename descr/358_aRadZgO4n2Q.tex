The video describes two (yet unproven) phenomena regarding the loci of X2 and X3 of pedals and antipedals of Poncelet triangles in the homothetic family with respect to a point P on the outer conic E (Steiner ellipse). Let its aspect ratio be a/b. Namely:

a) the locus of X3 of pedals is a straight line (that of X2 looks like a cardioid)
b) the locus of X2 of antipedals is a stationary point (that of X3 is an axis-aligned ellipse)
c) the locus of (b) over all P is an ellipse E2 concentric and axis-aligned with E, and with aspect ratio is b/a.

d) for a given Poncelet triangle, over all P on E, the locus of the vertices of antipedals are 3 distinct ellipses Z1,Z2,Z3, all with aspect ratio b/a. These and E2 all meet at a point. 
e) over Poncelet, the sum of the areas of the Zi is invariant!

Cheers!