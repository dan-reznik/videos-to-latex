Two a/b=1.5 elliptic billiards (black) are shown as well as their 5- and 6-periodic families in blue (left and right, respectively). Also shown is the confocal caustics (brown) and the foci f1 and f2 of the Poncelet pairs.

From each f1 drop N perpendiculars to each side of the N-periodic. 

Observation 1: the feet of all 2N said perpendiculars lie on a circle of radius equal to the major semi-axis of the caustic.

Let d_i, i=1,2,3 be their lengths. Repeat from f2, let e_i be their lengths.

Observation 2: for odd N, the product of the sum of d_i's with the sum of e_i's is invariant, i.e.:

(d1+d2+d3+...+eN)*(e1+e2+e3+...eN)

As seen in the video, this product is not invariant for N=6, or even N in general.

Observation 3 (triple conjunction): when feet of perpendiculars cross each other on their circular locus, they do so when one of the orbit vertices lies at one of the 4 intersections of the locus with the Elliptic Billiard.

Soundtrack: Villa-Lobos