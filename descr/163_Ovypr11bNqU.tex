The outer Apollonius' circle C of a triangle is tangent to and encompasses the excircles. Its center is X(970) and the radius is (r^2+s^2)/(4r), where r is ithe inradius of T and s its semiperimeter [1,2].

The video studies C of the anticomplementary triangle (ACT) of an N=3 bicentric (poristic) family of Poncelet triangles [3].

Over the family (i) X(10) of the ACT is stationary, (ii) the locus of its vertices is a pseudo Pascal Limaçon, and (iii) the locus of X(970) of the ACT is a segment collinear with X(1) and X(3) of the bicentric family. Note that the radius of C is not stable.

Note: X(970) of the ACT is X(10441) of the reference [4, Part 6]. So we are showing that over the poristic family the locus of this center is a segment on the OI axis.

You can try out the simulation here: https://bit.ly/38S2OZS

[1] E. Weisstein, "Apollonius' Circle", MathWorld, 2021., https://mathworld.wolfram.com/ApolloniusCircle.html
[2] Grinberg, D. and Yiu, P. "The Apollonius Circle as a Tucker Circle." Forum Geom. 2, 175-182, 2002. http://forumgeom.fau.edu/FG2002volume2/FG200222index.html
[3] B. Odehnal, "Poristic Loci of Triangle Centers", J. Geom. Graphics 15/1 (2011), 45-67.
[4] C. Kimberling, ETC Part 6 (Centers 10001 to 12000), https://faculty.evansville.edu/ck6/encyclopedia/ETCPart6.html