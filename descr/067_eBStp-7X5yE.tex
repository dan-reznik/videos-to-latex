The envelope to a family of lines is that curve to which every member is tangent, aka as the Caustic [1]. We are intersested in the envelopes generated by pairs of Triangle Centers over the 3-periodic family.

An a/b=1.618 elliptic billiard (EB) is shown (black) as well as its family of 3-periodics (triangular orbits, blue).

Also shown are X1 and X5 riding along their elliptic loci (red, green, respectively). The X1X5 line is shown purple. Its envelope (purple) is the 4-cuspid astroidal curve. Miraculously, this is simultaneously tangent to the X1 and X5 locus! Any idea how?

Also shown is the evolute [2] of the elliptic billiard,which is the envelope of inward-pointing normals. This is also astroidal and shown dashed blue.  This curve has a pleasant closed-form parametric expression in terms of a,b the ellipse's axes [2]:

x(t) = (a^2-b^2) cos(t)^3 / a
y(t) = (b^2-a^2) sin(t)^3 / b

Let P1(t) be a vertex of a 3-periodic. The line P1X1 is parallel to the ellipse normal at P1, therefore this family yields the involute.

The instantaneous X1X5 line is shown purple having the purple envelope as its caustics. The instantaneous P1X1 line is drawn gray, and its caustic is the ellipse evolute.

Suprisingly both the evolute and the X1X5 caustic touch the X1 locus on the same four spots (marked by blue dots). One of them is given by:

x1= c2 ((-b2 + d)/c2)^(3/2)/a
y1 = c2 ((a^2 - d)/c2)^(3/2)/b

where  a2=a*a, b2=b*b, d = Sqrt[a2^2 - a2*b2 + b2^2], and c2=a2-b2

The Orthocenter X4 is shown as an orange dot. When it is on one of the 3-periodic vertices, the 3-periodic is a right triangle. There doesn't seem to be any specific phenomena tied to such configurations.

A large gallery of Triangle Center pairs and their envelopes is available in [3].

[1] http://mathworld.wolfram.com/Envelope.html
[2] http://mathworld.wolfram.com/EllipseEvolute.html
[3] https://dan-reznik.github.io/Elliptical-Billiards-Triangular-Orbits/envelopes1618.html