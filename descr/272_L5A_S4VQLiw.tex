Poncelet's Porism states that given two real plane conics C and D, if an N-gon can be inscribed to D while simultaneously circumscribing D (called the ``caustic''), then a 1d family of said N-gons exists, starting from a vertex of P anywhere on P. This is actually a special case of a more general result,  known as "Poncelet's Closure Theorem" (PCT), illustrated in the video. It can be stated as follows (see [1]): 

 Let C and Di, i=1,...,M be M+1 distinct conics in the same pencil [2]. If an N-gon can be found inscribed in C such that each side is tangent to some a D_i, then a 1d porism such N-gons exists.

The video illustrates two 5-gon families interscribed between one external circle (black) and 4 internal caustics (gold). In the left case, the family is simple, while in the second it contains one self-intersection.

[1] Del Centina, A. "Poncelet's Porism: a long story of renewed discoveries, I" Arch. Hist. Exact Sci., 70(1), 2014.
[2] L. Halbeisen and N. Hungerbühler, "A Simple Proof of Poncelet's Theorem (on the Occasion of Its Bicentennial)", Am. Math. Monthly, 
Vol. 122, No. 6, pp. 537-55, 2015.