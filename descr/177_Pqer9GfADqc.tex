The family of 3-periodics T in the elliptic billiard conserves perimeter. Recall its Mittenpunkt X9 is stationary at the common center. Now consider its derived excentral triangles T'. Its symmedian point coincides with X9 of the 3-periodics and is therefore also stationary at the center.

Recall the tangential triangle [1] has sides which pass through those of a reference triangle and are tangent to the circumcircle. Its Gergonne point X7 coincides with the reference X6, and it is homothetic to the orthic at X25 [1] (*). Let a triangle have sidelengths a,b,c. Courtesy of Peter Moses, the homothecy ratio is given by [2]:

K = (2 a b c)/Sqrt[(a^2 - b^2 - c^2)*(a^2 + b^2 - c^2)*(a^2 - b^2 + c^2)]

Peter adds: the sign decides if the vertices of the two triangles are on the same side from the homothetic center X(25)

Let T'' denote the family of tangential triangles of the excentral family T', which we will regard as the "reference" family.

X7'' = X6' = X9 are all stationary at the center. The T'' are homothetic to the T at X25 of the T'. 

It turns out over 3-periodics in the elliptic billiard, K is invariant. This implies the perimeter of the T'' is conserved!

(*) in the video I wrongly state the center of homotecy is X6.

[1] E. Weisstein, "Tangential Triangle", MathWorld, 2021.
[2] P. Moses, Private Communication, 4-Feb-2021.