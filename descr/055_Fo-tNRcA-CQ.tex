Consider an ellipse with semi-axes a,b. Choose two fixed points on it:

v1 = [a cos(t1), b sin(t1)]
v2 = [a cos(t2), b sin(t2)]

Let a third point P(t) ride on the ellipse, and let T be the triangle {v1,v2,P(t)}. Observations:

a) the locus of the orthocenter X4 of T is *always* an upright ellipse (axis aligned with the original one, but with major axis vertical and minor horizontal).

b) pick an angle "k". over all v1,v2 such that t1+t2=k, the locus generated is the same ellipse up to translation!

c) the maximal (resp minimal) area elliptic locus occurs for k = 180 (resp. 0) degrees.

to do:

- where is the center of the elliptic locus
- what are its axes
- what is its translation (looks like a straight line) as one visits all t1+t2=k?

Having fun folks!