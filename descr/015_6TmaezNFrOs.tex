The video narrates several curious properties of the bicentric family [1] and its pedal polygons wrt to its limiting points l1 and l2 [2] i.e., the two points wrt which the inversion of the bicentric circle pair yields a concentric pair (I messed up in the video, sorry). The video shows that

a) the sum of cosines of bicentrics is conserved.
b) limiting pedals are inscribed in separate, touching Pascal Limaçons, one loopless and the other with a loop through l2.
c) the perimeter of both limiting pedals is invariant.
d) the sum of cosines of both limiting pedals is invariant (except for when N=4, where the l1-pedal has variable sum of cosines).
e) in the N=4 case,  the vertices of the l2-pedal are collinear.
f) in the N=3 case the bicentrics are the poristic family [3]. The sum of cosines of bicentrics, l1- and l2-pedals are the *same*. Their Gergonne point X7 is stationary.

For (f) we prepared a set of animations here: dan-reznik.github.io/ellipse-mounted-loci-p5js/?juke=6

Note: over the poristic family, the Gergonne point X7 moves along a circle [3].

[1] E. Weisstein, "Poncelet Porism", MathWorld, 2021. https://mathworld.wolfram.com/PonceletsPorism.html
[2] E. Weisstein, "Limiting Points", MathWorld, 2021. https://mathworld.wolfram.com/LimitingPoint.html
[3] B. Odehnal, "Poristic Loci of Triangle Centers, Journal for Geometry and Graphics 15(1) , 2010.