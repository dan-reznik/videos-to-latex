An a/b=1.5 elliptic billiard is shown (black) as well as its family of N=5 (pentagonal) non-intersecting orbits (blue). Also shown is the orbits' excentral (exterior) polygon P' (solid green) and P'', the latter's reflection about the center of the billiard (dashed green). It turns out if, for any N odd, the locus of intersection of the ith edge of P' with the jth edge of P'', j=(N-1)/2, is circular.

We call this stationary circle the Monge-Darboux Circle in homage to the great French Geometers Gaspard Monge (1746-1818) and Jean-Gaston Darboux (1842-1917) which studied closely related geometries (Monge's Orthoptic Circle and the Poncelet-Darboux Grid of edge-edge intersection loci).

More Info: https://dan-reznik.github.io/Elliptical-Billiards-Triangular-Orbits/