Shown is a family of triangles P1P2P3 inscribed in an outer circle C1 (black) with side P1P2 tangent to a 2nd circle C2 (light brown), side P1P3 to a 3rd circle C3 (dashed brown) in the pencil of (C1,C2), and with side P2P3 automatically tangent to to a circle C4 (dashed red) also in the pencil of C1,C2. They are called "quadricentric" since this family is defined by 4 non-concentric circles.

Left: the non-conic loci of the incenter X1 (dark green) and of the 3 excenters (light green), for the case where C4 (dashed red) has a greater-than-zero radius.

Right: C2C3 are chosen so that the envelope of C4 is a point. Even in this case the loci of the three excenter are non-conic (as is that of the incenter), though the locus of the excenter opposite to P1 is nearly a circle.