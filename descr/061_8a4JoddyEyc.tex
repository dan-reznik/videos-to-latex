Let a vertex P1 of a 3-periodic in en Elliptic Billiard (black) be parametrized as P1=[a cos(t), b sin(t)], and its copy P1', lying 90 degrees ahead as: P1'=[a cos(t+pi/2), b sin(t+pi/2)] = [a sin(t), - b cos(t)]. The 3-periodic initiating at P1 (resp. P1') is shown blue (resp. dashed blue).

The video shows the family of lines [P1,P1'] (red) and the caustic they envelope (green): an ellipse similar to the EB itself. The tangency point "C" is dynamically the midpoint between P1 and P1'.

The above can be proven via an affine transformation which takes the EB to a circle (multiply the x coordinate by b/a). In this new ambient, the billiard is a circle, and the 3-periodic and its forward doppleganger are equilaterals, and the internal caustic is the common incircle. By symmetry, these equilaterals will be tangent to the incircle at their midpoints. At the original space, the caustic is the inverse affine transform of the incircle.