An a/b=1.5 elliptic billiard is shown (black) as well as the 1d-family of 3-periodic billiard orbits (blue). Left is a zoomed out view (always showing the excentral triangle -- green -- in its entirety) and the right one zooms in on the billiard.

Also shown (red) is the Thomson Cubic [1] of the 3-periodics, given by the polyomial which a set of trilinears [x:y:z] on the plane of a triangle T satisfy [1]:

b c x (y^2-z^2) + c a y (z^2-x^2) + a b z (x^2-y^2)

where a,b,c are the sidelengths of T. This magical cubic passes through:

a) the vertices of T
b) the sides' midpoints [medial triangle is drawn dashed blue]
c) the altitudes' midpoints (not shown)
d) the excenters (vertices of the excentral triangle is drawn green)
e) Kimberling Centers Xi, i=1,2,3,4,6,9 (shown black)
f) Kimberling Centers Xj, j=57 [isogonal conj. of X9], 223, 282, 1073, 1249, 3341, 3342, 3343, 3344, 3349, 3350, 3351, 3352, 3356, 14481.

Because over the 3-periodic family X(9) is stationary at the Billiard center, the Thomson cubic is guaranteed to pass there. Questions we are asking:

1) are there any quantities associated with this cubic (e.g. some sort of discriminant) remain invariant over the family.
2) what are the 3 intersection points between the cubic and the elliptic billiard? Is this a known triangle? does it have invariant properties over the family? (e.g., area, perimeter, sum of sides squared, etc.)

[1] http://mathworld.wolfram.com/ThomsonCubic.html 
[2] https://bernard-gibert.pagesperso-orange.fr/Exemples/k002.html