Take the family of Poncelet N-periodics (blue polygons) interscribed in a pairof confocal ellipses (black and brown). Now invert its vertices with respect to a unit circle (dashed black) centered on one focus. This gives rise to an "inversive" polygon (pink), inscribed in a Pascal Limaçon (olive green). Elsewhere we showed this has (i) constant sum of spokes connecting the focus to its vertices, (ii) constant perimeter, and (iii) constant sum of cosines (except for N=4).

Here we consider the inversion of the *segments* of the original N-periodics wrt to said unit circle. Namely, each segment inverts to an arc (blue) circumscribed about the Limaçon. The fact that lines invert to circles with is well-known. What may be new is (i) the fact that the centers C1,C2,...CN, of said arcs are concyclic (the center is shown as C0), and that furthermore, the polygon (call it Q) formed by the Ci is bicentric, i.e., its caustic is also a circle (center shown as C0').

Furthermore, for any N, the following are invariant for this bicentric family:

- ratio of perimeter to area (though neither is constant)
- sum of cosines
- ratio of its area to that of the inversive polygon.