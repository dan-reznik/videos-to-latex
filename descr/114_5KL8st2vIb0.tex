Elliptic Billiards (EB) with a/b=1.618 (golden ratio) and a/b=1.982 are shown on left and right respectively (black). Also shown is their 1d family of 3-periodics (black triangles) and their orthic triangles (solid orange). Around the orthics we draw their Circumbilliards (solid orange), i.e., ellipses to which the orthics would be 3-periodics. These are centered on the orthic's Mittenpunkt X9*.

Also drawn is the locus of 3-periodics' Symmedian Point X6 (light blue) and that of X9*. We know the former is a quartic [1]. Notice the latter detaches from the former when the 3-periodic is obtuse. This implies the locus of X9* cannot be analytic!

On the right a/b is chosens such that the locus of X9* touches the top and bottom vertex of the EB. When the 3-periodic is an upright isosceles, X9* is there, and the orthic is an equilateral! This can be shown by observing both its Mittenpunkt and Incenter coincide at the EB's top vertex: the incenter of the orthic of an obtuse triangle is at the triangle's obtuse vertex [2], i.e., at the EB top vertex.

Note: when the orbit is a right-triangle, the orthic is degenerate: it becomes a segment congruent with the altitude of the obtuse vertex. X6 and X9* coincide and are the midpoints of said segment.

[1] Garcia et al., Loci of 3-periodics in an Elliptic Billiard: why so many ellipses? 2020, arXiv: https://arxiv.org/abs/2001.08041

[2] Reznik et al., The Ballet of Triangle Centers on the Elliptic Billiard. 2020,  arXiv: https://arxiv.org/abs/2002.00001