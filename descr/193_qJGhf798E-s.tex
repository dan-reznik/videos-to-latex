Left: A family of triangles T=P1P2P3 is shown inscribed in an outer (black) circle centered on X_3. Sides P1 P2 and P1P3 (blue) are tangent to a second inner circle (brown), centered on O'. Over the family, the third side (red) envelops a third, in-pencil circle (dashed red), centered on Oenv.

Main result: the locus of X1 (green) is a circle, for any choices of outer and inner circle.

Also shown are Bevan point X40 (orange), and the excentral barycenter X165, which lie on the X1X3 (OI) line. In fact, X40 (resp. X165) is the reflection of X1 about X3 with a scale of 1 (resp. 1/3). Therefore the loci of X40 and X165 are scaled versions of that of X1.

Right:  a similar setup but with a larger inner circle (brown) and wider distance between X3 and O'. Miraculously, all loci remain circular.