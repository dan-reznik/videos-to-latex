Shown is the family of 3-periodics in the elliptic billiard (blue triangle) whose vertices are bisected by the ellipse normals. The Cosine Circle (red) of the excentral triangles (green) [1] (aka., 2nd Lemoine) is stationary.

For any triangle the Symmedian X6 of the excentral coincides with the Mittenpunkt X9. Recall the Cosine Circle of a triangle passes through the 6 points of intersection of the "antiparallels" through the symmedian point X6. These are the sides of the tangential triangle, which are parallel to the orthic. Recall the reference triangle is the orthic of the excentral.

To construct it, take a first orbit vertex P1 and reflect it about X9, which lies stationary at the center of the billiard, yielding P1'. Compute the two intersections Q1 and Q2 of the tangent to the billiard at P1' with the orbit's excentral triangle (shown green). The excentral Cosine Circle is the locus of these points [1].

[1] http://mathworld.wolfram.com/CosineCircle.html
More Info: https://dan-reznik.github.io/Elliptical-Billiards-Triangular-Orbits/