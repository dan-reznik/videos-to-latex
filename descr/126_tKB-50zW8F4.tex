Take the 3-periodic family T (blue) in the elliptic billiard. Now compute its "focus-inversive" triangle T' (pink) whose vertices are inversions of the original with respect to a unit-circle (dashed black) centered on one focus. The T' are inscribed in Pascal's Limaçon (dashed olive green). That the perimeter of the T is constant is a classic elliptic billiard invariant. Strinkingly, the perimeter of the T' is also constant!

The video shows the loci of triangle centers Xk, k=1, 2, 3, 4, 5, 9, 10, 11, 20, 40, 100. They are all CIRCLES!!!

Not shown: The Gergonne Point X7 is conspicuously stationary on the x-axis, see [1]

In fact, from X1 thru X100, a whopping 28 Kimberling centers of the T' produce circular loci:

Xk, k = 1, 2, 3, 4, 5, 9, 8, 10, 11, 12, 20, 21, 35, 36, 40, 46, 55, 56, 57, 63, 65, 73, 78, 79, 80, 84, 90, 100.

Why so many circles ???

[1] D. Reznik, "Focus-Inversive Elliptic Billiard 3-periodics: Invariant-perimeter & inscribed in Pascal's Limaçon!", YouTube, 2020, https://youtu.be/LOJK5izTctI