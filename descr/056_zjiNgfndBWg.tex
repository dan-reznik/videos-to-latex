A triangle V1 V2 P(t) (blue) is inscribed in an ellipse E (black) with semi-axes a,b (in the video a/b=1.5). While V1,V2 are fixed, P(t) slides along E's boundary. Show are the loci of the barycenter X2 (brown), circumcenter X3 (red), orthocenter X4 (orange), collinear on the Euler line (dashed green). 

As it turns out, for any choice of fixed V1,V2, the locus of:

- X2 is an ellipse axis-aligned w E, with semi-axes a/3,b/3
- X3 is a segment
- Z4 is an ellipse axis-aligned w E, with aspect ratio equal to a/b.

An intermediate point X(ρ)=X2+ρ(X4-X2) is shown which is a fixed linear combination of X2 and X4 (in the video ρ=0.5, which makes X(ρ) be X381 [1]). For whichever choice of V1,V2 and ρ, the locus of X(ρ) is always an ellipse (green), in general not axis-aligned w E.

[1] C. Kimberling, "Encycl. of Triangle Centers", 2020. https://faculty.evansville.edu/ck6/encyclopedia/ETC.html