Consider the family of Poncelet 3-periodics (blue) inscribed in an ellipse (a,b) and circumscribed about a concentric, axis-aligned caustic (a/2, b/2). This is known as the "homothetic pair". We have shown elsewhere [1] that this family conserves area, sum of squared sidelengths, and therefore Brocard angle.

The video shows the following additional properties:

a) the loci of the Brocard points Ω1 and Ω2 are two tilted, symmetric ellipses similar to the ones in the pair. These can be interior (a/b below 2.23...) or exterior to the caustic.
b) the locus of the vertices of the First Brocard Triangle (FBT) [2] is a horizontal ellipse, also similar to the ones in the pair. This is always interior to the caustic.
c) the area of the FBT is invariant
d) the ratio of 3-periodic circumradius to that of the FBT is invariant

[1] D. Reznik, "Poncelet 3-Periodics in the Homothetic Pair conserve Brocard angle", YouTube, 2020. https://youtu.be/2fvGd8wioZY
[2] E. Weisstein, "First Brocard Triangle", Mathworld 2020. https://mathworld.wolfram.com/FirstBrocardTriangle.html