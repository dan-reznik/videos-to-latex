Shown is a "tricentric" Poncelet family of triangles P1P2P3 inscribed in an outer circle C with diameter R. Sides P1P3 and P1P3 (blue) are kept tangent to an internal circle C' (the first "caustic", brown), with radius r. In this case, and according to Poncelet's General Closure Thm, side P2P3 (red) will envelop a circle (dashed red) in the pencil of C, C', i.e., a second "caustic".

Shown are the loci of X1 (green) and X2 (purple) over variable "r". Amazingly, the former is always a circle while the latter is a loopy sextic. 

Notice that when "r" is such that the system is poristic (all sides are tangent to C'), the locus of X1 collapses to a point and that of X2 becomes a circle, whose radius and position were derived in [1].

[1] B. Odehnal, "Poristic Loci of Triangle Centers", J. Geom. Graphics 15/1 (2011), 45-67.