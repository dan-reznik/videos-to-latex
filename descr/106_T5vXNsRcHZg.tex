An a/b=1.5 elliptic billiard is shown as well as its family of triangular (N=3) orbits. For each orbit we show the Darboux line (connect X144 to X7 with X9 as its midpoint) and two hyperbolas associated with the triangle:

1) the Feuerbach Circumhyperbola [2] (shown brown), passing through the three vertices, the Incenter X1 and the Orthocenter X4 (shown brown) -- its center is on the Feuerbach point X11, and therefore on the caustic (not shown). This hyperbola also passes through the Mittenpunkt X9, stationary at the center of the billiard, as well as through  X7 (Gergonne), X8 (Nagel), X21 (Shiffler), X79, X80, X84 (isogonal conjugate of X40 see below), X90, etc. The locus of X4, known to be similar to the billiard, is shown in orange. 

2) the Excentral Hyperbola (shown brown), passing through the Incenter X1, 3 excenters and the Mittenpunkt X9. Its center lies on X100, the anticomplement of the Feuerbach, and therefore on the billiard. This hyperbola also passes through X40 (the Bevan point, circumcenter of excentral triangle and meetpoint of perpendiculars from excenters to each corresponding side). The locus of X40, known to be similar to the billiard is shown in blue. Prof. Igor Minevich, Rose-Hulman Institute of Technology, has pointed out this curve is the Jerabek Hyperbola [1] of the excentral triangle.

Notice both hyperbolas shown are rectangular (a=b) and have asymptotes parallel to the billiard. The latter property is not understood. Seems to be related to the fact that the axes of the X1-centered circumellipse are axis-aligned w/ Billiard (though the X2-centered one is not).

Also shown is the "Darboux Line" (in purple) with 3 of its points: X144 (Darboux Point) on one end, X7 (Gergonne Point) on the other, and X9 (Mittenpunkt) in the middle. We have found this line to be always tangent to the Excentral Hyperbola at X9.

[1] http://mathworld.wolfram.com/JerabekHyperbola.html
[2] http://mathworld.wolfram.com/FeuerbachHyperbola.html

More Info: https://dan-reznik.github.io/Elliptical-Billiards-Triangular-Orbits/