Shown is a continuous sweep of the 1d family of triangular orbits in an elliptical billiard. Point P1 in the orbit is computed with (a/b Cos(t), Sin(t)) for some t in 0 to 2pi. For every t, P2 is found numerically such that it yields a perfect triangular orbit.

It turns out the locus of the incenter (the intersection of all bisectors and/or the center of the inscribed circle, dashed green) is a perfect ellipse. This has been recently proven here https://arxiv.org/abs/1304.7588 and here ttps://www.groundai.com/project/centers-of-inscribed-circles-in-triangular-orbits-of-an-elliptic-billiard/

Also shown is the locus of the P1P2 "apothem" (the tangent point of the inscribed circle with the P1P2 side). Notice its locus is an involute with two internal loops, and that the other two apothems follow the same exact path, but out of phase by 1/3 and 2/3 of the cycle. The nature of this curve has not yet been studied.

More info: https://dan-reznik.github.io/Elliptical-Billiards-Triangular-Orbits/