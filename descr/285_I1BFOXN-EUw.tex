Family of N=3 Poncelet Polygons inscrived in an ellipse (a,b) and which circumscribe stationary a circle C centered on the origin. For N=3 to work, the radius of C must be: r = a*b/(a+b). We have noticed:

a) the circumradius is constant across the family, R = (a+b)/2.
b) the locus of the circumcenter X3 is circular and its radius R3 = (a-b)/2
c) the locus of the nine-point circle center is circular. 
d) the locus of the barycenter X2, orthocenter X4 is elliptic.
e) the locus of the feuerbach point X11 is the C (by definition as C is the fixed incircle of all polygons). 
f) [e] implies X1, the incenter, is statinary.
g) surprisingly, the locus of X100, the anticomplement of X11, is the ellipse.