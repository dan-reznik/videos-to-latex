Top and bottom pictures illustrate an interesting property of N=3 orbits in an elliptic billiard related to their anticomplementary (blue, top) and medial triangles (red, bottom).

Starting with the top picture, one can see that the anticomplementary triangle's incircle (blue) has its three contact points on the billiard (black ellipse), a surprising phenomenon.

Likewise, on the bottom picture the contact points of the orbit's incircle are on the medial triangle's circumbilliard (the circumconic centered on X(142), its moving Mittenpunkt, the complement (half-reflection about about X(2)) of the orbit's X(9), which is stationary.

Also shown are:

- the non-elliptic loci for vertices of the anticomplementary (dashed blue, top) and the medial triangles (dashed red, bottom).
- the N=3 caustic to the orbits (brown), swept by the Feuerbach point X(11).
- The Feuerbach of the anticomplementary (F bar), X(100) shown sweeping the billiard.
- Top: the Mittenpunkt X(9), the barycenter X(2), and the Gergonne point X(7) all lie on the same line. The latter coincides with the anticomplementary's Mittenpunkt, and is therefore the moving center of its circumbilliard (blue ellipse). Not shown: X(144), the "Darboux" point, which is collinear with the latter 3.
- Bottom: the Mittenpunkt X(9), the baricenter X(2), and X(142), the complement of the Mittenpunkt (shown as T) are collinear. The latter coincides with the Mittenpunkt of the medial triangle, i.e., it will be moving the center of its circumbilliard (red ellipse) 
- The four points: Darboux X(144), Mittenpunkt X(9), Barycenter X(2), X(142), and Gergonne X(7) are collinear in this order, and X(9) is the midpoint between X(144) and X(7).

The two pictures are not drawn to scale but are registered to the position of point p1 in the orbit.

https://dan-reznik.github.io/Elliptical-Billiards-Triangular-Orbits/