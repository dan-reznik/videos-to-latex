Consider a Poncelet family of N-gons interscribed between an outer conic E and an inner conic E'. Let ri be the ith exradius of an N-gon, i=1,...,N.  The video shows that for all N:

a) sum(1/ri) is invariant if E' is a circle.

b) sum(ri) is also invariant if E' is a concentric circle, or the pair (E,E') is a bicentric pair. 

For N=3 (a) is obvious since sum(1/ri)=1/r, where r is the (fixed) inradius, and (b) is a corollary of relation [1, eqn. (4)]:

r1+r2+r3=4R+r

noting that the circumradius R is fixed [2, Thm. 1] over this family (the inradius r is constant by definition).

Note: toward the last quarter of the video I was trying to say that sum(1/ri) though invariant whenever the caustic is a circle, it is only equal to 1/r if N=3. 

[1] E. Weisstein, "Exradius", MathWorld, 2021. https://mathworld.wolfram.com/Exradius.html
[2]  Ronaldo Garcia and Dan Reznik. Family ties: Relating Poncelet 3-periodics by their properties. J. Croatian Soc. for Geom. & Gr. (KoG), to appear, 2021. arXiv:2012.11270