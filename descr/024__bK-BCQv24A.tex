The Brocard Porism [0] is a 1d family of Poncelet 3-periodics (blue) inscribed in a circle (black) and circumscribed about the Brocard inellipse (also black). Parametrics for their vertices were obtained from [1].

The 1st, 2nd, and 7th (*) Brocard Triangles [2] are know to be inscribed in the Brocard circle (green), and are concyclic with both Brocard points Ω1, Ω2, and X3, and X6. The 5th Brocard is homothetic to the reference, and its circumcenter is X9821 [3]. Therefore the locus of the vertices of the 1st, 2nd, and 7th are the Brocard circle itself, whereas that of the 5th is a circle centered on X9821. 

The loci of 3rd, 4th, and 6th triangles (not shown) are complicated curves.

You can also visualize these phenomena on our interactive app [4] here: https://bit.ly/32GFvQu

Note (*): The 7th Brocard Triangle was invented with Peter Moses in Sept. 2020.

References. 
[0] R. Johnson, "Advanced Euclidean Geometry" (Chapt XVII), Dover, 1960.
[1] R. Garcia, Vertices of Brocard-Poristic Triangles, Private Comm., Sept 2020.
[2] B. Gibert, CTC, https://bernard-gibert.pagesperso-orange.fr/gloss/brocardtriangles.html
[3] C. Kimberling, ETC (Part 5), 2020. https://faculty.evansville.edu/ck6/encyclopedia/ETCPart5.html
[4] I. Darlan and D. Reznik, Loci of Ellipse-Mounted Triangles, 2020, https://dan-reznik.github.io/ellipse-mounted-triangles/