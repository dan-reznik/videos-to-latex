This is joint work w P. Roitman, A. Akopyan, S. Tabachnikov, H. Stachel, R. Garcia, and J. Koiller.

Shown is the family of N-periodics (blue) inscribed in an outer ellipse (black) and circumscribed about an inner, confocal caustic (brown). For this video N=5, but the obs herein apply to any N. Recall the confocal Poncelet pair is also known as the Elliptic Billiard since any trajectory (closed or not) is automatically bisected by the local normal.

Consider the set of segments (dashed blue) connecting one focus (say the left one, call it f1) to vertices P(i) of the N-periodic. Let d1(i) denote their lenghts. It turns out that for any N, the sum of 1/d1(i), i=1,...,N is invariant. 

Also drawn (dashed red) are the segments from f1 to the vertices P''(i) of the so-called "inner" polygon (red, defined by the points of tangency of the N-periodic with the caustic). 

The video shows the f1P(i) segments (solid red) inverted with respect to a unit-radius circle centered on f1 and f1P''(i) (solid blue) inverted wrt to same circle. These inversive spokes sweep two Pascal Limaçons (drawns black and brown, respectively). 

The inversive image of the N-periodic segments are circular arcs inscribed on the outer limaçon (brown, inv. image of caustic) and the inner one (black, inv. image of outer ellipse). Interestingly, the center of said arcs C1,C2,...,CN are concyclic on a base circle (dashed blue).