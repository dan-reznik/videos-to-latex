We explore the family of tangential triangles [1] to Poncelet N=3 "bicentrics" (aka "poristics"). We show that if X3 is interior (resp. exterior) to the incircle, the tangentials will be inscribed in an ellipse (resp. hyperbola). If X3 is on the incircle, i.e., r/R = sqrt(2)-1, the family is inscribed in a parabola w focus on X1' of the tangentials (X3 of the poristics).

A nice phenomenon is that the locus of their tangential orthocenter X4'  is an line ellipse (resp. hyperbola) if the bicentrics' X3 is interior (resp. exterior) to theincircle. If X3 is on the incircle, then the locus of X4' is an infinite, vertical line. 

You can play with the simulation here: https://bit.ly/3a4a1a5

[1] E. Weisstein, "Tangential Triangle", MathWorld 2021. https://mathworld.wolfram.com/TangentialTriangle.html