An elliptic billiard (EB) is shown with aspect ratio a/b=1.618...=golden ratio. Also shown is its family of 3-periodics (blue triangles). Also shown are Triangle Centers [1] X(48) and X(37143) [formerly the isotomic conjugate of X(30565)]. The former's locus is an upright convex curve (green, not a proper ellipse [3]), whereas the latter rides along the EB (it is a "swan" [4]). Also shown is the envelope [2] (purple) of the family of lines defined by the pair. Holy Mackerel, this looks like a sideways bat!

For the aspect ratio a/b=golden ratio chosen, we do not understand why:

(i) both centers move monotonically with respect to the 3-periodic family;
(ii) the envelope has 8 tangency points with the locus of X(48); 
(iii) when "C" passes thru these so does X(48). why?
(iv) four tangency points intersect the Elliptic Billiard precisely where the (non-elliptic) locus of X(48) intersects it;
(v) there are 8 cusps;
(vi) two of which coincide with the billiard's foci;
(vii) what kind of curves are those between the cusps? could they be elliptic arcs?

For additional harmonies when "C" lies at the cusps and/or tangency points of the envelope with X(48), visit [5].

[1] https://faculty.evansville.edu/ck6/encyclopedia/ETC.html
[2] https://mathworld.wolfram.com/Envelope.html
[3] https://arxiv.org/abs/2001.08041
[4] https://arxiv.org/abs/2002.00001
[5] https://dan-reznik.github.io/Elliptical-Billiards-Triangular-Orbits/envelopes1618.html