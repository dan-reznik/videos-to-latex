Two elliptic billiards are shown: a/b=1.24 (left), and a/b=1.58005 (right). The locus of X(59) is shown [1] over the family of N=3 orbits. This locus has four self-intersections, two on the x and two on the y axis. It is at least a sextic curve (intersect a line parallel and near the y axis w/ curve and get 6 intersections).

When a/b is less (resp. more) than sqrt(2*sqrt(2)-1)) = 1.352..., the N=3 family only contains acute (resp. both acute and obtuse) triangles [2]. Therefore the left (resp. right) billiard only contains acute (resp. both acute and obtuse) triangles.

The right billiard is very special as when X(59) crosses one of the self-intersections, the orbit triangle is a perfect right triangle (X(4), shown, will be at an alternate vertex).

Open challenges: let one vertex P1 of the orbit be give by P1(t) = (a cos(t), b sin(t)).

1) Compute expressions for t (in terms of a,b) such that X(59) is at the lower self-intersections w/ the y-axis.
2)  Compute an expression for a/b such that when X(59) is at the lower self-intersection with the y-axis, the orbit is a right-triangle. (we know a/b~1.58). "t" for this position should be obtainable from (1).

[1] Clark Kimberling, "Encyclopedia of Triangle Centers", https://faculty.evansville.edu/ck6/encyclopedia/ETC.html
[2] Dan Reznik, Ronaldo Garcia and Jair Koiller, "New Properties of Triangular Orbits in Elliptic Billiards", 2020. In preparation.