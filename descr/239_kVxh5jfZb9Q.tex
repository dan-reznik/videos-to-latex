The 5- and 6-periodic families in elliptic billiard are shown on the left and right, respectively. Let P(i), i=1,...N denote their vertices, interscribed in a confocal pair with foci f1 (left) and f2 (right). Also shown are the circumcircles (dashed purple) of triads P(i),P(i+1),f1 and the polygon (purple)  defined by their centers. Notice how this polygon is homothetic to the antipedal polygon (solid green) of the N-periodics wrt f1.

Let R(i) denote the circumradius of the above circumcircles. We have observed that the product of the R(i) is invariant when N=2 (mod 4).