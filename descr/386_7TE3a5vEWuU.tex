We've been interested in properties of N-periodics in the Elliptic Billiard (EB) [1,2]. 

Let P_i denote the vertices of an N-periodic, i=1,...,N. Consider a point M. Let the feet Q_i of perpendiculars dropped from M onto sides (P_i,P_{i+1}) be the vertices of the so-called "Pedal Polygon with respect to M".

If M is a focus of the EB, then over the family of N-periodics two remarkable things happen:

a) the Q_i move along a circle (shown dashed) concentric with the EB and of radius equal to the major axis of the confocal caustic to the N-periodics (not shown).

b) the caustic to the family of pedal polygons is _also_ a circle (shown solid) whose center lies on the major axis of the EB.

Therefore the family of pedal polygons with respect to a focus of the EB is Ponceletian, with outer and inner conics being the [a] and [b] non-concentric circles.

An explanation for the phenomenon has been kindly extended [3]: the polar image of an ellipse with respect to a circle with center at one focus is also circle.  So, the N-periodic between confocal ellipses is a dual to the Poncelet trajectory between two circles.

The video shows said pedal families for N=3,4,5,6.

Soundtrack: Alexander Borodin, "Polovetsian Dances"

References:

[1] Dan Reznik, Ronaldo Garcia, Jair Koiller, "Can the Elliptic Billiard Still Suprise Us?", Math. Intelligencer, Vol 42, Dec 2019, https://rdcu.be/b2cg1

[2] Dan Reznik, Ronaldo Garcia, Jair Koiller, "Forty New Invariants of N-Periodics in the Elliptic Billiard", April 2020, https://arxiv.org/abs/2004.12497

[3] Arseniy Akoypan, Private Communication, April, 2020.