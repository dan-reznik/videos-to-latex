Two instances of the Poristic Triangle family [2,3] discovered by William Chapple in 1746 [1]. This family (blue) has a fixed incircle (green) and fixed circumcircle (purple), centered on the Incenter X1, and Circumcenter X3, respectively. Chapple, and later Euler, discovered that:

d=|X1X3|=Sqrt[R(R-2r)].

Where r,R represent the radius and circumradius. Note this implies: d no smaller than 2 [1].

For both the left (resp. right) simulation, R=1, and  r=0.3627 (resp. r=0.4494).

Also shown are the Excentral Triangles (green) and the locus (orange) of their vertices: a circle centered on X40 and of radius 2R, discovered by Boris Odehnal [4].

The main phenomenon portrayed refers to the two ellipses shown:

1) green: C1 = The X1-centered circumconic to the poristic triangles. Its semiaxes are *invariant* over the family at R+d, and R-d where d=Sqrt[R(R-2r)], i.e., C1 is a fixed ellipse rotating around X1.

2) red: I3' = The X40-centered Inconic to the Excentral Triangles (i.e., centered on the latter's circumcenter). Its semiaxes are *also* invariant and equal to R-d and R+d. I.e., this is a fixed ellipse rotating around X40.

Notice both meet the circumcircle at X100. 

The main result: C1 is a 90-degree rotated copy of I3'.

More info in [5,6].

Soundtrack: Beethoven, "Pathétique", 3rd movement.

[1] William Chapple, Surveryor: https://en.wikipedia.org/wiki/William_Chapple_(surveyor)
[2] W. Gallatly, "The Modern Geometry of the Triangle", F. Hodgson, 1914.
[3] J. H. Weaver, "Invariants of a poristic system of triangles", Bull. Amer. Math. Soc., 33:2, 1927. https://projecteuclid.org/download/pdf_1/euclid.bams/1183492031
[4] Boris Odehnal, "Poristic Loci of Triangle Centers, Journal of Geometry and Graphics", 15(1), 2011. http://www.heldermann-verlag.de/jgg/jgg15/j15h1odeh.pdf
[5] D. Reznik and R. Garcia, "Circuminvariants of 3-Periodics in the Elliptic Billiard", arXiv: https://arxiv.org/abs/2004.02680
[6] Ronaldo Garcia and Dan Reznik, "Related by Similiarity: Poristic Triangles and 3-Periodics in the Elliptic Billiard", 2020, arXiv: https://arxiv.org/abs/2004.13509