A light ray (red arrow) injected in the vertical direction into a reflective ellipse from a varying point on its boundary. The injection angle with the vertical is varied from -90 to 90 degrees. A 200-reflection trajectory is shown, revealing certain well-known patterns, chiefly that all rays are tangent to a virtual confocal conic, known as the "caustic": an ellipse if the initial ray does not pass between the foci, and a hyperbola if it does [1].

For most injection angles, the trajectory is space-filling, however, for a dense set, it becomes N-periodic, i.e., polygonal with N sides. This happens when a certain quantity τ (the translation in the billiard map) divided by 2π is a rational number [1].

See [2] for a gallery of N-periodics obtained from this experiment and [3] for further experimental work. Also, we've uploaded 100s of videos on this channel on youtube.

[1] S. Tabachnikov, "Geometry and Billiards", 1991. http://www.personal.psu.edu/sot2/books/billiardsgeometry.pdf

[2] D. Reznik, "N-periodics in the elliptic billiard", 2011, http://www.personal.psu.edu/sot2/books/billiardsgeometry.pdf

[3] D. Reznik, R. Garcia, and J. Koiller, "Invariants of N-Perioics in the Elliptic Billiard", 2019.