Consider an a/b=1.5 elliptic billiard (black) and its family of N-periodics (blue), tangent to a confocal caustic (brown). Also shown are the foci f1 and f2, and the polygon P' (green) tangent to the N-periodics.

1) Top row: N=5

left: the product of sums of altitudes (dashed blue) from each focus to the sides of the N-periodic is constant. the feet of said altitudes lie on a first circle (red) whose radius is equal to the major axis of the caustic. this is only conserved for odd N.

middle: the sum of squared altitudes (dashed green) from either focus to the sides of P' is invariant. the feet of said altitudes lie on a second circle (red) whose radius is equal to the billiard's major axis.

right: the left and right pictures superimposed.

2) Bottom row: N=6

left (not conserved, N is even), middle, right: the same as above.

Soundtrack: Best of Francisco Tárrega

Note: companion video, for N=3,4: https://youtu.be/MvZhWbI6iB8