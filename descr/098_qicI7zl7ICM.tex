Erratum: many times in the video I say "focus locus" instead of "vertex locus".

In a previous video we saw that over Poncelet triangles (blue) inscribed in a circle (black) and circumscribing a concentric inellipse (brown), the locus of the vertices of inparabolas (magenta) with focus at a fixed point F on the circumcircle is a circle (solid green) through F and tangent to the inellipse.

A corollary is that the locus of the reflection of the focus on the vertex ("center" of directrix) is a double-radius circle (solid orange), also passing thru F, but centered at the tangency w the caustic.

This video features a few new observations:

a) over any Poncelet family  inscribed in a circle, the directrix rigidly rotates about point W, the reflection of the focus on the point of contact with the caustic. (in one family in the video, W is stationary over all F)

b) over F on the circumcircle, the locus of the *center* of the circular loci of the vertex and of the projection of F on the directrix) are ellipses (dashed green and orange) concentric and axis-aligned with the caustic.

If the inellipse has a focus coinciding with the circumcenter (this family are the excentral triangles to bicentrics), you get more harmonies: (i) the loci of both vertex and directrix center (dashed green and orange) are circles (I wonder if they are in the same pencil as the circumcircle). the latter is  externally tangent to the caustic. (ii) point W is stationary at the other focus of the caustic, over all Poncelet and over all F

Exercise: for each of the Poncelet families studied, what is the locus of W over all F on the circumcircle?