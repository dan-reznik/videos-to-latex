Research below done jointly w P. Roitman, J. Koiller, R. Garcia, and A. Akopyan.

Consider a family of N-periodics (blue, N=5 shown) polygons w vertices Pi in the elliptic Billiard (confocal pair with outer black ellipse and inner brown caustic). This family conserves perimeter L, joachimsthal's constant (not shown), and sum of cosines. Let di represent the length of "focal spokes" (dashed blue) connecting the left focus (f1) to each Pi. It turns out the sum of 1/di is conserved.

1) The "inversive" polygon (red) is obtained by inverting the Pi with respect to a unit-radius circle (dashed black) centered on a focus f1 (left in the video). Let its vertices be denoted Pi' and its (varying) area A'. This family conserves: sum of distances from f1 to its vertices (equivalent to sum of 1/di), sum of cosines, and perimeter. This family is inscribed in a Pascal Limaçon (not shown). 

2) The "polar" polygon (green) has sides which pass thru the Pi' and are perpendicular to Pi'-f1 (i.e., it is the antipedal polygon of the inversive wrt f1). Let Pi'' denote its vertices. This family is interscribed between two non-concentric circles (dashed green). It conserves its sum of cosines. Also invariant is the ratio of its area by A'.

3) The "dual" polygon (orange) is obtained by inverting the Pi'' with respect to a unit-radius circle centered on the "polar" incenter. This family is inscribed in a circle w center O and circumscribes an ellipse w one focus centered on O. It conserves the sum of sidelengths squared [1]. Also invariant is the ratio of its area by A'.

4) The "outer dual" polygon (pink) is obtained by inverting the Pi'' with respect to a unit-radius circle centered on its circumcenter. This family is inscribed in an ellipse and circumscribes a circle with center which coincides w one focus of said ellipse. No invariants have been identified for this family yet.

[1] A. Akopyan, Private Communication, Oct 12, 2020.