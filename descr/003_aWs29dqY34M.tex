Left: A family of 3-periodics (blue) in the elliptic billiard, interscribed in a pair of confocal ellipses (black and brown) is shown on the left which classically conserve perimeter and Joachimsthal's constant [1].  This family also conserves the sum of internal angle cosines; furthermore, the family of excentral polygon (green) conserves the product of its internal angle cosines [2,3,4]. The ratio of areas of the outer polygon to that of 3-periodics is also conserved [5]. The locus of the excenters is an ellipse (dashed green). 

Middle: the affine image of the former which sends the confocal caustic to a circle. This family also conserves the sum of its cosines [3]. Surprisingly it is equal to that of its confocal pre-image (left). Notice the image of the original excentrals (green) does not conserve its product of cosines (grayed out). Note these are not the excentrals of the current family.

Right: the affine image of the confocal 3-periodics (left) which sends the locus of the excenters to a circle (dashed green). Like its confocal pre-image, this also conserves the product of its cosines [3]. Surprisingly it is equal to that of its confocal pre-image (left). These triangles are *not* the excentrals of the current family (blue) which do not conserve their sum of cosines (grayed out).

Experimentally, we have also noticed that the confocal+incircle families sweep the same curve in 3-dimensional "cosine space", as do the outer+circumcircle families, see [8].

[1] S. Tabachnikov, "Geometry and Billiards", Student Mathematical Library, vol 30, American Mathematical Society, 2005. http://www.personal.psu.edu/sot2/book...​
[2] D. Reznik, R. Garcia, and J. Koiller, "Can the Elliptic Billiard still surprise us?", Math Intelligencer, 42, 2020. http://rdcu.be/b2cg1​
[3] A. Akopyan, R. Schwartz, and S., "Billiards in Ellipses Revisited", Eur. J. Math, 2020. 
[4] M. Bialy and S. Tabachnikov, "Dan Reznik's Identities and More",
Eur. J. Math., 2020.
[5] A.C. Chavez-Caliz, "More About Areas and Centers of Poncelet Polygons" , Arnold Math J., 2020.
[6] R. Schwartz,  "The Poncelet grid", Advances in Geometry, 7:2, 2007.
[7] M. Levi and  S. Tabachnikov, "The Poncelet Grid and Billiards in Ellipses",Am. Math. Monthly,  114:10, 2007.
[8] D. Jaud, D. Reznik, and R. Garcia, "Poncelet Plectra: Harmonious Properties of Cosine Space", arXiv:2104.13174 , 2020.

---

soundtrack: K. MacLeod, "Lachaim", licensed under a Creative Commons Attribution 4.0 license. https://creativecommons.org/licenses/by/4.0/

Source: http://incompetech.com/music/royalty-free/index.html?isrc=USUAN1100412

Artist: http://incompetech.com/