Recall the anticevian triangle T' of a reference triangle T wrt point X is such that T is the cevian of T' wrt X [1]. 

Here we extend this notion to N-gons, where N is odd.

First define the X-cevian polygon of an N-gon P as having vertices at the intersections of lines Pi X with sides of P opposite to Pi. For example, P3P4 is opposite to P1, P4P5 to P2, etc.

Now define P', the X-anticevian of P: it is such such that P is the X-cevian of P'. 

Unlike the cevian calculation, which is direct and local (it only requires a vertex and the opposing side), obtaining the vertices of the anticevian requires global information. A. Akopyan has suggested a very clever algorithm based on the composition of projectivities we shall explain elsewhere.

The video shows  the dynamic geometry of both the cevian and anticevian as a point X is dragged around in the plane of a regular pentagon.

When bounded (see below), the anticevian can be convex, concave, self-intersection.

The video also shows that if X is on a certain web of hyperbola-like curves (10 branches total), one or more vertices of the anticevian go to infinity. This web divides the plane into "cells". If X remains within one such cell, the vertices of the anticevian remain bounded, and its area function is continuous.

Note: in the video I say every such branch goes thru a side midpoint, of course this is wrong. In fact, every *pair* of branches has one branch which goes thru sides' midpoint.

The viewer is also invited to interact with a (slow) wolfram app we published on the web [2].

[1] E. Weisstein, Anticevian Triangle, Mathworld, 2021. https://mathworld.wolfram.com/AnticevianTriangle.html
[2] D. Reznik, Anticevian App, Wolfram Cloud, 2021. http://bit.ly/3oKUVf5