The vertices of the family of self-intersected 4-periodics (blue) in the elliptic billiard (black), tangent to a confocal hyperbolic caustic (brown), turn out to be concyclic with the two foci on a circle (dashed blue) whose center C rides on the billiard's minor axis. Therefore, the inversion of said vertices w respect to a unit circle (dashed black) centered on one focus produces four collinear points (a degenerate polygon of constant perimeter, pink), along the radical axis of the circle centered on C and the inversion one.

Also shown is the outer polygon (green) to said 4-periodic family, tangent to the billiard at the vertices. Its vertices are also concyclic with the foci on a distinct circle (dashed green) centered on C', which also translates on the billiard minor axis. Therefore the inversion of outer vertices with the aforementioned inversion circle also result in 4 collinear points (degenerate polygon of variable perimeter, dashed pink) along the radical axis of the circle centered on C' and the inversion one.

It turns out both aforementioned radical axis and/or degenerate inversive polygons are dynamically perpendicular.

The signed area of both 4-periodic and outer polygon is dynamically zero.