Three elliptic billiards are shown with aspect ratios a/b of 1.35, 1.486, 1.75. The family of 3-periodics for each billiard are shown as blue polygons.

X1, the incenter is shown as well as its elliptic locus (green).

Also shown are centers X100 and X88 whose locus is on the billiard boundary. Interestingly, X1, X100, and X88 are collinear [1].  Shown (purple) is the astroid-like envelope of the lines X1-X100, to which said line is instantaneously tangent, at a point "E".

In the simulation, orbit vertices move monotonically counterclockwise (CCW). Notice centers X1 (resp. X100) move monotonically CCW (resp. CW).

What happens to X88 is more complex. Define a constant a88=1.486.

a) a/b ﹤a88: X88 moves monotonically CW.
b) a/b = a88: X88 moves monotonically CW though it comes to a stop (velocity zero), when E is on the left/right vertex of the billiard.
c) a/b ﹥ a88 X88 motion has four phases: two CW portions and two CCW. The former (resp. latter) occur when E is inside (resp. outside) the billiard. As in [b], X88 comes to a stop when E is on the billiard boundary.

[1] Kimberling, C., "Encyclopedia of Triangle Centers", https://faculty.evansville.edu/ck6/encyclopedia/ETC.html