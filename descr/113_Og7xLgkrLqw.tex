Every triangle is associated with a Circumbilliard (CB) of which it is a 3-periodic orbit.

Three identical Elliptic Billiards (EB) are shown (black) in left, right top, and right bottom areas. Inside each is one sees the family of 3-periodics (black triangles).

Left: the (moving) CB of the Excentral Triangle (solid green) is shown centered on the latter's Mittenpunkt is X(168) [1]. Is locus (red) is ellipse-like tough not an ellipse. Also shown (dashed green) is the elliptic locus of the Excenters (the MacBeath Circumellipse of the Excentrals [2]), whose center i is X(9) and axes appear in [3].

Top Right: the CB of the Anticomplementary Triangle (ACT) (blue) is axis-aligned with the EB and centered on the Mittenpunkt od the ACT, i.e., the Gergonne Point X(7) [1]. Its whose locus (red) is an ellipse whose axes are given in [3]. Also shown (dashed blue) is the non-elliptic locus of the ACT vertices.

Bottom Right: the CB of the Medial Triangle (teal), also axis-aligned with the EB, and centered onthe Medial's Mittenpunkt, i.e,. X(142) [1]. Its locus is also an ellipse [3]. This follows from the fact X(142) is the midpoint of X(9)X(7) [1] with X(9) stationary over the family. The locus of the medial vertices is a dumbbell shaped curve (dashed teal).


[1] ETC X(i), i=9,7,142,168: https://faculty.evansville.edu/ck6/encyclopedia/ETC.html
[2] MathWorld, MacBeath Circumconic: https://mathworld.wolfram.com/MacBeathCircumconic.html 
[3] Garcia et al., "Why so many ellipses?", 2020 --  https://arxiv.org/abs/2001.08041