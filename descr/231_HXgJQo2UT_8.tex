Shown is the family of Poncelet 3-periodics (blue) inscribed in a circle and circumscribing a non-concentric ellipse, as as well as the circular loci of Xk, k=2,3,4,5,20. The left (resp. right) picture shows a choice of inner ellipse which contains (does not contain) the center of the outer circle. In the first (resp. second) case, all (resp. some) 3-periodics are acute, and the locus of X4 and X20 is interior (resp. both interior and exterior) to the outer circle).

A few remarks from [1].

a) The condition for the locus of a triangle center to be an ellipse is that it is a fixed linear combination of X2 and X3, i.e., it will lie on the Euler Line (dashed gray). In the present case (outer circle, inner ellipse), the loci degenerate to circles.

b) the centers of the ellipses are collinear (magenta line). 

c) If the caustic contains (resp. does not contain) the center of the outer circle, all (resp. some) 3-periodics are acute (resp. and some are obtuse).

[1] M Helman, D Laurain, R Garcia, and D Reznik, "Invariant Center Power and Elliptic Loci of Poncelet Triangles", 2021. arXiv:2102.09438