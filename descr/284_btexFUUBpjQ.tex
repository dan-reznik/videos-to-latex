An a/b=1.5 elliptic billiard (EB) is shown (black) as well as its foci f1 and f2, and its family of N-periodics (blue), N=5 is used without loss of generality. Also shown is the "tangential polygon" (green), tangent to the EB at the N-periodic vertices. 

For each of the 6 EBs drawn, a point "m" is chosen at a different location on the plane. For each case, N perpendiculars are dropped from "m" to the sides of the tangential polygon, aka. altitudes.

The video illustrates that independent of the position of "m" the sum of the square lenghts of said N altitudes is invariant (though its invariant value depends on m's location, with the minimum occurring at the EB center, top left). Notice the invariance is also true when "m" is exterior to the EB.

Soundtrack: Joaquín Rodrigo, "Concierto para Aranjuez"