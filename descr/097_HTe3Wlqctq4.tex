Given any triangle ABC, the focus of any inconic which is a parabola -- an inparabola -- must lie on the circumcircle.

Consider the family of Poncelet triangles (blue) inscribed in a circle (black) and circumscribing a concentric inellipse (brown).  Consider a fixed point F on the circumcircle; this prescribes a unique inparabola (magenta) w focus at F.

The video shows that with fixed F and over the Poncelet family, the locus of the vertex V of the inparabolas is a circle passing through F and tangent at a point T to the inellipse. The locus of the center of the directrix (foot of perp dropped from F to directrix) is a twice-sized circle also containing F and centered on T.

Proofs welcome!

Not shown:

(a) over N=3 bicentrics and the Brocard porism, the locus of the vertex is also a circle (though not touching the caustic). Liliana Gheorghe has simulated the case of the excentral family to N=3 bicentrics and reported the locus of the vertex there is also a circle.

Conjecture: for a Poncelet triangle family inscribed in a circle, the locus of the vertex of the inparabola wrt a fixed focus F on the circumcircle is a circle.

(b) the locus of the *center* of the circular locus of the vertex. Liliana Gheorghe has told me this is a conic homothetic to the inner one.

(c) the envelope of the directrix

Dan