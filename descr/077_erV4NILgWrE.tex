A family (blue) of triangles (left) and quadrilaterals (right) are shown which are Poncelet families interscribed in a pair of homothetic ellipses (only the outer one is shown). These can also be regarded as affine images of a family of regular polygons.

Also shown are their polar images (magenta) with respect to the left focus of the caustic (not shown). These are  inscribed in a (dashed magenta circle) and circumscribe a Brocard inellipse (not shown). Their Brocard points lie at the foci of the latter. This class of polygons is known as "harmonic" since they can also be regarded as the image of regular polygons under inversion [1,2,3].

The animation shows isocurves of are of the inversion of the harmonic family wrt to point on the plane, as they revolve.

[1] J. Casey, "A sequel to the first six books of the Elements of Euclid", Longman, Dublin, 1888

[2] T. Sharp, "Harmonic Polygons", The Mathematical Gazette
Vol. 29, No. 287 (Dec., 1945), pp. 210-213.

[3] A. Zazlavsky and A. Akopyan, "Геометрические свойства кривых
второго порядка" (Geometric properties of curves of second order), sec. 4.6, pp. 129. MCNMO Publishing House, Moscow, 2011.