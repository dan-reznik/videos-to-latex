We explore loci phenomena of related to the polar triangle with respect to circumparabolas over both N=3 bicentric and "homothetic" Poncelet families. The following observations are made

1) N=3 bicentrics (circumparabolas are isogonal images of fixed tangent to circumcircle)

a) we saw before the locus of circumparabolas follow a straight line
b) the envelope of circumparabola directrices is a parabola whose focus is the incenter of the bicentric family!
c) the locus of X2 of the polar triangle wrt circumparabola is a straight line parallel to (a)
c) the locus of X4 of the polar triangle is a parabola tangent to (b) at a mysterious point

2) Poncelet triangles of the "homothetic family" (circumparabolas are the isotomic image of a fixed tangent to the Steiner ellipse)

a) we saw before all circumparabolas are tangent to a reflection of the original tangent wrt center
b) the envelope of the directrix is also a parabola (could not show result due to a bug)
c) the locus of X2 of the polar triangle is a straight line parallel to the original tangent to the Steiner