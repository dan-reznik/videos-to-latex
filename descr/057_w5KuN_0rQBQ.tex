A triangle V1 V2 P(t) (blue) is inscribed in an ellipse E (black) with semi-axes a,b (in the video a/b=1.5). While V1,V2 are fixed, P(t) slides along E's boundary. Show are the elliptic loci of the barycenter X2 (brown), circumcenter X3 (red), orthocenter X4 (orange), collinear on the Euler line (dashed green) and of an point X(ρ)=X2+ρ(X4-X2) on the Euler line for variable ρ. 

As ρ varies, the locus of X(ρ) is a family of ellipse (in general not axis-aligned w E) whose center Oρ follows a straight line (dashed purple). Notice at ρ=-1/2 (at X3), and also at some second location, one of the axes of said locus vanishes.

[1] C. Kimberling, "Encycl. of Triangle Centers", 2020. https://faculty.evansville.edu/ck6/encyclopedia/ETC.html