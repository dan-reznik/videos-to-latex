Take two concentric ellipses E and E' (axis-aligned or not) which admit a family of Poncelet 3-periodics (Cayley's condition is satisfied). Let the common center be O.

Let C3 (resp. C5) denote the circumcircle (resp. Euler's circle) of the family. These are centered on X3 and X5, respectively. Let their radii be R and R5. It is well-known R5=R/2.

The following observations are made. If E' and E are not axis-aligned:

a) the locus of X3 is an ellipse concentric and axis-aligned with E'.
b) the locus of X5 is an ellipse concentric but not axis-aligned with E'.
c) the power of O wrt C3 is invariant over the family.
d) the power of O wrt C5 is invariant over the family.

Note: in (a), and (b) the axes of the loci of X3 and X5 become axis aligned with the pair (E,E') is axis aligned itself.

If the pair is non-concentric, the locus of both X3 and X5 are still ellipses, though in general neither is axis-aligned with either E or E'. Also the power of O wrt to either C3 or C5 is no longer constant.