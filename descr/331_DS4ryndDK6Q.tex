This demonstrates William Chapple's (1718-1781) Porism, aka. a poristic system triangles [4]. In a 1746 essay in The Gentleman' s Magazine Chapple writes [1]: when two circles are the incircle and circumcircle of a triangle, then there is an infinite family of triangles for which they are the incircle (shown green) and circumcircle (shown purple).

Note this significantly predates Poncelet' s own 1822 work in this area.

We use R=1 and d=0.5, and r=0.375. 

Also we illustrate (i) an invariant reported in [2], that the antiorthic axis (blue, known as L1) is stationary, and another one (ii) reported in [3], that the locus of the excenters is a circle (shown orange).

Note that the point X(1155) sits still at the intersection of the X1X3 with L1.

Also note that the ratio of areas between excentral (green) and reference (blue) triangle is invariant, as this is a corollary to r/R being constant. In fact Aexc/A=2R/r.

Notice the circumcircle (purple) is the 9-point circle of the excentral, which passes thru the vertices of the excentral medial triangle (shown dashed pink).

Sountrack: Mozart Concerto for Piano no 25 in C major- Allegretto

References:

[1] William Chapple, Surveyor, https://en.wikipedia.org/wiki/William_Chapple_(surveyor)
[2] J. H. Weaver, "Invariants of a poristic system of triangles", Bull. Amer. Math. Soc., 33:2, 1927. https://projecteuclid.org/download/pdf_1/euclid.bams/1183492031
[3] B. Odehnal, "Poristic Loci of Triangle Centers, Journal of Geometry and Graphics", 15(1), 2011. https://www.geometrie.tuwien.ac.at/odehnal/pltc.pdf
[4] W. Gallatly, "The Modern Geometry of the Triangle", F. Hodgson, 1914.