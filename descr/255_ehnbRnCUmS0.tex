After an idea by Peter Roitman [2].

Consider the family of N-periodics (blue) in the Elliptic Billiard E (black) w semiaxes a,b and foci f1,f2.

Let P(i), i=1,2,...,N denote its vertices, in the video N=5, a/b=2. Let A denote its (variable) area.

Consider the polygon (dark green) whose vertices Q(i) are the incenters of triads [P(i),P(i+1),f1], i=1,...N. Let A' denote its (variable) area. This is reminiscent of the so-called "Japanese Theorem" which we studied in [1] for a bicentric Poncelet pair. The incircles of each triad are shown (dashed green) as well as the "focal" spokes (lines connecting f1 to the vertices).

The video shows the following two new properties:

1) Over the family, the ratio A/A' is invariant, in fact for any choice of N and a/b.

2) The locus of the Q(i) is an ellipse E' (solid light green) centered on O1 to the left of the billiard center. E' is non-concentric, though  axis-aligned with E (their major axis are on the same line).

3) Not shown. In a separate video [2] we show a similar construction where instead of the incenters we consider the polygon of circumcenters of the same triads. It turns out the ratio of sum of inradii divided by the sum of circumradii is invariant.

[1] A. Akopyan and D. Reznik, "Non-Concentric Circular Poncelet Pair: Invariant Sum of Japanese Theorem Inradii", https://youtu.be/BEvdUUolUXI
[2] P. Roitman and D. Reznik, "Circumcircles of Focus with Consecutive Vertices Homothetic to Focus Antipedal", https://youtu.be/kVxh5jfZb9Q