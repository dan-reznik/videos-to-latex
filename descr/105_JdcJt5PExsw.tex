Let E be an elliptic billiard (on the video a/b=1.5).

On July 1st, 2019, Prof. Clark Kimberling also posted on ETC [1] results by one important contributor to ETC, Peter J.C. Moses, relating several new properties of E, including a list of several Kimberling centers that lie on E's boundary, specifically: 

X(190), X(651), X(655), X(658), X(660), X(662), X(673),
X(771), X(799), X(823), X(897), X(1156), X(1492),
X(1821), X(2349), X(2580), X(2581), X(3257), X(4598),
X(4599), X(4604), X(4606), X(4607), X(8052), X(20332),
X(23707), X(24624), X(27834), X(32680)

(we are not including X(100) and X(88) as the former was already on ETC and the latter had been known to us for a few months [3]).

The video shows the motion of all listed centers on the boundary of an a/b=1.5 ellipse, driven by a smooth CCW rotation of 3-periodic orbit (a triangle), which we have proven must have a stationary X(9) [2,3]. We have also created a Wolfram Mathematica interactive applet [4] of this experiment.

Note: colors represent whether X(i) is:

black: i in (1,1000)
red: i in (1001,3000)
blue: i greater than 3000

[1] faculty.evansville.edu/ck6/encyclopedia/ETC.html
[2] www.youtube.com/watch?v=AoCWcza95OA
[3] dan-reznik.github.io/Elliptical-Billiards-Triangular-Orbits/
[4] www.wolframcloud.com/objects/user-abf31092-d7c1-4e49-8701-dc65d547b021/peter%20moses%20points%20on%20X9-centered%20circumellipse

More Info: https://dan-reznik.github.io/Elliptical-Billiards-Triangular-Orbits/