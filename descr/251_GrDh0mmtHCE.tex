Shown is the family of 5-periodics (blue) interscribed in a confocal pair (elliptic billiard: black and brown ellipses).

Let the billiard major, minor axes define an xy reference system with (0,0) at the billiard center.

Define the Cremona-Inversive-Polygon (CIP, dashed blue) as having vertices at the images of the original ones unver the following Cremona transform [1]:

(x,y) ↦ (1/x,1/y)

The purple curve represents the image of the outer ellipse under this transformation. So the vertices of the CIP must lie on it.

Surprisingly, the signed area of this polygon is dynamically zero. Indeed, this is the case for any odd N (pending proof).

When N=3, the CIP has collinear vertices, i.e., it is a zero-area degenerate polygon, see: https://bit.ly/2JkYQzD

Interestingly, for the homothetic Poncelet family, the signed area of the CIP is zero for *all* N, except, for the N=4 case. See the N=3 case here: https://bit.ly/37dJS6z 

[1] "Cremona Transformation", Encyclopedia of Mathematics, https://encyclopediaofmath.org/wiki/Cremona_transformation