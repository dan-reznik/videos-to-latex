Consider a family of triangles T=ABC (blue) rigidly rotating about X(61), the isogonal conjugate of X(17). Consider a family T' (red) with vertices at the inversions of A,B,C with respect to an inversion circle (dashed green) centered at an arbitrary point O. The video illustrates phenomena pertaining to the locus of X'(61) of T', namely:

1) When the moving circumcircle C of T contains (resp. does not contain) O, X'(15) moves along an arc of an ellipse (resp. is stationary). Let R be the circumradius of T. There are 3 possibilities:
1.1) If |O-X(61)|  greater than R, O always exterior to C, X'(61) is a stationary point over the entire motion
1.2) If |O-X(61)|  less than R: (i) while O is exterior to C: X'(61) is stationary, (ii) while O is interior, X'(61) moves along a disjoint elliptic arc.
1.3) O is sufficiently close to X(61) such that O is always interior to C, X'(61) moves along a contiguous ellipse.

2) Similar phenomena happen when T is pivoting about its 2nd isodynamic point X(62), the isogonal conjugate of X(18), and one is tracking the X'(62) of T'.

Note: X(61) and X(62) are inversions of each other wrt Brocard Circle.