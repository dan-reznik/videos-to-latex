Consider an affine image of equilaterals rotating in the unit circle. These (i) are inscribed in an ellipse E, (ii) maintain their centroid X(2) stationary at the center of E, and (iii) have constant area. This family can also be regarded as the Poncelet family of 3-periodics interscribed in a pair of concentric, homothetic ellipses.

Interestingly, the family also conserves (iv) the sum of squared sidelengths, and (v) the sum of its internal angle cotangents.

Define the "Generalized Evolute Polygon" (GEP) of the affine family as having vertices Pi' = Pi + s Ri ni, where:

Pi = vertices of affine polygons, i=1,2,..,N
Ri = radius of curvature of E at Pi
ni = inward-pointing normal of E at Pi (shown as arrows)

Construction lines are shown (dashed gray) from each vertex Pi to each Pi' along ni, and measuring s Ri.

It can be shown that for any choice of "s" the area of the GEP is conserved, for any N greater than 2. In fact,  two values of "s" can be chosen such that the GEPs have zero signed area. See other videos on these polygons on our channel.

The video shows the N=3 case: the zero-area GEPs are segments with (collinear vertices), parallel to either the major or minor axis of E.

Said segments intersect at the "3rd Brocard Point" or X(76) as specified in [1]. Over the family, its locus is an ellipse (dashed pink).

[1] C. Kimberling, "The Third Brocard Point X(76)", Encyclopedia of Triangle Centers (ETC), 2021. https://faculty.evansville.edu/ck6/encyclopedia/ETC.html