Shown is an a/b=1.5 elliptic billiard (black) and its family of N=3 (triangular) orbits T (blue).

For every orbit, the excentral triangle T' [1] is shown (green) as well as C', its cosine circle [2] (red), also known as the Second Lemoine Circle, centered on the symmedian point of T', congruent with the Mittenpunkt X(9) of T, stationary for the family of orbits [3].

C' can be computed by intersecting the 3 antiparallels [3] of T', drawn through X(9), with the sides of T' (yielding 6 intersections, shown as red dots).

Antiparallels from a vertex will be in the direction of the tangent to the circumcircle of T' (dashed green) at the vertex. The center of the circumcircle of T' is shown at X(40), the Bevan point of the orbit.

Alternatively, the antiparallels can be obtained from as lines parallel to the sides of a triangle's orthic, as shown in [6]. Since the excentral's orthis is the orbit, one simply needs to draw lines parallel to the orbit sides which pass through the Mittenpunkt of the orbit.

The main result here is that the radius of C' is constant, and since its center is stationary, C' is stationary for all orbits.

Also shown is the cosine hexagon, a self-intersecting polygon, made up of the 6 points of intersection between antiparallels and the sides of T'. It turns out neither its perimeter nor its area are constant.

We've created an interactive applet for this experiment [4].

[1] Excentral Triangle: http://mathworld.wolfram.com/ExcentralTriangle.html
[2] Cosine Circle: http://mathworld.wolfram.com/CosineCircle.html
[3] Stationarity of the Mittenpukt for Triangular Orbits: https://dan-reznik.github.io/Elliptical-Billiards-Triangular-Orbits/
[4] Antiparallels:  http://mathworld.wolfram.com/Antiparallel.html
[5] Interactive Applet: https://www.wolframcloud.com/objects/user-abf31092-d7c1-4e49-8701-dc65d547b021/cosine%20circle%20of%20excentral%20triangle%20is%20stationary%20v1
[6] https://demonstrations.wolfram.com/TheSecondLemoineCircle/

More Info: https://dan-reznik.github.io/Elliptical-Billiards-Triangular-Orbits/