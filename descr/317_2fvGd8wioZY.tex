Imagine if Jean-Victor Poncelet (1788-1867), Jakob Steiner (1796-1863), and Jean Baptiste Henri Brocard (1845-1922) walked into a bar. A trifle this may be, but they would have laughed about it.

Is there a pair of ellipses which admits a 3-periodic poncelet family such that the Brocard angle ω [1] of the family is invariant (i.e., the family is "equibrocardal" (chapt XVII)[4])? This video shows that suprisingly, if a homothetic pair of concentric, axis-aligned ellipses is chosen (*), (a,b) and (a/2,b/2), such a magical phenomenon occurs. In fact, this family preserves area, sum of squared sidelengths (call it L2), and therefore Brocard angle ω = atan(4A/L2).

Note *: we call it a "Steiner pair" as well since the outer ellipse is the family's stationary Steiner Circumellipse and the inner one a stationary Steiner Inellipse, both centered on the Barycenter X2. 

Postscript (7-sept-2020): the locus of either Brocard point for the above triangle family is an angled ellipse. See it with our interactive app: https://bit.ly/3i1RLjP

[1] E. Weisstein, "Brocard Points", MathWorld. https://mathworld.wolfram.com/BrocardPoints.html
[2] E. Weisstein, "Steiner Circumellipse",  MathWorld. https://mathworld.wolfram.com/SteinerCircumellipse.html
[3] E. Weisstein, "Steiner Inellipse",  MathWorld. https://mathworld.wolfram.com/SteinerInellipse.html
[4] R. Johnson, Advanced Euclidean Geometry, Dover, 1960.