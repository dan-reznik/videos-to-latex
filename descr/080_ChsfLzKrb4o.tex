This video shows the family of generalized evolute polygons (GEP pink) who are an affine image of the family of regular 5-gons in a circle. These can also be regarded as the family of Poncelet N-periodics in the homothetic ellipse pair.

The vertices Qi of the GEP lie along ellipse normals at a distance proportional to the inverse of the curvature namely:

Qi = Pi + s ni (s/ki)

In the video s is chosen to be 1, so the Qi are the centers of curvature, i.e., they lie on the ellipse evolute (green curve).

In the video a/b=Sqrt(2) so the evolute just touches the upper and lower vertex of the ellipse. 

Both families are area invariant and conserve their sum of sidelengths squared,. It turns out the GEP do not conserve either quantity for the special case of N=4.