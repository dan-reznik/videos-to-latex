An a/b=1.5 elliptic billiard is shown as well as the family of N=3 (triangular) orbits. Also shown is the orbits' circumcircle (purple) and E1, the incenter-centered circumellipse (green). These two intersect at X100, the anticomplement of the Feuerbach point.

E1 has two very interesting properties: its axes are parallel to the billiards', and the ratio of its axis is constant. Note however neither of the axes have constant lengths nor is their product constant.

For the Steiner Circumellipse the relationship between the two axes (L1 and L2) remains fixed but via a linear equation: L1 = c0 + c1 L2. For the Steiner Inellipse (with axes lengths L1' and L2') the relationship is L1' = c0/2 + c1 L2', where c0 and c1 refer to the same quantities used for the Circumellipse.

More Info: https://dan-reznik.github.io/Elliptical-Billiards-Triangular-Orbits/