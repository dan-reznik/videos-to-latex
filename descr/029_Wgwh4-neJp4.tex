The Brocard porism is a 1d Poncelet family of triangles T=ABC (blue) inscribed in a circle Γ (black) and circumscribed about an inellipse E (black) known as the Brocard inellipse [1]. The Brocard points Ω1,Ω2 are stationary at the foci of E and the Brocard angle ω is invariant.

The Brocard circle K (green), which contains Ω1,Ω2,X3,X6 is stationary. The 2nd Brocard Triangle (purple) has vertices A'B'C' at the intersections of symmedians (cevians thru X6) with the Brocard circle [2]. It is therefore inscribed in K. Incredibly, the family of 2nd Brocard triangles is a new Brocard porism inscribed in K and circumscribed about a 2nd, smaller Brocard inellipse E' (not shown).

Also shown are the points of contact DEF of E to T: since X6 is the Brianchon point (perspector) of E [3], said points occur at the intersection of the symmedians with the sidelengths. Note also that the foci on an inellipse are isogonal conjugates, which is consistent with the fact that Ω1,Ω2 are such a pair. 

за здоро́вье!

[1] E. Weisstein, "Brocard Inellipse", Mathworld, 2020. https://mathworld.wolfram.com/BrocardInellipse.html
[2] E. Weisstein, "Second Brocard Triangle", Mathworld, 2020. https://mathworld.wolfram.com/SecondBrocardTriangle.html
[3] E. Weisstein, "Brianchon Point", Mathworld, 2020. https://mathworld.wolfram.com/BrianchonPoint.html