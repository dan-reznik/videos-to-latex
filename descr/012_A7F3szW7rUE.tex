Consider N-periodics in the elliptic billiard w/ foci at f1 and f2. A polar transformation wrt f1 sends the billiard family to a bicentric "polar" family where f1 is one of its limiting points [1]. Therefore, the invariant-perimeter f1-inversive polygon can be regarded as the pedal of the polar family wrt to f1. 

obs 1: the polar family  conserves the sum of cosines.
obs 2: the f1-inversives conserve perimeter. It also conserves sum of cosines (except when N=4).
obs 3: let l2 denote the 2nd limiting point of the bicentric pair. the pedal of the bicentrics wrt to l2 also conserves perimeter and sum of cosines (no exceptions).

Each of the two bicentric pedal families (wrt f1 or l2) rides on a separate limaçon of Pascal. The f1 one is loopless, whereas the second limaçon has a loop which goes thru l2.

obs 4: For N=4 the l2-bicentric-pedal degenerates to a segment
obs 5: for N=3, the f1- and l2-pedals are similar triangles (modulo reflection), and all their sum of cosines is equal to that of the bicentric (poristic) family. This implies all 3 families have identical and invariant r/R. Note that the  r/R of billiard 3-periodics is also invariant but different from the trio's.

[1] E. Weisstein, "Limiting Point", MathWorld, 2021. https://mathworld.wolfram.com/LimitingPoint.html