Left: 1d family of 3-periodic Poncelet triangles (blue) inscribed in an cicle and circumscribing an inner, concentric ellipse. It turns out (i) the locus of the family's 9-point circle center X5 and orthocenter X4 are concentric circles, (ii) the orthic (red) has fixed inradius and circumradius, and (iii) it conserves the product of cosines and sum of squared sidelengths. The family's Macbeath inconic (green, centered on X5 and w foci on X4 and X3), is rigidly rotating,

Right: It turns out said family of orthics is identical (up to rotation about X3) to the family of Poristic triangles, i.e., 3-periodic Poncelet triangles of fixed inradius and circumradius [1]. 

Since Poristics have been shown to be the image of 3-periodics in the elliptic Billiard under a varying affine transform [2], all invariants valid for the 3-periodic orbits in  the billiard (see [3]) will be valid for the orthic (e.g., sum of cosines). Likewise, those valid for the excentrals (product of cosines) will be valid for the original 3-periodics (between circle and ellipse).

The original inner ellipse is identical, up to rotation, to the X40-centered inellipse to the poristics, which is rigidly rotating [2] and has axes equal to R-d and R+d, where d=|X1-X3| of the poristics, or |X4-X5| of the original 3-periodics. The latter's MacBeath (green) appears here as the stationary caustic to the excentrals (also shown green).

Music: Djavan, Samurai

References:

[1] W. Gallatly, "The modern geometry of the triangle", Francis Hodgson, 1914.
[2] R. Garcia and D. Reznik, "Related by Similiarity: Poristic Triangles and 3-Periodics in the Elliptic Billiard", April 2020, https://arxiv.org/abs/2004.13509
[3] D. Reznik, R. Garcia and J. Koiller, "Forty New Invariants of N-Periodics in the Elliptic Billiard", https://arxiv.org/abs/2004.12497