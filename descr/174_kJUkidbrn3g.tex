Consider a fixed point M on an ellipse E w semi-axes (a,b) centered on the origin, and another point P(t)=[a cos t, b sin t] also on E.

Consider the family of chords from M to P(t). The locus of their midpoints is an ellipse E' (blue) centered on M/2 and with axes (a/2,b/2).

The video explores the locus (orange) of point F on E' farthest from M, over all M. These can be a two-branch, 4-hook discontinuous curve (resp. continuous 4-leaf clover) if a/b is greater than (resp. less than sqrt(2)).

this a/b threshold is related to the evolute pierceing out (resp. being completely interior) to the ellipse. At a/b = sqrt(2) the evolute top and bottom tips touch the ellipse at its top and bottom vertices. 

Contributions are welcome!