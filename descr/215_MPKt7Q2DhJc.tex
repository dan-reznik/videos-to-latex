Consider a 1d triangle family (blue) T(t)={P1,P2,P3(t)}, where P1 and P2 are fixed and P3(t) sweeps some curve. One can ask the following "inverse question": what should P3(t) be such that the locus of a triangle center X(n) is a curve of interest.

The video shows the the locus of P3(t) [red and dashed red] required for the incenter X(1) to sweep a circle (green and dashed green) of radius R centered on the midpoint of P1P2.

Circular X1 are shown for 3 values of R: 40%, 30%, and 20% of |P1-P2|.  Each correspond to a different dumbbell-shaped P3(t). These can be convex or concave depending on R. 

Not shown: as R approaches |P1-P2| the curve "explodes" to infinity.

Prof. Boris Odehnal (Universität für angewandte Kunst, Wien, Vienna), has generously contributed the following valuable information about this curve family [1]:

The mapping (P1,P2,X1) ⇒ P3 is (rational) cubic (but not birational). Therefore, the images of circles as traces of X1 are curves of degree 6 (as locus of P3, and tri-circular, rational, and -- I presume-- higher-order trochoids). The cubic mapping produces third vertices P3 on the ideal line for X1 on the Thales' circle over P1P2. The envelope of all incircles of all degenerate triangles is a nephroid (symm. wrt the x- and y-axis and cusps at (1,0) and (-1,0)). Points X1 inside the Thales circle P1P2 cause the flipping of P3.

[1] B. Odehnal, Private Communication, Dec, 2020.