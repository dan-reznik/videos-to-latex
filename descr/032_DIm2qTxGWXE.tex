Shown are 3-periodics (blue triangles) interscribed in a homothetic poncelet pair: outer elipse (a,b), and inner one (a/2, b/2). The video shows a few phenomena.

1. Over the family, the sum of squared sidelenghts L2, area A and therefore Brocard angle ω = L2/(4A) are all invariant. The last property renders the family "equibrocardal".

2. The loci of the Brocard points Ω1,Ω2 are two marvellous titled ellipses (red, green), similar to the ones in the pair.

3. The family's Brocard inellipse (purple) [1] is centered on X39 and  Ω1,Ω2 lie on its foci. amazingly, its aspect ratio is invariant. What this implies is that there is a variable similarity transform between this family and the so-called Brocard porism, where 3-periodics are interscribed between a circle (their circumcirle) and the (stationary) Brocard inellipse. The converse image in the Brocard porism is that it conserves the aspec ratio of its Steiner Circumellipse, see [2].

4. The locus of X39 is an ellipse (dashed purple) whose axes (a39,b39) are (Ronaldo Garcia, Aug. 2020): a39 = (c2 a)/(2 (a^2 + 3 b^2)), b39 = (c2 b)/(2 (3 a^2 + b^2)), where c2=a^2-b^2.

[1] E. Weisstein, "Brocard Inellipse", MathWorld, 2020. https://mathworld.wolfram.com/BrocardInellipse.html 
[2] D. Reznik, "Brocard Porism", Youtube, 2020. https://youtu.be/h3GZz7pcJp0