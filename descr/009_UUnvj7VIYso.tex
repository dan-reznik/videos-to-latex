Let C be a curve (in our case an ellipse, but it can be anything) and M a point in the plane. The pedal (resp. contrapedal) of C with respect to M is the locus of the foot of perpendiculars dropped from M onto tangents (resp. normals) to C passing through P on C, for all P on C [1,2].

The video makes an observation that may or may not be known:

Both the pedal and contrapedal curves of an ellipse E have *invariant* signed areas, so long as M rides on a circle concentric with E. It made me jump off my chair! The signed area counts self-intersecting loops of the curve as negative [3]. 

This is probably a known identity: let A, A_p, and A_c denote the area of the ellipse, the pedal curve, and the contra-pedal, the latter two with respect to the same point M. 

A_p = A + A_c

I.e., if the phenomenon of invariance with respect to all M on a concentric circle holds for the pedal its hould also hold for the contra-pedal.

P.S. - during the part I cover the "pedal"  I keep incorrectly calling it an "evolute". The ellipse evolute is a different beast [4], though the contra-pedal curve can also be computed as the pedal to the evolute. Sorry about the confusion.

[1] Mathworld, "Pedal Curve", https://mathworld.wolfram.com/PedalCurve.html
[2] Mathworld, "Antipedal Curve", https://mathworld.wolfram.com/ContrapedalCurve.html
[3] Mathworld, "Polygon Area", https://mathworld.wolfram.com/PolygonArea.html
[4] Mathworld, "Ellipse evolute", https://mathworld.wolfram.com/EllipseEvolute.html