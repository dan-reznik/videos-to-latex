An a/b Elliptic Billiard is shown (black) as well as its family of 3-periodic orbits (blue). Also drawn are two well known "right" (orthogonal axes) circumhyperbolas (pass through all vertices):

1) The Feuerbach Circumhyperbolas (brown), centered on X(11)
2) the Jerabek Circumhyperbola (purple) of the Excentral Triangle (green), centered on X(125) of the Excentral, i.e., X(100) of the orbit.

A rectangular circumhyperbola always passes through the orthocenter X(4) and has center on the nine-point circle [1].

- Therefore the Feuerbach will pass through X(4) and the Excentral Jerabek through X(4) of the Excentral, i.e., X(1) of the orbit.
- Both pass through X(9) and X(1), intimately connected with the Billiard.

Aditionally, we have found the following properties:

a) the asymptotes of both Hyperbolas are parallel to the Billiard axes.
b) the ratio of focal lengths is invariant!
c) the ratio of Hessian determinants is invariant!
d) The Feuerbach intersects the Billiard at X(1156).

[1] http://mathworld.wolfram.com/Circumhyperbola.html