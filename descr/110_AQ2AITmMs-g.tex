Shown is the family of N=3 orbits on an a/b=1.5 elliptic billiard (the billiard is the X(9)-centered circumellipse of the orbit family). Also shown are the X(1)- and X(2)-centered circumellipses, call them C1 and C2. The latter is also known as the Steiner Circumellipse (minimal area). The following properties were noticed:

a) C1's axes are aligned with the billiards (horizontal and vertical int he example). However, C2's can become slanted.
b) both C1 and C2 intersect the billiard at the orbit's vertices and that's obvious. However, C1's additional intersection occurs exactly at X(100) (shown as Fbar), the Feuerbach point of the anticomplementary triangle, and for this reason it coincides with the orbit's circumcircle (congruent with the nine-point circle of the anticomplementary triangle, used to obtain X(100)).
c) C2 has a 4th, separate, still not understood, intersection with the billiard.

https://dan-reznik.github.io/Elliptical-Billiards-Triangular-Orbits/