The Poristic Triangle family [2,3] was discovered by William Chapple in 1746 [1]. This family (blue) has a fixed incircle (green) and fixed circumcircle (purple) whose centers are the Incenter X1, and Circumcenter X3, respectively. Chapple, and later Euler, discovered that:

d=|X1X3|=Sqrt[R(R-2r)].

Where r,R represent the radius and circumradius, with R/r greate than 2.

Also shown are the Excentral Triangles (green) and the locus (orange) of their vertices: a circle centered on X40 and of radius 2R, discovered by Boris Odehnal [4].

The video superposed to the family two circumhyperbolas:

a) the Feuerbach Hyperbola F (light blue) of the poristics. This is a rectangular hyperbola centered on X11 [5]. Let λ denote its half focal length.

b) the Jerabek Hyperbola of the Excentrals. Also a rectangular hyperbola centered on their X125, i.e., X100 [6]. Let λ' denote its half focal length.

On the top area of the video one notices the ratio λ'/λ remains invariant over the poristic family and equal to Sqrt[2R/r], a result proven in [7].

Also shown (black) is the "circumbilliard"  to the poristics, i.e., a circumellipse inside which a triangle is a 3-periodic orbit [8]. Its axes are a9, b9. Notice the ratio a9/b9 is invariant over the porisitic family. Remarkably, the asymptotes of F and J' are parallel to its axes.

[1] William Chapple, Surveryor: https://en.wikipedia.org/wiki/William_Chapple_(surveyor)
[2] W. Gallatly, "The Modern Geometry of the Triangle", F. Hodgson, 1914.
[3] J. H. Weaver, "Invariants of a poristic system of triangles", Bull. Amer. Math. Soc., 33:2, 1927. https://www.ams.org/journals/bull/1927-33-02/S0002-9904-1927-04367-1/S0002-9904-1927-04367-1.pdf
[4] Boris Odehnal, "Poristic Loci of Triangle Centers, Journal of Geometry and Graphics", 15(1), 2011. 
[5] Feuerbach Hyperbola, https://mathworld.wolfram.com/FeuerbachHyperbola.html
[6] Jerabek Hyperbola, https://mathworld.wolfram.com/JerabekHyperbola.html
[7] Ronaldo Garcia and Dan Reznik, "Related by Similarity: the Poristic and 3-Periodic Triangle Families", April, 2020, arXiv.
[8] Dan Reznik and Ronaldo Garcia,  "The Circumbilliard: Any Triangle can be a 3-Periodic", April, 2020. https://arxiv.org/abs/2004.06776