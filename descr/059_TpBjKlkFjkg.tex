Consider the family of triangles T(t)=V1V2P(t) inscribed in an ellipse E with axes a,b. Namely, V1,V2 are fixed on the boundary of E and P(t), not shown, executes one revolution along it.

It turns out that provided a point X lies on the Euler line at some fixed linear combination of X2 and X4, its locus will be an ellipse over T(t). The video shows elliptic loci for triangle centers X2,X5,X381,X4.

Furthermore:

1) Over the family of parallel V1V2, said loci rigidly translate
2) For a given V1V2, the center of all loci are collinear.
3) When V1 and V2 are symmetric about the center of E, the centers of all loci collapse to a point.

In the video one can observe how the discrete family of loci change their relative position as a family of parallel V1V2 is traversed.