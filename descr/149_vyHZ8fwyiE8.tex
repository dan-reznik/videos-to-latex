Family of N=3 orbits in an a/b=1.5 elliptic billiard. Also shown are five points which we have found lie on the billiard via indirect transformations:

- The anti-feuerbach point X(100), shown as "F bar" (the feuerbach of the anticomplementary triangle)
- The vertices C1,C2,C3 of the anticomplementary triangle's contact triangle (touchpoints of its incircle with its sides)
- X(88), the isogonal conjugate of X(44). The latter is the intersection of the line from the origin through the incenter X(1) with the anti-orthic axis, shown blue. This axis is computed by intersecting corresponding sides on the orbit and excentral triangle (dashed green).
- The Tabachnikov point, red T, which lies at the geometric mean (in terms of radial length) between the Incenter and X44.

Note1: the antifeuerbach X(100), the incenter X(1), and X(88) are collinear
Note2: the mittenpunkt X(9), the incenter X(1), the Symmedian X(6), and X(44) are collinear.
Note3: the encyclopedia of triangular centers does say X(44) is "X(44) = inverse-in-circumconic-centered-at-X(9) of X(1)". Because the billiard is centered on the mittenpunkt X(9), this is a precise statement.

https://dan-reznik.github.io/Elliptical-Billiards-Triangular-Orbits/