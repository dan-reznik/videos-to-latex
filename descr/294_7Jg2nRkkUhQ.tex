Consider the family of Poncelet N-periodics inscribed in an external ellipse w axes (a,b) and circumscribed about an internal, concentric circle w radius r. For N=3, r = (a*b)/(a+b), but for N greater than 3 the required r is given by a more complicated (Cayley) equation, and is typically obtained numerically.

Left: For N=3 this family has a few suprising properties:
a) by definition, the inradius r is constant.
b) the circumradius R is invariant
c) therefore the ratio r/R is invartiant, so the family conserves the sum of cosines.
d) since both r and R are invariant so is the product r R. since the distance d from the circumcenter X3 to the incenter X1=O is  given by R^2 - d^2 = r R, then d is invariant. in the video (left), one can see this manifested in the circular locus of X3 about the origin.
e) the area A of a triangle is given by the product of the inradius r by the semiperimeter s. Since r is constant, so is A/s, or A/L, since L=2s.
f) The circumradius R is given by (a b c)/(4 r s), where a,b,c are the sidelengths. Since r / R is invariant, so is (a b c)/(4 s).

Right: For N=5 (and in general for N greater than 3)
a) this family still has a circular caustic or, the pseudo-inradius r is constant.
b) though there is no concept of circumradius R for N greater than 3, the family conserves the sum of cosines.
c) consider the Steiner Curvature Centroid K as a pseudo-circumcenter. its locus is circular about the origin (resp. the origin) if N is odd (resp. even).
d) A/L is invariant for all N.
e) The product of sidelengths divided by perimeter is invariant for odd N only.