Consider two ellipses in the plane E and E'. Pick a point P1 on E. Define P2 to be where a ray shot from P1 toward one of the tangents to E' meets E again. Now define P3 to be where a ray shot from P2 toward one of the tangents to E' meets E, etc.

Poncelet's porism (also known as his "closure theorem") asserts that if such an iteration ever returns to P1, say in N steps, i.e., forming a closed polygon (potentially self-intersecting), starting the iteration from any other point P1' on E is guaranteed to return to P1' is the same N steps. Since the boundary of E is 1d, the porism prescribes a 1d family o N-gons [1].

Show is a family of Poncelet 7-periodics (blue) "interscribed" between two ellipses (black), i.e., inscribed in a first ellipse while simultaneously circumscribing a second one. The ellipses shown are in "general position", i.e., neither are they concentric nor axis-parallel. For more information see our live page [2].

[1] O. Nash, "Poring over Poncelet", 2018. http://olivernash.org/2018/07/08/poring-over-poncelet/index.html 
[2] D. Reznik et al., "Poncelet Experiments: Media Artifacts", Observable, 2021. https://observablehq.com/@dan-reznik/poncelet-media-assets