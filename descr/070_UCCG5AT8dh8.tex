This is joint work with M. Helman, R. Garcia, and D. Laurain, who introduced us to M. Frégier's wonderful theorem [1,2].

In 1810 M. Frégier (French Mathematician) discovered a remarkable fact. Take a point M on an ellipse E and consider all pairs of rays at a right angle with each other emanating from M. Consider the family of chords defined by the intersections of these rays with E. All such chords pass thru a common point F, called the "Frégier Point" [2].

Let ϴ denote the angle between the pairs of rays, above ϴ=90 degreess. The video shows the following old/new facts:

a) Probably known: the locus of F over all M is an ellipse.

Let ϴ denote the angle between the pair of rays. When ϴ is not 90, the following observations could be new:

b) the family of chords envelops an ellipse E', which is in general non-concentric and not axis-aligned with E.
c) Over all M, the semi-axes of E' are variable as is their ratio.
d) Remarkably, the *area* of E' is invariant over M!!!
e) Over all M, the centers of E' span a 3rd ellipse, E'', this time concentric, axis-aligned and *homothetic* to E.

The video also shows a "real-time" discovery: when ϴ=60, the envelope of chords in [b] will be inscribed in the triangle formed by M, and the two rays emitted symmetrically from the normal (at 30 and -30 degrees) and the chord between the intersections.

Any help w/ proofs appreciated!

[1] D. Laurain, Private Communication, Jan 2021. 
[2] E. Weisstein, "Frégier's Theorem", MathWorld, 2020. https://mathworld.wolfram.com/FregiersTheorem.html