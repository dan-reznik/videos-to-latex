The family of poristic triangles [1] (blue) is shown with a fixed circumcircle (purple) and incircle (green), centered on the circumcenter X3 and incenter X1, and with radii R and r, respectively. Here r/R=0.36266.

Also shown is the excentral triangle (green). Its vertices (excenters) describe a circular locus (orange) of radius 2R, i.e., the circumcircle of the excentral triangles is stationary and centered on its X3, or the reference X40 [2].

Also shown is (i) E10, the X10 (spieker point) centered circumconic (pink) to the poristic triangles, (ii) E5', the circumconic to the excentrals (light blue) centered on their X5 (X3 of the reference). Interestingly, not only do both conserve the ratio of their axis' lengths (aspect ratio), this ratio is identical!

Other properties

 - the axes of E10 and E5' are parallel. these are also parallel to E9, the X9-centered circumellipse (aka, the Circumbilliard [3]).
- E10 contains X100.
- Other conics containing X100 and with parallel axes: E1, the X1-centered circumconic, and I3' and I5', the X3- and X5-centered inconics to the Excentral Triangle [3].

Sountrack: Chopin, "Polonaise"

References:

[1] W. Gallatly, "The Modern Geometry of the Triangle",  F. Hodgson, London 1914
[2] B. Odehnal, "Poristic Loci of Triangle Centers", 2011. http://www.heldermann-verlag.de/jgg/jgg15/j15h1odeh.pdf
[3] D. Reznik and R. Garcia, "The Circumbilliard: any Triangle has an Elliptic Billiard", 2020, in preparation.