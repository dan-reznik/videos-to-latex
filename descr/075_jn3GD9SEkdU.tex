This is a generalization of results regarding triangles in previous videos.

A family of harmonic polygons H (magenta) is shown inscribed in a circle. John Casey (1888) constructs them as the inversive images of regular N-gons with respect to some circle [1]. T. Sharp (1945) constructs them as projections of a regular polygon [2].  Harmonic polygons have a symmedian point K. If N is even these are the intersections of diagonals. If N is odd these are the intersections of lines from the vertices to the opposite points of contact with the ellipse enveloped by the sides (aka the Brocard inellipse).

The video shows a few curious properties of the inversive image H' of harmonic polygons (unproved) wrt to an inversion circle C with center Cinv.

1) Trivially H' is also harmonic (the composition of two inversions is an inversion).
2) The symmedian K' of H' is stationary and collinear with Cinv and K.
3) Amazingly, over rigid rotations of H about K, K' is stationary!
symmedian inversive image of the harmonic family wrt fo a fixed circle is harmonic

[1] John Casey, "A sequel to the first six books of the Elements of Euclid, containing an easy introduction to modern geometry, with numerous examples". Dublin: Hodges, Figgis & co., 1888.
[2] T. Sharp, "Harmonic Polygons", The Mathematical Gazette, Vol. 29, No. 287 (Dec., 1945), pp. 210-213.