The 1d family of 3-periodics in an Elliptic Billiard not only has a stationary Mittenpukt X(9), but it also conserves an incredible quantity: r/R, the ratio of inradius-to-circumradius. Conservation corollaries include: (i) the sum of orbit's cosines = 1+r/R; (ii) the product of excentral cosines = r/(4R); and (iii) the ratio of excentral-to-orbit areas = r/(2R). Here we visualize (i) and (ii).

LEFT: An a/b=1.5 elliptic biliard is shown as well as its N=3 family of orbits (blue). For each orbit the excentral polygon is also shown (green). The red dot at the center of the billiard represents the stationary Mittenpunkt X(9). 

MIDDLE: a 2d representation of both orbit and excentral polygons is shown: a first vertex is placed at (0,0), a second one at (1,0), and a third one at some (u,v) location on the plane such that this normalized triangle is similar to the orbit (blue) or excentral (green). Drawn in the background are level curves of r/R for such a (u,v) family of triangles. Notice the (u,v) tip of the orbit triangle follows a constant r/R level-curve, whereas the excentral one does not. Note r/R=1+cosA+cosB+cosC, i.e., r/R level curves are congruent to sum-of-cosine ones.

RIGHT: the same as LEFT except level curves are shown for the *product* of cosines. Notice the excentral arm follows a product level curve perfectly whereas the orbit one does not.

In summary, orbit triangles conserve the *sum* of their cosines whereas the excentrals conserve the *product* of their cosines.

More Info:  https://dan-reznik.github.io/Elliptical-Billiards-Triangular-Orbits/