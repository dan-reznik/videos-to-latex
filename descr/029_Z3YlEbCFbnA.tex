The Brocard Porism [1] is a 1d Poncelet family of 3-periodics (triangles) inscribed in a circle and circumscribing their Brocard Inellipse [2]. Remarkably, the whole family has stationary brocard points Ω1 and Ω2 and fixed Brocard angle ω (Chapter XVII)[3].

The vertices of the Second Brocard Triangle T2 [4,5] are obtained by intersecting cevians through X6 with the Brocard circle, centered on X182 [6], i.e., T2 is inscribed in said circle.

It turns out that over the Brocard Porism, the Brocard points of T2 are also stationary [7]. Since they are inscribed in a circle, these define a "nested" Brocard porism.

The video shows such 4 iterations of such a nesting, whereby successive T2s are computed. Notice:

(1) successive Brocard inellipses tend to a shrinking circle
(2) successive T2's approach an ever shkring equilateral circumscribing
(3) the center of Brocard circles converges to a limit point. Could this be a triangle center?  
(4) the left (resp. right) sequence of Brocard points lies on a circle (shown red) centered on O2 (resp. O1), the first (resp. second) Beltrami point (circumcircle inverses of the Brocard points). The two circles have the same radius.

References:
 
[1] R. Bradley & G. Smith, "On a Construction of Hagge", Forum Geometricorum, vol 7, pp. 231-–247, 2007, url forumgeom.fau.edu/FG2007volume7/FG200730.pdf
[2] E. Weisstein, "Brocard Inellipse", Mathworld, 2020. https://mathworld.wolfram.com/BrocardInellipse.html
[3] R. Johnson, "Advanced Euclidean Geometry", Dover, New York, 1960. bit.ly/33cbrvd
[4] E. Weisstein, "Second Brocard Triangle", Mathworld, 2020. https://mathworld.wolfram.com/SecondBrocardTriangle.html
[5] B. Gibert, "Brocard Triangles", CTC, 2020. https://bernard-gibert.pagesperso-orange.fr/gloss/brocardtriangles.html
[6] C. Kimberling, "X(182)", ETC, 2020. https://faculty.evansville.edu/ck6/encyclopedia/ETC.html
[7] R. Garcia and D. Reznik, "Loci of the Brocard Points over Selected Triangle Families", arXiv, in preparation.