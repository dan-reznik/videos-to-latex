The video shows the family of 3-periodic orbits (blue) in a hyperbolic billiard (black). The top branch is reflective (about the normal) while the bottom one is refractive, i.e., it reflects incoming rays about the tangent vector to the curve. Let P1 denote the vertex on the top branch and P2,P3 those on the bottom branch. 

Let dij denote the distance |Pi-Pj|, and ci the cosine of the 3-periodic triangle at vertex Pi.

As the top legend shows, this family conserves the quantity d12+d13-d23, and c1-c2-c3.

Compare with the standard elliptic billiard: it conserves L=d12+d13+d23, and the sum of cosines c1+c2+c3, i.e., for the hyperbolic billiard, those quantities associated with the P2P3 trajectory segment (the refracted one) must appear with their sign inverted for invariants to continue to hold.