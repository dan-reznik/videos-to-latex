The isogonal (or isotomic) conjugate of a line with respect to a triangle is a circumconic [1].

If an elliptic billiard is regarded as a stationary, X(9)-centered circumellipse to the 3-periodic family, we can analyze its isogonal and isotomic conjugate lines dynamically. These turn out to be [3] the orbits' Antiorthic Axis [4] and Line L(31) [5], respectively.

In particular, what is the envelope [2] of such straight lines over the family of 3-periodics?

The two lines can be constructed with any pair of Triangle Centers lying on them (Peter Moses provides many in [3]). For special reasons, we use:

a) L(1), Antiorthic Axis, X(44)X(513)
b) L(31), Line X(514)X(661) -- note: to be reviewed, but this is the Gergonne line of a triangle derived from the reference one. ACT?

The video shows the locus of the above Triangle Centers for an a/b=1.618=phi billiard, drawn black. Orbits are drawn blue, as well as the envelope of said lines.

Specifically:

Left: the non-elliptic locus of X(44) is shown red. The elliptic locus of X(1155) is shown green. Notice how the Antiorthic axis (dashed blue) is dynamically tangent to the latter, i.e., the locus of X(1155) is the envelope of the Antiorthic Axis.

Right: the non-elliptic locus of X(857) which lies on L(31) is shown red. The elliptic locus of X(908), also on L(31), is shown green. Notice how line L(31) (dashed blue) is dynamically tangent to the latter, i.e., the locus of X(908) is the envelop of L(31).

In both cases, "C" depicts where the envelope currently is.

[1] http://mathworld.wolfram.com/Circumconic.html
[2] http://mathworld.wolfram.com/Envelope.html
[3] Peter Moses, mentioed by Clark Kimberling, ETC under X(9), https://faculty.evansville.edu/ck6/encyclopedia/ETC.html
[4] http://mathworld.wolfram.com/AntiorthicAxis.html
[5] https://faculty.evansville.edu/ck6/encyclopedia/CentralLines.html