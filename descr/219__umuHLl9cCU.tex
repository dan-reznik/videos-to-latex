Consider an Elliptic Billiard (EB) with semiaxes a,b, its family of 3-periodics (triangles), and the loci of its triangle centers [1]. We know for example, that the Mittenpunkt X9 is stationary at the EB center [2]. 

When a/b≈2.81652, the locus of X140, the midpoint of X3 and X5, is a perfect *circle* of radius ≈0.593 b.

Also, when a/b≈2.077, the locus of X547, the midpoint of X2 and X5, is a perfect circle of radius ≈.678 b.

Not shown: when a/b≈1.669, the locus of X631, which is at 3/5 between X3 and X2, becomes a perfect circle with radius ≈ .1651 b. So we expect for there to be many more, but the rule is still not clear.

Mysteriously, these other bonafide Euler Line citizens, lying at midpoints of known Euler Line centers, can *never* produce circular loci:

* X376 (mid of X2,X20)
* X381 (mid of X4,X5)
* X548 (mid of X5,X20)
* X549 (mid of X2,X3)
* X550 (mid X3,X20)

So whatr gives? For more of our own research on the topic see [2].

[1] Clark Kimberling, Encyclopedia of Triangle Centers (ETC), https://faculty.evansville.edu/ck6/encyclopedia/ETC.html
[2] D. Reznik, R. Garcia, and J. Koiller, "N-Periodics in the Elliptic Billiard", https://dan-reznik.github.io/Elliptical-Billiards-Triangular-Orbits/videos.html