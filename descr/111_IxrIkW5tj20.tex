An a/b=1.5 (resp. 2.0) elliptic billiard is shown on the left (resp. right). Also shown is the family of 3-periodics (blue). The Excentral Triangles are shown (green) whose vertices are the Excenters. We have shown their locus is an ellipse congruent with the the MacBeath Circumellipse [2] and similar to the Incenter elliptic locus [1].

Also shown is the MacBeath Inellipse [3] of the Excentral Triangle (red). Its center and foci are X5, X4, and X3 respectively, which in terms of the reference triangle are X3, X1, and X40.

Suprisingly, the ratio of this magical Inellipse's semiaxes η'/η is invariant over the 3-periodic family.

Also shown are the following loci:

(1) Orange: Inellipse contact points (red dots) with the Excentral Triangle
(2) Pink: X1742, the Brianchon Point of the Inellipse. This is X(264) of the Excentral Triangle.

Soundtrack: Claude Debussy, "Première Arabesque"

References:

[1] Garcia R., Reznik, D., Koiller J., "Loci of 3-periodics in an Elliptic Billiard: why so many ellipses?", arxiv: https://arxiv.org/abs/2001.08041
[2] https://mathworld.wolfram.com/MacBeathCircumconic.html
[3] https://mathworld.wolfram.com/MacBeathInconic.html