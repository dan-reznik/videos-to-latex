An a/b=1.618 elliptic billiard (EB) is shown (black) as well as its family of 5-periodics (blue). The two foci are marked "f1" and "f2". Also shown is the confocal caustic (brown).

Left: the feet of perpendiculars dropped from either focus onto N-periodic sides are concyclic (dashed red circle). These define two N-sided bicentric polygons (transparent blue). Amazingly, the product of their areas is invariant over the N-periodic family. Notice our previous observation the product of sums of perpendicular lengths d_1, d_2 is invariant for odd N [1]

Right: the feet of perpendiculars dropped from either focus onto the sides of the polygon tangent to the EB at N-periodic are also bicentric (dashed red circle, of slightly higher radius than the former). They define two N-sided cyclic polygons (transparent green). Amazingly, the product of their areas is invariant over the N-periodic family. Notice our previous observation that the sum of squared perpendicular lengths e_1,e_2 is invariant from any point "m" in the plane (it is equal for f1 and f2) [2]

Also intringuing is the fact that the ratio of areas of the blue (left) polygons is dynamically the same as the ratio of areas of the green (right) ones, i.e., A[d1]/A[d2] = A[e1]/A[e2].

Not shown: for N even, the two polygons have equal areas (pairs of perpendiculars will be parallel). Of course in both cases the product (and in this case the ratio=1) will be preserved.

For both cases notice the "triple junction" of pairs of perpendicular feet with an N-periodic vertex.

Soundtrack: Heitor Villa-Lobos, "Suite Popular Brasileña" (02). Performed by Pablo De Giusto

[1] Concyclic Feet of Focal Pedals and Invariant Product of Sums of Lengths for odd N, https://youtu.be/OT-xAdbOp8o
[2] Sum of square altitudes from arbitrary point to N-periodic tangents is invariant, https://youtu.be/sw8pJFMV00w