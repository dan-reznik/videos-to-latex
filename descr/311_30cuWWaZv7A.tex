Take a pair of concentric, axis-aligned ellipses which are homothetic to one another. Consider their 1d family of Poncelet Polygons. The video showcases a few suprising facts for N=3,4,5,6, namely:

1) for all N, the following are invariant over the 1d Poncelet family:

1.1) the area of the Poncelet polygons. (system is affine image of two-circle, regular poly case)
1.2.) the sum of sidelengths squared (proved, stems from cyclic trigonometric groups)
1.3) the sum of cotangents of angles (needs proof!)

2) Let K denote the Steiner's Curvature Centroid (Krümmungs-Schwerpunkt) of the Poncelet polygons. This is the weighted average of vertices where weights are the sum of double angles of the vertices.

2.1) for even N (resp. odd N), K is stationary at O (resp. moves along an ellipse, needs proof).
2.2) for odd N, the area of the pedal polygon of the N-periodic with respect to K is invariant! (needs proof)

Note: Steiner showed in 1825 that the pedal polygon w/ respect to K has extremal area.

Note that for N=3, K is the circumcenter X3. The pedal to a triangle wrt to X3 is the medial, whose area is 1/4 that of the reference (the latter's area is constant).