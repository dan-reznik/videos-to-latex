Left: A family of N-periodics (blue) in the elliptic billiard, interscribed in a pair of confocal ellipses (black and brown) is shown on the left which classically conserve perimeter and Joachimsthal's constant [1]. Without loss of generality, N=5, and the aspect ratio of the billiard is 1.75. This family also conserves the sum of internal angle cosines; furthermore, the family of "outer" polygon (green), with sides are tangent to the outer ellipse at the N-periodic vertices) conserves the product of its internal angle cosines [2,3,4]. In fact, if N is odd, the ratio of areas of the outer polygon to that of N-periodics is also conserved [5]. By virtue of the Poncelet grid [6], the outer family is also Ponceletian, as it is inscribed in an ellipse (dashed green). 

Middle: a Poncelet family of 5-gons (blue) with a fixed incircle which is the affine image of the confocal family (left) such that the caustic is sent to a circle. This family also conserves the sum of its cosines [3]. Surprisingly it is equal to that of its confocal pre-image (left). Notice its outer polygon (green) does not conserve its product of cosines (grayed out).

Right: a Poncelet family of 5-gons (blue) which is the affine image of the confocal family (left) such that the ellipse (dashed green) to which the outer polygons are inscribed is sent to a circle (dashed green). Like its confocal pre-image, this family of outer polygons also conserves the product of its cosines [3]. Surprisingly it is equal to that of its confocal pre-image (left). Notice the N-periodics here (blue) do not conserve their sum of cosines (grayed out).

Experimentally, we have also noticed that the confocal+incircle families sweep the same curve in N-dimensional "cosine space", as do the outer+circumcircle families. Proof pending!

[1] S. Tabachnikov, "Geometry and Billiards", Student Mathematical Library, vol 30, American Mathematical Society, 2005. http://www.personal.psu.edu/sot2/book...​
[2] D. Reznik, R. Garcia, and J. Koiller, "Can the Elliptic Billiard still surprise us?", Math Intelligencer, 42, 2020. http://rdcu.be/b2cg1​
[3] A. Akopyan, R. Schwartz, and S., "Billiards in Ellipses Revisited", Eur. J. Math, 2020. 
[4] M. Bialy and S. Tabachnikov, "Dan Reznik's Identities and More",
Eur. J. Math., 2020.
[5] A.C. Chavez-Caliz, "More About Areas and Centers of Poncelet Polygons" , Arnold Math J., 2020.
[6] R. Schwartz,  "The Poncelet grid", Advances in Geometry, 7:2, 2007.
[7] M. Levi and  S. Tabachnikov, "The Poncelet Grid and Billiards in Ellipses",Am. Math. Monthly,  114:10, 2007.