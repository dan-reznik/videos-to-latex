Family of Poncelet 3-Periodics between two concentric conics: an external circle and an internal ellipse. Indeed, these are closely related to the family of Poristic Triangles [1,2,3]. Several remarkable invariants are observed:

- Orthic Inradius (the X1 of the orthic is the same as X4 of the 3-periodics, since these turn out to be always acute).
- Orthic Circumradius
- |X5-X3|
- Axes of the MacBeath Inellipse (shown olive green). Rotates ridgly about X5 (its foci are X3 and X4).
- product of cosines (hold for all N, the orthics conserve the *sum* of cosines). Note: there was a bug in the reporting of this quantity which appears non-constant on the video. I will soon post a video where the bug is fixed confriming the invariance of the product of cosines.
- sum of squared sidelenghts

Relation to the Poristics:

Consider a reference frame centered on the orthic's circumcienter (X5 of the 3-periodic), with one axis oriented toward the orthic incenter (X4 of the 3-periodic). With respect to this reference frame: 

- the orthic triangles (orange in the video) are none other than Chapple's Poristic Family [1].
- the Poncelet 3-periodics (blue on video) are are the excentrals to the Poristic Family.
- the MacBeath inellipse becomes stationary. It is the caustic of the poristic family of excentrals. The outer conic is a circle equal to our original one.
- the original inner ellipse (black) becomes a rigidly rotating inconic centered in the excentrals' circumcenter. See [4]

We have studied these in [3].

[1] Chapple's Porism, Wikipedia. https://en.wikipedia.org/wiki/Poncelet%27s_closure_theorem
[2] B. Odehnal, "Poristic Loci of Triangle Centers", 2011, https://www.geometrie.tuwien.ac.at/odehnal/pltc.pdf
[3] R. Garcia and D. Reznik, "Related by Similarity:  Poristic Triangles and 3-Periodics in the Elliptic Billiard", 2020. https://arxiv.org/abs/2004.13509
[4] D. Reznik, "Poristic Family: X1-Ctr Circumconic & X40-Centered (Excentral) Inconic: Identical Invariant Axes", YouTube. https://youtu.be/PGdQY7f626Y