Video shows a number of phenomena related to pedals, and inversions of N-periodics in the elliptic billiard. The main result is that there are two sequences of 3 operations which take the original N-periodic to a translated copy of itself. Theses are:

Path 1. pedal+pedal+inversion

1.1) the f1-pedal to N-periodics is a bicentric family inscribed in a circle whose radius is equal to the major axis of the caustic. To be proved: bicentric families conserve the sum of cosines.
1.2) the f1-pedal to (1.1), i.e., the "squared" f1-pedal, is equiperimeter and homothetic to the f2-inversive family. these families are inscribed in Pascal's Limaçon (not shown), and both have identical invariant sum of cosines.
1.3) the f1-inversion of (1.2) produces a copy of N-periodics translated by (-2c,0), where c is the original focal distance sqrt(a^2-b^2).

Path 2. inversion+antipedal+antipedal

2.1) the f1-inversion of N-periodics produces an invariant perimeter family whose invariant sum of cosines is identical to (1.2).
2.2) the f1-antipedal of (2,1) produces a bicentric family which conserves the sum of cosines, identical to the sum of cosines of (1.1). Additionally, the product of its area with that of (1.1) is constant.
2.3) the f1-antipedal of (2.2), i.e., the "squared" f1-antipedal produces a family identical to (1.3).

Errata:

- at 5:15: "the two perimeters are identical": what's meant is that they are both *invariant*
- at 14:18: the numerics are correct in that it is the *product* of areas of the two bicentric families, (i) f1-pedal and (ii) polar which is conserved. notice these families are "out of phase, mirrored" coopies of each other and their sum of cosines is identical.