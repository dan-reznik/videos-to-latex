Consider the 3-periodic Poncelet family(blue) inscribed in an external ellipse (black) and circumscribed about an internal concentric circle (brown). The excentral triangle (dashed green) is also shown, its vertices are known as the "excenters". The following properties are illustrated:

a) the circumradius of the family is constant, shown before in [1].
b) the sum of areas of the three circumellipses (contain the vertices) centered on the excenters is invariant, though each is variable.
c) This result has been generalized to 3 other concentric Poncelet families [2], namely, for the constant total area property to hold, the 3 circumellipses need to be centered on the vertices of the anticevian triangle wrt to the common center.

[1] D. Reznik, "Between an Ellipse and a Concentric Circle: Poncelet 3-Periodics Identical to Poristic Triangles", YouTube, July 2020. https://youtu.be/ML_AZoX736w

[2] D. Reznik, "Constant-Area Poncelet Propellers", 2020. https://youtu.be/crXxPJ93ZDk