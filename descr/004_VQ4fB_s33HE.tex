From Left to Right we see four animations (a Poncelet Quartet):

First: A family of Poncelet N-periodics (blue) in the elliptic billiard, i.e., interscribed in a pair of confocal ellipses (black and brown). We chose N=5 without loss of generality. Classical conservations include perimeter and Joachimsthal's constant [1]. This family also conserves the sum of internal angle cosines. Also shown are its ``outer´´ polygons (solid green) whose sides are tangent to the billiard at the N-periodic vertices; i.e., they run along each vertex's external bisector. The outer family conserves the product of internal angle cosines [2,3,4]. The ratio of areas of the outer polygon to that of 3-periodics is also conserved [5]. The locus of their vertices is an ellipse (dashed green), thanks to the Poncelet grid [6,7]. 

Second: the affine image of the billiard family which sends the caustic to a circle. This family also conserves the sum of its cosines [3], and it is equal to that of its confocal pre-image (left). Notice the image of the original outer polygons (dashed green) does not conserve its product of cosines (grayed out). Unlike the original outers, these do not have sides parallel to the external bisectors of corresponding N-periodic vertices. In N=3 parlance this would akin to saying "they are not excentral triangles".

Third: the affine image of the confocal family which sends the outer ellipse to a circle (black). Like confocal excentrals, this new family also conserves the product of cosines [3], and its value is equal to that of the confocal outer family. Also shown is the image of the confocal excentrals under the same affine transformation. This family circumscribes what is now a circle, i.e., it has a fixed incircle; like the previous one it conserves the sum of cosines, and surprisingly, it is equal to the sum conserved by the original confocal family (first) and the one with incircle (second), though none of these polygons is homothetic to one another nor do they have equal angle vectors (though we believe each sweeps the same curve in 5d angle or cosine space, see [8]).

Fourth: the affine image of billiard N-periodics (first) which sends the locus of outer vertices to a circle (dashed green). Like its confocal pre-image, this also conserves the product of its cosines [3], and it is equal to that of its confocal pre-image (first) and that of the family with incircle (third). 

[1] S. Tabachnikov, "Geometry and Billiards", Student Mathematical Library, vol 30, American Mathematical Society, 2005. http://www.personal.psu.edu/sot2/book...​
[2] D. Reznik, R. Garcia, and J. Koiller, "Can the Elliptic Billiard still surprise us?", Math Intelligencer, 42, 2020. http://rdcu.be/b2cg1​
[3] A. Akopyan, R. Schwartz, and S., "Billiards in Ellipses Revisited", Eur. J. Math, 2020. 
[4] M. Bialy and S. Tabachnikov, "Dan Reznik's Identities and More",
Eur. J. Math., 2020.
[5] A.C. Chavez-Caliz, "More About Areas and Centers of Poncelet Polygons" , Arnold Math J., 2020.
[6] R. Schwartz,  "The Poncelet grid", Advances in Geometry, 7:2, 2007.
[7] M. Levi and  S. Tabachnikov, "The Poncelet Grid and Billiards in Ellipses",Am. Math. Monthly,  114:10, 2007.
[8] D. Jaud, D. Reznik, and R. Garcia, "Poncelet Plectra: Harmonious Properties of Cosine Space", arXiv:2104.13174 , 2020.