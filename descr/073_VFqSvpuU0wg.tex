Let H (blue) be a harmonic polygon (blue) , and let H' (red) be a new polygon with vertices at the inversions of those of H wrt to a circle centered on a point C. H' is also harmonic [1, Section VI].

The video shows two potentially new observations which are likely generalizations of [3, Thm 11], to harmonic polygons.

a) the isocurves of Brocard angle of H' are circles in the pencil of the circumcircle and Brocard circle of H, also known as the Schoute pencil [2,3]. 
b) if C is on the Brocard circle or on the radical axis of the pencil, then the Brocard angle of H' is equal to that of H.

[1] J. Casey, "A Sequel to Euclid Elements", Longman, London, 1888.
[2] P. Pamfilos, "Symmedian", Geometrikon, http://users.math.uoc.gr/~pamfilos/eGallery/problems/Symmedian.pdf
[3] R. Johnson, "Directed Angles and Inversion, and a Proof of Schoute's Theorem", American Math. Monthly, 24:313–317, 1917.