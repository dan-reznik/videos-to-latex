1) start with the family of parabola-inscribed triangles T circumscribing an incircle centered on the focus of the parabola (X1 of the family). This family is actually the tangential triangles to a bicentric family whose circumcenter is on the incircle, see our previous video [1].

2) Then add to the above a new triangle family T' , the polar triangles of T wrt to the parabola, i.e., with sides bound by the tangents to the parabola at the vertices of T. A basic property of the parabola dictates that the circumcircle of any triangle bounded by tangents goes thru the focus, i.e., T'contains  X1 of the T. This "polar" family is Ponceletian, inscribed in a (dashed dark red) hyperbola, and circumscribing the original parabola

Here are a few curious properties of the  polar family, the most degenerate Poncelet family I have ever seen.

1) its Euler line goes thru the green parabola focus. Its X(26), a point normally not on the circumcircle, is on the circumcircle, and stationary on said focus. X(68) and X(110) are stationary on the left and right vertex of the hyperbola the family is inscribed to. X(161) is stationary on the left extreme of the incircle. Incidentally, X(110) is the focus of the Kiepert (in)Parabola, shown Magenta below. 

2) the loci of Xk, k={2, 3, 4, 5, 6, 20, 22, 23, 24, 25, 49, 51, 52, 54, 64, 66, 67, 69, 74,110,113,125,140,141,143,146,154,155,156,159,161,182,184,185,186, 193,195,... are straight lines parallel to the directrix (see a few below)

3) the loci of Xk, k=99, 107, 112, 249, 476, 691, 827, 907, 925, 930, 933, 935 are circles in a parabolic pencil: they all touch at a single point: X(110), the aforementioned focus of the Kiepert parabola, shown Magenta below).  Most of these centers lie on the circumcircle, but a few don't

4) though visually, the locus of X1 of T' looks like a vertical line, numerically, it is likely a parabola. The jury is still out on this one.

To do: what is the locus of the vertex of the Kiepert over the family?

[1] https://youtu.be/9thwJcfUBmM