The animation shows a family of triangles (blue) inscribed in an ellipse E (black). Triangle vertices are positioned so as to trisect E's perimeter (I had to use inverse elliptic functions to locate them).

Each inset shows an E with a different aspect ratio. In each case, the locus of the triangle centroid G (brown) and envelope of the sides (pink) is shown. For high enough a/b, the envelope can be contain self-intersections and/or cusps. The locus of G seems to be always convex.

Also indicated (at the top) are the variable perimeter (L) or area (A) of the trisecting triangle family. 

Open questions:

a) What kind of curve is the locus of G? Is it even algebraic? Can it be shown it is always convex?
b) What kind of curve is the envelope? At what aspect ratio does it become non-regular and/or self-intersecting?
c) Are there any Euclidean invariants manifested by the family? (funcions of angles, perimeters of derived triangles, area ratios, etc.)

Note: in a numerical experiment, we found out no triangle centers from X(1) to X(10k) sweep conic loci.