This is joint work with Profs Ronaldo Garcia ad Pedro Roitman.

The "Brocard porism" is a family of triangles with fixed Brocard points and invariant Brocard angle [1]. The caustic is known as the "Brocard Inellipse", centered on X39. Its foci coincide with the stationary Brocard points of the family [4]. The symmedian point X6 of the family is also stationary [3] . We have studied the relationship between the Homothetic and Brocard family in [5]. 

The video shows that an N greater than 3 "generalized" Brocard porism (GBP) can be constructed as the polar image of the Poncelet homothetic family wrt to a circle C centered on one of the internal foci.

That the homothetic family conserves area is a trivial fact. Other conservations such as the sum of squared sidelengths, and sum of internal angle cotangents is proved in an upcoming paper [2].

The video shows that though the GBP does not conserve area, it conserves sum of inverse squared sidelengths AND sum of its angle cotangents (distinct from the former).

Erratum 1: in a parallelogram consecutive angles are supplementary (not opposing).

PS.1 -- the video does not mention a curious fact. For the N=3 case, the focus of the homothetic caustic where C is centered coincides with the (stationary) symmedian point X6 of the Brocard porism. You can also visualize the N=3 case live in your browser here: https://bit.ly/3tkDiEG

PS.2 -- the reason for null sum of cotangents in the N=4 generalized brocard porism (GBP) is because cyclic polygons have supplementary opposing angles.




References:

[1] C. Bradley, "The geometry of the Brocard axis and associated conics", CJB/2011/170, 2011, http://people.bath.ac.uk/masgcs/Article116.pdf
[2] S. Galkin, R. Garcia, and D. Reznik, "Invariants of Affine Images of Regular Polygons", to appear, 2021.
[3] D. Reznik and R. Garcia, "An Infinite, Converging, Sequence of Brocard Porisms", 2020. https://arxiv.org/abs/2010.01391
[4] E. Weisstein, "Brocard Inellipse", MathWorld, 2021. https://mathworld.wolfram.com/BrocardInellipse.html
[5] D. Reznik and R. Garcia, "Related By Similarity II: Poncelet 3-Periodics in the Homothetic Pair and the Brocard Porism", Intl. J. of Geom., 10(4), 2021, pp, 18--31.