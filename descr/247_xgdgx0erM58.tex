In 1871 Cayley published a paper about a very curious property of octagons inscribed in conics, known as "Octagramma Mysticum", also explained here: https://arxiv.org/pdf/1209.4795.pdf

Namely, if an octagon is inscribed in an ellipse, the intersection of lines passing thru every other side will all land on the same conic, e.g., an outer ellipse. 

The video shows what happens when the octagon is an N-periodic *orbit* of an elliptic billiard (a/b=1.5).

- We found the loci of the intersection points will be an ellipse *confocal* with the billiard.

- The video shows that in fact N=5,6,7 will also produce elliptic foci which are confocal.

- We also found that intersections of  P(i,i+1) with P(i+3,i+4) also yield confocal elliptic loci, larger than the former ones.

https://dan-reznik.github.io/Elliptical-Billiards-Triangular-Orbits/