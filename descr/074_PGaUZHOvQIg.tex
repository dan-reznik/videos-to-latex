Steiner's porism [1] consists of a chain of touching circles moving (as their radii vary) in the interstice between to non-concentric circles. They are the inversive image of a "ball bearing" with identical circular disks between two concentric circles. In [2,3] it was stated that 4 harmonic polygons can be built out of Steiner's porism. Deeper connections with invariant symmetric polynomials are are explored in [4].

Let ri be the radii of the disks in a Steiner porism. The video shows sum(1/ri), sum(1/ri^2), sum(1/ri^3) is conserved.

The centers of the original disks are the vertices of a regular polygon. Their inversive image with respect to C is a harmonic polygon (green).

Also harmonic are the polygons (magenta) with vertices at the inversions of points of contact of the bearings with (i) the inner circle, (ii) outer circle, and (iii) with each other.

All said 4 harmonic families are distinct, and each conserves a different sum of cotangents of internal angles (sums of cot^2 and cot^3 are also conserved).


[1] E. Weisstein, "Steiner's Porism", MathWorld 2021. https://mathworld.wolfram.com/SteinersPorism.html

[2] G. Tarry & J. Neuberg, "Sur les polygones et les polyèdres harmoniques", Comptes rendus de l'Association française pour l'avancement des sciences, 1887. Note: extrait du Congrès de Nancy, 1886.

[3] T. C. Simmons, "A new Method for the Investigation of Harmonic Polygons, Proc London Math. Soc., Volume 1--18, number 1, pp. 289--304, 1886.

[4] R. Schwartz & S. Tabachnikov, "Descartes Circle Theorem, Steiner Porism, and Spherical Designs", Am. Math. Monthly, volume 127, Issue 3, 2020.