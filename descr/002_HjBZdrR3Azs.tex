Left: A family of elliptic billiard 3-periodics (blue), i.e., , interscribed in a confocal pair of  ellipses (black and brown). This classically conserves perimeter and Joachimsthal's constant [1].  It turns out it also conserves the sum of internal angle cosines; furthermore, the Ponceletian family of its excentral polygons (green, inscribed in the dashed-green ellipse) conserves the product of its internal angle cosines [2,3,4]. The ratio of areas of the outer polygon to that of 3-periodics is also conserved [5]. Curiously, the excentrals also conserve the ratio of squared sidelengths by the product of sidelengths.

Middle: the affine image of billiard 3-periodics which sends the confocal caustic to a circle. This family also conserves the circumradius (not shown) and therefore the sum of its cosines [3]. Surprisingly, the latter is equal to that of its confocal pre-image (left). No conservations are known for its outer polygon (dashed green). Note these are affine images of confocal  excentrals, but are *not* excentrals of the current family.

Right: affine image of the confocal 3-periodics which sends the elliptic billiard to a circle (black). This also conserves the product of its cosines [3]. Surprisingly it is equal to the value conserved by the confocal excentrals (solid green, left). The caustic to its outer family (solid green) is a cirlce, i.e., this conserves the sum of cosines. Surprisingly, it is the same quantity conserved by the original billiard family.

[1] S. Tabachnikov, "Geometry and Billiards", Student Mathematical Library, vol 30, American Mathematical Society, 2005. http://www.personal.psu.edu/sot2/book...​
[2] D. Reznik, R. Garcia, and J. Koiller, "Can the Elliptic Billiard still surprise us?", Math Intelligencer, 42, 2020. http://rdcu.be/b2cg1​​
[3] A. Akopyan, R. Schwartz, and S., "Billiards in Ellipses Revisited", Eur. J. Math, 2020. 
[4] M. Bialy and S. Tabachnikov, "Dan Reznik's Identities and More",
Eur. J. Math., 2020.
[5] A.C. Chavez-Caliz, "More About Areas and Centers of Poncelet Polygons" , Arnold Math J., 2020.
[6] R. Schwartz,  "The Poncelet grid", Advances in Geometry, 7:2, 2007.
[7] M. Levi and  S. Tabachnikov, "The Poncelet Grid and Billiards in Ellipses",Am. Math. Monthly,  114:10, 2007.