The family of N=3 (triangular) orbits (black) for an elliptic billiard (a/b=1.5) is shown.

For each orbit its anticomplementary triangle T' is also shown (blue). This triangle has sides parallel to the orbit opposite ones; the orbit's vertices are each of its sides' medians.

Both the 9-point circle (pink, center C bar) and incircle (green, center I bar) for T' are shown. The former is known to coincide with the orbit's circumcircle/center.

Also shown is the Feuerbach point "F bar" for T', at the touchpoint between circle C bar and I bar. This is known to be X(100) of the orbit triangle (anticomplement of the Feuerbach point).

The amazing property about F' is that it sweeps the elliptic billiard exactly. Hoever, his video shows two new properties:

a) the locus of the anticomplementary triangle is a non-elliptical curve.
b) the 3 contact points of the incircle of T' (intouchpoints) also sweep the billiard.

This seems dual with the property that the Feuerbach point of the orbit sweeps the caustic and the 3 extouch points (where the excircles touch the sides) sweep the caustic.

More info: https://dan-reznik.github.io/Elliptical-Billiards-Triangular-Orbits/