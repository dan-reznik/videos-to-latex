This is a continuation of joint with Prof Ronaldo Garcia and Hellmuth Stachel, see [2]. Let E be an ellipse, and let M be a point on it about which we will calculate a (deltoidal) Orthocaustic, aka Negative Pedal Curve [1]. Let P_i' the 3 cusps of said envelope and P_i their preimages on E [2]. 

Here we cover properties associated with the Osculating Circles centered at K_i and tangent to the Ellipse at the P_i.

Note: The center of an osculating circle is known to lie on the Evolute to the curve [3], and for the Ellipse, this is an astroid of known parametrization [4]. 

We have found:

1) The three osculating circles tangent to E at the P_i intersect at M. 
2) The K_i form a triangle T whose area is invariant over all M. 
3) The P_i's lie on the 3 osculating circles. Corollary: the lines P_i K_i are diameters of the osculating cirlces.
4) Said diameters concur at M.
5) Let T' be the triangle defined by the deltoid cusps P_i. Let A' be its area, and A be the area of T. A'/A=4, irrespective of E's dimensions.

[1] Orthocaustic, aka., negative pedal curve: https://mathworld.wolfram.com/NegativePedalCurve.html
[2] Steiner's Hat, https://youtu.be/LxADeM1-WHw
[3] Osculating Circle, https://mathworld.wolfram.com/OsculatingCircle.html
[4] Ellipse Evolute, https://mathworld.wolfram.com/EllipseEvolute.html