The video shows a Poncelet N=4 family (blue) inscribed in an outer circle D  (black), with sides tangent to two internal circles D1, and D2 (orange), such that C,D1,D2 are in the same pencil. The internal limiting point l2 is also drawn.  

What is interesting about this family is that it "automatically" closes: departing from a point P1 on C:

1) let P2 be the intersection with C of the cw tangent to D1 from P1. 
2) let P3 be the intersection with C of the cw tangent to D2 from P2.
3) let P4 be the intersection with C of the ccwtangent to D1 from P3.
4) let P5 be the intersection with C of the ccwtangent to D2 from P4.

Then P5 automatically meets P1, i.e., the polygon closes, for any choice of circles.

The reason is the (say positive) shift in the elliptic function in 1 (resp. 2) is symmetric to the one in 3 (resp. 4). This is because the shift by every application of the map is constant and only depends on the radius of the circle in the pencil.

Also shown is the polar image (light blue)of the N=4 family with respect to limiting point l1. This family has a dynamically null signed perimeter (refracted rays count negative) though its area is variable.