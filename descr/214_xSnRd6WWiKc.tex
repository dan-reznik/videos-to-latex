In both Top and Bottom animations an a/b=1.5 elliptic billiard is shown with a P1,P2,P3 orbit as well as its family of N=3 orbits.

Top: (i) the orbit's Feuerbach point X(11) (where incircle and 9-point circle meet) sweeps the caustic (proven). (ii) the three perpendicular dropped from the excircles onto the sides also sweep the caustic (Chasles' theorem, explained by you a few days ago). (iii) the anticomplement X(100) of the Feuerbach point (shown as an F with a bar on top) sweeps the billiard (proven). Fbar is simply a reflection of F about the barycenter (B), at twice the distance |FB|.

Bottom: the anticomplement of F is also the Feuerbach point of the orbit's anticomplementary triangle T', shown in blue for the same orbit position above. T' has sides parallel to the orbit, and passes thru the orbit's vertices (it's a similar triangle) at each of its sides midpoints (each of its sides is twice each of the orbit's).

To get its Feuerbach point, I compute the point of tangency between its incircle and its 9-point-circle (the latter is elegantly congruent w the orbit's circumcircle).

We noticed 3 amazing properties regarding the above picture:

a) non-ellipticity: the locus of the vertices of T' is not elliptic (drawn dashed blue, with humps on its north and bottom areas).
b) sweeps the billiard: the contact points of its incircle are magically where the sides of T' cross the billiard. Therefore, like Fbar, they will also sweep the billiard.
c) retrogade motion: all motion is driven by a CCW motion of P1 on the billiard. We notice Fbar moves along the billiard monotonically in the *opposite* direction. However, the three intouch points move mostly CCW but with retrograde phases, though they never leave the billiard. This is mesmerizing.

We think property (b) is dual to the fact that on the first picture, if you "drop perpendiculars" from the excircles onto the sides they are where the sides touch the caustic. Likewise, if we compute the feet from the incircle of T', we get the billiard (seems like a dual of Chasles' thm).

Schematically, with ":=:" representing some (projective?) duality or homeomorphic "action":

Orbit :=: Anticomplementary triangle
Caustic :=: Billiard  
Feuerbach of orbit computes caustic :=: Feuerbach of anticomplementary computes billiard
Extouchpoints of orbit compute caustic :=: Intouchpoints of anticomplementary compute the billiard

More info: https://dan-reznik.github.io/Elliptical-Billiards-Triangular-Orbits/