The "circumcevian" triangle A'B'C' of a reference ABC wrt point P has vertices at the intersections of cevians thru P with the circumcircle [1].

Here we define the "circumellipscevian" triangle T' of a reference T wrt to a point P and a second point Q. The vertices of T' are at the intersections of cevians thru P with the circumellipse of T centered on Q.

Let A (resp. A') denote the areas of T (resp. T').

Consider a pair of concentric ellipses (axis aligned or not) which admits a family of Poncelet 3-periodics, let O denote its center.

The video shows that over said family, A/A' is invariant if Q is chosen to be the anticomplement of O, i.e., the double-length reflection of O wrt to X2.

Note: the anticomplement of Xk, k=1,3,4,9 is Xj, j=8,4,20,7 [2]

[1] E. Weisstein, "Circumcevian Triangle", Mathworld, 2021.
[2] C. Kimberling, "Encycl. of Triangle Centers", 2021.