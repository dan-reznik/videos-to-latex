Let E be an ellipse w semi-axes a,b, and M a fixed point. The negative pedal curve of E with respect to M is the envelope of lines passing through a point P(t)=[a cos(t), b sin(t)] on the boundary of E, and perpendicular to P(t)-M, for t in [0,2pi). 

Turns out when M is also on the boundary of E, this curve is a 3-cusp, asymetric deltoid Δ affinely related to the Steiner Hypocycloid (aka Steiner Curve or Deltoid) [2]. A really cool observation is that the area of Δ is invariant over all M [3].

Let the cusps of Δ be labeled P_i', and their preimages (on E) P_i. The preimage is the t_i such that the line throuhg P(t_i) perpendicular to P(t_i) passes is tangent to the deltoid at one of its cusps.

Let C_2 denote the center of area of Δ. The video shows the animation of Δ as M slides across the boundary of E. Namely, we draw the locus of the P_i' (a two-lobed curve) for 3 aspect ratios a/b of the ellipse E. This instersects the ellipse at the points Z_j, j=1,2,3,4 shown. Also shown is the (elliptic) locus of C_2, intersecting the ellipse at the W_j.

Note: that at a/b=Sqrt(2) [center animatoin], the cusps pass through the center of E, and the two lobes of the locus of the P_i' touch.

Not shown: C_2, M, and the P_i are concyclic (5-point circle) [see companion video]

Notice whenever the cusps enter (or leave) the ellipse, they do so with the corresponding pre-image P_i on the same spot. A similar phenomenon occurs when C_2 enters (or leaves) E.

A full 3 rotations of M about the ellipse are necessary for the process to complete a full cycle.

Soundtrack: Saint-Preux, "Concerto pour une seule voix"

References:

[1] MathWorld, "Negative Pedal Curve", https://mathworld.wolfram.com/NegativePedalCurve.html
[2] MathWorld, "Steiner Deltoid",  https://mathworld.wolfram.com/SteinerDeltoid.html
[3] R. Garcia, D. Reznik, H. Stachel, M. Helman, "A Family of Constant-Area Deltoids Associated with the Ellipse", 2020. In progress.