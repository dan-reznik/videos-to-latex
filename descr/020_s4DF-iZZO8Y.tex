A known property is that for any triangle, the pedal triangles of the isodynamic points X15, X16 is an equilateral. Here we study the dynamic geometry of said pedals over a special family of triangles known as the Brocard porism. The main takeaways are that, over said family, (1) the locus of the barycenters are two separate circles, and (2) the ratio of their areas is conserved. Also shown is a property likely true for any triangle: (3) the equilateral barycenters are collinear with the symmedian point X6.

The Brocard porism is a 1d Poncelet 3-periodic family (blue) inscribed in a circle (black) and circumscribed about an ellipse (black) known as the Brocard inellipse. Its two Brocard points Ω1,Ω2 are stationary on the latter's foci and the Brocard angle ω is conserved. The Brocard midpoint X39 is stationary. Also stationary are X3 (by definition) and X6, and the isodynamic points X15 and X16.

Containing Ω1,Ω2,X3,X6 is the (stationary) Brocard circle (green). It is centered on X182. Inscribed in it is the second Brocard Triangle (dashed blue) whose vertices are the intersections of cevians through X6 with the Brocard circle. Strikingly, second Brocards are a new Brocard porism, notice its stationary, smaller, less eccentric Brocard inellipse (gray). Its isodynamic points X15', X16' coincide with those of the original family. Furthermore, its first (resp. second) Brocard points Ω1',Ω2' is concyclic with Ω1, X15, X16 (resp. Ω2, X15, X16). These circles (dashed red) are centered on the Beltrami points P(2) and U(2), each of which is the circumcircle-intervers of Ω1,Ω2, respectively.