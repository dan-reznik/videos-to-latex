A family of harmonic polygons H (magenta) is shown inscribed in a circle centered at O. John Casey (1888) constructs them as the inversive images of regular N-gons with respect to some circle [1]. T. Sharp (1945) constructs them as projections of a regular polygon [2].

Harmonic polygons have a symmedian point K. If N is even these are the intersections of diagonals. If N is odd these are the intersections of lines from the vertices to the opposite points of contact with the ellipse enveloped by the sides (aka the Brocard inellipse).

Define the Brocard circle as having KO as diameter. Let l1 and l2 be the limiting points of the pencil defined by the circumcircle and Brocard circle of a harmonic family. Consider a family H' (also harmonic) whose vertices are the inverses of those of H wrt to a circle centered somewhere.

The video shows that when a harmonic family is rotated about either limiting point l1 or l2, the corresponding limiting point of H' stays put.

[1] John Casey, "A sequel to the first six books of the Elements of Euclid, containing an easy introduction to modern geometry, with numerous examples". Dublin: Hodges, Figgis & co., 1888.
[2] T. Sharp, "Harmonic Polygons", The Mathematical Gazette, Vol. 29, No. 287 (Dec., 1945), pp. 210-213.