Left: the "poristic" family of triangles where both incircle and circumcircle are fixed. The vertices of the excentral triangle (dark green), i.e., the excenters sweep a circle with twice the radius of the circumcircle, a result proved in [1].

Right: the "tricentric" family -- P1P2 and P1P3 are tangent to a circle whose radius (in this case) is less than the poristic one. Notice P2P3 is no longer tangent to the inner circle, indeed its envelope is a third circle (dashed red) in the pencil of the circumcircle and inner circle. As shown in previous videos, both X1 and X40 sweep circles of the same radius, w centers symmetric about X3.

The new result here is the curious loci of the excenters: excenter P1' (opposite to P1) sweeps a circle (dark green) while that the other two (P2' and P3') sweep are non-conics (light green) which intersect on the X1X3 line.  

[1] B. Odehnal, "Poristic Loci of Triangle Centers", J. Geom. Graphics 15/1 (2011), 45-67.