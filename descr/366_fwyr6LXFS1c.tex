Let E be an ellipse w semi-axes a,b, and M a fixed point. The negative pedal curve of E with respect to M is the envelope of lines passing through a point P(t)=[a cos(t), b sin(t)] on the boundary of E, and perpendicular to P(t)-M, for t in [0,2pi). 

Turns out when M is also on the boundary of E, this curve is a 3-cusp, asymetric deltoid Δ affinely related to the Steiner Hypocycloid (aka Steiner Curve or Deltoid) [2]. A really cool observation is that the area of Δ is invariant over all M [3].

The video shows the evolution of Δ (and derived objects) as M slides across the boundary of E.  A full 3 rotations of M about the ellipse are necessary for the process to complete a full cycle.

The preimage is the t_i such that the line throuhg P(t_i) perpendicular to P(t_i) passes is tangent to the deltoid at one of its cusps. Let C_2 denote the center of area of Δ. 

A few phenomena shown include (see [3] for proofs):

Left: (i) the triangle T' defined by the 3 cusps P_i' of Δ has invariant area; (ii) its barycenter X2 coincides with the center of area C_2 of Δ; (iii); the cusp pre-images P_i (the line through P_i and perp to P_i-M passes through cusp P_i') are concyclic with M and C_2 on a 5-point circle K (orange); (iv) M and C2 are antipodes with respect to the center X3 of K; (v) the triangle T (orange) defined by the three pre-images has invariant area; (vi) its barycenter X_2 is stationary at the center of E which must be its Steiner Ellipse; (vii) the T have an elliptic caustic E' (dashed orange), whose axes are half those of E (the Steiner Inellipse); so T is an N=3 constant-area poncelet family. (viii) Since M is the intersection of E and K, M is the Steiner Point X99 of T, and C2 is X98, the reflection of X99 on X3. (ix) not shown: the lines P_iP_i' concur at C2.

Right: (i) define three circles K_i as passing through a cusp P_i', its pre-image P_i and M. These osculate E at the P_i. (ii) the centers P_i'' of the K_i define a triangle T'' whose area is invariant over M; (iii) the barycenter of X2'' of T'' coincides with the center X3 of K; (iv) T'' is a 1/4 size homothety of T', and M is there perspector (the lines P_i'P_i'' all concur at M).

So all of Δ, T', T, and T'' have invariant areas over all M. T'/T''=4. 

Soundtrack: Joe Dassin, "L'été Indien"

References:

[1] MathWorld, "Negative Pedal Curve", https://mathworld.wolfram.com/NegativePedalCurve.html
[2] MathWorld, "Steiner Deltoid",  https://mathworld.wolfram.com/SteinerDeltoid.html
[3] R. Garcia, D. Reznik, H. Stachel, M. Helman, "A Family of Constant-Area Deltoids Associated with the Ellipse", 2020. In progress.