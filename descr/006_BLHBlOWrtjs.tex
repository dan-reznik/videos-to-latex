This is a continuation of our previous video on the anticevian polygon. We demonstrate that the anticevian is obtained as the composition of projectivities. We also demonstrate what happens to this iteration when the starting point W is not chosen at the correct location of a first vertex Q1 of the anticevian. After N iterations, we get a non-closing path. However, if N additional iterations are executed, we are taken back to W. We show the correct anticevian is a degenerate case where the two wraparound paths are identical.

The viewer is encouraged to interact with this experiment by going to [2].

[1] D. Reznik, "The Anticevian Polygon I: Basic Phenomena", YouTube, 2021. https://youtu.be/FKvDfamTy-Y
[2] D. Reznik, Anticevian App, Wolfram Cloud, 2021. http://bit.ly/3oKUVf5