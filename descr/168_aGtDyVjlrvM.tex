A reuleaux triangle is known for having constant width [1]. It is a closed shape comprising 3 circular arcs. In this video we explore:

1) The negative pedal (antipedal) curve of its boundary with respect to a point M on the boundary. Here we notice that it is a 5-cusp curve with many interesting properties. Proofs are welcome!

2) Billiard trajectories in its interior, showing some beautiful regimes, see [2,3,4] for a proper theoretical basis.

[1] https://mathworld.wolfram.com/ReuleauxTriangle.html
[2] Daniel Bezdek and Karoly Bezdek, "Shortest billiard trajectories", Oct, 2018, https://arxiv.org/pdf/1110.4324.pdf
[3] D. Genin and S. Tabachnikov, "On configuration spaces of plane polygons, sub-Riemannian geometry and periodic orbits of outer billiards", J. of Modern Dynamics 1(2), 2006, https://arxiv.org/abs/math/0604388
[4] E. Gutkin, "Billiard Caustics, floating in neutral equilibrium, and the isoperimetric inequality", 2012, http://bcc.impan.pl/12Ergodic/uploads/gutkin.pdf