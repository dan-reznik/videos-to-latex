An elliptic billiard (black) is shown as well as the family of 3-periodic orbits (blue), let P1,P2,P3 denote its vertices.

Consider two parabolas confocal, coaxial parabolas Π1, Π2, (red, green), with focus on P1 and axis along P2P3 . Further, Π1 (resp. Π2) contain P2 (resp. P3). Label the parabola vertices V and V'. An amazing invariant is that over the 3-periodic family, the intervertex distance |V-V'| remains invariant!

Since the parabolas share their focal axes, V,P1,V' are collinear.

Recall a parabola is defined by 4 constraints. For Π1 these are: (i,ii) focus is on P1=(x1,y1). (iii) contains P2, (iv) focal axis parallel to P2P3. Similarly for Π2.

Claim 1: Π1,Π2 are tangent to E at P3,P3, respectively. To see this note P1P2P3 is a billiard 3-periodic, i.e., each vertex is bisected by the ellipse normal. For Π1, one can regard P1P2 as a ray emanating from the focus which will be reflected parallel to its focal axis. Since the latter was made parallel to P2P3, such a ray will be reflected along P2P3, therefore, the normal to Π1 at P2 coincides with ellipse's.

Claim 2: d=|V-V'| is invariant (value is shown at the top of the video, for a=1.5, b=1 of the elliptic billiard). It can be shown d=L/2, where L is the 3-periodic invariant perimeter!

Also shown is the area between the two parabolas, not constant.

Notice the two parabolas become identical (up to 90 degree rotation) when P1 is at a vertex of the elliptic billiard.