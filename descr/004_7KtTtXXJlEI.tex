Left: family of Poncelet N-periodics (blue) in the elliptic billiard (black). N=5 case shown without loss of generality. Also shown is the outer polygon (solid green) whose sides are tangent to the billiard at the N-periodic vertices. The Poncelet grid implies this family will be inscribed in an ellipse (dashed green), i.e., the outer polygons are also a Poncelet family. While N-periodics conserve the sum of cosines, the outer family conserves the product of its cosines.

Right: the affine image of the confocal family which sends the billiard ellipse to a circle. A curious "role reversal" takes place here: (i) the circle-inscribed affine image of billiard N-periodics (blue) now conserves the product of cosines, and its value is equal to that conserved by the outer polygons in the pre-image. (ii) The affine image of the outer family (solid green) contains a fixed incircle. It too conserves the sum of its cosines, and its value is the same as the sum of cosines conserved by the original affine image.

Note that due to the affine transform there are no homothetic polygons.

Experiments also show that the (confocal, affine outer) pair not only conserves the same sum of cosines, but sweep the same curve in 5-dimensional cosine space. Likewise for the (confocal outer, confocal affine) -- not only these conserve the same product of cosines but sweep the same curve in 5-dim cosine space.