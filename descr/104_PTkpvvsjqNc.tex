Two elliptic billiards are shown (top: a/b=1.5, bottom: a/b=2) as well as their family of N=3 (triangular) orbits. Also shown are E1 and E2: the incenter (green) and barycenter (light brown) circumellipses, respectively. These meet at the three orbit vertices as well as on a 4th point interior to the billiard (red dot). The locus of this intersection is shown as a red curve for both cases. We make the following observations:

1) The axes of E1 remain parallel to the billiard's
2) The ratio of the axes of E1 is constant and equal to:

(-a^2+2*b^2+2*delta)/((2*a^2-1+2*delta)*a^4)

where delta = sqrt(a^4-a^2+1)

It turns out the line X(75) to X(77) pass through the point of intersection.

3) the axes of E2 are in general not parallel to the billiard's
4) the ratio of axes of E2 is: length1 = c0 + c1 length2, with c0 and c1 remaining constant for the whole family.
5) the locus of the intersection of E1 and E2 is interior to the billiard

More Info:  https://dan-reznik.github.io/Elliptical-Billiards-Triangular-Orbits/