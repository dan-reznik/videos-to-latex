An a/b=1.5 elliptic billiard is shown as well as the family of N=3 (triangular) orbits.

Also shown are the Steiner Circumellipse, centered on the barycenter X2 and intersecting the billiard at X190.

Also shown is the Steiner Inellipse, tangent to the billiard at its medians.

More Info:  https://dan-reznik.github.io/Elliptical-Billiards-Triangular-Orbits/

For the circumellipse, the relationship between the two axes (L1 and L2) is a fixed linear function:

L1 = c0 + c1 L2

For the Inellipse (with axes lengths L1' and L2') the relationship is L1' = c0/2 + c1 L2', where c0 and c1 refer to the same quantities used for the Circumellipse.

Notice that unlike the X(1)-centered circumellipse, whose axes always remain parallel to the billiard, both Steiner conics have axes which in general are not aligned with the billiard.