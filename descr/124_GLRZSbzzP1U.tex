Consider a reference triangle T (blue) rigidly rotating about some point P in the plane (interior or exterior to T).

Let C be a fixed inversion circle (dashed green) centered at O anywhere.

Now consider a triangle T' (red) whose vertices are inversions of T vertices wrt C. Let X6' denote its symmedian point.

The video shows the case where P=X6, i.e., T rotates about its own symmedian point. In this case, X6' is *suprisingly* stationary. Also shown are the loci of the vertices of T (resp. T'), which are the light blue (resp. orange) circles.

Oher observations: for any pivot P and/or inversion circle C:

a) O,X6,X6' are collinear
b) the locus of X6' is a conic (it degenerates o a point only when P=X16).

Finally, the same properties above apply to X'(k), k=3,15,16,61,62. The loci of X'15,X'16 in particular are always circles. The loci of said X'(k) collapse to a point when P=X(k).