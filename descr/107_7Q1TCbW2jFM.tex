An a/b=1.5 elliptic billiard is shown as well as its family of N=3 (triangular) orbits. Also drawn (green) are the orbits' excentral triangles [1], and the latter's Jerabek Hyperbola [1], a circumhyperbola which passes through the excentral's vertices, its circumcenter, orthocenter, and symmedian point. In terms of the orbit triangle, these correspond, respectively, to: the excenters, bevan point X(40), incenter X(1), and mittenpunk X(9). The *center* of the Jerabek Hyperbola is X(125), which is also X(100) of its orthic (since the excentral is acute) [3], and following a remark by Prof Igor Minevich, Department of Mathematics, Rose-Hulman Institute of Technology.

Also shown is the excentral's circumellipse E' to which it is a billiard orbit, i.e., it's circumbilliard, whose center is congruent with its Mittenpunkt X(168) [4], shown in red. Notice the axes of E' are not parallel to those of the billiard.

[1] http://mathworld.wolfram.com/ExcentralTriangle.html
[2] http://mathworld.wolfram.com/JerabekHyperbola.html
[3] https://faculty.evansville.edu/ck6/encyclopedia/ETC.html
[4] https://dan-reznik.github.io/Elliptical-Billiards-Triangular-Orbits/

More Info: https://dan-reznik.github.io/Elliptical-Billiards-Triangular-Orbits/