\documentclass[12pt]{amsart}
\input{setup}

\title{Videos of Experiments (2019--2022)}

\author{Dan S. Reznik}
\author{Ronaldo A. Garcia}

\begin{document}

\maketitle

\tableofcontents
\section{Affine Images (6)}

\subsection{Affine images of billiard N-periodics IIa: N=5 quartet \& trio (part 02)}
\label{vid:buhDTsjpvRQ}
\noindent N=5, 2m25s (4/2021). 
\begin{center}\includegraphics[width=.5\textwidth]{pics/buhDTsjpvRQ.jpg} \\ 
\href{https://youtu.be/buhDTsjpvRQ}{\url{youtu.be/buhDTsjpvRQ}}\end{center}
% Left: A family of N-periodics (blue) in the elliptic billiard, interscribed in a pair of confocal ellipses (black and brown) is shown on the left which classically conserve perimeter and Joachimsthal's constant [1]. Without loss of generality, N=5, and the aspect ratio of the billiard is 1.75. This family also conserves the sum of internal angle cosines; furthermore, the family of "outer" polygon (green), with sides are tangent to the outer ellipse at the N-periodic vertices) conserves the product of its internal angle cosines [2,3,4]. In fact, if N is odd, the ratio of areas of the outer polygon to that of N-periodics is also conserved [5]. By virtue of the Poncelet grid [6], the outer family is also Ponceletian, as it is inscribed in an ellipse (dashed green). 

Middle: a Poncelet family of 5-gons (blue) with a fixed incircle which is the affine image of the confocal family (left) such that the caustic is sent to a circle. This family also conserves the sum of its cosines [3]. Surprisingly it is equal to that of its confocal pre-image (left). Notice its outer polygon (green) does not conserve its product of cosines (grayed out).

Right: a Poncelet family of 5-gons (blue) which is the affine image of the confocal family (left) such that the ellipse (dashed green) to which the outer polygons are inscribed is sent to a circle (dashed green). Like its confocal pre-image, this family of outer polygons also conserves the product of its cosines [3]. Surprisingly it is equal to that of its confocal pre-image (left). Notice the N-periodics here (blue) do not conserve their sum of cosines (grayed out).

Experimentally, we have also noticed that the confocal+incircle families sweep the same curve in N-dimensional "cosine space", as do the outer+circumcircle families. Proof pending!

[1] S. Tabachnikov, "Geometry and Billiards", Student Mathematical Library, vol 30, American Mathematical Society, 2005. http://www.personal.psu.edu/sot2/book...​
[2] D. Reznik, R. Garcia, and J. Koiller, "Can the Elliptic Billiard still surprise us?", Math Intelligencer, 42, 2020. http://rdcu.be/b2cg1​
[3] A. Akopyan, R. Schwartz, and S., "Billiards in Ellipses Revisited", Eur. J. Math, 2020. 
[4] M. Bialy and S. Tabachnikov, "Dan Reznik's Identities and More",
Eur. J. Math., 2020.
[5] A.C. Chavez-Caliz, "More About Areas and Centers of Poncelet Polygons" , Arnold Math J., 2020.
[6] R. Schwartz,  "The Poncelet grid", Advances in Geometry, 7:2, 2007.
[7] M. Levi and  S. Tabachnikov, "The Poncelet Grid and Billiards in Ellipses",Am. Math. Monthly,  114:10, 2007.
\subsection{Affine images of billiard N-periodics Ib: N=3 trio (part 02)}
\label{vid:aWs29dqY34M}
\noindent N=3, 2m25s (4/2021). 
\begin{center}\includegraphics[width=.5\textwidth]{pics/aWs29dqY34M.jpg} \\ 
\href{https://youtu.be/aWs29dqY34M}{\url{youtu.be/aWs29dqY34M}}\end{center}
% Left: A family of 3-periodics (blue) in the elliptic billiard, interscribed in a pair of confocal ellipses (black and brown) is shown on the left which classically conserve perimeter and Joachimsthal's constant [1].  This family also conserves the sum of internal angle cosines; furthermore, the family of excentral polygon (green) conserves the product of its internal angle cosines [2,3,4]. The ratio of areas of the outer polygon to that of 3-periodics is also conserved [5]. The locus of the excenters is an ellipse (dashed green). 

Middle: the affine image of the former which sends the confocal caustic to a circle. This family also conserves the sum of its cosines [3]. Surprisingly it is equal to that of its confocal pre-image (left). Notice the image of the original excentrals (green) does not conserve its product of cosines (grayed out). Note these are not the excentrals of the current family.

Right: the affine image of the confocal 3-periodics (left) which sends the locus of the excenters to a circle (dashed green). Like its confocal pre-image, this also conserves the product of its cosines [3]. Surprisingly it is equal to that of its confocal pre-image (left). These triangles are *not* the excentrals of the current family (blue) which do not conserve their sum of cosines (grayed out).

Experimentally, we have also noticed that the confocal+incircle families sweep the same curve in 3-dimensional "cosine space", as do the outer+circumcircle families, see [8].

[1] S. Tabachnikov, "Geometry and Billiards", Student Mathematical Library, vol 30, American Mathematical Society, 2005. http://www.personal.psu.edu/sot2/book...​
[2] D. Reznik, R. Garcia, and J. Koiller, "Can the Elliptic Billiard still surprise us?", Math Intelligencer, 42, 2020. http://rdcu.be/b2cg1​
[3] A. Akopyan, R. Schwartz, and S., "Billiards in Ellipses Revisited", Eur. J. Math, 2020. 
[4] M. Bialy and S. Tabachnikov, "Dan Reznik's Identities and More",
Eur. J. Math., 2020.
[5] A.C. Chavez-Caliz, "More About Areas and Centers of Poncelet Polygons" , Arnold Math J., 2020.
[6] R. Schwartz,  "The Poncelet grid", Advances in Geometry, 7:2, 2007.
[7] M. Levi and  S. Tabachnikov, "The Poncelet Grid and Billiards in Ellipses",Am. Math. Monthly,  114:10, 2007.
[8] D. Jaud, D. Reznik, and R. Garcia, "Poncelet Plectra: Harmonious Properties of Cosine Space", arXiv:2104.13174 , 2020.

---

soundtrack: K. MacLeod, "Lachaim", licensed under a Creative Commons Attribution 4.0 license. https://creativecommons.org/licenses/by/4.0/

Source: http://incompetech.com/music/royalty-free/index.html?isrc=USUAN1100412

Artist: http://incompetech.com/
\subsection{Affine images of billiard N-periodics Ia: N=3 quartet}
\label{vid:EJzqsELkPN4}
\noindent N=3, 2m25s (4/2021). 
\begin{center}\includegraphics[width=.5\textwidth]{pics/EJzqsELkPN4.jpg} \\ 
\href{https://youtu.be/EJzqsELkPN4}{\url{youtu.be/EJzqsELkPN4}}\end{center}
% From Left to Right:

First: A family of Poncelet 3-periodics (blue) in the elliptic billiard, i.e., interscribed in a pair of confocal ellipses (black and brown). Classical conservations include perimeter and Joachimsthal's constant [1].  This family also conserves the sum of internal angle cosines. Also shown are the excentral polygons (solid green); these conserve the product of internal angle cosines [2,3,4]. The ratio of areas of the excentral to that of 3-periodics is also conserved [5]. The locus of the excenters is an ellipse (dashed green). 

Second: the affine image of confocal family which sends the caustic to a circle. This family also conserves the sum of its cosines [3]. Surprisingly it is equal to that of its confocal pre-image (left). Notice the image of the original excentrals (green) does not conserve its product of cosines (grayed out). Note these are not the excentrals of the current family.

Third: the affine image of the confocal family which sends the outer ellipse to a circle (black). Like confocal excentrals, this also conserves the product of cosines [3]. Surprisingly it is equal to that of the excentrals. Also shown is the image of the confocal excentrals under the same affine transformation. This family has an incircle and like the previous one conserves its sum of cosines. Surprisingly, it is equal to the sum conserved by the original confocal family (first) and the one with incircle (second).

Fourth: the affine image of the confocal 3-periodics (first) which sends the locus of the excenters to a circle (dashed green). Like its confocal pre-image, this also conserves the product of its cosines [3]. Surprisingly it is equal to that of its confocal pre-image (first) and that of the family with incircle (third). These triangles are *not* the excentrals of the current family (blue) which do not conserve their sum of cosines (grayed out).

Experimentally, we have also noticed that the confocal+incircle families sweep the same curve in 3-dimensional "cosine space", as do the outer+circumcircle families, see [8].

[1] S. Tabachnikov, "Geometry and Billiards", Student Mathematical Library, vol 30, American Mathematical Society, 2005. http://www.personal.psu.edu/sot2/book...​
[2] D. Reznik, R. Garcia, and J. Koiller, "Can the Elliptic Billiard still surprise us?", Math Intelligencer, 42, 2020. http://rdcu.be/b2cg1​
[3] A. Akopyan, R. Schwartz, and S., "Billiards in Ellipses Revisited", Eur. J. Math, 2020. 
[4] M. Bialy and S. Tabachnikov, "Dan Reznik's Identities and More",
Eur. J. Math., 2020.
[5] A.C. Chavez-Caliz, "More About Areas and Centers of Poncelet Polygons" , Arnold Math J., 2020.
[6] R. Schwartz,  "The Poncelet grid", Advances in Geometry, 7:2, 2007.
[7] M. Levi and  S. Tabachnikov, "The Poncelet Grid and Billiards in Ellipses",Am. Math. Monthly,  114:10, 2007.
[8] D. Jaud, D. Reznik, and R. Garcia, "Poncelet Plectra: Harmonious Properties of Cosine Space", arXiv:2104.13174 , 2020.
\subsection{Affine images of billiard N-periodics IIa: N=5 quartet \& trio (part 01)}
\label{vid:VQ4fB_s33HE}
\noindent N=5, 2m25s (4/2021). 
\begin{center}\includegraphics[width=.5\textwidth]{pics/VQ4fB_s33HE.jpg} \\ 
\href{https://youtu.be/VQ4fB_s33HE}{\url{youtu.be/VQ4fB\_s33HE}}\end{center}
% From Left to Right we see four animations (a Poncelet Quartet):

First: A family of Poncelet N-periodics (blue) in the elliptic billiard, i.e., interscribed in a pair of confocal ellipses (black and brown). We chose N=5 without loss of generality. Classical conservations include perimeter and Joachimsthal's constant [1]. This family also conserves the sum of internal angle cosines. Also shown are its ``outer´´ polygons (solid green) whose sides are tangent to the billiard at the N-periodic vertices; i.e., they run along each vertex's external bisector. The outer family conserves the product of internal angle cosines [2,3,4]. The ratio of areas of the outer polygon to that of 3-periodics is also conserved [5]. The locus of their vertices is an ellipse (dashed green), thanks to the Poncelet grid [6,7]. 

Second: the affine image of the billiard family which sends the caustic to a circle. This family also conserves the sum of its cosines [3], and it is equal to that of its confocal pre-image (left). Notice the image of the original outer polygons (dashed green) does not conserve its product of cosines (grayed out). Unlike the original outers, these do not have sides parallel to the external bisectors of corresponding N-periodic vertices. In N=3 parlance this would akin to saying "they are not excentral triangles".

Third: the affine image of the confocal family which sends the outer ellipse to a circle (black). Like confocal excentrals, this new family also conserves the product of cosines [3], and its value is equal to that of the confocal outer family. Also shown is the image of the confocal excentrals under the same affine transformation. This family circumscribes what is now a circle, i.e., it has a fixed incircle; like the previous one it conserves the sum of cosines, and surprisingly, it is equal to the sum conserved by the original confocal family (first) and the one with incircle (second), though none of these polygons is homothetic to one another nor do they have equal angle vectors (though we believe each sweeps the same curve in 5d angle or cosine space, see [8]).

Fourth: the affine image of billiard N-periodics (first) which sends the locus of outer vertices to a circle (dashed green). Like its confocal pre-image, this also conserves the product of its cosines [3], and it is equal to that of its confocal pre-image (first) and that of the family with incircle (third). 

[1] S. Tabachnikov, "Geometry and Billiards", Student Mathematical Library, vol 30, American Mathematical Society, 2005. http://www.personal.psu.edu/sot2/book...​
[2] D. Reznik, R. Garcia, and J. Koiller, "Can the Elliptic Billiard still surprise us?", Math Intelligencer, 42, 2020. http://rdcu.be/b2cg1​
[3] A. Akopyan, R. Schwartz, and S., "Billiards in Ellipses Revisited", Eur. J. Math, 2020. 
[4] M. Bialy and S. Tabachnikov, "Dan Reznik's Identities and More",
Eur. J. Math., 2020.
[5] A.C. Chavez-Caliz, "More About Areas and Centers of Poncelet Polygons" , Arnold Math J., 2020.
[6] R. Schwartz,  "The Poncelet grid", Advances in Geometry, 7:2, 2007.
[7] M. Levi and  S. Tabachnikov, "The Poncelet Grid and Billiards in Ellipses",Am. Math. Monthly,  114:10, 2007.
[8] D. Jaud, D. Reznik, and R. Garcia, "Poncelet Plectra: Harmonious Properties of Cosine Space", arXiv:2104.13174 , 2020.
\subsection{Affine images of billiard N-periodics Ib: N=3 trio (part 01)}
\label{vid:HjBZdrR3Azs}
\noindent N=3, 2m25s (4/2021). 
\begin{center}\includegraphics[width=.5\textwidth]{pics/HjBZdrR3Azs.jpg} \\ 
\href{https://youtu.be/HjBZdrR3Azs}{\url{youtu.be/HjBZdrR3Azs}}\end{center}
% Left: A family of elliptic billiard 3-periodics (blue), i.e., , interscribed in a confocal pair of  ellipses (black and brown). This classically conserves perimeter and Joachimsthal's constant [1].  It turns out it also conserves the sum of internal angle cosines; furthermore, the Ponceletian family of its excentral polygons (green, inscribed in the dashed-green ellipse) conserves the product of its internal angle cosines [2,3,4]. The ratio of areas of the outer polygon to that of 3-periodics is also conserved [5]. Curiously, the excentrals also conserve the ratio of squared sidelengths by the product of sidelengths.

Middle: the affine image of billiard 3-periodics which sends the confocal caustic to a circle. This family also conserves the circumradius (not shown) and therefore the sum of its cosines [3]. Surprisingly, the latter is equal to that of its confocal pre-image (left). No conservations are known for its outer polygon (dashed green). Note these are affine images of confocal  excentrals, but are *not* excentrals of the current family.

Right: affine image of the confocal 3-periodics which sends the elliptic billiard to a circle (black). This also conserves the product of its cosines [3]. Surprisingly it is equal to the value conserved by the confocal excentrals (solid green, left). The caustic to its outer family (solid green) is a cirlce, i.e., this conserves the sum of cosines. Surprisingly, it is the same quantity conserved by the original billiard family.

[1] S. Tabachnikov, "Geometry and Billiards", Student Mathematical Library, vol 30, American Mathematical Society, 2005. http://www.personal.psu.edu/sot2/book...​
[2] D. Reznik, R. Garcia, and J. Koiller, "Can the Elliptic Billiard still surprise us?", Math Intelligencer, 42, 2020. http://rdcu.be/b2cg1​​
[3] A. Akopyan, R. Schwartz, and S., "Billiards in Ellipses Revisited", Eur. J. Math, 2020. 
[4] M. Bialy and S. Tabachnikov, "Dan Reznik's Identities and More",
Eur. J. Math., 2020.
[5] A.C. Chavez-Caliz, "More About Areas and Centers of Poncelet Polygons" , Arnold Math J., 2020.
[6] R. Schwartz,  "The Poncelet grid", Advances in Geometry, 7:2, 2007.
[7] M. Levi and  S. Tabachnikov, "The Poncelet Grid and Billiards in Ellipses",Am. Math. Monthly,  114:10, 2007.
\subsection{Affine images of billiard N-periodics IIb: N=5 role reversal}
\label{vid:7KtTtXXJlEI}
\noindent N=5, 2m25s (4/2021). 
\begin{center}\includegraphics[width=.5\textwidth]{pics/7KtTtXXJlEI.jpg} \\ 
\href{https://youtu.be/7KtTtXXJlEI}{\url{youtu.be/7KtTtXXJlEI}}\end{center}
% \input{descr/006_7KtTtXXJlEI}

\section{Anticevian (2)}

\subsection{The Anticevian Polygon I: Basic Phenomena}
\label{vid:FKvDfamTy-Y}
\noindent N=5, 13m18s (1/2021). 
\begin{center}\includegraphics[width=.5\textwidth]{pics/FKvDfamTy-Y.jpg} \\ 
\href{https://youtu.be/FKvDfamTy-Y}{\url{youtu.be/FKvDfamTy-Y}}\end{center}
% Recall the anticevian triangle T' of a reference triangle T wrt point X is such that T is the cevian of T' wrt X [1]. 

Here we extend this notion to N-gons, where N is odd.

First define the X-cevian polygon of an N-gon P as having vertices at the intersections of lines Pi X with sides of P opposite to Pi. For example, P3P4 is opposite to P1, P4P5 to P2, etc.

Now define P', the X-anticevian of P: it is such such that P is the X-cevian of P'. 

Unlike the cevian calculation, which is direct and local (it only requires a vertex and the opposing side), obtaining the vertices of the anticevian requires global information. A. Akopyan has suggested a very clever algorithm based on the composition of projectivities we shall explain elsewhere.

The video shows  the dynamic geometry of both the cevian and anticevian as a point X is dragged around in the plane of a regular pentagon.

When bounded (see below), the anticevian can be convex, concave, self-intersection.

The video also shows that if X is on a certain web of hyperbola-like curves (10 branches total), one or more vertices of the anticevian go to infinity. This web divides the plane into "cells". If X remains within one such cell, the vertices of the anticevian remain bounded, and its area function is continuous.

Note: in the video I say every such branch goes thru a side midpoint, of course this is wrong. In fact, every *pair* of branches has one branch which goes thru sides' midpoint.

The viewer is also invited to interact with a (slow) wolfram app we published on the web [2].

[1] E. Weisstein, Anticevian Triangle, Mathworld, 2021. https://mathworld.wolfram.com/AnticevianTriangle.html
[2] D. Reznik, Anticevian App, Wolfram Cloud, 2021. http://bit.ly/3oKUVf5
\subsection{The Anticevian Polygon II: Cyclical Projectivities}
\label{vid:BLHBlOWrtjs}
\noindent N=5, 9m18s (1/2021). 
\begin{center}\includegraphics[width=.5\textwidth]{pics/BLHBlOWrtjs.jpg} \\ 
\href{https://youtu.be/BLHBlOWrtjs}{\url{youtu.be/BLHBlOWrtjs}}\end{center}
% \input{descr/008_BLHBlOWrtjs}

\section{Area Invariants (3)}

\subsection{Amazing Ellipse Pedal and Contrapedal Curves: area invariance for all pedal points on a circle}
\label{vid:UUnvj7VIYso}
\noindent N=n/a, 7m20s (6/2020). 
\begin{center}\includegraphics[width=.5\textwidth]{pics/UUnvj7VIYso.jpg} \\ 
\href{https://youtu.be/UUnvj7VIYso}{\url{youtu.be/UUnvj7VIYso}}\end{center}
% Let C be a curve (in our case an ellipse, but it can be anything) and M a point in the plane. The pedal (resp. contrapedal) of C with respect to M is the locus of the foot of perpendiculars dropped from M onto tangents (resp. normals) to C passing through P on C, for all P on C [1,2].

The video makes an observation that may or may not be known:

Both the pedal and contrapedal curves of an ellipse E have *invariant* signed areas, so long as M rides on a circle concentric with E. It made me jump off my chair! The signed area counts self-intersecting loops of the curve as negative [3]. 

This is probably a known identity: let A, A_p, and A_c denote the area of the ellipse, the pedal curve, and the contra-pedal, the latter two with respect to the same point M. 

A_p = A + A_c

I.e., if the phenomenon of invariance with respect to all M on a concentric circle holds for the pedal its hould also hold for the contra-pedal.

P.S. - during the part I cover the "pedal"  I keep incorrectly calling it an "evolute". The ellipse evolute is a different beast [4], though the contra-pedal curve can also be computed as the pedal to the evolute. Sorry about the confusion.

[1] Mathworld, "Pedal Curve", https://mathworld.wolfram.com/PedalCurve.html
[2] Mathworld, "Antipedal Curve", https://mathworld.wolfram.com/ContrapedalCurve.html
[3] Mathworld, "Polygon Area", https://mathworld.wolfram.com/PolygonArea.html
[4] Mathworld, "Ellipse evolute", https://mathworld.wolfram.com/EllipseEvolute.html
\subsection{Regular Polygons: the Signed Area of the Antipedal Polygon Vanishes along a Circle?}
\label{vid:9PZ6_bHz2UE}
\noindent N=3,4,5,6,7,8, 18m4s (6/2020). 
\begin{center}\includegraphics[width=.5\textwidth]{pics/9PZ6_bHz2UE.jpg} \\ 
\href{https://youtu.be/9PZ6_bHz2UE}{\url{youtu.be/9PZ6\_bHz2UE}}\end{center}
% \input{descr/010_9PZ6_bHz2UE}
\subsection{Steiner's Krümmungs-Schwerpunkt implies Area-Invariant Interpolated Pedal Curve over Circles}
\label{vid:gR8Axe823_M}
\noindent N=n/a, 1m6s (9/2020). 
\begin{center}\includegraphics[width=.5\textwidth]{pics/gR8Axe823_M.jpg} \\ 
\href{https://youtu.be/gR8Axe823_M}{\url{youtu.be/gR8Axe823\_M}}\end{center}
% \input{descr/011_gR8Axe823_M}

\section{Bicentric Family (10)}

\subsection{Non-Concentric Circular Poncelet Pair: Invariant Sum of Japanese Theorem Inradii (A. Akopyan)}
\label{vid:BEvdUUolUXI}
\noindent N=5,6, 1m47s (10/2020). 
\begin{center}\includegraphics[width=.5\textwidth]{pics/BEvdUUolUXI.jpg} \\ 
\href{https://youtu.be/BEvdUUolUXI}{\url{youtu.be/BEvdUUolUXI}}\end{center}
% \input{descr/012_BEvdUUolUXI}
\subsection{Bicentric Family I: Properties of Pedals, Polars, \& Inversions of N-Periodics in Elliptic Billiard}
\label{vid:jhXDKRFLpVk}
\noindent N=3,4,5, PT19M (2/2021). 
\begin{center}\includegraphics[width=.5\textwidth]{pics/jhXDKRFLpVk.jpg} \\ 
\href{https://youtu.be/jhXDKRFLpVk}{\url{youtu.be/jhXDKRFLpVk}}\end{center}
% \input{descr/013_jhXDKRFLpVk}
\subsection{Bicentric Family II: invariant perimeter limiting point pedals}
\label{vid:A7F3szW7rUE}
\noindent N=3,4,5, 13m21s (2/2021). 
\begin{center}\includegraphics[width=.5\textwidth]{pics/A7F3szW7rUE.jpg} \\ 
\href{https://youtu.be/A7F3szW7rUE}{\url{youtu.be/A7F3szW7rUE}}\end{center}
% \input{descr/014_A7F3szW7rUE}
\subsection{Bicentric Family III: Equiperimeter ``Limiting'' Pedal Polygons to the Bicentric Family}
\label{vid:6TmaezNFrOs}
\noindent N=3,4,5, 25m53s (2/2021). 
\begin{center}\includegraphics[width=.5\textwidth]{pics/6TmaezNFrOs.jpg} \\ 
\href{https://youtu.be/6TmaezNFrOs}{\url{youtu.be/6TmaezNFrOs}}\end{center}
% The video narrates several curious properties of the bicentric family [1] and its pedal polygons wrt to its limiting points l1 and l2 [2] i.e., the two points wrt which the inversion of the bicentric circle pair yields a concentric pair (I messed up in the video, sorry). The video shows that

a) the sum of cosines of bicentrics is conserved.
b) limiting pedals are inscribed in separate, touching Pascal Limaçons, one loopless and the other with a loop through l2.
c) the perimeter of both limiting pedals is invariant.
d) the sum of cosines of both limiting pedals is invariant (except for when N=4, where the l1-pedal has variable sum of cosines).
e) in the N=4 case,  the vertices of the l2-pedal are collinear.
f) in the N=3 case the bicentrics are the poristic family [3]. The sum of cosines of bicentrics, l1- and l2-pedals are the *same*. Their Gergonne point X7 is stationary.

For (f) we prepared a set of animations here: dan-reznik.github.io/ellipse-mounted-loci-p5js/?juke=6

Note: over the poristic family, the Gergonne point X7 moves along a circle [3].

[1] E. Weisstein, "Poncelet Porism", MathWorld, 2021. https://mathworld.wolfram.com/PonceletsPorism.html
[2] E. Weisstein, "Limiting Points", MathWorld, 2021. https://mathworld.wolfram.com/LimitingPoint.html
[3] B. Odehnal, "Poristic Loci of Triangle Centers, Journal for Geometry and Graphics 15(1) , 2010.
\subsection{Bicentric Family IV: Constant-Perimeter Limiting Point Pedals in the N=4 Case}
\label{vid:fZe6elRTfeA}
\noindent N=4, 10m45s (3/2021). 
\begin{center}\includegraphics[width=.5\textwidth]{pics/fZe6elRTfeA.jpg} \\ 
\href{https://youtu.be/fZe6elRTfeA}{\url{youtu.be/fZe6elRTfeA}}\end{center}
% The video describes properties of the Poncelet Bicentric family of quadrilaterals interscribed by two circles which can be calculated via Fuss's formula for N=4 [1, Eqn. (39)]. Let l1,l2 denote the two limiting points [2] of the circumcircle-incircle pair of the family. A well know result about bicentric polygons is that their diagonals meet at one of the limiting points.

The video showcases a few curious properties of the bicentrics' pedal polygons with respect to either l1 and l2, to be sure:

a) the l1-pedal has constant perimeter and its vertices sweep a loopless Pascal limaçon.
b) all 4 vertices of the l2-pedal are dynamically collinear (zero area) and sweep a Pascal limaçon whose loop has a node at l2. The perimeter of the l2-pedal is also constant and its sum of cosines is invariant and equal to 4.

[1] E. Weisstein, "Poncelet Porism", MathWorld, 2021. https://mathworld.wolfram.com/PonceletsPorism.html

[2] E. Weisstein, "Limiting Points, MathWorld, 2021. https://mathworld.wolfram.com/LimitingPoint.html
\subsection{Zero-Area N=4 Bicentric Pedals}
\label{vid:hwx1i-W6yLQ}
\noindent N=4, 8m37s (3/2021). 
\begin{center}\includegraphics[width=.5\textwidth]{pics/hwx1i-W6yLQ.jpg} \\ 
\href{https://youtu.be/hwx1i-W6yLQ}{\url{youtu.be/hwx1i-W6yLQ}}\end{center}
% \input{descr/021_hwx1i-W6yLQ}
\subsection{The Jacobi-Poncelet Bicentric family: 3 derived constant perimeter families}
\label{vid:8m21fCz8eX4}
\noindent N=5, 4m20s (3/2021). 
\begin{center}\includegraphics[width=.5\textwidth]{pics/8m21fCz8eX4.jpg} \\ 
\href{https://youtu.be/8m21fCz8eX4}{\url{youtu.be/8m21fCz8eX4}}\end{center}
% \input{descr/020_8m21fCz8eX4}
\subsection{Peripheral triangles: circular locus of incenters and invariant sum of inradii}
\label{vid:HnqqaqDf2mo}
\noindent N=4,5, 2m26s (7/2021). 
\begin{center}\includegraphics[width=.5\textwidth]{pics/HnqqaqDf2mo.jpg} \\ 
\href{https://youtu.be/HnqqaqDf2mo}{\url{youtu.be/HnqqaqDf2mo}}\end{center}
% The video shows an N=4 (left) and N=5 (right) family of polygons (blue) interscribed between two circles C (black) and C' (brown), also known as bicentric polygons. Let Pi=1,...,N denote the vertices of each family. Let Ti = P(i)P(i+1)P(i+2) denote N "peripheral" subtriangles (dashed blue), where indices are computed (mod N). The video shows three properties:

a) the sum of internal angle cosines of a bicentric family is conserved (proved in [1]).
b) the sum of the inradii r(i) of the T(i) is conserved -- unproved for N greater than 4
c) the locus of the incenters of the T(i) is a circle (dashed red).
d) (Sept 1, 2021): the sum of 1/r(i) is also conserved!

[1] P. Roitman, R. Garcia, and D. Reznik, "New Invariants of Poncelet-Jacobi Bicentric Polygons",  Arnold Mathematical Journal, 2021 (to appear).
\subsection{Peripheral triangles: Constant Inradius Sum, Japanese-Style Triangles, Incenter Circular Loci}
\label{vid:TGwlfBUtKrs}
\noindent N=4,5,6,7, 2m26s (7/2021). 
\begin{center}\includegraphics[width=.5\textwidth]{pics/TGwlfBUtKrs.jpg} \\ 
\href{https://youtu.be/TGwlfBUtKrs}{\url{youtu.be/TGwlfBUtKrs}}\end{center}
% This is a result privately communicated to us by A. Akopyan, which we simulate here. A first video of this phemomenon appeared in [1].

The so-called "bicentric Poncelet" family is 1d family of polygons inscribed in an outer circle while simultaneously circumscribing a second inner circle. Recently we proved that  this family conserves the sum of its internal angle cosines [3].

Let the vertices of polygons in the family be labeled P1, P2, ..., PN. We call "japanese-style" triangulation (in reference to the much famed "japanese theorem" [2]) a subdivision in of an N-gon into N-2 triangles obtained as P(1)P(i)P(i+1), where i in [2,N-1] The original japanese theorem accepts any triangulation, but here we will stick to the just described "fan" style triangulation (all triangles share P1).

The video shows two phenomena. Over the bicentric family:

a) the sum of inradii is constant.
b) the locus of sum incenters (all in the N=4,5 cases, and some in the N=6,7 cases) are circles (solid reds). The remaining loci are non-conics (dashed red).

[1] https://youtu.be/BEvdUUolUXI
[2] https://mathworld.wolfram.com/JapaneseTheorem.html
[3] https://arxiv.org/abs/2103.11260 -- to appear, Arnold Math. J. 2021
\subsection{Circular locus of the covertices of an ellipse ``mounted'' to bicentric triangles}
\label{vid:wvNUdJbyHZA}
\noindent N=3, 3m55s (12/2021). 
\begin{center}\includegraphics[width=.5\textwidth]{pics/wvNUdJbyHZA.jpg} \\ 
\href{https://youtu.be/wvNUdJbyHZA}{\url{youtu.be/wvNUdJbyHZA}}\end{center}
% \input{descr/017_wvNUdJbyHZA}

\section{Brocard (12)}

\subsection{Poncelet 3-Periodics of Homothetic Pair: Elliptic Loci of Brocard Pts + Vertices of 1st Brocard Tri}
\label{vid:13i3JGY-fK4}
\noindent N=3, 1m37s (9/2020). 
\begin{center}\includegraphics[width=.5\textwidth]{pics/13i3JGY-fK4.jpg} \\ 
\href{https://youtu.be/13i3JGY-fK4}{\url{youtu.be/13i3JGY-fK4}}\end{center}
% Consider the family of Poncelet 3-periodics (blue) inscribed in an ellipse (a,b) and circumscribed about a concentric, axis-aligned caustic (a/2, b/2). This is known as the "homothetic pair". We have shown elsewhere [1] that this family conserves area, sum of squared sidelengths, and therefore Brocard angle.

The video shows the following additional properties:

a) the loci of the Brocard points Ω1 and Ω2 are two tilted, symmetric ellipses similar to the ones in the pair. These can be interior (a/b below 2.23...) or exterior to the caustic.
b) the locus of the vertices of the First Brocard Triangle (FBT) [2] is a horizontal ellipse, also similar to the ones in the pair. This is always interior to the caustic.
c) the area of the FBT is invariant
d) the ratio of 3-periodic circumradius to that of the FBT is invariant

[1] D. Reznik, "Poncelet 3-Periodics in the Homothetic Pair conserve Brocard angle", YouTube, 2020. https://youtu.be/2fvGd8wioZY
[2] E. Weisstein, "First Brocard Triangle", Mathworld 2020. https://mathworld.wolfram.com/FirstBrocardTriangle.html
\subsection{It takes 2 to tango: Brocard-Poncelet Porism, stationary Brocard Points and invariant Brocard Angle}
\label{vid:JANPPLET0so}
\noindent N=3, 2m26s (9/2020). 
\begin{center}\includegraphics[width=.5\textwidth]{pics/JANPPLET0so.jpg} \\ 
\href{https://youtu.be/JANPPLET0so}{\url{youtu.be/JANPPLET0so}}\end{center}
% Given an ellipse (a,b) (black) two Poncelet porisms can be set up for 2 families of 3-periodics (red and green triangles) of fixed circumcircle (and equal circumradius) such that E is their common caustic and their Brocard points (and their midpoint X39) are common and stationary. Also stationary for each family are X3, X6 and X182 (wrongly labeled X186 on video), the latter the center of the respective Brocard circles (shown dashed red and green).

It does take two to tango!

Acknowledgements: the vertices of isosceles configurations used to compute the porism were derived by Prof Ronaldo Garcia, on Sept 8, 2020.
\subsection{Joined at the hip: Brocard Porism, Steiner Ellipses, and the Homothetic Poncelet Pair}
\label{vid:h3GZz7pcJp0}
\noindent N=3, 4m1s (9/2020). 
\begin{center}\includegraphics[width=.5\textwidth]{pics/h3GZz7pcJp0.jpg} \\ 
\href{https://youtu.be/h3GZz7pcJp0}{\url{youtu.be/h3GZz7pcJp0}}\end{center}
% \input{descr/027_h3GZz7pcJp0}
\subsection{The Poncelet Homothetic Pair contains an Aspect-Ratio Invariant Brocard Inellipse}
\label{vid:DIm2qTxGWXE}
\noindent N=3, 1m37s (9/2020). 
\begin{center}\includegraphics[width=.5\textwidth]{pics/DIm2qTxGWXE.jpg} \\ 
\href{https://youtu.be/DIm2qTxGWXE}{\url{youtu.be/DIm2qTxGWXE}}\end{center}
% \input{descr/032_DIm2qTxGWXE}
\subsection{Brocard Porism: Locus of 1st, 2nd, 5th, and 7th Brocard Triangles' Vertices are Circles}
\label{vid:_bK-BCQv24A}
\noindent N=3, 2m26s (9/2020). 
\begin{center}\includegraphics[width=.5\textwidth]{pics/_bK-BCQv24A.jpg} \\ 
\href{https://youtu.be/_bK-BCQv24A}{\url{youtu.be/\_bK-BCQv24A}}\end{center}
% The Brocard Porism [0] is a 1d family of Poncelet 3-periodics (blue) inscribed in a circle (black) and circumscribed about the Brocard inellipse (also black). Parametrics for their vertices were obtained from [1].

The 1st, 2nd, and 7th (*) Brocard Triangles [2] are know to be inscribed in the Brocard circle (green), and are concyclic with both Brocard points Ω1, Ω2, and X3, and X6. The 5th Brocard is homothetic to the reference, and its circumcenter is X9821 [3]. Therefore the locus of the vertices of the 1st, 2nd, and 7th are the Brocard circle itself, whereas that of the 5th is a circle centered on X9821. 

The loci of 3rd, 4th, and 6th triangles (not shown) are complicated curves.

You can also visualize these phenomena on our interactive app [4] here: https://bit.ly/32GFvQu

Note (*): The 7th Brocard Triangle was invented with Peter Moses in Sept. 2020.

References. 
[0] R. Johnson, "Advanced Euclidean Geometry" (Chapt XVII), Dover, 1960.
[1] R. Garcia, Vertices of Brocard-Poristic Triangles, Private Comm., Sept 2020.
[2] B. Gibert, CTC, https://bernard-gibert.pagesperso-orange.fr/gloss/brocardtriangles.html
[3] C. Kimberling, ETC (Part 5), 2020. https://faculty.evansville.edu/ck6/encyclopedia/ETCPart5.html
[4] I. Darlan and D. Reznik, Loci of Ellipse-Mounted Triangles, 2020, https://dan-reznik.github.io/ellipse-mounted-triangles/
\subsection{Rusian-doll nesting of Brocard porisms: concyclic sequence of Brocard points and the Beltrami points}
\label{vid:Z3YlEbCFbnA}
\noindent N=3, 3m25s (9/2020). 
\begin{center}\includegraphics[width=.5\textwidth]{pics/Z3YlEbCFbnA.jpg} \\ 
\href{https://youtu.be/Z3YlEbCFbnA}{\url{youtu.be/Z3YlEbCFbnA}}\end{center}
% \input{descr/029_Z3YlEbCFbnA}
\subsection{Russian-Doll nesting of Brocard porisms courtesy of the second Brocard triangle}
\label{vid:T7c4CDHIk7s}
\noindent N=3, 3m13s (9/2020). 
\begin{center}\includegraphics[width=.5\textwidth]{pics/T7c4CDHIk7s.jpg} \\ 
\href{https://youtu.be/T7c4CDHIk7s}{\url{youtu.be/T7c4CDHIk7s}}\end{center}
% \input{descr/030_T7c4CDHIk7s}
\subsection{Brocard Porism: equilateral Isodynamic Pedals have invariant area ratio + circular centroidal locus}
\label{vid:s4DF-iZZO8Y}
\noindent N=3, 4m52s (9/2020). 
\begin{center}\includegraphics[width=.5\textwidth]{pics/s4DF-iZZO8Y.jpg} \\ 
\href{https://youtu.be/s4DF-iZZO8Y}{\url{youtu.be/s4DF-iZZO8Y}}\end{center}
% \input{descr/022_s4DF-iZZO8Y}
\subsection{Continuous Family of Brocard Porisms with Stationary Isodynamic Points $X_{15}$ and $X_{16}$}
\label{vid:jY_8zxBljuk}
\noindent N=3, 1m20s (9/2020). 
\begin{center}\includegraphics[width=.5\textwidth]{pics/jY_8zxBljuk.jpg} \\ 
\href{https://youtu.be/jY_8zxBljuk}{\url{youtu.be/jY\_8zxBljuk}}\end{center}
% A parameter t in 0 to pi/3 is used to parametrize a continous family of Brocard porisms, whereby the Brocard angles Ω1,Ω2 lie on two circular arcs (red) centered on the (fixed) Beltrami points P(2) and U(2).

Left: samples of the family of inellipses (gray) are shown having an elliptic envelope (thick black), with foci on the isodynamic points X15 (and X16, above the page, not shown). Picking a t amounts to selecting one inellipse and circumcircle (thick blue). The Brocard circle (dashed blue) is centered on X183, and contains both Brocard points Ω1,Ω2, X3 and X6. Both the latter and the circumcircle go start as an infinite circle (t=0), converging to X15 at t=pi/3. Notice the Brocard circle cuts the inellipse exactly where it touches the envelope. Notice the (fixed) Brocard angle ω of each porism is equal to t/2.

Right: inifnite nesting of the set of all circumcircles (gray) of the family. Notice these are perpendicular to the two Beltrami circles (red).
\subsection{The Family of Second Brocard Triangles in the Brocard Porism}
\label{vid:Wgwh4-neJp4}
\noindent N=3, 2m50s (9/2020). 
\begin{center}\includegraphics[width=.5\textwidth]{pics/Wgwh4-neJp4.jpg} \\ 
\href{https://youtu.be/Wgwh4-neJp4}{\url{youtu.be/Wgwh4-neJp4}}\end{center}
% \input{descr/031_Wgwh4-neJp4}
\subsection{Brocard Porism: Family of Second Brocard Triangles is a second Brocard Porism}
\label{vid:MprJtB4UW9s}
\noindent N=3, 2m50s (10/2020). 
\begin{center}\includegraphics[width=.5\textwidth]{pics/MprJtB4UW9s.jpg} \\ 
\href{https://youtu.be/MprJtB4UW9s}{\url{youtu.be/MprJtB4UW9s}}\end{center}
% The Brocard porism is a 1d Poncelet family of triangles T=ABC (blue) inscribed in a circle Γ (black) and circumscribed about an inellipse E (black) known as the Brocard inellipse [1]. The Brocard points Ω1,Ω2 are stationary at the foci of E and the Brocard angle ω is invariant. Also stationary are X15, and X16, the first and second Isodynamic points [4], though only the first one is shown. The Brocard circle K (green) contains Ω1,Ω2,X3,X6 and is stationary.

The 2nd Brocard Triangle T' (gold) has vertices A'B'C' at the intersections of symmedians (cevians thru X6) with the Brocard circle [2]. It is therefore inscribed in it.

One key observation borne out by the video is that the Brocard points  Ω1',Ω2' of T' are *also* stationary over the porism. Another known fact is that the isodynamic points of T' are congruent with those of T and are therefore also stationary.

But the main implication is that the family of 2nd Brocard triangles is a new Brocard porism inscribed in K and circumscribed about a 2nd, smaller Brocard inellipse E' (dashed black) of lower eccentricity than E. Interestingly, it can be shown the Brocard circle K' (dashed green) of this family is properly contained in K. 

If second Brocard triangles are computed recursively,  one obtains an infinite sequence of ever-smaller Brocard porisms which converge to the isodynamic point X15 common to all of them. Furthermore,  the sequence of Brocard circles K, K', K'', ..., form a Russian-doll (matryoshka) nesting which also shrinks to X15.

Note: also shown are the points of contact DEF of E to T: since X6 is the Brianchon point (perspector) of E [3], said points occur at the intersection of the symmedians with the sidelengths. Note also that the foci on an inellipse are isogonal conjugates, which is consistent with the fact that Ω1,Ω2 are such a pair. 

за здоро́вье!

[1] E. Weisstein, "Brocard Inellipse", Mathworld, 2020. 
[2] E. Weisstein, "Second Brocard Triangle", Mathworld, 2020. 
[3] E. Weisstein, "Brianchon Point", Mathworld, 2020. 
[4] E. Weisstein, "Isodynamic Points", Mathworld, 2020.
\subsection{Invariants of the Generalized Brocard Porism}
\label{vid:UYI_lBubKXA}
\noindent N=3,4,5,6,7, 10m43s (3/2021). 
\begin{center}\includegraphics[width=.5\textwidth]{pics/UYI_lBubKXA.jpg} \\ 
\href{https://youtu.be/UYI_lBubKXA}{\url{youtu.be/UYI\_lBubKXA}}\end{center}
% This is joint work with Profs Ronaldo Garcia ad Pedro Roitman.

The "Brocard porism" is a family of triangles with fixed Brocard points and invariant Brocard angle [1]. The caustic is known as the "Brocard Inellipse", centered on X39. Its foci coincide with the stationary Brocard points of the family [4]. The symmedian point X6 of the family is also stationary [3] . We have studied the relationship between the Homothetic and Brocard family in [5]. 

The video shows that an N greater than 3 "generalized" Brocard porism (GBP) can be constructed as the polar image of the Poncelet homothetic family wrt to a circle C centered on one of the internal foci.

That the homothetic family conserves area is a trivial fact. Other conservations such as the sum of squared sidelengths, and sum of internal angle cotangents is proved in an upcoming paper [2].

The video shows that though the GBP does not conserve area, it conserves sum of inverse squared sidelengths AND sum of its angle cotangents (distinct from the former).

Erratum 1: in a parallelogram consecutive angles are supplementary (not opposing).

PS.1 -- the video does not mention a curious fact. For the N=3 case, the focus of the homothetic caustic where C is centered coincides with the (stationary) symmedian point X6 of the Brocard porism. You can also visualize the N=3 case live in your browser here: https://bit.ly/3tkDiEG

PS.2 -- the reason for null sum of cotangents in the N=4 generalized brocard porism (GBP) is because cyclic polygons have supplementary opposing angles.




References:

[1] C. Bradley, "The geometry of the Brocard axis and associated conics", CJB/2011/170, 2011, http://people.bath.ac.uk/masgcs/Article116.pdf
[2] S. Galkin, R. Garcia, and D. Reznik, "Invariants of Affine Images of Regular Polygons", to appear, 2021.
[3] D. Reznik and R. Garcia, "An Infinite, Converging, Sequence of Brocard Porisms", 2020. https://arxiv.org/abs/2010.01391
[4] E. Weisstein, "Brocard Inellipse", MathWorld, 2021. https://mathworld.wolfram.com/BrocardInellipse.html
[5] D. Reznik and R. Garcia, "Related By Similarity II: Poncelet 3-Periodics in the Homothetic Pair and the Brocard Porism", Intl. J. of Geom., 10(4), 2021, pp, 18--31.

\section{Cayley-Poncelet (3)}

\subsection{Cayley-Poncelet Phenomena I: Finding an Ellipse Pair in General Position which admits 3-Periodics}
\label{vid:virCpDtEvJU}
\noindent N=3, 5m38s (1/2021). 
\begin{center}\includegraphics[width=.5\textwidth]{pics/virCpDtEvJU.jpg} \\ 
\href{https://youtu.be/virCpDtEvJU}{\url{youtu.be/virCpDtEvJU}}\end{center}
% \input{descr/035_virCpDtEvJU}
\subsection{Cayley-Poncelet Phenomena II: Invariant Power of Center wrt Circumrcircle and Euler's Circle}
\label{vid:4xsm_hQU-dE}
\noindent N=3, 11m24s (1/2021). 
\begin{center}\includegraphics[width=.5\textwidth]{pics/4xsm_hQU-dE.jpg} \\ 
\href{https://youtu.be/4xsm_hQU-dE}{\url{youtu.be/4xsm\_hQU-dE}}\end{center}
% Take two concentric ellipses E and E' (axis-aligned or not) which admit a family of Poncelet 3-periodics (Cayley's condition is satisfied). Let the common center be O.

Let C3 (resp. C5) denote the circumcircle (resp. Euler's circle) of the family. These are centered on X3 and X5, respectively. Let their radii be R and R5. It is well-known R5=R/2.

The following observations are made. If E' and E are not axis-aligned:

a) the locus of X3 is an ellipse concentric and axis-aligned with E'.
b) the locus of X5 is an ellipse concentric but not axis-aligned with E'.
c) the power of O wrt C3 is invariant over the family.
d) the power of O wrt C5 is invariant over the family.

Note: in (a), and (b) the axes of the loci of X3 and X5 become axis aligned with the pair (E,E') is axis aligned itself.

If the pair is non-concentric, the locus of both X3 and X5 are still ellipses, though in general neither is axis-aligned with either E or E'. Also the power of O wrt to either C3 or C5 is no longer constant.
\subsection{Area Invariants of Poncelet N-Periodics and their Polar Polygons}
\label{vid:2TgmJ-YHydQ}
\noindent N=3,4,5,6,7,8, 12m33s (3/2021). 
\begin{center}\includegraphics[width=.5\textwidth]{pics/2TgmJ-YHydQ.jpg} \\ 
\href{https://youtu.be/2TgmJ-YHydQ}{\url{youtu.be/2TgmJ-YHydQ}}\end{center}
% \input{descr/034_2TgmJ-YHydQ}

\section{Convex Combinations (3)}

\subsection{Barycenter with Median, and Incenter with Intouchpoint}
\label{vid:3Gr3Nh5-jHs}
\noindent N=3, PT41S (5/2019). 
\begin{center}\includegraphics[width=.5\textwidth]{pics/3Gr3Nh5-jHs.jpg} \\ 
\href{https://youtu.be/3Gr3Nh5-jHs}{\url{youtu.be/3Gr3Nh5-jHs}}\end{center}
% \input{descr/037_3Gr3Nh5-jHs}
\subsection{Excenter and its corresponding Extouch point}
\label{vid:OD8Ah0hf8yQ}
\noindent N=3, PT41S (5/2019). 
\begin{center}\includegraphics[width=.5\textwidth]{pics/OD8Ah0hf8yQ.jpg} \\ 
\href{https://youtu.be/OD8Ah0hf8yQ}{\url{youtu.be/OD8Ah0hf8yQ}}\end{center}
% The three excenters of the family of 3-periodic orbits in elliptic billiards describe the same elliptic locus. Shown is the locus of the convex combination of one excenter with its corresponding extouch point (where the excircle touches a side), which happens to also be elliptic and congruent with the caustic.  Apparently, and unlike other convex combinations (barycenter and median, orthocenter and altitude foot, incenter and contact point), in this case all intermediate loci are elliptic.

For more information: https://dan-reznik.github.io/Elliptical-Billiards-Triangular-Orbits/
\subsection{Orthocenter with one altitude foot, and Circumcenter with median}
\label{vid:HZFjkWD_CnE}
\noindent N=3, PT41S (5/2019). 
\begin{center}\includegraphics[width=.5\textwidth]{pics/HZFjkWD_CnE.jpg} \\ 
\href{https://youtu.be/HZFjkWD_CnE}{\url{youtu.be/HZFjkWD\_CnE}}\end{center}
% \input{descr/039_HZFjkWD_CnE}

\section{Cross-Ratio Invariants (6)}

\subsection{Poncelet Cross-Ratio Invariants I: exploring N=5, 6, 7, 8}
\label{vid:vgHoLM5pg6o}
\noindent N=5,6,7,8, 37m29s (2/2021). 
\begin{center}\includegraphics[width=.5\textwidth]{pics/vgHoLM5pg6o.jpg} \\ 
\href{https://youtu.be/vgHoLM5pg6o}{\url{youtu.be/vgHoLM5pg6o}}\end{center}
% Recall the cross-ratio of 4 collinear points A,B,C,D  = (AC)(BD)/((BC)(AD)) [1]. There are 24 ways to order the four points, but only 6 groups of 4 combinations each yield distinct results, though all interrelated by simple expressions, see [1].

The video introduces several (possibly known) invariants in the sum (and/or product) of cross ratios defined over 3 "diagonals" in Poncelet N-Periodics. We cover N=5,6,7,8.

Comments appreciated!

[1] E. Weisstein, "Cross-Ratio", MathWorld, 2021. https://mathworld.wolfram.com/CrossRatio.html
\subsection{Poncelet Cross-Ratio Invariants II: 3 diagonals in the N=5 case}
\label{vid:XqpYVrXbK5s}
\noindent N=5, 7m42s (2/2021). 
\begin{center}\includegraphics[width=.5\textwidth]{pics/XqpYVrXbK5s.jpg} \\ 
\href{https://youtu.be/XqpYVrXbK5s}{\url{youtu.be/XqpYVrXbK5s}}\end{center}
% \input{descr/041_XqpYVrXbK5s}
\subsection{Poncelet Cross-Ratio Invariants III: 3 diagonals in the N=5 self-intersected case}
\label{vid:4bd0YhQZMPM}
\noindent N=5, 8m58s (2/2021). 
\begin{center}\includegraphics[width=.5\textwidth]{pics/4bd0YhQZMPM.jpg} \\ 
\href{https://youtu.be/4bd0YhQZMPM}{\url{youtu.be/4bd0YhQZMPM}}\end{center}
% \input{descr/042_4bd0YhQZMPM}
\subsection{Poncelet Cross-Ratio Invariants IV: N=6 and N=8 and unit cross-ratio products}
\label{vid:i6krQu5Ls1E}
\noindent N=6,8, 13m20s (2/2021). 
\begin{center}\includegraphics[width=.5\textwidth]{pics/i6krQu5Ls1E.jpg} \\ 
\href{https://youtu.be/i6krQu5Ls1E}{\url{youtu.be/i6krQu5Ls1E}}\end{center}
% \input{descr/043_i6krQu5Ls1E}
\subsection{Poncelet Cross-Ratio Invariants V: Bicentric Pentagons, Pentagrams and their Projective Images}
\label{vid:b-WIfLej_yY}
\noindent N=5, 23m55s (2/2021). 
\begin{center}\includegraphics[width=.5\textwidth]{pics/b-WIfLej_yY.jpg} \\ 
\href{https://youtu.be/b-WIfLej_yY}{\url{youtu.be/b-WIfLej\_yY}}\end{center}
% This is not a Wicca video! We are actually describing interesting cross-ratio [1] related invariants manifested by over the Poncelet family of bicentric 5-gons (simple or self-intersected) [2]. Cross-ratios are measured on four collinear points obtained from certain diagonal (or chordal) arrangements between vertices in the trajectory. We show that the sum and product of cyclically-defined cross ratios are simultaneously invariant for 4 out 6 labeling of the collinear points.

Maybe this is a Wicca video after all...

[1] E. Weisstein, "Cross Ratio", MathWorld 2021.
[2] E. Weisstein, "Poncelet Porism", MathWorld 2021.
\subsection{Unit Product of Cross-Ratios in a Hexagon Independent of Vertices}
\label{vid:heiqfRTQ2Mc}
\noindent N=6, 11m50s (2/2021). 
\begin{center}\includegraphics[width=.5\textwidth]{pics/heiqfRTQ2Mc.jpg} \\ 
\href{https://youtu.be/heiqfRTQ2Mc}{\url{youtu.be/heiqfRTQ2Mc}}\end{center}
% \input{descr/045_heiqfRTQ2Mc}

\section{Early Results (6)}

\subsection{Mittenpunkt is stationary at center of billiard II}
\label{vid:AoCWcza95OA}
\noindent N=3, 1m13s (5/2019). 
\begin{center}\includegraphics[width=.5\textwidth]{pics/AoCWcza95OA.jpg} \\ 
\href{https://youtu.be/AoCWcza95OA}{\url{youtu.be/AoCWcza95OA}}\end{center}
% Amazing geometric phenomenon, discovered experimentally.

The mittenpunkt (where lines from each excenter to its corresponding median meet) of all triangular orbits in an elliptic billiard is STATIONARY at the billiard's center.

The triangle point which led to this amazing May 7th, 2019 discovery was identified in 1836 by Christian Heinrich von Nagel as the symmedian point of the excentral triangle. https://en.wikipedia.org/wiki/Mittenpunkt

More info: https://dan-reznik.github.io/Elliptical-Billiards-Triangular-Orbits/
\subsection{Mittenpunkt is stationary at center of billiard I}
\label{vid:tMrBqfRBYik}
\noindent N=3, 3m13s (6/2019). 
\begin{center}\includegraphics[width=.5\textwidth]{pics/tMrBqfRBYik.jpg} \\ 
\href{https://youtu.be/tMrBqfRBYik}{\url{youtu.be/tMrBqfRBYik}}\end{center}
% The family of N=3 (triangular) orbits is shown in an elliptic billiard with a/b=1.5. The mittenpunkt X(9) of a triangle is the point of concurrence of lines drawn from each excenter through the corresponding side's median. For all orbits these are stationary at the origin.

More info: https://dan-reznik.github.io/Elliptical-Billiards-Triangular-Orbits/
\subsection{Feuerbach Point Sweeps Billiard and its Anti-Complement and Extouch Points sweep caustic}
\label{vid:TXdg7tUl8lc}
\noindent N=3, 4m49s (7/2019). 
\begin{center}\includegraphics[width=.5\textwidth]{pics/TXdg7tUl8lc.jpg} \\ 
\href{https://youtu.be/TXdg7tUl8lc}{\url{youtu.be/TXdg7tUl8lc}}\end{center}
% a/b=1.5 elliptic billiard shown as well as its family of N=3 orbits. Also shown are the X(1)-centered incircle (green), X(5)-centered nine-point circle (pink), and their point of contact, the Feuerbach point X(11). Also shown the X(100), the anticomplement of X(11), i.e., the Feuerbach point of the orbit's anticomplementary triangle. Equivalently, twice the reflection of X(11) about X(2). Not shown: X(9) lies stationary at the center of the billiard!

More Info:  https://dan-reznik.github.io/Elliptical-Billiards-Triangular-Orbits/
\subsection{Loci of Vertices of Medial, Intouch and Feuerbach Triangles is not elliptic}
\label{vid:OGvCQbYqJyI}
\noindent N=3, 4m49s (7/2019). 
\begin{center}\includegraphics[width=.5\textwidth]{pics/OGvCQbYqJyI.jpg} \\ 
\href{https://youtu.be/OGvCQbYqJyI}{\url{youtu.be/OGvCQbYqJyI}}\end{center}
% An a/b=1.5 elliptic billiard is shown with its family of N=3 (triangular) orbits. Also shown are the non-elliptic loci of the medians, the intouch points, and the external feuerbach points.

More information: https://dan-reznik.github.io/Elliptical-Billiards-Triangular-Orbits/
Interactive Applet: https://editor.p5js.org/dreznik/full/i1Lin7lt7
\subsection{Conservation of Sum and Product of Cosines}
\label{vid:P8ykpE_ZbZ8}
\noindent N=3, 3m13s (7/2019). 
\begin{center}\includegraphics[width=.5\textwidth]{pics/P8ykpE_ZbZ8.jpg} \\ 
\href{https://youtu.be/P8ykpE_ZbZ8}{\url{youtu.be/P8ykpE\_ZbZ8}}\end{center}
% The 1d family of 3-periodics in an Elliptic Billiard not only has a stationary Mittenpukt X(9), but it also conserves an incredible quantity: r/R, the ratio of inradius-to-circumradius. Conservation corollaries include: (i) the sum of orbit's cosines = 1+r/R; (ii) the product of excentral cosines = r/(4R); and (iii) the ratio of excentral-to-orbit areas = r/(2R). Here we visualize (i) and (ii).

LEFT: An a/b=1.5 elliptic biliard is shown as well as its N=3 family of orbits (blue). For each orbit the excentral polygon is also shown (green). The red dot at the center of the billiard represents the stationary Mittenpunkt X(9). 

MIDDLE: a 2d representation of both orbit and excentral polygons is shown: a first vertex is placed at (0,0), a second one at (1,0), and a third one at some (u,v) location on the plane such that this normalized triangle is similar to the orbit (blue) or excentral (green). Drawn in the background are level curves of r/R for such a (u,v) family of triangles. Notice the (u,v) tip of the orbit triangle follows a constant r/R level-curve, whereas the excentral one does not. Note r/R=1+cosA+cosB+cosC, i.e., r/R level curves are congruent to sum-of-cosine ones.

RIGHT: the same as LEFT except level curves are shown for the *product* of cosines. Notice the excentral arm follows a product level curve perfectly whereas the orbit one does not.

In summary, orbit triangles conserve the *sum* of their cosines whereas the excentrals conserve the *product* of their cosines.

More Info:  https://dan-reznik.github.io/Elliptical-Billiards-Triangular-Orbits/
\subsection{Elliptic Loci of $X_{1}$ to $X_{5}$ and Euler Line}
\label{vid:sMcNzcYaqtg}
\noindent N=3, 4m49s (7/2019). 
\begin{center}\includegraphics[width=.5\textwidth]{pics/sMcNzcYaqtg.jpg} \\ 
\href{https://youtu.be/sMcNzcYaqtg}{\url{youtu.be/sMcNzcYaqtg}}\end{center}
% \input{descr/047_sMcNzcYaqtg}

\section{Ellipse-Inscribed Triangles (8)}

\subsection{Ellipse-Mounted Triangles: Elliptic locus of the Orthocenter $X_{4}$ and suprising area invariance}
\label{vid:Fo-tNRcA-CQ}
\noindent N=3, 9m17s (7/2020). 
\begin{center}\includegraphics[width=.5\textwidth]{pics/Fo-tNRcA-CQ.jpg} \\ 
\href{https://youtu.be/Fo-tNRcA-CQ}{\url{youtu.be/Fo-tNRcA-CQ}}\end{center}
% Consider an ellipse with semi-axes a,b. Choose two fixed points on it:

v1 = [a cos(t1), b sin(t1)]
v2 = [a cos(t2), b sin(t2)]

Let a third point P(t) ride on the ellipse, and let T be the triangle {v1,v2,P(t)}. Observations:

a) the locus of the orthocenter X4 of T is *always* an upright ellipse (axis aligned with the original one, but with major axis vertical and minor horizontal).

b) pick an angle "k". over all v1,v2 such that t1+t2=k, the locus generated is the same ellipse up to translation!

c) the maximal (resp minimal) area elliptic locus occurs for k = 180 (resp. 0) degrees.

to do:

- where is the center of the elliptic locus
- what are its axes
- what is its translation (looks like a straight line) as one visits all t1+t2=k?

Having fun folks!
\subsection{Circle-Mounted Triangles: Surprising Loci of the Brocard Points}
\label{vid:Ms8jC9yOKU4}
\noindent N=3, 2m47s (7/2020). 
\begin{center}\includegraphics[width=.5\textwidth]{pics/Ms8jC9yOKU4.jpg} \\ 
\href{https://youtu.be/Ms8jC9yOKU4}{\url{youtu.be/Ms8jC9yOKU4}}\end{center}
% This video explores the loci of the two Brocard points [1] for a family of triangles with two stationary vertices on the circumference of a circle and a third one free to revolve around the same circle.

Indeed a few experimental suprises were in store!
\subsection{Loci of Ellipse-Inscribed Triangles I: Basic Phenomena}
\label{vid:zjiNgfndBWg}
\noindent N=3, 3m3s (10/2020). 
\begin{center}\includegraphics[width=.5\textwidth]{pics/zjiNgfndBWg.jpg} \\ 
\href{https://youtu.be/zjiNgfndBWg}{\url{youtu.be/zjiNgfndBWg}}\end{center}
% A triangle V1 V2 P(t) (blue) is inscribed in an ellipse E (black) with semi-axes a,b (in the video a/b=1.5). While V1,V2 are fixed, P(t) slides along E's boundary. Show are the loci of the barycenter X2 (brown), circumcenter X3 (red), orthocenter X4 (orange), collinear on the Euler line (dashed green). 

As it turns out, for any choice of fixed V1,V2, the locus of:

- X2 is an ellipse axis-aligned w E, with semi-axes a/3,b/3
- X3 is a segment
- Z4 is an ellipse axis-aligned w E, with aspect ratio equal to a/b.

An intermediate point X(ρ)=X2+ρ(X4-X2) is shown which is a fixed linear combination of X2 and X4 (in the video ρ=0.5, which makes X(ρ) be X381 [1]). For whichever choice of V1,V2 and ρ, the locus of X(ρ) is always an ellipse (green), in general not axis-aligned w E.

[1] C. Kimberling, "Encycl. of Triangle Centers", 2020. https://faculty.evansville.edu/ck6/encyclopedia/ETC.html
\subsection{Loci of Ellipse-Inscribed Triangles II: $X_\rho$ slides merrily along the Euler line}
\label{vid:w5KuN_0rQBQ}
\noindent N=3, 2m12s (10/2020). 
\begin{center}\includegraphics[width=.5\textwidth]{pics/w5KuN_0rQBQ.jpg} \\ 
\href{https://youtu.be/w5KuN_0rQBQ}{\url{youtu.be/w5KuN\_0rQBQ}}\end{center}
% A triangle V1 V2 P(t) (blue) is inscribed in an ellipse E (black) with semi-axes a,b (in the video a/b=1.5). While V1,V2 are fixed, P(t) slides along E's boundary. Show are the elliptic loci of the barycenter X2 (brown), circumcenter X3 (red), orthocenter X4 (orange), collinear on the Euler line (dashed green) and of an point X(ρ)=X2+ρ(X4-X2) on the Euler line for variable ρ. 

As ρ varies, the locus of X(ρ) is a family of ellipse (in general not axis-aligned w E) whose center Oρ follows a straight line (dashed purple). Notice at ρ=-1/2 (at X3), and also at some second location, one of the axes of said locus vanishes.

[1] C. Kimberling, "Encycl. of Triangle Centers", 2020. https://faculty.evansville.edu/ck6/encyclopedia/ETC.html
\subsection{Loci of Ellipse-Inscribed Triangles III: family of V1V2 parallels causes rigid locus translation}
\label{vid:zFOeENDJRho}
\noindent N=3, 2m48s (10/2020). 
\begin{center}\includegraphics[width=.5\textwidth]{pics/zFOeENDJRho.jpg} \\ 
\href{https://youtu.be/zFOeENDJRho}{\url{youtu.be/zFOeENDJRho}}\end{center}
% Consider the family of triangles T(t)=V1V2P(t) inscribed in an ellipse E with axes a,b. Namely, V1,V2 are fixed on the boundary of E and P(t) executes one revolution along it.

The video shows the loci of triangle centers X2,X3,X4,X5 of T(t) [1] which are all ellipses (except X3 which is a segment). Specifically, we can observe how these loci change over parallel V1V2 (P(t) is not shown).

What is observed is that as V1V2 is translated, the loci also translate. Interestingly, the motion of their center is a line (magenta) which passes thru the center of E.

[1] C. Kimberling, "Encycl. of Triangle Centers", 2020. https://faculty.evansville.edu/ck6/encyclopedia/ETC.html
\subsection{Loci of Ellipse-Inscribed Triangles IV: Multiple Loci Over Parallel V1V2}
\label{vid:TpBjKlkFjkg}
\noindent N=3, 1m46s (10/2020). 
\begin{center}\includegraphics[width=.5\textwidth]{pics/TpBjKlkFjkg.jpg} \\ 
\href{https://youtu.be/TpBjKlkFjkg}{\url{youtu.be/TpBjKlkFjkg}}\end{center}
% Consider the family of triangles T(t)=V1V2P(t) inscribed in an ellipse E with axes a,b. Namely, V1,V2 are fixed on the boundary of E and P(t), not shown, executes one revolution along it.

It turns out that provided a point X lies on the Euler line at some fixed linear combination of X2 and X4, its locus will be an ellipse over T(t). The video shows elliptic loci for triangle centers X2,X5,X381,X4.

Furthermore:

1) Over the family of parallel V1V2, said loci rigidly translate
2) For a given V1V2, the center of all loci are collinear.
3) When V1 and V2 are symmetric about the center of E, the centers of all loci collapse to a point.

In the video one can observe how the discrete family of loci change their relative position as a family of parallel V1V2 is traversed.
\subsection{Loci of Ellipse-Inscribed Triangles V: Circular Loci if V1V2 Horizontal or Vertical for Certain $\rho$}
\label{vid:nLeKvxcicNY}
\noindent N=3, 2m49s (10/2020). 
\begin{center}\includegraphics[width=.5\textwidth]{pics/nLeKvxcicNY.jpg} \\ 
\href{https://youtu.be/nLeKvxcicNY}{\url{youtu.be/nLeKvxcicNY}}\end{center}
% Consider the family of triangles T(t)=V1V2P(t) inscribed in an ellipse E with axes a,b. Namely, V1,V2 are fixed on the boundary of E and P(t), not shown, executes one revolution along it. Consider the locus of a a point X a fixed linear combination of the barycenter X2 and orthocenter X4, i.e.:

X = X2 + (X4-X2) ρ

The video shows an intriguing result: only if V1V2 is either horizontal or vertical can said locus be a circles. Specifically, two values of ρ for each configuration accomplish this.
\subsection{Ellipse-Inscribed Triangles VI: Envelope of $X_{4}$ Loci is Area-Invariant and Cousin of Pascal's Limaçon}
\label{vid:sPQrz7ddRfA}
\noindent N=3, 3m5s (10/2020). 
\begin{center}\includegraphics[width=.5\textwidth]{pics/sPQrz7ddRfA.jpg} \\ 
\href{https://youtu.be/sPQrz7ddRfA}{\url{youtu.be/sPQrz7ddRfA}}\end{center}
% Consider the triangle family T(t)=V1V2P(t) inscribed in an ellipse E (black) of semiaxes a,b. Let V1 and V2 be fixed on E while P(t) executes one revolution on it.

1) Let X be a triangle center. Over the T(t), the locus of X will be an ellipse if X is a fixed linear combination of the barycenter X2 and orthocenter X4. (X will necessarily lie on the Euler Line).

2) Consider the case where X=X4. In this case, it can be shown that for any choice of V1,V2, the locus is an upright ellipse passing thru said points, which is axis-aligned with E and has aspect ratio of b/a.

3) Now pick a V1. Over all possible (fixed) placements of V2, one obtains a family of elliptic loci w the features in (2). It can be shown their centers will lie on an ellipse (red) whose aspect ratio is also a/b.  The *envelope* of said family will be a (generally) non-convex shape (pink), which is the affine image of Pascal's Limaçon [1].

The video shows said locus family as one varies V1, as well as the locus of their centers (an red ellipse) and their varying envelope (pink). Surpringly, the envelope is area-invariant over all V1, and this is valid for any triangle center X.

Note: the origin of the term "Limaçon" is the latin for snail, "limax".

[1] E. Weisstein, "Pascal's Limaçon", Mathworld, 2020. https://mathworld.wolfram.com/Limacon.html

\section{Ellipse Echoes (1)}

\subsection{Ellipse Echoes I: Circular wavefronts released into circular and elliptic cavities}
\label{vid:LQnNLMhH9EE}
\noindent N=n/a, 12m27s (1/2021). 
\begin{center}\includegraphics[width=.5\textwidth]{pics/LQnNLMhH9EE.jpg} \\ 
\href{https://youtu.be/LQnNLMhH9EE}{\url{youtu.be/LQnNLMhH9EE}}\end{center}
% These are early experiments with the geometry of circular wavefronts released from a boundary or interior point of a circular or elliptic cavity.

These can be tried at: https://dan-reznik.github.io/ellipse-echo-p5js/

\section{Envelopes (9)}

\subsection{Envelope of Antiorthic and Gergonne Lines}
\label{vid:Q7l6_Z4IyEI}
\noindent N=3, 2m43s (2/2020). 
\begin{center}\includegraphics[width=.5\textwidth]{pics/Q7l6_Z4IyEI.jpg} \\ 
\href{https://youtu.be/Q7l6_Z4IyEI}{\url{youtu.be/Q7l6\_Z4IyEI}}\end{center}
% The isogonal (or isotomic) conjugate of a line with respect to a triangle is a circumconic [1].

If an elliptic billiard is regarded as a stationary, X(9)-centered circumellipse to the 3-periodic family, we can analyze its isogonal and isotomic conjugate lines dynamically. These turn out to be [3] the orbits' Antiorthic Axis [4] and Line L(31) [5], respectively.

In particular, what is the envelope [2] of such straight lines over the family of 3-periodics?

The two lines can be constructed with any pair of Triangle Centers lying on them (Peter Moses provides many in [3]). For special reasons, we use:

a) L(1), Antiorthic Axis, X(44)X(513)
b) L(31), Line X(514)X(661) -- note: to be reviewed, but this is the Gergonne line of a triangle derived from the reference one. ACT?

The video shows the locus of the above Triangle Centers for an a/b=1.618=phi billiard, drawn black. Orbits are drawn blue, as well as the envelope of said lines.

Specifically:

Left: the non-elliptic locus of X(44) is shown red. The elliptic locus of X(1155) is shown green. Notice how the Antiorthic axis (dashed blue) is dynamically tangent to the latter, i.e., the locus of X(1155) is the envelope of the Antiorthic Axis.

Right: the non-elliptic locus of X(857) which lies on L(31) is shown red. The elliptic locus of X(908), also on L(31), is shown green. Notice how line L(31) (dashed blue) is dynamically tangent to the latter, i.e., the locus of X(908) is the envelop of L(31).

In both cases, "C" depicts where the envelope currently is.

[1] http://mathworld.wolfram.com/Circumconic.html
[2] http://mathworld.wolfram.com/Envelope.html
[3] Peter Moses, mentioed by Clark Kimberling, ETC under X(9), https://faculty.evansville.edu/ck6/encyclopedia/ETC.html
[4] http://mathworld.wolfram.com/AntiorthicAxis.html
[5] https://faculty.evansville.edu/ck6/encyclopedia/CentralLines.html
\subsection{Evolute of Elliptic Billiard and Envelope of $X_{1}$-$X_{5}$}
\label{vid:eBStp-7X5yE}
\noindent N=3, 4m1s (2/2020). 
\begin{center}\includegraphics[width=.5\textwidth]{pics/eBStp-7X5yE.jpg} \\ 
\href{https://youtu.be/eBStp-7X5yE}{\url{youtu.be/eBStp-7X5yE}}\end{center}
% The envelope to a family of lines is that curve to which every member is tangent, aka as the Caustic [1]. We are intersested in the envelopes generated by pairs of Triangle Centers over the 3-periodic family.

An a/b=1.618 elliptic billiard (EB) is shown (black) as well as its family of 3-periodics (triangular orbits, blue).

Also shown are X1 and X5 riding along their elliptic loci (red, green, respectively). The X1X5 line is shown purple. Its envelope (purple) is the 4-cuspid astroidal curve. Miraculously, this is simultaneously tangent to the X1 and X5 locus! Any idea how?

Also shown is the evolute [2] of the elliptic billiard,which is the envelope of inward-pointing normals. This is also astroidal and shown dashed blue.  This curve has a pleasant closed-form parametric expression in terms of a,b the ellipse's axes [2]:

x(t) = (a^2-b^2) cos(t)^3 / a
y(t) = (b^2-a^2) sin(t)^3 / b

Let P1(t) be a vertex of a 3-periodic. The line P1X1 is parallel to the ellipse normal at P1, therefore this family yields the involute.

The instantaneous X1X5 line is shown purple having the purple envelope as its caustics. The instantaneous P1X1 line is drawn gray, and its caustic is the ellipse evolute.

Suprisingly both the evolute and the X1X5 caustic touch the X1 locus on the same four spots (marked by blue dots). One of them is given by:

x1= c2 ((-b2 + d)/c2)^(3/2)/a
y1 = c2 ((a^2 - d)/c2)^(3/2)/b

where  a2=a*a, b2=b*b, d = Sqrt[a2^2 - a2*b2 + b2^2], and c2=a2-b2

The Orthocenter X4 is shown as an orange dot. When it is on one of the 3-periodic vertices, the 3-periodic is a right triangle. There doesn't seem to be any specific phenomena tied to such configurations.

A large gallery of Triangle Center pairs and their envelopes is available in [3].

[1] http://mathworld.wolfram.com/Envelope.html
[2] http://mathworld.wolfram.com/EllipseEvolute.html
[3] https://dan-reznik.github.io/Elliptical-Billiards-Triangular-Orbits/envelopes1618.html
\subsection{Envelope of 3-Periodic Vertex with Triangle Center}
\label{vid:bRY61RdxCkM}
\noindent N=3, 2m41s (2/2020). 
\begin{center}\includegraphics[width=.5\textwidth]{pics/bRY61RdxCkM.jpg} \\ 
\href{https://youtu.be/bRY61RdxCkM}{\url{youtu.be/bRY61RdxCkM}}\end{center}
% Consider the family of 3-periodics in an Elliptic Billiard with axes a,b [1]. Let each triangular 3-periodic be defined by vertices P1,P2,P3, where P1(t) = [a cos(t), b sin(t)], t=[0,2π) is used as a parametrization. 

Take a Kimberling Center [2] Xi, and consider the family of lines defined by [P1(t),Xi(t)].

The video shows the envelope [3] of such a family for various Xi, i=1,2,3,4,5,6,7,8,10,11,12,20, over a/b=[1.,2).

Notable cases include:

a) P1X1: the evolute [4] of the billiard boundary. as these are always parallel to the normal at P1
b) P1X5: spiderman? why does it explote?

More info in [5].

Note: the lower right animation of X(20) is incorrectly showing X(12). To be corrected.

[1] https://dan-reznik.github.io/Elliptical-Billiards-Triangular-Orbits/videos.html
[2] https://faculty.evansville.edu/ck6/encyclopedia/ETC.html
[3] http://mathworld.wolfram.com/Envelope.html
[4] http://mathworld.wolfram.com/Evolute.html
[5] https://dan-reznik.github.io/Elliptical-Billiards-Triangular-Orbits/envelopes1618.html
\subsection{Evolute Triangles of P1(t) with $X_i$}
\label{vid:DhqDdMAlBZM}
\noindent N=3, 8m1s (2/2020). 
\begin{center}\includegraphics[width=.5\textwidth]{pics/DhqDdMAlBZM.jpg} \\ 
\href{https://youtu.be/DhqDdMAlBZM}{\url{youtu.be/DhqDdMAlBZM}}\end{center}
% Given a 3-periodic T=Pi, i=1,2,3, elliptic billiard w axes a,b. Let P1(t) = [a cos(t), b sin(t) ], and X be a Triangle Center of T. Explicit expressions for P2(t) and P3(t) can be found here [1].

We define the "Evolute Triangle" of (T,Xi) by the instantaenous tangency points of P1(t)X_i, P2(t)X_i, P3(t)X_i to their common envelope [2]. These are shown in pink for i=1,3,5,20, for an a/b=1.45. We've found that for higher a/b some of the envelopes become too large and/or non-compact.

Note: the envelope of (T,X1) is the evolute of the billiard [3].
 
[1] Ronaldo Garcia, "Elliptic Billiards and Ellipses Associated to the 3-Periodic Orbits", Am. Math. Monthly, Volume 126, 2019 - Issue 6
[2] http://mathworld.wolfram.com/Envelope.html
[3] http://mathworld.wolfram.com/EllipseEvolute.html
\subsection{Elliptic Envelope of P1(t) with P1(t+$\pi/2$)}
\label{vid:8a4JoddyEyc}
\noindent N=3, 3m13s (3/2020). 
\begin{center}\includegraphics[width=.5\textwidth]{pics/8a4JoddyEyc.jpg} \\ 
\href{https://youtu.be/8a4JoddyEyc}{\url{youtu.be/8a4JoddyEyc}}\end{center}
% Let a vertex P1 of a 3-periodic in en Elliptic Billiard (black) be parametrized as P1=[a cos(t), b sin(t)], and its copy P1', lying 90 degrees ahead as: P1'=[a cos(t+pi/2), b sin(t+pi/2)] = [a sin(t), - b cos(t)]. The 3-periodic initiating at P1 (resp. P1') is shown blue (resp. dashed blue).

The video shows the family of lines [P1,P1'] (red) and the caustic they envelope (green): an ellipse similar to the EB itself. The tangency point "C" is dynamically the midpoint between P1 and P1'.

The above can be proven via an affine transformation which takes the EB to a circle (multiply the x coordinate by b/a). In this new ambient, the billiard is a circle, and the 3-periodic and its forward doppleganger are equilaterals, and the internal caustic is the common incircle. By symmetry, these equilaterals will be tangent to the incircle at their midpoints. At the original space, the caustic is the inverse affine transform of the incircle.
\subsection{Envelope of 3-Periodic P1 and reflected P2 is Elliptic}
\label{vid:GJgiUulX1aU}
\noindent N=3, 6m1s (3/2020). 
\begin{center}\includegraphics[width=.5\textwidth]{pics/GJgiUulX1aU.jpg} \\ 
\href{https://youtu.be/GJgiUulX1aU}{\url{youtu.be/GJgiUulX1aU}}\end{center}
% An a/b=1.5 Elliptic Billiard is shown as well as its family of  3-periodics (blue) and their caustic (brown). A copy of the latter (reflected about the Billiard Center) is also shown (dashed blue). The [P1, -P2] family of lines (red) is instantaneous tangent (at "C") to a non-confocal elliptic caustic (green). Note [P1,-P2] is vertical/horizontal at when the 3-periodics are isosceles, suggesting a simple method to compute the axis of the (green) caustic.
\subsection{The Bat-Envelope of $X_{48}$ and $X_{37143}$}
\label{vid:Kr93eFZnB_U}
\noindent N=3, 1m21s (3/2020). 
\begin{center}\includegraphics[width=.5\textwidth]{pics/Kr93eFZnB_U.jpg} \\ 
\href{https://youtu.be/Kr93eFZnB_U}{\url{youtu.be/Kr93eFZnB\_U}}\end{center}
% An elliptic billiard (EB) is shown with aspect ratio a/b=1.618...=golden ratio. Also shown is its family of 3-periodics (blue triangles). Also shown are Triangle Centers [1] X(48) and X(37143) [formerly the isotomic conjugate of X(30565)]. The former's locus is an upright convex curve (green, not a proper ellipse [3]), whereas the latter rides along the EB (it is a "swan" [4]). Also shown is the envelope [2] (purple) of the family of lines defined by the pair. Holy Mackerel, this looks like a sideways bat!

For the aspect ratio a/b=golden ratio chosen, we do not understand why:

(i) both centers move monotonically with respect to the 3-periodic family;
(ii) the envelope has 8 tangency points with the locus of X(48); 
(iii) when "C" passes thru these so does X(48). why?
(iv) four tangency points intersect the Elliptic Billiard precisely where the (non-elliptic) locus of X(48) intersects it;
(v) there are 8 cusps;
(vi) two of which coincide with the billiard's foci;
(vii) what kind of curves are those between the cusps? could they be elliptic arcs?

For additional harmonies when "C" lies at the cusps and/or tangency points of the envelope with X(48), visit [5].

[1] https://faculty.evansville.edu/ck6/encyclopedia/ETC.html
[2] https://mathworld.wolfram.com/Envelope.html
[3] https://arxiv.org/abs/2001.08041
[4] https://arxiv.org/abs/2002.00001
[5] https://dan-reznik.github.io/Elliptical-Billiards-Triangular-Orbits/envelopes1618.html
\subsection{Envelopes of Sides of Derived Triangles}
\label{vid:SJrgWtdX8xU}
\noindent N=3, 2m1s (3/2020). 
\begin{center}\includegraphics[width=.5\textwidth]{pics/SJrgWtdX8xU.jpg} \\ 
\href{https://youtu.be/SJrgWtdX8xU}{\url{youtu.be/SJrgWtdX8xU}}\end{center}
% Consider the family of 3-periodics in an Elliptic Billiard with aspect ratio a/b. The envelope of its sides is a confocal caustic. The video shows the envelope of the sides 16 well-known derived triangles, as the aspect ratio is varied. Equivalently, shown is the envelope of the line family defined by two consecutive vertices of a derived triangle.
\subsection{Envelope of Simson Lines from $X_{100}$ and $X_{99}$ to two N=3 Poncelet Families}
\label{vid:79veSHrElb4}
\noindent N=3, 4m5s (6/2020). 
\begin{center}\includegraphics[width=.5\textwidth]{pics/79veSHrElb4.jpg} \\ 
\href{https://youtu.be/79veSHrElb4}{\url{youtu.be/79veSHrElb4}}\end{center}
% Both ellipses shown have a/b=1.5.

Left: 3-periodics (blue) in the Elliptic Billiard (EB, black). Recall these have stationary Mittenpunkt X9, and that X100 lies at the intersection of the circumcircle with the X9-centered circumellipse, i.e., the EB. Over the family of 3-periodics, the envelope of Simson lines (red) with respect o X100 is an astroid-like curve (purple). The envelope of lines (dashed red) passing through X100 and perpendicular to the aforementioned Simson family is a "cushion shaped" curve (purple) w four cusps, tangent to the EB at the four vertices. Also shown is the EB's evolute (dashed black).

Right: the N=3 poncelet family (blue) inscribed in their Steiner Ellipse (a,b) semiaxes and centered on X2, with the Steiner Inellipse (a/b,b/2) as their caustic has constant area. Recall X99 lies at the intersection of the Steiner Ellipse with the Circumcircle. Over this family, the envelope of Simson lines (red) with respect o X100 is also an astroid-like curve (purple). The envelope of lines (dashed red) passing through X99 and perpendicular to the aforementioned Simson family is also a "cushion shaped" curve (purple) w four cusps, tangent to the EB at the four vertices. Also shown is the EB's evolute (dashed black).

\section{Frégier (3)}

\subsection{Frégier Phenomena I: Area-Invariant Envelope of Chords}
\label{vid:UCCG5AT8dh8}
\noindent N=n/a, 16m15s (1/2021). 
\begin{center}\includegraphics[width=.5\textwidth]{pics/UCCG5AT8dh8.jpg} \\ 
\href{https://youtu.be/UCCG5AT8dh8}{\url{youtu.be/UCCG5AT8dh8}}\end{center}
% This is joint work with M. Helman, R. Garcia, and D. Laurain, who introduced us to M. Frégier's wonderful theorem [1,2].

In 1810 M. Frégier (French Mathematician) discovered a remarkable fact. Take a point M on an ellipse E and consider all pairs of rays at a right angle with each other emanating from M. Consider the family of chords defined by the intersections of these rays with E. All such chords pass thru a common point F, called the "Frégier Point" [2].

Let ϴ denote the angle between the pairs of rays, above ϴ=90 degreess. The video shows the following old/new facts:

a) Probably known: the locus of F over all M is an ellipse.

Let ϴ denote the angle between the pair of rays. When ϴ is not 90, the following observations could be new:

b) the family of chords envelops an ellipse E', which is in general non-concentric and not axis-aligned with E.
c) Over all M, the semi-axes of E' are variable as is their ratio.
d) Remarkably, the *area* of E' is invariant over M!!!
e) Over all M, the centers of E' span a 3rd ellipse, E'', this time concentric, axis-aligned and *homothetic* to E.

The video also shows a "real-time" discovery: when ϴ=60, the envelope of chords in [b] will be inscribed in the triangle formed by M, and the two rays emitted symmetrically from the normal (at 30 and -30 degrees) and the chord between the intersections.

Any help w/ proofs appreciated!

[1] D. Laurain, Private Communication, Jan 2021. 
[2] E. Weisstein, "Frégier's Theorem", MathWorld, 2020. https://mathworld.wolfram.com/FregiersTheorem.html
\subsection{Frégier Phenomena II: Envelope of Chords over M}
\label{vid:mJ6opTPFAO4}
\noindent N=n/a, 11m50s (1/2021). 
\begin{center}\includegraphics[width=.5\textwidth]{pics/mJ6opTPFAO4.jpg} \\ 
\href{https://youtu.be/mJ6opTPFAO4}{\url{youtu.be/mJ6opTPFAO4}}\end{center}
% This is a continuation of the previous video.

Let M be a point on the boundary of an ellipse E. Let n be the ellipse normal at M.  Let Q1 and Q2 be the intersections of tow rays shot from M, along rays r1 and r2 (blue in the video), which are copies of n rotated by β+α and β-α. 

Before we sawo that when α=90, for fixed M, and over all β, the envelope of Q1Q2 is a point, known as the Frégier point [2]

The video shows the envelope of Q1Q2 for various combinations of fixed α,β, but over all M on E. These can contain cusps, stationary points, or be regular.
\subsection{Frégier Phenomena III: Circular Envelopes and a New Invariant for a Poncelet Family}
\label{vid:AzNXeBU2NTI}
\noindent N=n/a, 18m11s (1/2021). 
\begin{center}\includegraphics[width=.5\textwidth]{pics/AzNXeBU2NTI.jpg} \\ 
\href{https://youtu.be/AzNXeBU2NTI}{\url{youtu.be/AzNXeBU2NTI}}\end{center}
% We observe the dynamic geometry of elliptic envelopes over 4 different Poncelet families and discover a neat invariant. Let me know if you have ideas on how to prove it.

\section{Harmonic Polygons (6)}

\subsection{Inversive-Area Iso-Contours of Harmonic Polygons (obtained as Polar Images of the Homothetic family)}
\label{vid:erV4NILgWrE}
\noindent N=3,4, 2m26s (8/2021). 
\begin{center}\includegraphics[width=.5\textwidth]{pics/erV4NILgWrE.jpg} \\ 
\href{https://youtu.be/erV4NILgWrE}{\url{youtu.be/erV4NILgWrE}}\end{center}
% A family (blue) of triangles (left) and quadrilaterals (right) are shown which are Poncelet families interscribed in a pair of homothetic ellipses (only the outer one is shown). These can also be regarded as affine images of a family of regular polygons.

Also shown are their polar images (magenta) with respect to the left focus of the caustic (not shown). These are  inscribed in a (dashed magenta circle) and circumscribe a Brocard inellipse (not shown). Their Brocard points lie at the foci of the latter. This class of polygons is known as "harmonic" since they can also be regarded as the image of regular polygons under inversion [1,2,3].

The animation shows isocurves of are of the inversion of the harmonic family wrt to point on the plane, as they revolve.

[1] J. Casey, "A sequel to the first six books of the Elements of Euclid", Longman, Dublin, 1888

[2] T. Sharp, "Harmonic Polygons", The Mathematical Gazette
Vol. 29, No. 287 (Dec., 1945), pp. 210-213.

[3] A. Zazlavsky and A. Akopyan, "Геометрические свойства кривых
второго порядка" (Geometric properties of curves of second order), sec. 4.6, pp. 129. MCNMO Publishing House, Moscow, 2011.
\subsection{Inversive Image of Pivoting Harmonic Family Part I: Stationary Inversive Symmedian/Lemoine Point}
\label{vid:jn3GD9SEkdU}
\noindent N=6, 9m53s (8/2021). 
\begin{center}\includegraphics[width=.5\textwidth]{pics/jn3GD9SEkdU.jpg} \\ 
\href{https://youtu.be/jn3GD9SEkdU}{\url{youtu.be/jn3GD9SEkdU}}\end{center}
% This is a generalization of results regarding triangles in previous videos.

A family of harmonic polygons H (magenta) is shown inscribed in a circle. John Casey (1888) constructs them as the inversive images of regular N-gons with respect to some circle [1]. T. Sharp (1945) constructs them as projections of a regular polygon [2].  Harmonic polygons have a symmedian point K. If N is even these are the intersections of diagonals. If N is odd these are the intersections of lines from the vertices to the opposite points of contact with the ellipse enveloped by the sides (aka the Brocard inellipse).

The video shows a few curious properties of the inversive image H' of harmonic polygons (unproved) wrt to an inversion circle C with center Cinv.

1) Trivially H' is also harmonic (the composition of two inversions is an inversion).
2) The symmedian K' of H' is stationary and collinear with Cinv and K.
3) Amazingly, over rigid rotations of H about K, K' is stationary!
symmedian inversive image of the harmonic family wrt fo a fixed circle is harmonic

[1] John Casey, "A sequel to the first six books of the Elements of Euclid, containing an easy introduction to modern geometry, with numerous examples". Dublin: Hodges, Figgis & co., 1888.
[2] T. Sharp, "Harmonic Polygons", The Mathematical Gazette, Vol. 29, No. 287 (Dec., 1945), pp. 210-213.
\subsection{Inversive Image of Pivoting Harmonic Family Part II: Rigid Rotations about Harmonic Limiting Points}
\label{vid:9GwfELl-tHk}
\noindent N=6, 13m44s (8/2021). 
\begin{center}\includegraphics[width=.5\textwidth]{pics/9GwfELl-tHk.jpg} \\ 
\href{https://youtu.be/9GwfELl-tHk}{\url{youtu.be/9GwfELl-tHk}}\end{center}
% A family of harmonic polygons H (magenta) is shown inscribed in a circle centered at O. John Casey (1888) constructs them as the inversive images of regular N-gons with respect to some circle [1]. T. Sharp (1945) constructs them as projections of a regular polygon [2].

Harmonic polygons have a symmedian point K. If N is even these are the intersections of diagonals. If N is odd these are the intersections of lines from the vertices to the opposite points of contact with the ellipse enveloped by the sides (aka the Brocard inellipse).

Define the Brocard circle as having KO as diameter. Let l1 and l2 be the limiting points of the pencil defined by the circumcircle and Brocard circle of a harmonic family. Consider a family H' (also harmonic) whose vertices are the inverses of those of H wrt to a circle centered somewhere.

The video shows that when a harmonic family is rotated about either limiting point l1 or l2, the corresponding limiting point of H' stays put.

[1] John Casey, "A sequel to the first six books of the Elements of Euclid, containing an easy introduction to modern geometry, with numerous examples". Dublin: Hodges, Figgis & co., 1888.
[2] T. Sharp, "Harmonic Polygons", The Mathematical Gazette, Vol. 29, No. 287 (Dec., 1945), pp. 210-213.
\subsection{Harmonic Polygons, Part III: Isocurves of Inversive Brocard Angle are Circles of the Schoute Pencil}
\label{vid:VFqSvpuU0wg}
\noindent N=6, 15m13s (8/2021). 
\begin{center}\includegraphics[width=.5\textwidth]{pics/VFqSvpuU0wg.jpg} \\ 
\href{https://youtu.be/VFqSvpuU0wg}{\url{youtu.be/VFqSvpuU0wg}}\end{center}
% Let H (blue) be a harmonic polygon (blue) , and let H' (red) be a new polygon with vertices at the inversions of those of H wrt to a circle centered on a point C. H' is also harmonic [1, Section VI].

The video shows two potentially new observations which are likely generalizations of [3, Thm 11], to harmonic polygons.

a) the isocurves of Brocard angle of H' are circles in the pencil of the circumcircle and Brocard circle of H, also known as the Schoute pencil [2,3]. 
b) if C is on the Brocard circle or on the radical axis of the pencil, then the Brocard angle of H' is equal to that of H.

[1] J. Casey, "A Sequel to Euclid Elements", Longman, London, 1888.
[2] P. Pamfilos, "Symmedian", Geometrikon, http://users.math.uoc.gr/~pamfilos/eGallery/problems/Symmedian.pdf
[3] R. Johnson, "Directed Angles and Inversion, and a Proof of Schoute's Theorem", American Math. Monthly, 24:313–317, 1917.
\subsection{Harmonic Polygons, Part IV: Four Harmonic Inversive Images of Steiner's Porism}
\label{vid:PGaUZHOvQIg}
\noindent N=5, 2m26s (9/2021). 
\begin{center}\includegraphics[width=.5\textwidth]{pics/PGaUZHOvQIg.jpg} \\ 
\href{https://youtu.be/PGaUZHOvQIg}{\url{youtu.be/PGaUZHOvQIg}}\end{center}
% Steiner's porism [1] consists of a chain of touching circles moving (as their radii vary) in the interstice between to non-concentric circles. They are the inversive image of a "ball bearing" with identical circular disks between two concentric circles. In [2,3] it was stated that 4 harmonic polygons can be built out of Steiner's porism. Deeper connections with invariant symmetric polynomials are are explored in [4].

Let ri be the radii of the disks in a Steiner porism. The video shows sum(1/ri), sum(1/ri^2), sum(1/ri^3) is conserved.

The centers of the original disks are the vertices of a regular polygon. Their inversive image with respect to C is a harmonic polygon (green).

Also harmonic are the polygons (magenta) with vertices at the inversions of points of contact of the bearings with (i) the inner circle, (ii) outer circle, and (iii) with each other.

All said 4 harmonic families are distinct, and each conserves a different sum of cotangents of internal angles (sums of cot^2 and cot^3 are also conserved).


[1] E. Weisstein, "Steiner's Porism", MathWorld 2021. https://mathworld.wolfram.com/SteinersPorism.html

[2] G. Tarry & J. Neuberg, "Sur les polygones et les polyèdres harmoniques", Comptes rendus de l'Association française pour l'avancement des sciences, 1887. Note: extrait du Congrès de Nancy, 1886.

[3] T. C. Simmons, "A new Method for the Investigation of Harmonic Polygons, Proc London Math. Soc., Volume 1--18, number 1, pp. 289--304, 1886.

[4] R. Schwartz & S. Tabachnikov, "Descartes Circle Theorem, Steiner Porism, and Spherical Designs", Am. Math. Monthly, volume 127, Issue 3, 2020.
\subsection{Exploring the Dynamic Geometry and Conservations of the Steiner-Soddy and Harmonic Poncelet Porims}
\label{vid:1zrp1HMeUbk}
\noindent N=3,4,5,6, 22m43s (1/2022). 
\begin{center}\includegraphics[width=.5\textwidth]{pics/1zrp1HMeUbk.jpg} \\ 
\href{https://youtu.be/1zrp1HMeUbk}{\url{youtu.be/1zrp1HMeUbk}}\end{center}
% The video explores the dynamic geometry and conservations in a Poncelet porism whose vertices are the centers of circles in a Steiner chain with N circles. This family is the polar image of the harmonic family with respect to its circumcircle. One of the main observations is that the sum of tangents of half angles to the power of k, with k less than N is conserved. See [1,2,3].

References:

[1] Richard Schwartz and Sergei Tabachnikov,  "'Descartes Circle Theorem, Steiner Porism, and Spherical Designs", American Math Monthly Vol 127 Issue 3 (2020)
[2] Ronaldo Garcia, Dan Reznik, Pedro Roitman, "New Properties and Invariants of Harmonic Polygons", 2021. https://arxiv.org/abs/2112.02545
[3] Ronaldo Garcia, Liliana Gheorghe, Dan Reznik, "Exploring the Steiner-Soddy Porism", Proceedings of the Intl. Conf. on Geom. & Graphics, São Paulo, Brazil, 2022. https://arxiv.org/abs/2201.02222

\section{Homothetic Pair (5)}

\subsection{Homothetic Poncelet Pair: Invariant-Area Evolute Polygons}
\label{vid:JCj0q7_hlA8}
\noindent N=5, 3m21s (11/2020). 
\begin{center}\includegraphics[width=.5\textwidth]{pics/JCj0q7_hlA8.jpg} \\ 
\href{https://youtu.be/JCj0q7_hlA8}{\url{youtu.be/JCj0q7\_hlA8}}\end{center}
% Consider an affine image of regular polygons rotating in a circle, identical to Poncelet N-periodics in the homothetic ellipse pair (black and brown ellipses). Let Pi denote its vertices, i=1,...N.

This family conserves (i) sum sidelengths squared, (ii) area, and (iii) sum of cotangents, see [1,2].

Here we show a new, curious phenomenon.

Define the Generalized Evolute Polygon (GEP, pink) with vertices Pi' along the ellipse normals at Pi, at a distance s r from the Pi, where s is a real number and r is the radius of curvature at Pi. 

Pi' = Pi + s r n

Note: when s=0 (resp. s=1), the Pi' sweep the ellipse (resp. the ellipse evolute [3]).

It turns out for any choice of s, the signed area of the GEPs is invariant over the family.

The video shows the N=5 family for four values of s={1/4, 1/2, 3/4, 1}. Also shown is a circle (dashed pink) centered on P1' and passing through P1. When s=1 (lower right), this circle is the classic osculating circle at P1.

Postscript: the GEPs also conserve sum of sidelengths squared except when N=4 or 6.

[1] https://youtu.be/2PdsC3CcqaE
[2] https://youtu.be/30cuWWaZv7A
[3] E. Weisstein, "Ellipse Evolute", MathWorld 2021. https://mathworld.wolfram.com/EllipseEvolute.html
\subsection{Homothetic Poncelet Pair: Invariant-Area N=5 Evolute Polygons and the Ellipse Evolute}
\label{vid:ChsfLzKrb4o}
\noindent N=5, 1m45s (11/2020). 
\begin{center}\includegraphics[width=.5\textwidth]{pics/ChsfLzKrb4o.jpg} \\ 
\href{https://youtu.be/ChsfLzKrb4o}{\url{youtu.be/ChsfLzKrb4o}}\end{center}
% This video shows the family of generalized evolute polygons (GEP pink) who are an affine image of the family of regular 5-gons in a circle. These can also be regarded as the family of Poncelet N-periodics in the homothetic ellipse pair.

The vertices Qi of the GEP lie along ellipse normals at a distance proportional to the inverse of the curvature namely:

Qi = Pi + s ni (s/ki)

In the video s is chosen to be 1, so the Qi are the centers of curvature, i.e., they lie on the ellipse evolute (green curve).

In the video a/b=Sqrt(2) so the evolute just touches the upper and lower vertex of the ellipse. 

Both families are area invariant and conserve their sum of sidelengths squared,. It turns out the GEP do not conserve either quantity for the special case of N=4.
\subsection{Homothetic Poncelet Pair: Zero-Area Evolute Polygons}
\label{vid:3nvXYFoI5Wg}
\noindent N=3,5,6,8, 3m19s (11/2020). 
\begin{center}\includegraphics[width=.5\textwidth]{pics/3nvXYFoI5Wg.jpg} \\ 
\href{https://youtu.be/3nvXYFoI5Wg}{\url{youtu.be/3nvXYFoI5Wg}}\end{center}
% It can be shown the Poncelet family in the homothetic pair conserves (i) sum sidelengths squared, (ii) area, and (iii) sum of cotangents, see [1,2]. Let Pi, i=1,...,N denote its vertices.

Define the " generalized evolute polygon" (GEP, pink) as having vertices Pi' such that:

Pi' = Pi + s ri ni,    i=1,...N

where: s is a scalar, ni is the inward unit normal at Pi, and ri is the radius of curvature at Pi, respectively.

Note: when s=0 (resp. s=1), the Pi' sweep the ellipse (resp. the ellipse evolute). 

It turns out that for any choice of "s", the GEP conserves area (exception: N=4, see below). In fact, one can always choose two "s" such that the signed area is dynamically zeto.

The video shows one such zero-signed-area GEP for N=3,5,6,8 cases.  A few observations:

a) the zero-area GEPs for N=3 are segments (see [1])
b) the "s" required for N=3 and N=6 are the same
c) for N=4 (not shown), the area of the GEP variable.

[1] https://youtu.be/OFA_j25R8ks
\subsection{Zero-Area N=3 Homothetic Evolute Polygons}
\label{vid:f80QaYs5_J4}
\noindent N=3, 3m21s (12/2020). 
\begin{center}\includegraphics[width=.5\textwidth]{pics/f80QaYs5_J4.jpg} \\ 
\href{https://youtu.be/f80QaYs5_J4}{\url{youtu.be/f80QaYs5\_J4}}\end{center}
% Consider the affine image of a family of equilateral triangles rotating in a unit circle. These can also be regarded as a  family of Poncelet 3-periodics (blue) in the homothetic pai pair (black and brown). Let Pi, i=1,..,N denote its vertices. Here N=3.

Define the generalized evolute polygon (GEP) as having vertices at:

Pi' = Pi + s ri ni, where ri is the radius of curvature at Pi, ni is the inward-pointing normal at Pi, and s is a fixed scalar of choice.

Claim: for any s, and over the family, the area of the GEP is constant. In fact this is holds for all N greater than 2.

Furthermore, one can select two values "s" for which the area of the N=3 GEP is zero; in each case, the GEP collapses to a segment parallel to either major or minor axes, see [1].

The video also shows that the vertices is a 3-region pseudo-lemniscate (dashed pink).

[1] D. Reznik, "3-GEP zero-area meet at X(76)" -- https://www.youtube.com/watch?v=OFA_j25R8ks&feature=youtu.be
\subsection{Zero area N=3 evolute polygons are segments which intersect at $X_{76}$}
\label{vid:OFA_j25R8ks}
\noindent N=3, 1m27s (3/2021). 
\begin{center}\includegraphics[width=.5\textwidth]{pics/OFA_j25R8ks.jpg} \\ 
\href{https://youtu.be/OFA_j25R8ks}{\url{youtu.be/OFA\_j25R8ks}}\end{center}
% Consider an affine image of equilaterals rotating in the unit circle. These (i) are inscribed in an ellipse E, (ii) maintain their centroid X(2) stationary at the center of E, and (iii) have constant area. This family can also be regarded as the Poncelet family of 3-periodics interscribed in a pair of concentric, homothetic ellipses.

Interestingly, the family also conserves (iv) the sum of squared sidelengths, and (v) the sum of its internal angle cotangents.

Define the "Generalized Evolute Polygon" (GEP) of the affine family as having vertices Pi' = Pi + s Ri ni, where:

Pi = vertices of affine polygons, i=1,2,..,N
Ri = radius of curvature of E at Pi
ni = inward-pointing normal of E at Pi (shown as arrows)

Construction lines are shown (dashed gray) from each vertex Pi to each Pi' along ni, and measuring s Ri.

It can be shown that for any choice of "s" the area of the GEP is conserved, for any N greater than 2. In fact,  two values of "s" can be chosen such that the GEPs have zero signed area. See other videos on these polygons on our channel.

The video shows the N=3 case: the zero-area GEPs are segments with (collinear vertices), parallel to either the major or minor axis of E.

Said segments intersect at the "3rd Brocard Point" or X(76) as specified in [1]. Over the family, its locus is an ellipse (dashed pink).

[1] C. Kimberling, "The Third Brocard Point X(76)", Encyclopedia of Triangle Centers (ETC), 2021. https://faculty.evansville.edu/ck6/encyclopedia/ETC.html

\section{Hyperbolic Billiard (2)}

\subsection{3-Periodics in a Hyperbolic Billiard}
\label{vid:kaUdX0eGDec}
\noindent N=3, 1m30s (12/2020). 
\begin{center}\includegraphics[width=.5\textwidth]{pics/kaUdX0eGDec.jpg} \\ 
\href{https://youtu.be/kaUdX0eGDec}{\url{youtu.be/kaUdX0eGDec}}\end{center}
% The video shows the family of 3-periodic orbits (blue) in a hyperbolic billiard (black). The top branch is reflective (about the normal) while the bottom one is refractive, i.e., it reflects incoming rays about the tangent vector to the curve. Let P1 denote the vertex on the top branch and P2,P3 those on the bottom branch. 

Let dij denote the distance |Pi-Pj|, and ci the cosine of the 3-periodic triangle at vertex Pi.

As the top legend shows, this family conserves the quantity d12+d13-d23, and c1-c2-c3.

Compare with the standard elliptic billiard: it conserves L=d12+d13+d23, and the sum of cosines c1+c2+c3, i.e., for the hyperbolic billiard, those quantities associated with the P2P3 trajectory segment (the refracted one) must appear with their sign inverted for invariants to continue to hold.
\subsection{A triangle, its elliptic billiard, and three associated hyperbolic billiards}
\label{vid:mkdv4401-zY}
\noindent N=3, 2m52s (12/2020). 
\begin{center}\includegraphics[width=.5\textwidth]{pics/mkdv4401-zY.jpg} \\ 
\href{https://youtu.be/mkdv4401-zY}{\url{youtu.be/mkdv4401-zY}}\end{center}
% A family of 3-periodics (blue) is shown in the elliptic billiard (black). Also shown are 3 associated circum-hyperbolas (light blue, pink, orange). Each is tangent at two vertices to the internal bisector, and at a third vertex, tangent to the external bisector. In this sense each hyperbola is a "billiard" to the 3-periodic: it undergoes 2 "reflections" and one "refraction".

\section{IMPA 33o CBM (4)}

\subsection{Fenômenos Euclidianos em Famílias Ponceletianas, Aula 01 -- Apresentação}
\label{vid:84sG5siQWwg}
\noindent N=n/a, PT1H1M23S (6/2021). 
\begin{center}\includegraphics[width=.5\textwidth]{pics/84sG5siQWwg.jpg} \\ 
\href{https://youtu.be/84sG5siQWwg}{\url{youtu.be/84sG5siQWwg}}\end{center}
% Aula 01 (Apresentação) do nosso minicurso introdutório no 33o Colóquio Brasileiro de Matemática, IMPA, Rio de Janeiro, a ser realizado em Ago de 2021. Mais infos sobre o evento: https://impa.br/en_US/eventos-do-impa/2021-2/33cbm/
\subsection{Fenômenos Euclidianos em Famílias Ponceletianas, Aula 03 -- Revisão Geometria dos Triângulos}
\label{vid:AKmvd1uCZ9o}
\noindent N=n/a, PT1H17M58S (7/2021). 
\begin{center}\includegraphics[width=.5\textwidth]{pics/AKmvd1uCZ9o.jpg} \\ 
\href{https://youtu.be/AKmvd1uCZ9o}{\url{youtu.be/AKmvd1uCZ9o}}\end{center}
% \input{descr/087_AKmvd1uCZ9o}
\subsection{Fenômenos Euclidianos em Famílias Ponceletianas, Aula 04 -- Invariantes Bilhar Elíptico N=3}
\label{vid:ti1rkBK62ck}
\noindent N=n/a, 36m8s (7/2021). 
\begin{center}\includegraphics[width=.5\textwidth]{pics/ti1rkBK62ck.jpg} \\ 
\href{https://youtu.be/ti1rkBK62ck}{\url{youtu.be/ti1rkBK62ck}}\end{center}
% \input{descr/088_ti1rkBK62ck}
\subsection{Fenômenos Euclidianos em Famílias Ponceletianas, Aula 08 -- Famílias Concêntricas}
\label{vid:fMC4nmQ6OQE}
\noindent N=n/a, PT54M (7/2021). 
\begin{center}\includegraphics[width=.5\textwidth]{pics/fMC4nmQ6OQE.jpg} \\ 
\href{https://youtu.be/fMC4nmQ6OQE}{\url{youtu.be/fMC4nmQ6OQE}}\end{center}
% \input{descr/089_fMC4nmQ6OQE}

\section{In- and Circum-parabolas (10)}

\subsection{Surprising Loci of Inparabolas over Poncelet triangles inscribed in a circle: Part I}
\label{vid:HTe3Wlqctq4}
\noindent N=3, 6m30s (9/2021). 
\begin{center}\includegraphics[width=.5\textwidth]{pics/HTe3Wlqctq4.jpg} \\ 
\href{https://youtu.be/HTe3Wlqctq4}{\url{youtu.be/HTe3Wlqctq4}}\end{center}
% Given any triangle ABC, the focus of any inconic which is a parabola -- an inparabola -- must lie on the circumcircle.

Consider the family of Poncelet triangles (blue) inscribed in a circle (black) and circumscribing a concentric inellipse (brown).  Consider a fixed point F on the circumcircle; this prescribes a unique inparabola (magenta) w focus at F.

The video shows that with fixed F and over the Poncelet family, the locus of the vertex V of the inparabolas is a circle passing through F and tangent at a point T to the inellipse. The locus of the center of the directrix (foot of perp dropped from F to directrix) is a twice-sized circle also containing F and centered on T.

Proofs welcome!

Not shown:

(a) over N=3 bicentrics and the Brocard porism, the locus of the vertex is also a circle (though not touching the caustic). Liliana Gheorghe has simulated the case of the excentral family to N=3 bicentrics and reported the locus of the vertex there is also a circle.

Conjecture: for a Poncelet triangle family inscribed in a circle, the locus of the vertex of the inparabola wrt a fixed focus F on the circumcircle is a circle.

(b) the locus of the *center* of the circular locus of the vertex. Liliana Gheorghe has told me this is a conic homothetic to the inner one.

(c) the envelope of the directrix

Dan
\subsection{Family of tangential triangles to N=3 bicentrics: locus of vertices and of $X_{4}$}
\label{vid:9thwJcfUBmM}
\noindent N=3, 9m8s (9/2021). 
\begin{center}\includegraphics[width=.5\textwidth]{pics/9thwJcfUBmM.jpg} \\ 
\href{https://youtu.be/9thwJcfUBmM}{\url{youtu.be/9thwJcfUBmM}}\end{center}
% We explore the family of tangential triangles [1] to Poncelet N=3 "bicentrics" (aka "poristics"). We show that if X3 is interior (resp. exterior) to the incircle, the tangentials will be inscribed in an ellipse (resp. hyperbola). If X3 is on the incircle, i.e., r/R = sqrt(2)-1, the family is inscribed in a parabola w focus on X1' of the tangentials (X3 of the poristics).

A nice phenomenon is that the locus of their tangential orthocenter X4'  is an line ellipse (resp. hyperbola) if the bicentrics' X3 is interior (resp. exterior) to theincircle. If X3 is on the incircle, then the locus of X4' is an infinite, vertical line. 

You can play with the simulation here: https://bit.ly/3a4a1a5

[1] E. Weisstein, "Tangential Triangle", MathWorld 2021. https://mathworld.wolfram.com/TangentialTriangle.html
\subsection{The amazing family of polar triangles derived from a parabola-inscribed Poncelet family}
\label{vid:L2UpEHFQ6CY}
\noindent N=3, 23m14s (9/2021). 
\begin{center}\includegraphics[width=.5\textwidth]{pics/L2UpEHFQ6CY.jpg} \\ 
\href{https://youtu.be/L2UpEHFQ6CY}{\url{youtu.be/L2UpEHFQ6CY}}\end{center}
% 1) start with the family of parabola-inscribed triangles T circumscribing an incircle centered on the focus of the parabola (X1 of the family). This family is actually the tangential triangles to a bicentric family whose circumcenter is on the incircle, see our previous video [1].

2) Then add to the above a new triangle family T' , the polar triangles of T wrt to the parabola, i.e., with sides bound by the tangents to the parabola at the vertices of T. A basic property of the parabola dictates that the circumcircle of any triangle bounded by tangents goes thru the focus, i.e., T'contains  X1 of the T. This "polar" family is Ponceletian, inscribed in a (dashed dark red) hyperbola, and circumscribing the original parabola

Here are a few curious properties of the  polar family, the most degenerate Poncelet family I have ever seen.

1) its Euler line goes thru the green parabola focus. Its X(26), a point normally not on the circumcircle, is on the circumcircle, and stationary on said focus. X(68) and X(110) are stationary on the left and right vertex of the hyperbola the family is inscribed to. X(161) is stationary on the left extreme of the incircle. Incidentally, X(110) is the focus of the Kiepert (in)Parabola, shown Magenta below. 

2) the loci of Xk, k={2, 3, 4, 5, 6, 20, 22, 23, 24, 25, 49, 51, 52, 54, 64, 66, 67, 69, 74,110,113,125,140,141,143,146,154,155,156,159,161,182,184,185,186, 193,195,... are straight lines parallel to the directrix (see a few below)

3) the loci of Xk, k=99, 107, 112, 249, 476, 691, 827, 907, 925, 930, 933, 935 are circles in a parabolic pencil: they all touch at a single point: X(110), the aforementioned focus of the Kiepert parabola, shown Magenta below).  Most of these centers lie on the circumcircle, but a few don't

4) though visually, the locus of X1 of T' looks like a vertical line, numerically, it is likely a parabola. The jury is still out on this one.

To do: what is the locus of the vertex of the Kiepert over the family?

[1] https://youtu.be/9thwJcfUBmM
\subsection{Focus Locus Hocus Pocus: Circumparabolas of the Bicentric Family}
\label{vid:_7gv3Pqed6M}
\noindent N=3, 9m34s (9/2021). 
\begin{center}\includegraphics[width=.5\textwidth]{pics/_7gv3Pqed6M.jpg} \\ 
\href{https://youtu.be/_7gv3Pqed6M}{\url{youtu.be/\_7gv3Pqed6M}}\end{center}
% A circumparabola is the isogonal image of a tangent to the circumcircle. If you hold that tangent fixed, you can observe the family of circumparabolas induced by a poncelet family inscribed in a fixed circumcircle. The video shows that the locus of the focus, vertex, and directrix "center" are high-degree curves over Poncelet families with a caustic which is (i) a concentric inellipse, (ii) the Brocard inellipse.

Amazingly, the locus of the focus is a *straight line* if the family under consideration is Chapple's porism, i.e., N=3 bicentric triangles (caustic is a circle).
\subsection{Poncelet ``homothetic'' triangle family and its family of circumparabolas}
\label{vid:DKd7kjnVVTc}
\noindent N=3, 6m47s (9/2021). 
\begin{center}\includegraphics[width=.5\textwidth]{pics/DKd7kjnVVTc.jpg} \\ 
\href{https://youtu.be/DKd7kjnVVTc}{\url{youtu.be/DKd7kjnVVTc}}\end{center}
% The video explores the family of circumparabolas to Poncelet triangles in the "homothetic family". Circumparabolas are constructed as isotomic images of a line L tangent to the Steiner of the family. The video shows that (i)  all circumparabolas remain tangent to a reflection of L about the common center, and (ii) that the locus of the focus, vertex, and "center" of the directrix are curves of high degree.

Thanks to Bernard Gibert for pointing out that the focus locus is of degree 9.
\subsection{Surprising Loci of Inparabolas over Poncelet triangles inscribed in a circle: Part II}
\label{vid:qicI7zl7ICM}
\noindent N=3, 21m18s (10/2021). 
\begin{center}\includegraphics[width=.5\textwidth]{pics/qicI7zl7ICM.jpg} \\ 
\href{https://youtu.be/qicI7zl7ICM}{\url{youtu.be/qicI7zl7ICM}}\end{center}
% Erratum: many times in the video I say "focus locus" instead of "vertex locus".

In a previous video we saw that over Poncelet triangles (blue) inscribed in a circle (black) and circumscribing a concentric inellipse (brown), the locus of the vertices of inparabolas (magenta) with focus at a fixed point F on the circumcircle is a circle (solid green) through F and tangent to the inellipse.

A corollary is that the locus of the reflection of the focus on the vertex ("center" of directrix) is a double-radius circle (solid orange), also passing thru F, but centered at the tangency w the caustic.

This video features a few new observations:

a) over any Poncelet family  inscribed in a circle, the directrix rigidly rotates about point W, the reflection of the focus on the point of contact with the caustic. (in one family in the video, W is stationary over all F)

b) over F on the circumcircle, the locus of the *center* of the circular loci of the vertex and of the projection of F on the directrix) are ellipses (dashed green and orange) concentric and axis-aligned with the caustic.

If the inellipse has a focus coinciding with the circumcenter (this family are the excentral triangles to bicentrics), you get more harmonies: (i) the loci of both vertex and directrix center (dashed green and orange) are circles (I wonder if they are in the same pencil as the circumcircle). the latter is  externally tangent to the caustic. (ii) point W is stationary at the other focus of the caustic, over all Poncelet and over all F

Exercise: for each of the Poncelet families studied, what is the locus of W over all F on the circumcircle?
\subsection{Loci of a family of parabola-inscribed equilaterals and their polar triangles}
\label{vid:5nnPWQGp1tE}
\noindent N=3, 7m4s (10/2021). 
\begin{center}\includegraphics[width=.5\textwidth]{pics/5nnPWQGp1tE.jpg} \\ 
\href{https://youtu.be/5nnPWQGp1tE}{\url{youtu.be/5nnPWQGp1tE}}\end{center}
% \input{descr/093_5nnPWQGp1tE}
\subsection{Loci of circumparabolas of an equilateral triangle and associated family of polar triangles}
\label{vid:51jSJmaZ8Vk}
\noindent N=3, 13m49s (10/2021). 
\begin{center}\includegraphics[width=.5\textwidth]{pics/51jSJmaZ8Vk.jpg} \\ 
\href{https://youtu.be/51jSJmaZ8Vk}{\url{youtu.be/51jSJmaZ8Vk}}\end{center}
% In this video we consider circumparabolas to a fixed equilateral triangle T which are isogonal images of a line tangent to the circumcircle at a point Q. We show that:

a) the locus of the focus is a non-intersecting 3-branched infinite curve
b) the envelope of the directrix is a 3-branch infinite curve w 3 self-intersections
b) the circumcenter of the polar triangle is collinear with the circumcenter of T and Q (this is only true if T is equilateral)
c) the locus of X4 of the polar of T wrt to the circumparabolas coincides with (b)

We also showcase several loci of several other triangle centers of the polar triangle. The locus of X99 is a 3-leaf clover which is tangent to (a). At a special location o Q on the circumcircle, X99 coincides with the focus of the associated circumparabola.
\subsection{More Circumparabola Loci over Poncelet Triangle: the case of the Polar triangle}
\label{vid:tTTIs_zxU3U}
\noindent N=3, 13m36s (10/2021). 
\begin{center}\includegraphics[width=.5\textwidth]{pics/tTTIs_zxU3U.jpg} \\ 
\href{https://youtu.be/tTTIs_zxU3U}{\url{youtu.be/tTTIs\_zxU3U}}\end{center}
% We explore loci phenomena of related to the polar triangle with respect to circumparabolas over both N=3 bicentric and "homothetic" Poncelet families. The following observations are made

1) N=3 bicentrics (circumparabolas are isogonal images of fixed tangent to circumcircle)

a) we saw before the locus of circumparabolas follow a straight line
b) the envelope of circumparabola directrices is a parabola whose focus is the incenter of the bicentric family!
c) the locus of X2 of the polar triangle wrt circumparabola is a straight line parallel to (a)
c) the locus of X4 of the polar triangle is a parabola tangent to (b) at a mysterious point

2) Poncelet triangles of the "homothetic family" (circumparabolas are the isotomic image of a fixed tangent to the Steiner ellipse)

a) we saw before all circumparabolas are tangent to a reflection of the original tangent wrt center
b) the envelope of the directrix is also a parabola (could not show result due to a bug)
c) the locus of X2 of the polar triangle is a straight line parallel to the original tangent to the Steiner
\subsection{General Circle-Inscribed Poncelet Triangles and Loci of the Inparabola Vertex}
\label{vid:Y_Mh4zPtwWM}
\noindent N=3, 8m13s (10/2021). 
\begin{center}\includegraphics[width=.5\textwidth]{pics/Y_Mh4zPtwWM.jpg} \\ 
\href{https://youtu.be/Y_Mh4zPtwWM}{\url{youtu.be/Y\_Mh4zPtwWM}}\end{center}
% This video explores the locus of the vertex V of inparabolas to a generic circle-inscribed Poncelet family of triangles with focus on a fixed point F on the circumcircle.

Recall classically known facts (thanks to Liliana Gheorghe for letting us know). The Simson line of triangle wrt a point F is tangent to the inparabola w focus F at its vertex V. Therefore the latter is the projection of F on the Simson line.

We show that:

a) over Poncelet, for fixed F, the locus of V is a circle
b) over all F, the locus of the center of (a) is an ellipse
c) over Poncelet, the envelope of Simson lines is a point W which is the reflection of F on the center of (a).
d) Over all F, the locus of W in (c) is an ellipse concentric with the inconic.

\section{Inconics \& Circumconics (22)}

\subsection{Every triangle has a unique Circumbilliard}
\label{vid:vSCnorIJ2X8}
\noindent N=3, 1m13s (5/2019). 
\begin{center}\includegraphics[width=.5\textwidth]{pics/vSCnorIJ2X8.jpg} \\ 
\href{https://youtu.be/vSCnorIJ2X8}{\url{youtu.be/vSCnorIJ2X8}}\end{center}
% Given a triangle ABC it will be associated with a unique circumellipse E which passes thru each of A,B,C, and is locally tangent (perpendicular) to the bisectors of A,B, and C. We call this the "circumbilliard", as ABC will be an N=3 orbit of E. Because we know the billiard will be centered on the triangle's Mittenpunkt X(9), another way to say this is: the circumbilliard is the circumellipse centered on X(9).

In the animation two vertices are fixed, and a third one moves along a sinusoidal curve. For each configuration the circumbilliard is shown, centered on M, the Mittenpunkt, as well as its axes' orientation.

More info: https://dan-reznik.github.io/Elliptical-Billiards-Triangular-Orbits/
\subsection{Circumbilliard of anticomplementary triangle}
\label{vid:18RyUdh8qLk}
\noindent N=3, 1m37s (6/2019). 
\begin{center}\includegraphics[width=.5\textwidth]{pics/18RyUdh8qLk.jpg} \\ 
\href{https://youtu.be/18RyUdh8qLk}{\url{youtu.be/18RyUdh8qLk}}\end{center}
% An elliptic billiard (a/b=1.5) is shown as well as its family of N=3 orbits. Also shown is the orbit's anticomplementary triangle T', the (non-elliptic locus of its vertices -- dashed blue), the latter's 9-point circle (pink, congruent w/ the orbits circumcircle), and its incircle (green). The latter's intouchpoints happen to be on the billiard itself. Also shown is the circumbilliard for T'. We have found this to be an axis-aligned ellipse (a 2x scaled version of the billiard) whose center (its mittenpunkt M') moves along an ellipse -- M' is the Gergonne Point X(7). Also shown are the Feuerbach point of the orbit (F) and that of T' (F bar).

https://dan-reznik.github.io/Elliptical-Billiards-Triangular-Orbits/
\subsection{The $X_{1}$- and $X_{2}$-centered circumellipses I}
\label{vid:AQ2AITmMs-g}
\noindent N=3, 1m37s (6/2019). 
\begin{center}\includegraphics[width=.5\textwidth]{pics/AQ2AITmMs-g.jpg} \\ 
\href{https://youtu.be/AQ2AITmMs-g}{\url{youtu.be/AQ2AITmMs-g}}\end{center}
% Shown is the family of N=3 orbits on an a/b=1.5 elliptic billiard (the billiard is the X(9)-centered circumellipse of the orbit family). Also shown are the X(1)- and X(2)-centered circumellipses, call them C1 and C2. The latter is also known as the Steiner Circumellipse (minimal area). The following properties were noticed:

a) C1's axes are aligned with the billiards (horizontal and vertical int he example). However, C2's can become slanted.
b) both C1 and C2 intersect the billiard at the orbit's vertices and that's obvious. However, C1's additional intersection occurs exactly at X(100) (shown as Fbar), the Feuerbach point of the anticomplementary triangle, and for this reason it coincides with the orbit's circumcircle (congruent with the nine-point circle of the anticomplementary triangle, used to obtain X(100)).
c) C2 has a 4th, separate, still not understood, intersection with the billiard.

https://dan-reznik.github.io/Elliptical-Billiards-Triangular-Orbits/
\subsection{Peter Moses' Points on the $X_{9}$-centered circumellipse I}
\label{vid:JdcJt5PExsw}
\noindent N=3, 3m13s (7/2019). 
\begin{center}\includegraphics[width=.5\textwidth]{pics/JdcJt5PExsw.jpg} \\ 
\href{https://youtu.be/JdcJt5PExsw}{\url{youtu.be/JdcJt5PExsw}}\end{center}
% Let E be an elliptic billiard (on the video a/b=1.5).

On July 1st, 2019, Prof. Clark Kimberling also posted on ETC [1] results by one important contributor to ETC, Peter J.C. Moses, relating several new properties of E, including a list of several Kimberling centers that lie on E's boundary, specifically: 

X(190), X(651), X(655), X(658), X(660), X(662), X(673),
X(771), X(799), X(823), X(897), X(1156), X(1492),
X(1821), X(2349), X(2580), X(2581), X(3257), X(4598),
X(4599), X(4604), X(4606), X(4607), X(8052), X(20332),
X(23707), X(24624), X(27834), X(32680)

(we are not including X(100) and X(88) as the former was already on ETC and the latter had been known to us for a few months [3]).

The video shows the motion of all listed centers on the boundary of an a/b=1.5 ellipse, driven by a smooth CCW rotation of 3-periodic orbit (a triangle), which we have proven must have a stationary X(9) [2,3]. We have also created a Wolfram Mathematica interactive applet [4] of this experiment.

Note: colors represent whether X(i) is:

black: i in (1,1000)
red: i in (1001,3000)
blue: i greater than 3000

[1] faculty.evansville.edu/ck6/encyclopedia/ETC.html
[2] www.youtube.com/watch?v=AoCWcza95OA
[3] dan-reznik.github.io/Elliptical-Billiards-Triangular-Orbits/
[4] www.wolframcloud.com/objects/user-abf31092-d7c1-4e49-8701-dc65d547b021/peter%20moses%20points%20on%20X9-centered%20circumellipse

More Info: https://dan-reznik.github.io/Elliptical-Billiards-Triangular-Orbits/
\subsection{Peter Moses' Points on the $X_{9}$-centered circumellipse II}
\label{vid:86TZzDuRNN0}
\noindent N=3, 12m49s (7/2019). 
\begin{center}\includegraphics[width=.5\textwidth]{pics/86TZzDuRNN0.jpg} \\ 
\href{https://youtu.be/86TZzDuRNN0}{\url{youtu.be/86TZzDuRNN0}}\end{center}
% The simulation shows the family of triangular orbits (black rotating triangle) in an a/b=1.5 elliptic billiard E. Notice the orbits' Mittenpunkt X(9) is stationary at the billiard's center [1,3].

I) Moses' Points

Also shown are 29 points discovered by Peter J.C. Moses which lie on the boundary of E, reported on ETC [2] on July, 1st 2019, to be sure:

X(i), for i = 190, 651, 655, 658, 660, 662, 673, 771, 799, 823, 897, 1156, 1492, 1821, 2349, 2580, 2581, 3257, 4598, 4599, 4604, 4606, 4607, 8052, 20332, 23707, 24624, 27834, 32680.

X(100) and X(88), also shown, have been known to lie on E both from ETC [2] and experimentation by the authors [3]. ETC also reports X(100), X(1), and X(88) lie on a single line (shown dashed green).

II) Main Result

Also shown is E's anticomplementary triangle T' (in blue) as well as its incircle, centered on X(8), the Nagel Point of T, the anticomplement of X(1), the incenter of T. The contact triangle T'' of T', shown green, is perspective with T' via X(144), which we call the "Darboux Point".

The amazing property is that for any T, the vertices of T'' will be on E. Notice none of the other triangular points drawn on the boundary track the vertices of T''. To interact with this geometry please go to [4].

III) The Darboux Axis

Also shown is the fact that points Xi = 7,142,2,9,144 are collinear (dashed purple line). We call this axis the "Darboux Axis". Each point has a very interesting property in connection with the geometry shown, namely:

a) X(9), the Mittenpunkt, is stationary for all orbits and is congruent with the center of the billiard
b) X(1), the Barycenter is the point about which the orbit is reflected (and scaled by 2) to generate the anticomplementary triangle.
c) X(144), the "Darboux" point, is the perspector of the orbit's anticomplementary triangle T' and its contact triangle T'', as well as the intouch triangle (not shown).
d) X(7), the Gergonne Point, is the anticomplement of X(9), and therefore the (moving) Mittenpunkt of T', i.e., it will be center of its circumbilliard.
e) X(142), is the complement (reflect on X(2) and divide by 2) of X(9) and will therefore be the (moving) Mittenpunkt of the orbit's medial triangle (not shown), i.e., the center of its circumbilliard [5].

The following facts are true about the above points:

- X(9) is the midpoint of X(7) X(144)
- X(142) is the midpoint between X(7) and X(9)
- X(9) to X(2) is 1/3 of X(9) to X(7).
- X(142) is the mdpoint of X(9) and X(7)

[1] https://www.youtube.com/watch?v=tMrBqfRBYik
[2] https://faculty.evansville.edu/ck6/encyclopedia/ETC.html
[3] https://dan-reznik.github.io/Elliptical-Billiards-Triangular-Orbits/
[4] https://www.wolframcloud.com/objects/user-abf31092-d7c1-4e49-8701-dc65d547b021/peter%20moses%20points%20on%20X9-centered%20circumellipse
[5] https://www.youtube.com/watch?v=gwfx6LDJnsE

More Info: https://dan-reznik.github.io/Elliptical-Billiards-Triangular-Orbits/
\subsection{The $X_{100}$-centered Excentral Jerabek Hyperbola}
\label{vid:uS0V1YjmEyY}
\noindent N=3, 6m25s (7/2019). 
\begin{center}\includegraphics[width=.5\textwidth]{pics/uS0V1YjmEyY.jpg} \\ 
\href{https://youtu.be/uS0V1YjmEyY}{\url{youtu.be/uS0V1YjmEyY}}\end{center}
% Shown is the family of triangular orbits for an a/b=1.25 elliptic billiard. If you demand a hyperbola H pass through the Mittenpunkt X(9), the Incenter X(1), and the three excenters (shown green), H's center will fall squarely on X(100), the anticomplement of the Feuerbach point, which is always on the billiard. Furthermore, the asymptotes of H will be parallel to the billiard's.

Per Prof Igor Minevich, this hyperbola is the Jerabek Hyperbola of the Excentral Triangle.

More Info: https://dan-reznik.github.io/Elliptical-Billiards-Triangular-Orbits/
\subsection{The Feuerbach and Excentral Hyperbolas I}
\label{vid:T5vXNsRcHZg}
\noindent N=3, 6m25s (7/2019). 
\begin{center}\includegraphics[width=.5\textwidth]{pics/T5vXNsRcHZg.jpg} \\ 
\href{https://youtu.be/T5vXNsRcHZg}{\url{youtu.be/T5vXNsRcHZg}}\end{center}
% \input{descr/107_T5vXNsRcHZg}
\subsection{The Jerabek Hyperbola and Circumbilliard of the Excentral Triangle}
\label{vid:7Q1TCbW2jFM}
\noindent N=3, 6m25s (7/2019). 
\begin{center}\includegraphics[width=.5\textwidth]{pics/7Q1TCbW2jFM.jpg} \\ 
\href{https://youtu.be/7Q1TCbW2jFM}{\url{youtu.be/7Q1TCbW2jFM}}\end{center}
% An a/b=1.5 elliptic billiard is shown as well as its family of N=3 (triangular) orbits. Also drawn (green) are the orbits' excentral triangles [1], and the latter's Jerabek Hyperbola [1], a circumhyperbola which passes through the excentral's vertices, its circumcenter, orthocenter, and symmedian point. In terms of the orbit triangle, these correspond, respectively, to: the excenters, bevan point X(40), incenter X(1), and mittenpunk X(9). The *center* of the Jerabek Hyperbola is X(125), which is also X(100) of its orthic (since the excentral is acute) [3], and following a remark by Prof Igor Minevich, Department of Mathematics, Rose-Hulman Institute of Technology.

Also shown is the excentral's circumellipse E' to which it is a billiard orbit, i.e., it's circumbilliard, whose center is congruent with its Mittenpunkt X(168) [4], shown in red. Notice the axes of E' are not parallel to those of the billiard.

[1] http://mathworld.wolfram.com/ExcentralTriangle.html
[2] http://mathworld.wolfram.com/JerabekHyperbola.html
[3] https://faculty.evansville.edu/ck6/encyclopedia/ETC.html
[4] https://dan-reznik.github.io/Elliptical-Billiards-Triangular-Orbits/

More Info: https://dan-reznik.github.io/Elliptical-Billiards-Triangular-Orbits/
\subsection{Invariants of the Steiner Circum and Inconics}
\label{vid:YQpX1eZ6O0I}
\noindent N=3, 4m49s (7/2019). 
\begin{center}\includegraphics[width=.5\textwidth]{pics/YQpX1eZ6O0I.jpg} \\ 
\href{https://youtu.be/YQpX1eZ6O0I}{\url{youtu.be/YQpX1eZ6O0I}}\end{center}
% An a/b=1.5 elliptic billiard is shown as well as the family of N=3 (triangular) orbits.

Also shown are the Steiner Circumellipse, centered on the barycenter X2 and intersecting the billiard at X190.

Also shown is the Steiner Inellipse, tangent to the billiard at its medians.

More Info:  https://dan-reznik.github.io/Elliptical-Billiards-Triangular-Orbits/

For the circumellipse, the relationship between the two axes (L1 and L2) is a fixed linear function:

L1 = c0 + c1 L2

For the Inellipse (with axes lengths L1' and L2') the relationship is L1' = c0/2 + c1 L2', where c0 and c1 refer to the same quantities used for the Circumellipse.

Notice that unlike the X(1)-centered circumellipse, whose axes always remain parallel to the billiard, both Steiner conics have axes which in general are not aligned with the billiard.
\subsection{Invariants of the $X_{1}$-centered circumellipse}
\label{vid:82gYh_3hIe4}
\noindent N=3, 4m49s (7/2019). 
\begin{center}\includegraphics[width=.5\textwidth]{pics/82gYh_3hIe4.jpg} \\ 
\href{https://youtu.be/82gYh_3hIe4}{\url{youtu.be/82gYh\_3hIe4}}\end{center}
% An a/b=1.5 elliptic billiard is shown as well as the family of N=3 (triangular) orbits. Also shown is the orbits' circumcircle (purple) and E1, the incenter-centered circumellipse (green). These two intersect at X100, the anticomplement of the Feuerbach point.

E1 has two very interesting properties: its axes are parallel to the billiards', and the ratio of its axis is constant. Note however neither of the axes have constant lengths nor is their product constant.

For the Steiner Circumellipse the relationship between the two axes (L1 and L2) remains fixed but via a linear equation: L1 = c0 + c1 L2. For the Steiner Inellipse (with axes lengths L1' and L2') the relationship is L1' = c0/2 + c1 L2', where c0 and c1 refer to the same quantities used for the Circumellipse.

More Info: https://dan-reznik.github.io/Elliptical-Billiards-Triangular-Orbits/
\subsection{Locus of Intersection of $X_{1}$- and $X_{2}$-centered circumellipses}
\label{vid:PTkpvvsjqNc}
\noindent N=3, 6m24s (7/2019). 
\begin{center}\includegraphics[width=.5\textwidth]{pics/PTkpvvsjqNc.jpg} \\ 
\href{https://youtu.be/PTkpvvsjqNc}{\url{youtu.be/PTkpvvsjqNc}}\end{center}
% Two elliptic billiards are shown (top: a/b=1.5, bottom: a/b=2) as well as their family of N=3 (triangular) orbits. Also shown are E1 and E2: the incenter (green) and barycenter (light brown) circumellipses, respectively. These meet at the three orbit vertices as well as on a 4th point interior to the billiard (red dot). The locus of this intersection is shown as a red curve for both cases. We make the following observations:

1) The axes of E1 remain parallel to the billiard's
2) The ratio of the axes of E1 is constant and equal to:

(-a^2+2*b^2+2*delta)/((2*a^2-1+2*delta)*a^4)

where delta = sqrt(a^4-a^2+1)

It turns out the line X(75) to X(77) pass through the point of intersection.

3) the axes of E2 are in general not parallel to the billiard's
4) the ratio of axes of E2 is: length1 = c0 + c1 length2, with c0 and c1 remaining constant for the whole family.
5) the locus of the intersection of E1 and E2 is interior to the billiard

More Info:  https://dan-reznik.github.io/Elliptical-Billiards-Triangular-Orbits/
\subsection{The $X_{1}$- and $X_{2}$-centered circumellipses I}
\label{vid:P_Io7HsWGnQ}
\noindent N=3, 6m25s (1/2020). 
\begin{center}\includegraphics[width=.5\textwidth]{pics/P_Io7HsWGnQ.jpg} \\ 
\href{https://youtu.be/P_Io7HsWGnQ}{\url{youtu.be/P\_Io7HsWGnQ}}\end{center}
% An a/b=1.5 Elliptic Billiard (black) is shown. The family of triangular orbits (blue) have a stationary Mittenpunkt X(9) at the EB center [1]. For each orbit we draw E1 and E2, the Incenter X(1)- and Barycenter X(2)-centered Circumellipses. The latter is known as the Steiner Circumellipse, least-area amongst all Circumellipses. It is known that X(100) (resp. X(190)) lie on the X(9)-centered Circumellipse (our Billiard). Additionally, we have detected / proven [2] the following invariants over the N=3 family:

- E1's axes are parallel to the EB
- E1's axes ratio is constant
- E2's axes are only parallel when orbit triangle is isosceles (a vertex is on one of the EB vertices).
- P12, the fourth intersection of E1 with E2 is colinear with X(75), X(77) (not shown). Viewer proofs are welcome! 

[1] Reznik et al., "Can the Elliptic Billiard Still Surprise Us?", The Mathematical Intelligencer, 2020. ArXiv: https://arxiv.org/abs/1911.01515
[2] Reznik et al., "New Properties of Triangular Orbits in Elliptic Billiards", 2020, in preparation.
\subsection{The Feuerbach and Excentral Hyperbolas II}
\label{vid:Pz4tUijYZCA}
\noindent N=3, 4m49s (1/2020). 
\begin{center}\includegraphics[width=.5\textwidth]{pics/Pz4tUijYZCA.jpg} \\ 
\href{https://youtu.be/Pz4tUijYZCA}{\url{youtu.be/Pz4tUijYZCA}}\end{center}
% An a/b Elliptic Billiard is shown (black) as well as its family of 3-periodic orbits (blue). Also drawn are two well known "right" (orthogonal axes) circumhyperbolas (pass through all vertices):

1) The Feuerbach Circumhyperbolas (brown), centered on X(11)
2) the Jerabek Circumhyperbola (purple) of the Excentral Triangle (green), centered on X(125) of the Excentral, i.e., X(100) of the orbit.

A rectangular circumhyperbola always passes through the orthocenter X(4) and has center on the nine-point circle [1].

- Therefore the Feuerbach will pass through X(4) and the Excentral Jerabek through X(4) of the Excentral, i.e., X(1) of the orbit.
- Both pass through X(9) and X(1), intimately connected with the Billiard.

Aditionally, we have found the following properties:

a) the asymptotes of both Hyperbolas are parallel to the Billiard axes.
b) the ratio of focal lengths is invariant!
c) the ratio of Hessian determinants is invariant!
d) The Feuerbach intersects the Billiard at X(1156).

[1] http://mathworld.wolfram.com/Circumhyperbola.html
\subsection{The Yff Parabola, Contact Triangle \& Loci of Vertex and Axis Foot I}
\label{vid:BaQmt3hHVtw}
\noindent N=3, 8m1s (2/2020). 
\begin{center}\includegraphics[width=.5\textwidth]{pics/BaQmt3hHVtw.jpg} \\ 
\href{https://youtu.be/BaQmt3hHVtw}{\url{youtu.be/BaQmt3hHVtw}}\end{center}
% \input{descr/119_BaQmt3hHVtw}
\subsection{The Yff Parabola, Contact Triangle \& Loci of Vertex and Axis Foot II}
\label{vid:Sm9g5lqhZbk}
\noindent N=3, 8m1s (2/2020). 
\begin{center}\includegraphics[width=.5\textwidth]{pics/Sm9g5lqhZbk.jpg} \\ 
\href{https://youtu.be/Sm9g5lqhZbk}{\url{youtu.be/Sm9g5lqhZbk}}\end{center}
% An a/b=1.5 Elliptic Billiard (EB) is shown (blue) as well as the family of 3-periodic orbits (blue).

Shown (green) is the Yff Parabola [1] whose focus is X101, a point always exterior to the EB. This parabola is an inconic since it touches the reference triangle (the orbit) one each side (or an extension thereof).

The touchpoints define de Yff Contact Triangle [2] (red), whose area is always twice that of the reference triangle (it is the Cevian triangle of a point -- X190 -- on the Steiner circumellipse [2])! Interesting, X190 always on the Elliptic Billiard (the X9-centered circumellipse).

Another way to say this is that X190 is the perspector (or Brianchon Point) of the inconic, i.e., lines drawn from each orbit vertex to the contact points meet at X190. 

Also shown is the parabola's directrix (black) known to pass through the Mittenpunkt X9 and the Orthocenter X4. Recall the former is stationary at the Billiard center.

An interesting observation (proofs accepted!) is that when X4 is on the Billiard (this happens when the orbit is a right-triangle), X101 will be on one side of the Yff contact triangle! At this position, the axis of the Yff parabola (dashed black) passes through the Mittenpunkt X9.

The parabola vertex V is at X(3234) [3]. Shown is its astroid-like locus (orange) over the 3-periodic family. What curve is that?

Also shown is the four-petal (purple) locus of "F", where the parabola axis meets the directrix. What cureve is that?

[1] http://mathworld.wolfram.com/YffParabola.html

[2] http://mathworld.wolfram.com/YffContactTriangle.html

[3] Peter Moses (via Clark Kimberling), private communication, Feb 2020.
\subsection{Circumbilliards of Triangles Derived from 3-Periodics}
\label{vid:Og7xLgkrLqw}
\noindent N=3, 8m1s (3/2020). 
\begin{center}\includegraphics[width=.5\textwidth]{pics/Og7xLgkrLqw.jpg} \\ 
\href{https://youtu.be/Og7xLgkrLqw}{\url{youtu.be/Og7xLgkrLqw}}\end{center}
% \input{descr/113_Og7xLgkrLqw}
\subsection{Orthic Circumbilliard \& Locus of Its Mittenpunkt}
\label{vid:5KL8st2vIb0}
\noindent N=3, 1m1s (3/2020). 
\begin{center}\includegraphics[width=.5\textwidth]{pics/5KL8st2vIb0.jpg} \\ 
\href{https://youtu.be/5KL8st2vIb0}{\url{youtu.be/5KL8st2vIb0}}\end{center}
% \input{descr/118_5KL8st2vIb0}
\subsection{Feuerbach and Excentral Jerabek Circumhyperbolas: Invariant Focal Length Ratio II}
\label{vid:QZN82WDTGGY}
\noindent N=3, 3m1s (3/2020). 
\begin{center}\includegraphics[width=.5\textwidth]{pics/QZN82WDTGGY.jpg} \\ 
\href{https://youtu.be/QZN82WDTGGY}{\url{youtu.be/QZN82WDTGGY}}\end{center}
% \input{descr/117_QZN82WDTGGY}
\subsection{Feuerbach and Excentral Jerabek Circumhyperbolas: Invariant Focal Length Ratio I}
\label{vid:ewioM6-nCpY}
\noindent N=3, 4m1s (4/2020). 
\begin{center}\includegraphics[width=.5\textwidth]{pics/ewioM6-nCpY.jpg} \\ 
\href{https://youtu.be/ewioM6-nCpY}{\url{youtu.be/ewioM6-nCpY}}\end{center}
% An a/b=1.5 Elliptic Billiard (EB) is shown as well as the 1d-family of 3-periodics (blue). These envelop a confocal caustic (brown). The Excentral Triangle is shown (solid green).

The Feuerbach Circumyperbola is rectangular and passes through X1, X4 and X9 (among many other Triangle Centers) and is centered on X11 [1]. Let F be said hyperbola computed dynamically over the 3-periodic family (blue).  Since over the family X9 is stationary at the EB center [2],  F passes there too. X11 is on the Caustic. Also shown is X1156, the intersection of F and the EB [3]. Surprisingly, the asymptotes of F are always parallel to the EB axes.

The Jerabek Circumhyperbola is rectangular and passes through X1, X3, X4, X6 (among others). Its center is X125 [4]. Let Jexc be the Jerabek Hyperbola of the Excentral Triangle (green). X1,X3,X4,X6,X125 of the Excentral correspond to X4,X40,X1,X9,X100 of the reference triangle, in our case, 3-periodics. Notice its center, X100, is always on the EB. Lile F, the asymptotes of Jexc are parallel to the EB axes.

Let F' (resp. Jexc') be a copy of F (resp. Jexc) translated by X11 (resp. X100), so its center coincides with the EB's. These are shown dashed blue (resp. dashed green). The focal axes and foci (f and f') of both F' and Jexc' are shown as a diagonal through the EB center. Notice X1156=-X100 as it is the reflection of X100 on X9 (at the origin). Therefore Jexc' intersects the EB at X1156. Also, F' and J' are given by x y = k, and x y = k', respectively.

Notice the focal length f of both F and F' is the same, as is f' of Jexc and Jexc'. The foremost property illustrated is that over the 3-periodic family the the ratio f'/f is invariant over the family [3] (see top area of video).

Additionally, when f,f' are maximal, the following has been detected experimentally [3]:

1) Jexc' is tangent to E9 at +/- X100.
2) F' is tangent to the Caustic at +/- X11.
3) k' = a/b.

Soundtrack: Edvard Grieg (Peer Gynt): Suite No.1, Op. 46, Morning Mood.

References:

[1] Feuerbach Hyperbola: https://mathworld.wolfram.com/FeuerbachHyperbola.html
[2] Reznik D., Garcia R., Koiller J., "Can the Elliptic Billiard still surprise us?", Math Intelligencer, vol 42, 2019, https://rdcu.be/b2cg1
[3] Reznik D., Garcia R., Koiller J., "Invariants of Circumconics of 3-Periodics in the Elliptic Billiard", 2020. in preparation
[4] Jerabek Hyperbola: https://mathworld.wolfram.com/JerabekHyperbola.html
\subsection{Excentral MacBeath Inconic: Invariant Aspect Ratio}
\label{vid:IxrIkW5tj20}
\noindent N=3, 4m1s (4/2020). 
\begin{center}\includegraphics[width=.5\textwidth]{pics/IxrIkW5tj20.jpg} \\ 
\href{https://youtu.be/IxrIkW5tj20}{\url{youtu.be/IxrIkW5tj20}}\end{center}
% An a/b=1.5 (resp. 2.0) elliptic billiard is shown on the left (resp. right). Also shown is the family of 3-periodics (blue). The Excentral Triangles are shown (green) whose vertices are the Excenters. We have shown their locus is an ellipse congruent with the the MacBeath Circumellipse [2] and similar to the Incenter elliptic locus [1].

Also shown is the MacBeath Inellipse [3] of the Excentral Triangle (red). Its center and foci are X5, X4, and X3 respectively, which in terms of the reference triangle are X3, X1, and X40.

Suprisingly, the ratio of this magical Inellipse's semiaxes η'/η is invariant over the 3-periodic family.

Also shown are the following loci:

(1) Orange: Inellipse contact points (red dots) with the Excentral Triangle
(2) Pink: X1742, the Brianchon Point of the Inellipse. This is X(264) of the Excentral Triangle.

Soundtrack: Claude Debussy, "Première Arabesque"

References:

[1] Garcia R., Reznik, D., Koiller J., "Loci of 3-periodics in an Elliptic Billiard: why so many ellipses?", arxiv: https://arxiv.org/abs/2001.08041
[2] https://mathworld.wolfram.com/MacBeathCircumconic.html
[3] https://mathworld.wolfram.com/MacBeathInconic.html
\subsection{Excentral $X_{3}$-Centered \& MacBeath Inconics: Invariant Aspect Ratio}
\label{vid:CHbrZvx1I8w}
\noindent N=NA, 4m1s (4/2020). 
\begin{center}\includegraphics[width=.5\textwidth]{pics/CHbrZvx1I8w.jpg} \\ 
\href{https://youtu.be/CHbrZvx1I8w}{\url{youtu.be/CHbrZvx1I8w}}\end{center}
% \input{descr/115_CHbrZvx1I8w}
\subsection{$X_{3}$-Centered Excentral Inconic: Invariant Aspect Ratio}
\label{vid:ojxzOS1Sjjo}
\noindent N=3, 4m1s (4/2020). 
\begin{center}\includegraphics[width=.5\textwidth]{pics/ojxzOS1Sjjo.jpg} \\ 
\href{https://youtu.be/ojxzOS1Sjjo}{\url{youtu.be/ojxzOS1Sjjo}}\end{center}
% \input{descr/121_ojxzOS1Sjjo}

\section{Inversions of Pivoting Triangles (9)}

\subsection{Isocurves of distance between triangle center in reference triangle vs in inversive (or polar) image II}
\label{vid:BHG4yMSgi8k}
\noindent N=3, 26m29s (8/2021). 
\begin{center}\includegraphics[width=.5\textwidth]{pics/BHG4yMSgi8k.jpg} \\ 
\href{https://youtu.be/BHG4yMSgi8k}{\url{youtu.be/BHG4yMSgi8k}}\end{center}
% Let X be a triangle center of a triangle T. Let T' (resp T'')  be the inversive  (resp. polar) image of T wrt to a unit circle centered on a point P=(x,y). Let X' and (resp. X'') be triangle center X(k) of T' (resp. T'').

Every frame of the video shows isocurves of |X'-X| (left) and |X''-X| (right) for a particular k in the 1 to 200 range.
\subsection{Circular isocurves of distance between Symmedian of a reference triangle and its inversive image}
\label{vid:uoAocIk8LYI}
\noindent N=3, 2m41s (8/2021). 
\begin{center}\includegraphics[width=.5\textwidth]{pics/uoAocIk8LYI.jpg} \\ 
\href{https://youtu.be/uoAocIk8LYI}{\url{youtu.be/uoAocIk8LYI}}\end{center}
% Let T' be a triangle (dashed red) whose vertices are inversions of those of a reference triangle T (solid red) with respect to a circle C (solid green) centered on O.

Let X6 (resp. X6') be the symmedian point of T (resp. T').

The fact that O,X6,X6' are collinear is known.

The video shows that if O follows a circle (dashed green) centered on X6, then X6' follows another concentric circle (dashed red).
\subsection{Isocurves of distance between triangle center in reference triangle vs in inversive (or polar) image I}
\label{vid:SvG8ogSzZh4}
\noindent N=3, 26m17s (8/2021). 
\begin{center}\includegraphics[width=.5\textwidth]{pics/SvG8ogSzZh4.jpg} \\ 
\href{https://youtu.be/SvG8ogSzZh4}{\url{youtu.be/SvG8ogSzZh4}}\end{center}
% Let X be a triangle center of a triangle T. Let T' (resp T'')  be the inversive  (resp. polar) image of T wrt to a unit circle centered on a point P=(x,y). Let X' and (resp. X'') be triangle center X(k) of T' (resp. T'').

Every frame of the video shows isocurves of |X'-X| (left) and |X''-X| (right) for a particular k in the 1 to 200 range.

Remarkably, the isoscurves of distance from X(6) in the reference to the X(6) in the inversive image are circles... Many other triangle centers have this property too.
\subsection{Stationary Symmedian Point of the Inversive Image of an $X_{6}$-Pivoting Triangle}
\label{vid:GLRZSbzzP1U}
\noindent N=3, 2m1s (8/2021). 
\begin{center}\includegraphics[width=.5\textwidth]{pics/GLRZSbzzP1U.jpg} \\ 
\href{https://youtu.be/GLRZSbzzP1U}{\url{youtu.be/GLRZSbzzP1U}}\end{center}
% \input{descr/130_GLRZSbzzP1U}
\subsection{Inversive Image of Pivoting Triangle, Part I: Stationary Symmedian Point $X_{6}$ of Inversive}
\label{vid:mI4BcoUnb6U}
\noindent N=3, 9m44s (8/2021). 
\begin{center}\includegraphics[width=.5\textwidth]{pics/mI4BcoUnb6U.jpg} \\ 
\href{https://youtu.be/mI4BcoUnb6U}{\url{youtu.be/mI4BcoUnb6U}}\end{center}
% Consider a family of triangles T=ABC (blue) rigidly rotating about a point P in the plane. Now consider a family T' (red) with vertices at the inversions of A,B,C with respect to an inversion circle (dashed green) centered at an arbitrary point O. The video illustrates a few observations. Let X(k) [resp. X'(k)] denote the kth triangle center of T [resp. T'].

1) For any choice of O, if P=X(6) of T, then (i) O,X(6),X'(6) are collinear, and (ii) X'(6) is stationary
.2) For any choice of P,O, the locus of X'(6), i.e., the X(6) of T', is always an *ellipse* with minor axis along OP.

Note: P need not be a triangle center of T, it can be any point in the plane.

Related phenomena happen with X'(k), k=3,15,16,61,62,187, to be described in subsequent videos.
\subsection{Inversive Image of Pivoting Triangle, Part II: Conic Loci of $X_{3}$ and $X_{6}$ of Inversive}
\label{vid:qVCjQfXkK_k}
\noindent N=3, 25m9s (8/2021). 
\begin{center}\includegraphics[width=.5\textwidth]{pics/qVCjQfXkK_k.jpg} \\ 
\href{https://youtu.be/qVCjQfXkK_k}{\url{youtu.be/qVCjQfXkK\_k}}\end{center}
% Consider a family of triangles T=ABC (blue) rigidly rotating about a point P in the plane. Now consider a family T' (red) with vertices at the inversions of A,B,C with respect to an inversion circle (dashed green) centered at an arbitrary point O. The video illustrates a few observations. Let X(k) [resp. X'(k)] denote the kth triangle center of T [resp. T'].

1) Trivial: for any choice of O, if P=X(3) of T, then (i) O,X(3),X'(3) are collinear, and (ii) X'(3) is stationary.
2) For any choice of P, the locus of X'(3) is always a conic with major axis along OP. The *type* (ellipse, parabola, hyperbola) of conic depends on the choice of O.
3) For any choice of O, if P=X(6) of T, then (i) O,X(6),X'(6) are collinear, and (ii) X'(6) is stationary
.4) For any choice of P,O, the locus of X'(6), i.e., the X(6) of T', is always an ellipse with minor axis along OP.

Related phenomena happen with X'(k), k=15,16,61,62,187, to be described in a new video.
\subsection{Inversive Image of Pivoting Triangle, Part III: Piecewise Point-Circular Loci of $X'_{15}$ and $X'_{16}$}
\label{vid:iFEHMSELf7U}
\noindent N=3, 31m44s (8/2021). 
\begin{center}\includegraphics[width=.5\textwidth]{pics/iFEHMSELf7U.jpg} \\ 
\href{https://youtu.be/iFEHMSELf7U}{\url{youtu.be/iFEHMSELf7U}}\end{center}
% \input{descr/125_iFEHMSELf7U}
\subsection{Inversive Image of Pivoting Triangle, Part IV: Piecewise Point-Elliptic Loci of $X'_{61}$ and $X'_{62}$}
\label{vid:HPp5I_kCf0g}
\noindent N=3, 9m41s (8/2021). 
\begin{center}\includegraphics[width=.5\textwidth]{pics/HPp5I_kCf0g.jpg} \\ 
\href{https://youtu.be/HPp5I_kCf0g}{\url{youtu.be/HPp5I\_kCf0g}}\end{center}
% Consider a family of triangles T=ABC (blue) rigidly rotating about X(61), the isogonal conjugate of X(17). Consider a family T' (red) with vertices at the inversions of A,B,C with respect to an inversion circle (dashed green) centered at an arbitrary point O. The video illustrates phenomena pertaining to the locus of X'(61) of T', namely:

1) When the moving circumcircle C of T contains (resp. does not contain) O, X'(15) moves along an arc of an ellipse (resp. is stationary). Let R be the circumradius of T. There are 3 possibilities:
1.1) If |O-X(61)|  greater than R, O always exterior to C, X'(61) is a stationary point over the entire motion
1.2) If |O-X(61)|  less than R: (i) while O is exterior to C: X'(61) is stationary, (ii) while O is interior, X'(61) moves along a disjoint elliptic arc.
1.3) O is sufficiently close to X(61) such that O is always interior to C, X'(61) moves along a contiguous ellipse.

2) Similar phenomena happen when T is pivoting about its 2nd isodynamic point X(62), the isogonal conjugate of X(18), and one is tracking the X'(62) of T'.

Note: X(61) and X(62) are inversions of each other wrt Brocard Circle.
\subsection{Inversive Image of Pivoting Triangle, Part V: Loci of the Isodynamic Midpoint $X'_{187}$}
\label{vid:qZFd0SwJ8xg}
\noindent N=3, 16m43s (8/2021). 
\begin{center}\includegraphics[width=.5\textwidth]{pics/qZFd0SwJ8xg.jpg} \\ 
\href{https://youtu.be/qZFd0SwJ8xg}{\url{youtu.be/qZFd0SwJ8xg}}\end{center}
% \input{descr/127_qZFd0SwJ8xg}

\section{Inversive Poncelet (20)}

\subsection{Elliptic Billiard N-Periodics: invariant sum of inverse focal distances \& inversive Pascal Limaçon}
\label{vid:FmWq1YiAs5o}
\noindent N=5, 2m26s (10/2020). 
\begin{center}\includegraphics[width=.5\textwidth]{pics/FmWq1YiAs5o.jpg} \\ 
\href{https://youtu.be/FmWq1YiAs5o}{\url{youtu.be/FmWq1YiAs5o}}\end{center}
% The observations below were co-developed with P. Roitman, A. Akopyan, S. Tabachnikov, H. Stachel, R. Garcia, and J. Koiller.

Consider the family of N-periodics (blue) in an elliptic billiard (black). Consider the so-called "focal spokes", i.e., segments connecting each vertex to one of the foci (dashed blue). Let r1(i) denote their lenghts. It turns out the sum of 1/r1(i), i=1,...,N is invariant over the family! 

The inversion of an ellipse with respect to a focus is Pascal's Limaçon [1]. So the inverted focal spokes can be visualized inscribed in said Limaçon (orange).

By lateral symmetry, the sum of inverse focal distances to the other focus will also be invariant and of the same value. Invariance of both sums 1/r1(i) and 1/r2(i) implies that the sum of [r1(i)+r2(i)]/(r1(i) r2(i)] is invariant. Since r1(i)+r2(i) is constant (property of the ellipse), the sum of 1/[r1(i) r2(i)] is also invariant. 

A known relation is that the curvature k(i) of the ellipse at point P(i) is equal to a b (r1 r2)^(-3/2) [2], where a,b are the semiaxes. Therefore the sum of curvatures k(i)^(2/3) is also invariant. Prof. Hellmuth Stachel derived the following expression for the sum of k(i)^2/3:

sum of k(i)^(2/3) = L/(2 J (a b)^(4/3))

where L,J are the invariant perimeter and Joachimsthal's constant.

Note: the video shows an N=5 family in an a/b=golden ratio billiard, but the above invariants work for all N and aspect ratios.

Audio track: Valse Musette in homage to Blaise Pascal (1623–1662)

[1] R. Ferréol, "Limaçon de Pascal", Mathcurve, 2020. https://bit.ly/2GO9EF1
[2] J. Calvert, "Ellipse", Math. Index, 2005. http://mysite.du.edu/~jcalvert/math/ellipse.htm
\subsection{Inversive Elliptic Billiard N-Periodics are Circular Arcs Interscribed between two Pascal Limaçons}
\label{vid:bFsehskizls}
\noindent N=5, 2m29s (10/2020). 
\begin{center}\includegraphics[width=.5\textwidth]{pics/bFsehskizls.jpg} \\ 
\href{https://youtu.be/bFsehskizls}{\url{youtu.be/bFsehskizls}}\end{center}
% \input{descr/144_bFsehskizls}
\subsection{A Rose in the Elliptic Billiard: the Constant-Perimeter, Focus-Inversive Family}
\label{vid:wkstGKq5jOo}
\noindent N=5, 2m45s (10/2020). 
\begin{center}\includegraphics[width=.5\textwidth]{pics/wkstGKq5jOo.jpg} \\ 
\href{https://youtu.be/wkstGKq5jOo}{\url{youtu.be/wkstGKq5jOo}}\end{center}
% \input{descr/147_wkstGKq5jOo}
\subsection{Focus-Inversive N=3 Family in the Elliptic Billiard: Pascal Limaçon-Inscribed Billiard Triangles II}
\label{vid:LOJK5izTctI}
\noindent N=3, 1m48s (10/2020). 
\begin{center}\includegraphics[width=.5\textwidth]{pics/LOJK5izTctI.jpg} \\ 
\href{https://youtu.be/LOJK5izTctI}{\url{youtu.be/LOJK5izTctI}}\end{center}
% If you take the N=3 family of elliptic billiard trajectories T (blue triangles, X9 stationary at billiard center), and invert the vertices with respect a unit-radius circle (dashed black) centered on one of the foci, you get a new "focus-inversive" triangles T' (pink). 

The T' is inscribed a Pascal Limaçon (classical result, the focus-inversive image of the billiard ellipse).

Surprisingly,  the T', like their parents T, display invariant perimeter and invariant sum of cosines. Their mittenpunkt X9' describe a circular locus whereas their Gergonne point X7' is stationary!

Also drawn is the moving circumellipse (orange) of T' centered on X9', which we call a "circumbilliard". The lengths a',b' of its semiaxes' id also invariant. I.e., the T' are a billiard family with respect to this rigidly rotating family of ellipses.
\subsection{Focus-Inversive N=3 Family in the Elliptic Billiard: Pascal Limaçon-Inscribed Billiard Triangles I}
\label{vid:Y-j5eXqKGQE}
\noindent N=3, 1m44s (10/2020). 
\begin{center}\includegraphics[width=.5\textwidth]{pics/Y-j5eXqKGQE.jpg} \\ 
\href{https://youtu.be/Y-j5eXqKGQE}{\url{youtu.be/Y-j5eXqKGQE}}\end{center}
% If you take the N=3 family of elliptic billiard trajectories T (blue triangles, X9 stationary at billiard center. Let its vertices be denoted Pi. Perform an inversion of the Pi with respect a unit-radius circle (dashed black) centered on one of the foci (say left, f1) to get a new "focus-inversive" triangles T' (pink) w vertices Pi'. 

White the Pi are inscribed in an ellipse, the Pi' are inscribed in a Pascal Limaçon (classical result, the focus-inversive image of the billiard ellipse).

Let r_i (resp. r_i') be the distance |Pi-f1| (resp. |P_i'-f1). By construction r_i'=1/r_i. It turns out the sum of the r_i' is invariant! The r_i (resp. r_i') focal "spokes" are shown dashed (resp. solid) blue.

Also, the T', like its parent T, displays invariant perimeter and invariant sum of cosines. Their mittenpunkt X9' describe a circular locus whereas their Gergonne point X7' is stationary!

Also drawn is the moving circumellipse (orange) of T' centered on X9', which we call a "circumbilliard". The lengths a',b' of its semiaxes' id also invariant. I.e., the T' are a billiard family with respect to this rigidly rotating family of ellipses.
\subsection{Invariants of Inversive, Polar, and Dual Polygons derived from N-Periodics in the Elliptic Billiard}
\label{vid:qyAHOW32NXY}
\noindent N=5, 1m51s (10/2020). 
\begin{center}\includegraphics[width=.5\textwidth]{pics/qyAHOW32NXY.jpg} \\ 
\href{https://youtu.be/qyAHOW32NXY}{\url{youtu.be/qyAHOW32NXY}}\end{center}
% Research below done jointly w P. Roitman, J. Koiller, R. Garcia, and A. Akopyan.

Consider a family of N-periodics (blue, N=5 shown) polygons w vertices Pi in the elliptic Billiard (confocal pair with outer black ellipse and inner brown caustic). This family conserves perimeter L, joachimsthal's constant (not shown), and sum of cosines. Let di represent the length of "focal spokes" (dashed blue) connecting the left focus (f1) to each Pi. It turns out the sum of 1/di is conserved.

1) The "inversive" polygon (red) is obtained by inverting the Pi with respect to a unit-radius circle (dashed black) centered on a focus f1 (left in the video). Let its vertices be denoted Pi' and its (varying) area A'. This family conserves: sum of distances from f1 to its vertices (equivalent to sum of 1/di), sum of cosines, and perimeter. This family is inscribed in a Pascal Limaçon (not shown). 

2) The "polar" polygon (green) has sides which pass thru the Pi' and are perpendicular to Pi'-f1 (i.e., it is the antipedal polygon of the inversive wrt f1). Let Pi'' denote its vertices. This family is interscribed between two non-concentric circles (dashed green). It conserves its sum of cosines. Also invariant is the ratio of its area by A'.

3) The "dual" polygon (orange) is obtained by inverting the Pi'' with respect to a unit-radius circle centered on the "polar" incenter. This family is inscribed in a circle w center O and circumscribes an ellipse w one focus centered on O. It conserves the sum of sidelengths squared [1]. Also invariant is the ratio of its area by A'.

4) The "outer dual" polygon (pink) is obtained by inverting the Pi'' with respect to a unit-radius circle centered on its circumcenter. This family is inscribed in an ellipse and circumscribes a circle with center which coincides w one focus of said ellipse. No invariants have been identified for this family yet.

[1] A. Akopyan, Private Communication, Oct 12, 2020.
\subsection{Focus-Inversive Polygons' Equi-Area Pedal Polygons (wrt foci)}
\label{vid:0L2uMk2xyKk}
\noindent N=3, 3m23s (10/2020). 
\begin{center}\includegraphics[width=.5\textwidth]{pics/0L2uMk2xyKk.jpg} \\ 
\href{https://youtu.be/0L2uMk2xyKk}{\url{youtu.be/0L2uMk2xyKk}}\end{center}
% \input{descr/139_0L2uMk2xyKk}
\subsection{Invariant Inversive Perimeter (all N) and Area Product (odd N)}
\label{vid:bTkbdEPNUOY}
\noindent N=5, 3m23s (10/2020). 
\begin{center}\includegraphics[width=.5\textwidth]{pics/bTkbdEPNUOY.jpg} \\ 
\href{https://youtu.be/bTkbdEPNUOY}{\url{youtu.be/bTkbdEPNUOY}}\end{center}
% \input{descr/141_bTkbdEPNUOY}
\subsection{Invariant Area Ratio Between Focus-Inversive Polygons for all N}
\label{vid:eG4UCgMkKl8}
\noindent N=5, 2m25s (10/2020). 
\begin{center}\includegraphics[width=.5\textwidth]{pics/eG4UCgMkKl8.jpg} \\ 
\href{https://youtu.be/eG4UCgMkKl8}{\url{youtu.be/eG4UCgMkKl8}}\end{center}
% \input{descr/140_eG4UCgMkKl8}
\subsection{Centers of Inversive Arcs area a Bicentric Poncelet Family with Invariants}
\label{vid:mXkk_4RYrnU}
\noindent N=5, 3m24s (10/2020). 
\begin{center}\includegraphics[width=.5\textwidth]{pics/mXkk_4RYrnU.jpg} \\ 
\href{https://youtu.be/mXkk_4RYrnU}{\url{youtu.be/mXkk\_4RYrnU}}\end{center}
% \input{descr/131_mXkk_4RYrnU}
\subsection{Invariant Inversive perimeter and N=6 a/b=2 Null Antipedal Area I}
\label{vid:fOAES-CzjNI}
\noindent N=6, 3m20s (10/2020). 
\begin{center}\includegraphics[width=.5\textwidth]{pics/fOAES-CzjNI.jpg} \\ 
\href{https://youtu.be/fOAES-CzjNI}{\url{youtu.be/fOAES-CzjNI}}\end{center}
% An a/b=2 elliptic billiard is shown (black) as well as its 6-Periodic family (blue). Also shown is the family's inversive (pink) and antipedal polygons (green) wrt to the left focus. These two polygons are "duals" of each other [1].

We have shown elsewhere [2] that for all N, the perimeter of the inversive polygon is invariant.

The video shows that for this special choice of a/b and N, the signed area of the antipedal is dynamically zero.

For no other choice of N and billiard aspect ratio have we been able to (experimentally) zero the area of the antipedal over the N-periodic family.

[1] A. Akopyan and A. Zaslavski, "Geometry of Conics", American Mathematical Society, Providence, 2007.

[2] D. Reznik et al, "Elliptic Billiard: Invariant Perimeter of Focus-Inversive Polygon", 2020. https://youtu.be/wkstGKq5jOo
\subsection{Loci of Invariant Inversive perimeter and N=6 a/b=2 Null Antipedal Area}
\label{vid:HMhZW_kWLGw}
\noindent N=6, 3m21s (10/2020). 
\begin{center}\includegraphics[width=.5\textwidth]{pics/HMhZW_kWLGw.jpg} \\ 
\href{https://youtu.be/HMhZW_kWLGw}{\url{youtu.be/HMhZW\_kWLGw}}\end{center}
% \input{descr/145_HMhZW_kWLGw}
\subsection{Circles Galore I: Loci of Focus-Inversive 3-Periodics in the Elliptic Billiard (11 notable centers)}
\label{vid:tKB-50zW8F4}
\noindent N=3, 1m46s (11/2020). 
\begin{center}\includegraphics[width=.5\textwidth]{pics/tKB-50zW8F4.jpg} \\ 
\href{https://youtu.be/tKB-50zW8F4}{\url{youtu.be/tKB-50zW8F4}}\end{center}
% \input{descr/132_tKB-50zW8F4}
\subsection{Circles Galore II: 29 Loci of Focus-Inversive 3-Periodics in the Elliptic Billiard}
\label{vid:srjm23nQbMc}
\noindent N=3, 2m32s (11/2020). 
\begin{center}\includegraphics[width=.5\textwidth]{pics/srjm23nQbMc.jpg} \\ 
\href{https://youtu.be/srjm23nQbMc}{\url{youtu.be/srjm23nQbMc}}\end{center}
% The focus-inversive family (pink) is obtained by performing inversions of 3-periodic (blue) vertices in the elliptic billiard (black) with respect to a unit-circle (dashed black) centered on a focus. Suprisingly, the focus-inversive family has invariant perimeter and some of cosines. Furthermore, its Gergonne point X(7) is stationary.

The video shows the suprisingly circular loci of the focus-inversive X(k), k=1, 2, 3, 4, 5, 8, 9, 10, 11, 12, 20, 21, 35, 36, 40, 46, 55, 56, 57, 63, 65, 72, 78, 79, 80, 84, 90, 100. The plot label shows their radii.

What gives?
\subsection{Circles Galore III: Loci of Focus-Inversive 3-Periodics in the Elliptic Billiard (9 notable centers)}
\label{vid:OAD2hpCRgCI}
\noindent N=3, 3m22s (11/2020). 
\begin{center}\includegraphics[width=.5\textwidth]{pics/OAD2hpCRgCI.jpg} \\ 
\href{https://youtu.be/OAD2hpCRgCI}{\url{youtu.be/OAD2hpCRgCI}}\end{center}
% Shown is the family of 3-periodics (blue) in the elliptic billiard E (black). This family has invariant perimeter, sum of cosines, and the Mittenpunkt X(9), not shown, is stationary at the center of E.

Also shown is the derived family of "focus-inversive" triangles (pink), whose vertices are inversions of the 3-periodic ones wrt a unit circle (dashed black) centered on a focus of E (left in video).

That the focus-inversive family is inscribed in Pascal's Limaçon (dashed pink) is easy to show (inversion of ellipse wrt to its focus). Unexpectedly (and unproven) is the fact that both the perimeter and sum of cosines are invariant.

To boot, the Gergonne point X(7) is stationary.

The main theme of the animation is the following surprising observation:

the loci of X(k), k=1,2,3,4,5,8,9,10,11 of the inversives are all circles whose centers lie on the x axis (E's major axis).

In fact, between X(1) and X(100) the following centers of the focus-inversive family are circles [1]:

X(k), k=1, 2, 3, 4, 5, 8, 9, 10, 11, 12, 20, 21, 35, 36, 40, 46, 55, 56, 57, 63, 65, 72, 78, 79, 80, 84, 90, 100.

[1] https://youtu.be/srjm23nQbMc
\subsection{Elliptic Billiard Focus-Inversive N-periodics: Loci of Vertex, Perimeter, Area Centroids are Circles}
\label{vid:hjzW84ZZApA}
\noindent N=5, 1m45s (11/2020). 
\begin{center}\includegraphics[width=.5\textwidth]{pics/hjzW84ZZApA.jpg} \\ 
\href{https://youtu.be/hjzW84ZZApA}{\url{youtu.be/hjzW84ZZApA}}\end{center}
% \input{descr/135_hjzW84ZZApA}
\subsection{Self-Intersected 5-periodics in the Elliptic Billiard: Loci of Focus-Inversive Centroids are Circles}
\label{vid:7bzID9SVwqM}
\noindent N=5, 1m44s (11/2020). 
\begin{center}\includegraphics[width=.5\textwidth]{pics/7bzID9SVwqM.jpg} \\ 
\href{https://youtu.be/7bzID9SVwqM}{\url{youtu.be/7bzID9SVwqM}}\end{center}
% This experiment was inspired by [1].

Shown is the family of self-intersected 5-periodics (blue) in the Elliptic Billiard E (black). Let Pi denote its vertices. The focus-inversive polygon (pink) has vertices Pi' at the inversions of the Pi wrt unit circle (dashed black) centered on a focus f1 of E. Turns out this new family has invariant perimeter and sum of cosines, for all N.

On a previous video [2] we showed that the loci of C0,C1,C2 (vertex, perimeter, and area centroids) of the non-intersected focus-inversive family were circles. Turns out this property holds for self-intersected N-periodics as well!

Santé!

[1] R. Schwartz and S. Tabachnikov. "Centers of mass of Poncelet polygons, 200 years after". The Mathematical Intelligencer, 38:29–34, 2016.

[2] D. Reznik, "Elliptic Billiard Focus-Inversive N-periodics: Loci of Vertex, Perimeter, Area Centroids are Circles", YouTube, 2020, https://youtu.be/hjzW84ZZApA
\subsection{Inversive Symmedian I: 4 Families with Stationary Symmedian Points}
\label{vid:F0YuAx2Dy6M}
\noindent N=3, 18m25s (2/2021). 
\begin{center}\includegraphics[width=.5\textwidth]{pics/F0YuAx2Dy6M.jpg} \\ 
\href{https://youtu.be/F0YuAx2Dy6M}{\url{youtu.be/F0YuAx2Dy6M}}\end{center}
% Consider the family of excentral triangles T' derived from 3-periodics in the elliptic billiard. This is a family of acute triangles. Its sides will be tangent to the billiard at the vertices (since bisectors to 3-periodics are parallel to ellipse normals at the vertices). Its symmedian point X6 will be stationary since it coincides with the (stationary) mittenpunkt X9 of the reference 3-periodics [3,4]. Furthermore, the locus of its vertices will be an ellipse E' [1].

The video shows that if you invert the T' family wrt a unit circle centered on

(i) the common center
(ii) a focus of the elliptic billiard (caustic to the excentrals)
(iii) a focus of the ellipse E' to which the excentrals are inscribed

Their individual symmedians X6 will also be stationary.

This experiment can be seen live with our app [2] here: https://bit.ly/3teCEcY

[1] R. Garcia, "Elliptic Billiards and Ellipses Associated to the 3-Periodic Orbits", American Mathematical Monthly, 126, number 6, 2019.

[2] I. Darlan and D. Reznik, "Loci of Ellipse-Mounted Triangle Centers", 2020. https://dan-reznik.github.io/ellipse-mounted-triangles/

[3] E. Weisstein, "Mittenpunkt", MathWorld, 2021. https://mathworld.wolfram.com/Mittenpunkt.html

[4]  E. Weisstein, "Symmedian Point", MathWorld, 2021. https://mathworld.wolfram.com/SymmedianPoint.html
\subsection{Inversive Symmedian II: Stationary Inversive $X_{6}$ in Excentral Family}
\label{vid:vHUsFzvnRu4}
\noindent N=3, 7m28s (2/2021). 
\begin{center}\includegraphics[width=.5\textwidth]{pics/vHUsFzvnRu4.jpg} \\ 
\href{https://youtu.be/vHUsFzvnRu4}{\url{youtu.be/vHUsFzvnRu4}}\end{center}
% \input{descr/149_vHUsFzvnRu4}
\subsection{Inversive Symmedian III: Stationary Cosine Circle of Inversive Triangles}
\label{vid:wwX_QfkjVi0}
\noindent N=3, 8m53s (2/2021). 
\begin{center}\includegraphics[width=.5\textwidth]{pics/wwX_QfkjVi0.jpg} \\ 
\href{https://youtu.be/wwX_QfkjVi0}{\url{youtu.be/wwX\_QfkjVi0}}\end{center}
% \input{descr/150_wwX_QfkjVi0}

\section{Isodynamic Points (4)}

\subsection{Isodynamic points I: construction via Apollonius' circles}
\label{vid:L0pr8PuPavk}
\noindent N=3, 10m36s (11/2021). 
\begin{center}\includegraphics[width=.5\textwidth]{pics/L0pr8PuPavk.jpg} \\ 
\href{https://youtu.be/L0pr8PuPavk}{\url{youtu.be/L0pr8PuPavk}}\end{center}
% This shows a Geogebra construction of the two isodynamic points of a triangle, denoted X15 and X16 on ETC.

Mathworld: https://mathworld.wolfram.com/IsodynamicPoints.html
Geogebra: https://www.geogebra.org/classic/enpmnszs
ETC: https://faculty.evansville.edu/ck6/encyclopedia/ETC.html
\subsection{Isodynamic points II: centers of Apollonius' circles are collinear \& reciprocity wrt circumcircle}
\label{vid:Bqd7mHHDRnU}
\noindent N=3, 6m17s (11/2021). 
\begin{center}\includegraphics[width=.5\textwidth]{pics/Bqd7mHHDRnU.jpg} \\ 
\href{https://youtu.be/Bqd7mHHDRnU}{\url{youtu.be/Bqd7mHHDRnU}}\end{center}
% Video shows the 3 Apollonius' circles of a triangle have collinear centers (since they belong to the same centers). Recall their intersection are the two isodynamic points X15 and X16. Also shown is the fact that the latter two are reciprocals with respect to the circumcircle.
\subsection{Isodynamic points III: constructing $X_{15}$ with barycentric coordinates}
\label{vid:uUSKWHrctOQ}
\noindent N=3, 12m8s (11/2021). 
\begin{center}\includegraphics[width=.5\textwidth]{pics/uUSKWHrctOQ.jpg} \\ 
\href{https://youtu.be/uUSKWHrctOQ}{\url{youtu.be/uUSKWHrctOQ}}\end{center}
% This video shows that a triangle center can be easily found in geogebra with the TriangleCenter() command; we also show that we can also locate using its barycentrics (which appear on Kimberling's ETC).
\subsection{Isodynamic Points IV: flank triangles between hexagons erected on a triangle's sides have common $X_{16}$}
\label{vid:e3MkijszDEA}
\noindent N=3, 7m19s (11/2021). 
\begin{center}\includegraphics[width=.5\textwidth]{pics/e3MkijszDEA.jpg} \\ 
\href{https://youtu.be/e3MkijszDEA}{\url{youtu.be/e3MkijszDEA}}\end{center}
% The video shows a very curious observation. Erect regular hexagons Ha,Hb, Hc on the sides of a triangle T=ABC and consider the 3 flank triangles Fa,Fb,Fc between them 

1) Let Th have vertices at the centroids of the hexagons. Th is perpsective with T at X13. The circumcenter Th is X627.
2) Let Tf have vertices at the X15 of the flanks. Tf is perpsective with T at X6. The circumcenter of Tf is (to be determined)
3) The 3 X16's of the Tf coincide at a single point!

geogebra: https://www.geogebra.org/classic/wgjtq5b7

Cheers!

\section{Isogonal and Isotomic (2)}

\subsection{Antiorthic Axis and 5 points on the Billiard}
\label{vid:vyHZ8fwyiE8}
\noindent N=3, 3m13s (6/2019). 
\begin{center}\includegraphics[width=.5\textwidth]{pics/vyHZ8fwyiE8.jpg} \\ 
\href{https://youtu.be/vyHZ8fwyiE8}{\url{youtu.be/vyHZ8fwyiE8}}\end{center}
% Family of N=3 orbits in an a/b=1.5 elliptic billiard. Also shown are five points which we have found lie on the billiard via indirect transformations:

- The anti-feuerbach point X(100), shown as "F bar" (the feuerbach of the anticomplementary triangle)
- The vertices C1,C2,C3 of the anticomplementary triangle's contact triangle (touchpoints of its incircle with its sides)
- X(88), the isogonal conjugate of X(44). The latter is the intersection of the line from the origin through the incenter X(1) with the anti-orthic axis, shown blue. This axis is computed by intersecting corresponding sides on the orbit and excentral triangle (dashed green).
- The Tabachnikov point, red T, which lies at the geometric mean (in terms of radial length) between the Incenter and X44.

Note1: the antifeuerbach X(100), the incenter X(1), and X(88) are collinear
Note2: the mittenpunkt X(9), the incenter X(1), the Symmedian X(6), and X(44) are collinear.
Note3: the encyclopedia of triangular centers does say X(44) is "X(44) = inverse-in-circumconic-centered-at-X(9) of X(1)". Because the billiard is centered on the mittenpunkt X(9), this is a precise statement.

https://dan-reznik.github.io/Elliptical-Billiards-Triangular-Orbits/
\subsection{Isotomic and Isogonal Conjugates of Billiard with respect to the 3-periodic family}
\label{vid:C0fIMK6fuAU}
\noindent N=3, 3m12s (7/2019). 
\begin{center}\includegraphics[width=.5\textwidth]{pics/C0fIMK6fuAU.jpg} \\ 
\href{https://youtu.be/C0fIMK6fuAU}{\url{youtu.be/C0fIMK6fuAU}}\end{center}
% Shown is an a/b=1.5 elliptic billiard and its family of N=3 (triangular) orbits. Also shown are the billiard's excentral (green) and anticomplementary (blue) triangles.

The isogonal conjugate of the billiard is the orbit's antiorthic axis, shown as a green line. Equivalently, the billiard's isogonal conjugate is the trilinear polar of X1, the perspector of the orbit and its excentral.

The isotomic conjugate of the billiard, shown blue, is the Gergonne line (L55) of its anticomplementary triangle. Equivalently, the billiard's isotomic conjugate is the trilinear polar of X144, the perspector of the the orbit's anticomplementary triangle and the latter's contact triangle (shown dashed blue).

https://dan-reznik.github.io/Elliptical-Billiards-Triangular-Orbits/

\section{Isogonals over Poncelet (2)}

\subsection{Extending A. Skutin's result I: locus of isogonal conjugate of fixed point over Poncelet triangles}
\label{vid:x0953ASLkuk}
\noindent N=3, 9m46s (10/2021). 
\begin{center}\includegraphics[width=.5\textwidth]{pics/x0953ASLkuk.jpg} \\ 
\href{https://youtu.be/x0953ASLkuk}{\url{youtu.be/x0953ASLkuk}}\end{center}
% This video explores a result in [1], namely, that over circle-inscribed Poncelet triangles (w arbitrary nested elliptic caustic), the locus of the isogonal conjugate of a fixed point P is a circle.

[1] A. Skutin, "On Rotation of a Isogonal Point", Journal of Classical Geometry, vol. 2, 2013. https://jcgeometry.org/Articles/Volume2/JCG2013V2pp66-67.pdf
\subsection{Extending A. Skutin's result II: conic frontiers of isogonal loci \& confocal envelope of line loci}
\label{vid:o9iHWbk4aPc}
\noindent N=3, 9m50s (10/2021). 
\begin{center}\includegraphics[width=.5\textwidth]{pics/o9iHWbk4aPc.jpg} \\ 
\href{https://youtu.be/o9iHWbk4aPc}{\url{youtu.be/o9iHWbk4aPc}}\end{center}
% This is a continuation of the work by A. Skutin. In [1] he proves an observation by Akopyan: given a circle-inscribed Poncelet family of triangles, the locus of the isogonal conjugate of a fixed point P wrt triangles in the family is a circle. In a previous video [2] we extended this to consider Poncelet triangles interscribed between two nested ellipses E,E' in general position.

video shows:

(1) the frontiers of P such that the locus of the isogonal of P over Poncelet is an ellipse, parabola, or hyperbola. In particular

(2) the locus of P such that the isogonal locus is a rectangular hyperbola is a circle! Finally, we show that if P  is on the outer conic

(3) the straigh-line isogonal loci envelop an ellipse which is confocal with E'.

[1] A. Skutin, "On Rotation of a Isogonal Point", Journal of Classical Geometry, vol. 2, 2013. https://jcgeometry.org/Articles/Volume2/JCG2013V2pp66-67.pdf


[2]  D. Reznik,  "Extending Sktutin's Result, Part I", https://youtu.be/x0953ASLkuk

\section{Locus App (5)}

\subsection{Loci of Ellipse-Inscribed Triangles: Part 01 - Intro to the App}
\label{vid:o63QTcpDqNA}
\noindent N=3, 11m14s (11/2020). 
\begin{center}\includegraphics[width=.5\textwidth]{pics/o63QTcpDqNA.jpg} \\ 
\href{https://youtu.be/o63QTcpDqNA}{\url{youtu.be/o63QTcpDqNA}}\end{center}
% Proofs for the ellipticity of the locus of several triangle centers over billiard 3-periodics include: incenter [1,2], barycenter [4], circumcenter [3] and these and 26+ additional triangle centers [5].

A few useful links

Access the app: https://dan-reznik.github.io/ellipse-mounted-loci-p5js/
GitHub: https://github.com/dan-reznik/ellipse-mounted-loci-p5js
Learn more: https://dan-reznik.github.io/ellipse-mounted-triangles/
Example with four loci: https://bit.ly/3kUYVqQ
Slides: bit.ly/2IZZqm1

[1] O. Romaskevich, "On the incenters of triangular orbits on elliptic billiards", Enseign. Math., 60:3 , 2014, pages = 247--255.
[2] R. Garcia, "Elliptic Billiards and Ellipses Associated to the 3-Periodic Orbits", American Mathematical Monthly, 126:06, pages 491--504, 2019.
[3] C. Fierobe, "On the circumcenters of triangular orbits in elliptic billiard", https://arxiv.org/pdf/1807.11903.pdf
[4] R. Schwartz and S. Tabachnikov, "Centers of mass of Poncelet polygons, 200 years after",
Math. Intelligencer, 38:2, 2016, pages 29-34. 
[5] R. Garcia, D. Reznik, and J. Koiller, "Loci of 3-periodics in an Elliptic Billiard: why so many ellipses?", 2020, https://arxiv.org/abs/2001.08041
\subsection{Loci of Ellipse-Inscribed Triangles: Part 02 - The Homothetic Family}
\label{vid:1i4ys4TFw48}
\noindent N=3, 14m44s (11/2020). 
\begin{center}\includegraphics[width=.5\textwidth]{pics/1i4ys4TFw48.jpg} \\ 
\href{https://youtu.be/1i4ys4TFw48}{\url{youtu.be/1i4ys4TFw48}}\end{center}
% We go over the capability to change "mounts", i.e., triangle families. Specifically, we switch between billiard and homothetic families. We also explore the "tandem" feature to change mounts across all channels simultaneously.

Four Homothetic Circles: https://bit.ly/2J5ncNZ

A few useful links:

Access the app: https://dan-reznik.github.io/ellipse-mounted-loci-p5js/
GitHub: https://github.com/dan-reznik/ellipse-mounted-loci-p5js
Learn more: https://dan-reznik.github.io/ellipse-mounted-triangles/
Example with four loci: https://bit.ly/3kUYVqQ
Slides: bit.ly/2IZZqm1
\subsection{Loci of Ellipse-Inscribed Triangles: Part 03 - Derived Triangles}
\label{vid:DSeZXMnuirA}
\noindent N=3, 14m35s (11/2020). 
\begin{center}\includegraphics[width=.5\textwidth]{pics/DSeZXMnuirA.jpg} \\ 
\href{https://youtu.be/DSeZXMnuirA}{\url{youtu.be/DSeZXMnuirA}}\end{center}
% Here you learn how to select a *derived* triangle under observation, e.g., anticomplementary, excentral, medial, orthic, etc.

Orthic experiment: https://bit.ly/3ftjBFd

A few useful links:

Access the app: https://dan-reznik.github.io/ellipse-mounted-loci-p5js/
GitHub: https://github.com/dan-reznik/ellipse-mounted-loci-p5js
Learn more: https://dan-reznik.github.io/ellipse-mounted-triangles/
Example with four loci: https://bit.ly/3kUYVqQ
Slides: bit.ly/2IZZqm1
\subsection{Loci of Ellipse-Inscribed Triangles: Part 04 - Locus Type}
\label{vid:QIcx89W6J_k}
\noindent N=3, 21m27s (11/2020). 
\begin{center}\includegraphics[width=.5\textwidth]{pics/QIcx89W6J_k.jpg} \\ 
\href{https://youtu.be/QIcx89W6J_k}{\url{youtu.be/QIcx89W6J\_k}}\end{center}
% The elliptic locus of the excenters over elliptic billiard 3-periodics was proved in [1]. The ellipse+two straight line locus of the excenters over the focus-mounted family was proved in [2].

Locus of excenters for the focus-mounted family: https://bit.ly/2V4nuqx

A few useful links: 

Access the app: https://dan-reznik.github.io/ellipse-mounted-loci-p5js/
GitHub: https://github.com/dan-reznik/ellipse-mounted-loci-p5js
Learn more: https://dan-reznik.github.io/ellipse-mounted-triangles/
Example with four loci: https://bit.ly/3kUYVqQ
Slides: bit.ly/2IZZqm1

References:

[1] R. Garcia, "Elliptic Billiards and Ellipses Associated to the 3-Periodic Orbits", American Mathematical Monthly, 126:06, pages 491--504, 2019.
[2]  J. Dykstra, C. Peterson, A. Rall. E. Shadduck, "Orbiting Vertex: Follow That Triangle Center!", 2006, available online at: bit.ly/2JaWMK9
\subsection{Segment Locus of the Center of a Certain Apollonius' Circle}
\label{vid:Ovypr11bNqU}
\noindent N=3, 7m31s (9/2021). 
\begin{center}\includegraphics[width=.5\textwidth]{pics/Ovypr11bNqU.jpg} \\ 
\href{https://youtu.be/Ovypr11bNqU}{\url{youtu.be/Ovypr11bNqU}}\end{center}
% The outer Apollonius' circle C of a triangle is tangent to and encompasses the excircles. Its center is X(970) and the radius is (r^2+s^2)/(4r), where r is ithe inradius of T and s its semiperimeter [1,2].

The video studies C of the anticomplementary triangle (ACT) of an N=3 bicentric (poristic) family of Poncelet triangles [3].

Over the family (i) X(10) of the ACT is stationary, (ii) the locus of its vertices is a pseudo Pascal Limaçon, and (iii) the locus of X(970) of the ACT is a segment collinear with X(1) and X(3) of the bicentric family. Note that the radius of C is not stable.

Note: X(970) of the ACT is X(10441) of the reference [4, Part 6]. So we are showing that over the poristic family the locus of this center is a segment on the OI axis.

You can try out the simulation here: https://bit.ly/38S2OZS

[1] E. Weisstein, "Apollonius' Circle", MathWorld, 2021., https://mathworld.wolfram.com/ApolloniusCircle.html
[2] Grinberg, D. and Yiu, P. "The Apollonius Circle as a Tucker Circle." Forum Geom. 2, 175-182, 2002. http://forumgeom.fau.edu/FG2002volume2/FG200222index.html
[3] B. Odehnal, "Poristic Loci of Triangle Centers", J. Geom. Graphics 15/1 (2011), 45-67.
[4] C. Kimberling, ETC Part 6 (Centers 10001 to 12000), https://faculty.evansville.edu/ck6/encyclopedia/ETCPart6.html

\section{Misc (24)}

\subsection{Elliptic Billiards in Brazil}
\label{vid:PHitZFbps8M}
\noindent N=4, 2m25s (7/2019). 
\begin{center}\includegraphics[width=.5\textwidth]{pics/PHitZFbps8M.jpg} \\ 
\href{https://youtu.be/PHitZFbps8M}{\url{youtu.be/PHitZFbps8M}}\end{center}
% - Inner (blue): caustic to an N=4 orbit in an a/b=1.5 elliptic billiard.
- Mid (yellow): an N=4 orbit (billiard not shown), a parallelogram.
- Outer (green): the excentral polygon to the orbit, tangent to the billiard at the orbit vertices. A perfect rectangle.
- Outer (dotted): circular locus of the excentral polygon vertices.

---

Audio: National Anthem of Brazil

More Info: https://dan-reznik.github.io/Elliptical-Billiards-Triangular-Orbits/
\subsection{Loci of Outer Napoleon Equilateral Construction I}
\label{vid:70-E-NZrNCQ}
\noindent N=3, 4m48s (7/2019). 
\begin{center}\includegraphics[width=.5\textwidth]{pics/70-E-NZrNCQ.jpg} \\ 
\href{https://youtu.be/70-E-NZrNCQ}{\url{youtu.be/70-E-NZrNCQ}}\end{center}
% An a/b=1.5 elliptic billiard is shown along with its family of N=3 (triangular) orbits. For each orbit 3 equilaterals are erected on its sides. The locus of their outer vertices is shown red. Also shown (purple) is the equilateral triangle formed by centroids of each triangle erected.

More Info:  https://dan-reznik.github.io/Elliptical-Billiards-Triangular-Orbits/
\subsection{Loci of Outer Napoleon Equilateral Construction II}
\label{vid:wwuvau5GeqY}
\noindent N=3, 4m48s (7/2019). 
\begin{center}\includegraphics[width=.5\textwidth]{pics/wwuvau5GeqY.jpg} \\ 
\href{https://youtu.be/wwuvau5GeqY}{\url{youtu.be/wwuvau5GeqY}}\end{center}
% An a/b=1.5 elliptic billiard is shown along with its family of N=3 (triangular) orbits. For each orbit 3 equilaterals are erected on its sides (green). The sequence of the latter centroids produces the "outer napoleon equilateral" (shown purple).

The following loci are shown:

- dashed green: the apparently (but non-) elliptic locus of their outer vertices is shown dashed green
- dashed blue: the non-elliptic locus of the side medians, above which the altitudes of the side equilaterals are erected. - 
- dashed purple: the non-elliptic locus of the vertices of the central outer napoleon equilateral.
- solid purple: the elliptic locus of the central equilateral's centroid.

More information: https://dan-reznik.github.io/Elliptical-Billiards-Triangular-Orbits/
Interactive Applet: https://editor.p5js.org/dreznik/full/i1Lin7lt7
\subsection{The Miquel Point of the Extouch and Excentral Triangles I}
\label{vid:CKaV_AKZc1U}
\noindent N=3, 3m12s (7/2019). 
\begin{center}\includegraphics[width=.5\textwidth]{pics/CKaV_AKZc1U.jpg} \\ 
\href{https://youtu.be/CKaV_AKZc1U}{\url{youtu.be/CKaV\_AKZc1U}}\end{center}
% \input{descr/167_CKaV_AKZc1U}
\subsection{The Miquel Point of the Extouch and Excentral Triangles II}
\label{vid:jDWwrUWVmjg}
\noindent N=3, 6m24s (7/2019). 
\begin{center}\includegraphics[width=.5\textwidth]{pics/jDWwrUWVmjg.jpg} \\ 
\href{https://youtu.be/jDWwrUWVmjg}{\url{youtu.be/jDWwrUWVmjg}}\end{center}
% An a/b=1.5 elliptic billiard is shown with its family of N=3 (triangular) orbits.

LEFT: for each orbit the three Miquel circles are shown in red as computed with respect to the extouch points (which we know sweep the orbits' caustic). Also shown (red dot) is the locus of the concurrence point of these three circles, congruent with X(40), known as the Miquel Point. Notice its upright elliptic trajectory.

RIGHT: the excentral triangle (green) is shown for each orbit. The Miquel construction is now computed with respect to the orbits' vertices. Notice the sideways elliptic locus of the Miquel Point. This is identified with X(1) the incenter of the orbit.

Audio Credits: Paulinho Nogueira (guitar), "Odeon" (Ernesto Nazareth, Francisco Mignone)
\subsection{An invariant in the parabolic pair associated with the N=3 family}
\label{vid:VpDrCPG6th0}
\noindent N=3, 1m37s (12/2019). 
\begin{center}\includegraphics[width=.5\textwidth]{pics/VpDrCPG6th0.jpg} \\ 
\href{https://youtu.be/VpDrCPG6th0}{\url{youtu.be/VpDrCPG6th0}}\end{center}
% \input{descr/169_VpDrCPG6th0}
\subsection{Non-monotonic $X_{88}$ and the $X_{1}$-$X_{100}$ envelope}
\label{vid:nJLp--JjDZU}
\noindent N=3, 2m49s (2/2020). 
\begin{center}\includegraphics[width=.5\textwidth]{pics/nJLp--JjDZU.jpg} \\ 
\href{https://youtu.be/nJLp--JjDZU}{\url{youtu.be/nJLp--JjDZU}}\end{center}
% Three elliptic billiards are shown with aspect ratios a/b of 1.35, 1.486, 1.75. The family of 3-periodics for each billiard are shown as blue polygons.

X1, the incenter is shown as well as its elliptic locus (green).

Also shown are centers X100 and X88 whose locus is on the billiard boundary. Interestingly, X1, X100, and X88 are collinear [1].  Shown (purple) is the astroid-like envelope of the lines X1-X100, to which said line is instantaneously tangent, at a point "E".

In the simulation, orbit vertices move monotonically counterclockwise (CCW). Notice centers X1 (resp. X100) move monotonically CCW (resp. CW).

What happens to X88 is more complex. Define a constant a88=1.486.

a) a/b ﹤a88: X88 moves monotonically CW.
b) a/b = a88: X88 moves monotonically CW though it comes to a stop (velocity zero), when E is on the left/right vertex of the billiard.
c) a/b ﹥ a88 X88 motion has four phases: two CW portions and two CCW. The former (resp. latter) occur when E is inside (resp. outside) the billiard. As in [b], X88 comes to a stop when E is on the billiard boundary.

[1] Kimberling, C., "Encyclopedia of Triangle Centers", https://faculty.evansville.edu/ck6/encyclopedia/ETC.html
\subsection{The Thomson Cubic of 3-periodics}
\label{vid:uNHIZXgZDOs}
\noindent N=3, 4m49s (2/2020). 
\begin{center}\includegraphics[width=.5\textwidth]{pics/uNHIZXgZDOs.jpg} \\ 
\href{https://youtu.be/uNHIZXgZDOs}{\url{youtu.be/uNHIZXgZDOs}}\end{center}
% \input{descr/177_uNHIZXgZDOs}
\subsection{Locus and elliptic envelope of excircle tangents' hexagon (side touchpoints)}
\label{vid:4XMTSvZtTJo}
\noindent N=3, 8m1s (3/2020). 
\begin{center}\includegraphics[width=.5\textwidth]{pics/4XMTSvZtTJo.jpg} \\ 
\href{https://youtu.be/4XMTSvZtTJo}{\url{youtu.be/4XMTSvZtTJo}}\end{center}
% An a/b=1.618 Elliptic Billiard (EB) is shown (black) as well as its family of 3-periodics (blue triangles). Also shown are the Excircles (green) centered on the vertices of the Excentral Triangle (green). The elliptic locus of the Excenters [1] is shown dashed green. Excircles touch the 3-periodic at three "Exouchpoints". Extensions of 3-periodic sides (dashed purple) are tangent to the Excircles at 6 points; these form a 6-gon (purple). The video shows the locus of alternate vertices of said 6-gon, each forming a bean-shaped curve (red and dashed red). Also shown is the (non-constant) area and perimeter of the tangents' hexagon.

Also shown (dashed purple) is the locus of two consecutive points on the hexagon. These have been numerically found to be a perfect ellipse whose aspect ratio is identical to the EB's.
\subsection{Six-Point Conic passes through Sideline Tangents to Excircles}
\label{vid:kVtTR-aINX4}
\noindent N=3, 4m1s (4/2020). 
\begin{center}\includegraphics[width=.5\textwidth]{pics/kVtTR-aINX4.jpg} \\ 
\href{https://youtu.be/kVtTR-aINX4}{\url{youtu.be/kVtTR-aINX4}}\end{center}
% \input{descr/176_kVtTR-aINX4}
\subsection{Reuleaux Triangle: Properties of Negative Pedal Curve, and Exploring its Billiard Trajectories}
\label{vid:aGtDyVjlrvM}
\noindent N=n/a, 18m24s (6/2020). 
\begin{center}\includegraphics[width=.5\textwidth]{pics/aGtDyVjlrvM.jpg} \\ 
\href{https://youtu.be/aGtDyVjlrvM}{\url{youtu.be/aGtDyVjlrvM}}\end{center}
% \input{descr/175_aGtDyVjlrvM}
\subsection{Horizontal-Vertical Billiard in a Rhombus and Parallelogram: are there N-Periodics?}
\label{vid:XKCYL-8hEVA}
\noindent N=n/a, 7m56s (6/2020). 
\begin{center}\includegraphics[width=.5\textwidth]{pics/XKCYL-8hEVA.jpg} \\ 
\href{https://youtu.be/XKCYL-8hEVA}{\url{youtu.be/XKCYL-8hEVA}}\end{center}
% \input{descr/171_XKCYL-8hEVA}
\subsection{Pascal's Limaçon as Envelope of Circles}
\label{vid:495A_ZjgcyE}
\noindent N=n/a, 2m55s (11/2020). 
\begin{center}\includegraphics[width=.5\textwidth]{pics/495A_ZjgcyE.jpg} \\ 
\href{https://youtu.be/495A_ZjgcyE}{\url{youtu.be/495A\_ZjgcyE}}\end{center}
% \input{descr/174_495A_ZjgcyE}
\subsection{Elliptic Billiard 3-Periodics: Invariants of the Focal Hyperbola}
\label{vid:_ydXSm-QplM}
\noindent N=3, 1m45s (11/2020). 
\begin{center}\includegraphics[width=.5\textwidth]{pics/_ydXSm-QplM.jpg} \\ 
\href{https://youtu.be/_ydXSm-QplM}{\url{youtu.be/\_ydXSm-QplM}}\end{center}
% Consider the family of 3-periodics (blue) with vertices P1,P2,P3 in the elliptic billiard with foci f1 and f2. Let the "focal" hyperbola (red) be defined dynamically by the quintuple (P1,P2,P3,f1,f2). The locus of the hyperbola center (red dot) is an ellipse (dashed red).

Consider a cartesian system with origin on the center of the elliptic billiard and x,y axis along its major and minor axis. Consider the following 5-parameter implicit for a general conic:

cxx x^2 + cyy y^2 + cxy x y + cx x + cy y + 1 = 0

where cxx,cyy,cxy,cx,cy are (variable) coefficients.

Surprisingly, over the hyperbola family, cxx and cyy are non-zero and invariant and cx = 0. The only varying parameters are cxy and cy.
\subsection{Ellipse Maximal Distance Chords I: locus of the farthest midpoint can be a 4-leaf clover}
\label{vid:kJUkidbrn3g}
\noindent N=n/a, 10m23s (1/2021). 
\begin{center}\includegraphics[width=.5\textwidth]{pics/kJUkidbrn3g.jpg} \\ 
\href{https://youtu.be/kJUkidbrn3g}{\url{youtu.be/kJUkidbrn3g}}\end{center}
% \input{descr/181_kJUkidbrn3g}
\subsection{Ellipse Maximal Distance Chords II: the Astroidal Evolute}
\label{vid:YvoyN46biq8}
\noindent N=n/a, 12m37s (1/2021). 
\begin{center}\includegraphics[width=.5\textwidth]{pics/YvoyN46biq8.jpg} \\ 
\href{https://youtu.be/YvoyN46biq8}{\url{youtu.be/YvoyN46biq8}}\end{center}
% \input{descr/182_YvoyN46biq8}
\subsection{Circumellipscevian Triangle: area ratio invariant in N=3 Concentric Poncelet}
\label{vid:4Q1uouMQzXU}
\noindent N=3, 11m49s (1/2021). 
\begin{center}\includegraphics[width=.5\textwidth]{pics/4Q1uouMQzXU.jpg} \\ 
\href{https://youtu.be/4Q1uouMQzXU}{\url{youtu.be/4Q1uouMQzXU}}\end{center}
% \input{descr/179_4Q1uouMQzXU}
\subsection{Excentral-Tangential Family of Elliptic Billiard 3-Periodics: invariant perimeter and stationary $X_{7}$}
\label{vid:Pqer9GfADqc}
\noindent N=3, 10m44s (2/2021). 
\begin{center}\includegraphics[width=.5\textwidth]{pics/Pqer9GfADqc.jpg} \\ 
\href{https://youtu.be/Pqer9GfADqc}{\url{youtu.be/Pqer9GfADqc}}\end{center}
% The family of 3-periodics T in the elliptic billiard conserves perimeter. Recall its Mittenpunkt X9 is stationary at the common center. Now consider its derived excentral triangles T'. Its symmedian point coincides with X9 of the 3-periodics and is therefore also stationary at the center.

Recall the tangential triangle [1] has sides which pass through those of a reference triangle and are tangent to the circumcircle. Its Gergonne point X7 coincides with the reference X6, and it is homothetic to the orthic at X25 [1] (*). Let a triangle have sidelengths a,b,c. Courtesy of Peter Moses, the homothecy ratio is given by [2]:

K = (2 a b c)/Sqrt[(a^2 - b^2 - c^2)*(a^2 + b^2 - c^2)*(a^2 - b^2 + c^2)]

Peter adds: the sign decides if the vertices of the two triangles are on the same side from the homothetic center X(25)

Let T'' denote the family of tangential triangles of the excentral family T', which we will regard as the "reference" family.

X7'' = X6' = X9 are all stationary at the center. The T'' are homothetic to the T at X25 of the T'. 

It turns out over 3-periodics in the elliptic billiard, K is invariant. This implies the perimeter of the T'' is conserved!

(*) in the video I wrongly state the center of homotecy is X6.

[1] E. Weisstein, "Tangential Triangle", MathWorld, 2021.
[2] P. Moses, Private Communication, 4-Feb-2021.
\subsection{Apresentação do Curso 33o CBM IMPA-2021
``Invariantes Ponceletianas: um Passeio Experimental''}
\label{vid:UakZhTIQVro}
\noindent N=n/a, 11m52s (3/2021). 
\begin{center}\includegraphics[width=.5\textwidth]{pics/UakZhTIQVro.jpg} \\ 
\href{https://youtu.be/UakZhTIQVro}{\url{youtu.be/UakZhTIQVro}}\end{center}
% \input{descr/178_UakZhTIQVro}
\subsection{Interpolation of pedal and contrapedal curves}
\label{vid:0SW_tBUeNKg}
\noindent N=n/a, 1m21s (5/2021). 
\begin{center}\includegraphics[width=.5\textwidth]{pics/0SW_tBUeNKg.jpg} \\ 
\href{https://youtu.be/0SW_tBUeNKg}{\url{youtu.be/0SW\_tBUeNKg}}\end{center}
% \input{descr/185_0SW_tBUeNKg}
\subsection{Ellipse-inscribed triangles with perimeter-trisecting vertices. What is the locus of the centroid?}
\label{vid:bGAE4IgBPN4}
\noindent N=3, 4m52s (8/2021). 
\begin{center}\includegraphics[width=.5\textwidth]{pics/bGAE4IgBPN4.jpg} \\ 
\href{https://youtu.be/bGAE4IgBPN4}{\url{youtu.be/bGAE4IgBPN4}}\end{center}
% \input{descr/183_bGAE4IgBPN4}
\subsection{Cramer-Castillon \& Poncelet: surprising Brocard and Lemoine harmonies}
\label{vid:Aw4U2Ah1Byc}
\noindent N=3, 29m32s (9/2021). 
\begin{center}\includegraphics[width=.5\textwidth]{pics/Aw4U2Ah1Byc.jpg} \\ 
\href{https://youtu.be/Aw4U2Ah1Byc}{\url{youtu.be/Aw4U2Ah1Byc}}\end{center}
% This video describes many interesting harmonies which ensue when Cramer-Castillon (CC) triangles [1] are controlled by the vertices of certain N=3 Poncelet families. To compute the 2 CC triangles, we used the method due to Carnot, explained here [2].

[1] Cramer-Castillon Problem https://en.wikipedia.org/wiki/Cramer%E2%80%93Castillon_problem
[2] https://www.e-periodica.ch/cntmng?pid=edm-001%3A2006%3A61%3A%3A72
\subsection{Sixty Pascal Lines over Certain N=6 Poncelet Families}
\label{vid:qB-tGP6dQkk}
\noindent N=6, 10m29s (10/2021). 
\begin{center}\includegraphics[width=.5\textwidth]{pics/qB-tGP6dQkk.jpg} \\ 
\href{https://youtu.be/qB-tGP6dQkk}{\url{youtu.be/qB-tGP6dQkk}}\end{center}
% \input{descr/187_qB-tGP6dQkk}
\subsection{Prof. Pamfilos' Construction of 6 points on a parabola}
\label{vid:ElxjUgHmKNU}
\noindent N=3, 3m35s (10/2021). 
\begin{center}\includegraphics[width=.5\textwidth]{pics/ElxjUgHmKNU.jpg} \\ 
\href{https://youtu.be/ElxjUgHmKNU}{\url{youtu.be/ElxjUgHmKNU}}\end{center}
% \input{descr/186_ElxjUgHmKNU}

\section{Multiple Caustics (16)}

\subsection{Bicentrics with Two Caustics I: circular locus of $X_{1}$}
\label{vid:OM7uilfdGgk}
\noindent N=3, 2m26s (7/2021). 
\begin{center}\includegraphics[width=.5\textwidth]{pics/OM7uilfdGgk.jpg} \\ 
\href{https://youtu.be/OM7uilfdGgk}{\url{youtu.be/OM7uilfdGgk}}\end{center}
% Top left: one triangle T=P1 P2 P3 in the poristic family with fixed circumcircle C (black) and incircle C' (brown).

Top right: the radius of  C' is reduced and only two sides of T remain tangent to C'. The third side (dashed blue) envelops a third circle C" (dashed red) which is on the pencil of C and C'. The incenter X1 is shown at its current position.

Bottom left: the radius of C' is now increased beyond so that the envelope of the third side (dashed red) is interior to C'.

Bottom right: at a special choice for the radius of C' the envelope collapses to a limiting point l1 of the C,C' pair. The radical axis (vertical dashed gray) of the pair is also shown, perpendicular to and passing through the midpoint of segment l1 l2.
\subsection{Bicentrics with Two Caustics II: circular Loci of $X_{1}$, $X_{40}$, $X_{165}$}
\label{vid:qJGhf798E-s}
\noindent N=3, 2m26s (7/2021). 
\begin{center}\includegraphics[width=.5\textwidth]{pics/qJGhf798E-s.jpg} \\ 
\href{https://youtu.be/qJGhf798E-s}{\url{youtu.be/qJGhf798E-s}}\end{center}
% Left: A family of triangles T=P1P2P3 is shown inscribed in an outer (black) circle centered on X_3. Sides P1 P2 and P1P3 (blue) are tangent to a second inner circle (brown), centered on O'. Over the family, the third side (red) envelops a third, in-pencil circle (dashed red), centered on Oenv.

Main result: the locus of X1 (green) is a circle, for any choices of outer and inner circle.

Also shown are Bevan point X40 (orange), and the excentral barycenter X165, which lie on the X1X3 (OI) line. In fact, X40 (resp. X165) is the reflection of X1 about X3 with a scale of 1 (resp. 1/3). Therefore the loci of X40 and X165 are scaled versions of that of X1.

Right:  a similar setup but with a larger inner circle (brown) and wider distance between X3 and O'. Miraculously, all loci remain circular.
\subsection{Bicentrics with Two Caustics III: sextic loci of $X_{2}$, $X_{4}$, $X_{5}$}
\label{vid:6Fqp6Z1Q-0A}
\noindent N=3, 2m1s (7/2021). 
\begin{center}\includegraphics[width=.5\textwidth]{pics/6Fqp6Z1Q-0A.jpg} \\ 
\href{https://youtu.be/6Fqp6Z1Q-0A}{\url{youtu.be/6Fqp6Z1Q-0A}}\end{center}
% \input{descr/194_6Fqp6Z1Q-0A}
\subsection{Bicentrics with Two Caustics IV: loci of $X_{1}$ (circle) and $X_{2}$ (sextic), varying caustic}
\label{vid:3dnsWPlAmxE}
\noindent N=3, 1m48s (7/2021). 
\begin{center}\includegraphics[width=.5\textwidth]{pics/3dnsWPlAmxE.jpg} \\ 
\href{https://youtu.be/3dnsWPlAmxE}{\url{youtu.be/3dnsWPlAmxE}}\end{center}
% Shown is a "tricentric" Poncelet family of triangles P1P2P3 inscribed in an outer circle C with diameter R. Sides P1P3 and P1P3 (blue) are kept tangent to an internal circle C' (the first "caustic", brown), with radius r. In this case, and according to Poncelet's General Closure Thm, side P2P3 (red) will envelop a circle (dashed red) in the pencil of C, C', i.e., a second "caustic".

Shown are the loci of X1 (green) and X2 (purple) over variable "r". Amazingly, the former is always a circle while the latter is a loopy sextic. 

Notice that when "r" is such that the system is poristic (all sides are tangent to C'), the locus of X1 collapses to a point and that of X2 becomes a circle, whose radius and position were derived in [1].

[1] B. Odehnal, "Poristic Loci of Triangle Centers", J. Geom. Graphics 15/1 (2011), 45-67.
\subsection{Bicentrics with Two Caustics V: 1 circle, 2 non-conic loci of the excenters (also: $X_{1}$, $X_{40}$ circular)}
\label{vid:qdqIuT-Qk6k}
\noindent N=3, 2m26s (7/2021). 
\begin{center}\includegraphics[width=.5\textwidth]{pics/qdqIuT-Qk6k.jpg} \\ 
\href{https://youtu.be/qdqIuT-Qk6k}{\url{youtu.be/qdqIuT-Qk6k}}\end{center}
% Left: the "poristic" family of triangles where both incircle and circumcircle are fixed. The vertices of the excentral triangle (dark green), i.e., the excenters sweep a circle with twice the radius of the circumcircle, a result proved in [1].

Right: the "tricentric" family -- P1P2 and P1P3 are tangent to a circle whose radius (in this case) is less than the poristic one. Notice P2P3 is no longer tangent to the inner circle, indeed its envelope is a third circle (dashed red) in the pencil of the circumcircle and inner circle. As shown in previous videos, both X1 and X40 sweep circles of the same radius, w centers symmetric about X3.

The new result here is the curious loci of the excenters: excenter P1' (opposite to P1) sweeps a circle (dark green) while that the other two (P2' and P3') sweep are non-conics (light green) which intersect on the X1X3 line.  

[1] B. Odehnal, "Poristic Loci of Triangle Centers", J. Geom. Graphics 15/1 (2011), 45-67.
\subsection{Confocals with Two Caustics I: locus of the incenter $X_{1}$ for 4 different caustics}
\label{vid:C14TL430UBc}
\noindent N=3, 2m26s (7/2021). 
\begin{center}\includegraphics[width=.5\textwidth]{pics/C14TL430UBc.jpg} \\ 
\href{https://youtu.be/C14TL430UBc}{\url{youtu.be/C14TL430UBc}}\end{center}
% \input{descr/197_C14TL430UBc}
\subsection{Confocals with Two Caustics II: locus of the incenter $X_{1}$ over confocal caustic sweep}
\label{vid:kCY6KHFDV2M}
\noindent N=3, 1m42s (7/2021). 
\begin{center}\includegraphics[width=.5\textwidth]{pics/kCY6KHFDV2M.jpg} \\ 
\href{https://youtu.be/kCY6KHFDV2M}{\url{youtu.be/kCY6KHFDV2M}}\end{center}
% \input{descr/198_kCY6KHFDV2M}
\subsection{Confocals with Two Caustics III: loci of excenters: two elliptic, one non-conic}
\label{vid:AisIrfn4IGg}
\noindent N=3, 2m26s (7/2021). 
\begin{center}\includegraphics[width=.5\textwidth]{pics/AisIrfn4IGg.jpg} \\ 
\href{https://youtu.be/AisIrfn4IGg}{\url{youtu.be/AisIrfn4IGg}}\end{center}
% \input{descr/199_AisIrfn4IGg}
\subsection{Half N=4 Poncelet Triangles: elliptic locus of barycenter over 4 different families}
\label{vid:6yNod1LFVrY}
\noindent N=3, 2m26s (7/2021). 
\begin{center}\includegraphics[width=.5\textwidth]{pics/6yNod1LFVrY.jpg} \\ 
\href{https://youtu.be/6yNod1LFVrY}{\url{youtu.be/6yNod1LFVrY}}\end{center}
% \input{descr/201_6yNod1LFVrY}
\subsection{Poncelet Triangles with Smoothly-Varying Circular Caustic: the amazing loci of $X_{11}$ and $X_{59}$}
\label{vid:2VqgB6KvP2g}
\noindent N=3, 1m47s (7/2021). 
\begin{center}\includegraphics[width=.5\textwidth]{pics/2VqgB6KvP2g.jpg} \\ 
\href{https://youtu.be/2VqgB6KvP2g}{\url{youtu.be/2VqgB6KvP2g}}\end{center}
% Shown are the loci of isogonal conjugates X(11) and X(59) over a Poncelet family of triangles P1P2P3 inscribed in a fixed outer ellipse, with two sides P1P2 and P1P3 (blue) tangent to an internal concentric circle (brown) of smoothly changing radius. The envelope of P2P3 (not shown) is, according to Poncelet's general theorem, an ellipse in the pencil of E and E'.

Notice the mesmerizing topological changes to both loci. From larvae to chrysalis to butterfly and vice-versa!
\subsection{Bicentrics with Three Caustics I: Non-Conic Loci of $X_i$, i=1, 2, 6, 8, 10}
\label{vid:cvB0A7LmlZc}
\noindent N=3, 2m26s (7/2021). 
\begin{center}\includegraphics[width=.5\textwidth]{pics/cvB0A7LmlZc.jpg} \\ 
\href{https://youtu.be/cvB0A7LmlZc}{\url{youtu.be/cvB0A7LmlZc}}\end{center}
% Shown is a family of triangles P1P2P3 inscribed in an outer circle C1 (black) with side P1P2 tangent to a 2nd circle C2 (light brown), side P1P3 to a 3rd circle C3 (dashed brown) in the pencil of (C1,C2), and with side P2P3 automatically tangent to to a circle C4 (dashed red) also in the pencil of C1,C2. They are called "quadricentric" since this family is defined by 4 non-concentric circles.

Left: the non-conic loci of the incenter X1 and the barycenter X2. Experimentally the former is always convex.

Right: the non-conic loci of the symmedian point X6, the Nagel point X8, and the Spieker center X10.
\subsection{Confocals with Two Caustics IV: two notable loci of the excenters (N=4, 6 caustics)}
\label{vid:wB9bVkY9rqU}
\noindent N=3, 2m26s (7/2021). 
\begin{center}\includegraphics[width=.5\textwidth]{pics/wB9bVkY9rqU.jpg} \\ 
\href{https://youtu.be/wB9bVkY9rqU}{\url{youtu.be/wB9bVkY9rqU}}\end{center}
% \input{descr/200_wB9bVkY9rqU}
\subsection{Bicentrics with Three Caustics II -- Non-conic loci of excenters and  incenter}
\label{vid:A_-U2VvM5kY}
\noindent N=3, 2m26s (7/2021). 
\begin{center}\includegraphics[width=.5\textwidth]{pics/A_-U2VvM5kY.jpg} \\ 
\href{https://youtu.be/A_-U2VvM5kY}{\url{youtu.be/A\_-U2VvM5kY}}\end{center}
% Shown is a family of triangles P1P2P3 inscribed in an outer circle C1 (black) with side P1P2 tangent to a 2nd circle C2 (light brown), side P1P3 to a 3rd circle C3 (dashed brown) in the pencil of (C1,C2), and with side P2P3 automatically tangent to to a circle C4 (dashed red) also in the pencil of C1,C2. They are called "quadricentric" since this family is defined by 4 non-concentric circles.

Left: the non-conic loci of the incenter X1 (dark green) and of the 3 excenters (light green), for the case where C4 (dashed red) has a greater-than-zero radius.

Right: C2C3 are chosen so that the envelope of C4 is a point. Even in this case the loci of the three excenter are non-conic (as is that of the incenter), though the locus of the excenter opposite to P1 is nearly a circle.
\subsection{Poncelet Triangle Families with Multiple Caustics (derived from the Poristic and Confocal Families)}
\label{vid:8HXgkuY-nFQ}
\noindent N=3, 2m26s (7/2021). 
\begin{center}\includegraphics[width=.5\textwidth]{pics/8HXgkuY-nFQ.jpg} \\ 
\href{https://youtu.be/8HXgkuY-nFQ}{\url{youtu.be/8HXgkuY-nFQ}}\end{center}
% \input{descr/202_8HXgkuY-nFQ}
\subsection{Bicentrics with Three Caustics III: loci of incenter \& barycenter over 4 possible triangle choices}
\label{vid:E1Rcu38MePQ}
\noindent N=3, 4m52s (7/2021). 
\begin{center}\includegraphics[width=.5\textwidth]{pics/E1Rcu38MePQ.jpg} \\ 
\href{https://youtu.be/E1Rcu38MePQ}{\url{youtu.be/E1Rcu38MePQ}}\end{center}
% \input{descr/190_E1Rcu38MePQ}
\subsection{Bicentrics with Three Caustics IV: loci of symmedian \& Nagel points over 4 triangle choices}
\label{vid:u1_uANWDNr8}
\noindent N=3, 3m13s (7/2021). 
\begin{center}\includegraphics[width=.5\textwidth]{pics/u1_uANWDNr8.jpg} \\ 
\href{https://youtu.be/u1_uANWDNr8}{\url{youtu.be/u1\_uANWDNr8}}\end{center}
% Consider a family of triangles P1P2P3 inscribed in an outer circle C (black), with P1P2 (resp. P1P3) tangent to a 1st (resp. 2nd) inner circular caustic C', brown (resp. C'', dashed brown). The last side, P2P3 will, by virtue of Poncelet's Closure Theorem, envelop yet a 3rd inner circular caustic C''' (dashed red).

There are four ways to choose the side of the 1st and 2nd caustic P1P2 and P1P3 will be tangent to. Interestingly, over the four choices, P2P3 will envelop only two distinct 3rd caustics C'''.

The video shows all choices superposed. In the left (resp. right) one can see the four distinct loci traced out by the symmedian point X6 (resp. Nagel point X8). Note: two of the X6 loci look fused, but there are actually 4 distinct ones.

\section{N-Point Porisms (3)}

\subsection{N-Point Porism I: three-point quasi-porisms}
\label{vid:ui0YmSqR-vI}
\noindent N=9,18,21, 5m15s (10/2021). 
\begin{center}\includegraphics[width=.5\textwidth]{pics/ui0YmSqR-vI.jpg} \\ 
\href{https://youtu.be/ui0YmSqR-vI}{\url{youtu.be/ui0YmSqR-vI}}\end{center}
% The video explores the quasi-porisms obtained when a sequence of chords is made to pass in round-robin fashion thru three fixed points in the interior of a circle, said points lying on the vertices of an equilateral whose circumcircle is concentric w said outer circle.

We obtain quasi-porisms when N is a multiple of 3 and greater than 8.
\subsection{N-Point Porism II: 3-point chord iteration: porism iff one point lies on mystery ellipse}
\label{vid:hvTYkkvhePQ}
\noindent N=9, 5m33s (10/2021). 
\begin{center}\includegraphics[width=.5\textwidth]{pics/hvTYkkvhePQ.jpg} \\ 
\href{https://youtu.be/hvTYkkvhePQ}{\url{youtu.be/hvTYkkvhePQ}}\end{center}
% \input{descr/205_hvTYkkvhePQ}
\subsection{N-Point Porism III: 3-point chord iteration \& elliptic loci of third point for various N}
\label{vid:a6zoq1YNPrw}
\noindent N=3,6,9,12,15,18, 21m58s (11/2021). 
\begin{center}\includegraphics[width=.5\textwidth]{pics/a6zoq1YNPrw.jpg} \\ 
\href{https://youtu.be/a6zoq1YNPrw}{\url{youtu.be/a6zoq1YNPrw}}\end{center}
% \input{descr/206_a6zoq1YNPrw}

\section{N=3 Loci (28)}

\subsection{Locus of several triangular centers is elliptic}
\label{vid:f84W2aVnMpU}
\noindent N=3, 1m13s (5/2019). 
\begin{center}\includegraphics[width=.5\textwidth]{pics/f84W2aVnMpU.jpg} \\ 
\href{https://youtu.be/f84W2aVnMpU}{\url{youtu.be/f84W2aVnMpU}}\end{center}
% \input{descr/217_f84W2aVnMpU}
\subsection{Triangular orbits' extouchpoints, Feuerbach point \& its anti-complement I}
\label{vid:1gYb5Y3-rQI}
\noindent N=3, 1m13s (5/2019). 
\begin{center}\includegraphics[width=.5\textwidth]{pics/1gYb5Y3-rQI.jpg} \\ 
\href{https://youtu.be/1gYb5Y3-rQI}{\url{youtu.be/1gYb5Y3-rQI}}\end{center}
% The Feuerbach point (shown as brown "F") is the tangency point between the incircle and the 9-point circle (not shown). The extouch points (green, t12, t23, t31) are the tangency points between each excenter and its corresponding side. All these four points describe an elliptic locus which is internally caustic to the orbits

More info: https://dan-reznik.github.io/Elliptical-Billiards-Triangular-Orbits/
\subsection{Locus of vertices of Feuerbach Triangle is non-elliptic}
\label{vid:YPz0_xbit2I}
\noindent N=3, 1m13s (5/2019). 
\begin{center}\includegraphics[width=.5\textwidth]{pics/YPz0_xbit2I.jpg} \\ 
\href{https://youtu.be/YPz0_xbit2I}{\url{youtu.be/YPz0\_xbit2I}}\end{center}
% \input{descr/218_YPz0_xbit2I}
\subsection{Locus of Bevan Point $X_{40}$ is similar to billiard}
\label{vid:NwPioKleiyU}
\noindent N=3, 1m12s (6/2019). 
\begin{center}\includegraphics[width=.5\textwidth]{pics/NwPioKleiyU.jpg} \\ 
\href{https://youtu.be/NwPioKleiyU}{\url{youtu.be/NwPioKleiyU}}\end{center}
% \input{descr/212_NwPioKleiyU}
\subsection{Triangular orbits' extouchpoints, Feuerbach point \& its anti-complement I}
\label{vid:NNOktM8tvPM}
\noindent N=3, 3m13s (6/2019). 
\begin{center}\includegraphics[width=.5\textwidth]{pics/NNOktM8tvPM.jpg} \\ 
\href{https://youtu.be/NNOktM8tvPM}{\url{youtu.be/NNOktM8tvPM}}\end{center}
% Elliptic billiard (a/b=1.5) shown black as well as its family of triangular (N=3) orbits. The three extouchpoints (where excircles touch each side of the triangle) have been bound to both sweep the caustic (in opposite direction) and to be congruent with tangency points between each side and caustic. The locus of the Feuerbach point F (where incircle and 9-pt circle touch) also sweeps the confocal caustic, albeit in a direction opposite to the extouchpoints. Finally, the anticomplement of F (shown as F bar), which is twice a reflection of F about the baricenter, has been found to sweep the billiard.

More info: https://dan-reznik.github.io/Elliptical-Billiards-Triangular-Orbits/
\subsection{Locus of Excentral and Anticomplementary Triangles and Objects I}
\label{vid:50dyxWJhfN4}
\noindent N=3, 1m37s (6/2019). 
\begin{center}\includegraphics[width=.5\textwidth]{pics/50dyxWJhfN4.jpg} \\ 
\href{https://youtu.be/50dyxWJhfN4}{\url{youtu.be/50dyxWJhfN4}}\end{center}
% The family of N=3 (triangular) orbits (black) for an elliptic billiard (a/b=1.5) is shown.

For each orbit its anticomplementary triangle T' is also shown (blue). This triangle has sides parallel to the orbit opposite ones; the orbit's vertices are each of its sides' medians.

Both the 9-point circle (pink, center C bar) and incircle (green, center I bar) for T' are shown. The former is known to coincide with the orbit's circumcircle/center.

Also shown is the Feuerbach point "F bar" for T', at the touchpoint between circle C bar and I bar. This is known to be X(100) of the orbit triangle (anticomplement of the Feuerbach point).

The amazing property about F' is that it sweeps the elliptic billiard exactly. Hoever, his video shows two new properties:

a) the locus of the anticomplementary triangle is a non-elliptical curve.
b) the 3 contact points of the incircle of T' (intouchpoints) also sweep the billiard.

This seems dual with the property that the Feuerbach point of the orbit sweeps the caustic and the 3 extouch points (where the excircles touch the sides) sweep the caustic.

More info: https://dan-reznik.github.io/Elliptical-Billiards-Triangular-Orbits/
\subsection{Locus of Excentral and Anticomplementary Triangles and Objects II}
\label{vid:xSnRd6WWiKc}
\noindent N=3, 9m37s (6/2019). 
\begin{center}\includegraphics[width=.5\textwidth]{pics/xSnRd6WWiKc.jpg} \\ 
\href{https://youtu.be/xSnRd6WWiKc}{\url{youtu.be/xSnRd6WWiKc}}\end{center}
% In both Top and Bottom animations an a/b=1.5 elliptic billiard is shown with a P1,P2,P3 orbit as well as its family of N=3 orbits.

Top: (i) the orbit's Feuerbach point X(11) (where incircle and 9-point circle meet) sweeps the caustic (proven). (ii) the three perpendicular dropped from the excircles onto the sides also sweep the caustic (Chasles' theorem, explained by you a few days ago). (iii) the anticomplement X(100) of the Feuerbach point (shown as an F with a bar on top) sweeps the billiard (proven). Fbar is simply a reflection of F about the barycenter (B), at twice the distance |FB|.

Bottom: the anticomplement of F is also the Feuerbach point of the orbit's anticomplementary triangle T', shown in blue for the same orbit position above. T' has sides parallel to the orbit, and passes thru the orbit's vertices (it's a similar triangle) at each of its sides midpoints (each of its sides is twice each of the orbit's).

To get its Feuerbach point, I compute the point of tangency between its incircle and its 9-point-circle (the latter is elegantly congruent w the orbit's circumcircle).

We noticed 3 amazing properties regarding the above picture:

a) non-ellipticity: the locus of the vertices of T' is not elliptic (drawn dashed blue, with humps on its north and bottom areas).
b) sweeps the billiard: the contact points of its incircle are magically where the sides of T' cross the billiard. Therefore, like Fbar, they will also sweep the billiard.
c) retrogade motion: all motion is driven by a CCW motion of P1 on the billiard. We notice Fbar moves along the billiard monotonically in the *opposite* direction. However, the three intouch points move mostly CCW but with retrograde phases, though they never leave the billiard. This is mesmerizing.

We think property (b) is dual to the fact that on the first picture, if you "drop perpendiculars" from the excircles onto the sides they are where the sides touch the caustic. Likewise, if we compute the feet from the incircle of T', we get the billiard (seems like a dual of Chasles' thm).

Schematically, with ":=:" representing some (projective?) duality or homeomorphic "action":

Orbit :=: Anticomplementary triangle
Caustic :=: Billiard  
Feuerbach of orbit computes caustic :=: Feuerbach of anticomplementary computes billiard
Extouchpoints of orbit compute caustic :=: Intouchpoints of anticomplementary compute the billiard

More info: https://dan-reznik.github.io/Elliptical-Billiards-Triangular-Orbits/
\subsection{3-Periodics and Derived Triangles}
\label{vid:xyroRTEVNDc}
\noindent N=3, 3m12s (6/2019). 
\begin{center}\includegraphics[width=.5\textwidth]{pics/xyroRTEVNDc.jpg} \\ 
\href{https://youtu.be/xyroRTEVNDc}{\url{youtu.be/xyroRTEVNDc}}\end{center}
% A black elliptic billiard (a/b=1.5) is shown as well as the family of its N=3 orbits (blue). The confocal caustic to all triangular orbits for this billiard is shown in dashed black. Also shown are 6 triangles derived from the orbit: the medial (red), orthic (orange), intouch (green), excentral (dashed blue), extouch (dashed green), anticomplementary (dashed red) and feuerbach (brown).

https://dan-reznik.github.io/Elliptical-Billiards-Triangular-Orbits/
\subsection{Anticomplementary triangle intouchpoints I}
\label{vid:NzGKU75-Fuo}
\noindent N=3, 3m13s (6/2019). 
\begin{center}\includegraphics[width=.5\textwidth]{pics/NzGKU75-Fuo.jpg} \\ 
\href{https://youtu.be/NzGKU75-Fuo}{\url{youtu.be/NzGKU75-Fuo}}\end{center}
% \input{descr/208_NzGKU75-Fuo}
\subsection{Anticomplementary triangle intouchpoints II}
\label{vid:gwfx6LDJnsE}
\noindent N=3, 3m13s (6/2019). 
\begin{center}\includegraphics[width=.5\textwidth]{pics/gwfx6LDJnsE.jpg} \\ 
\href{https://youtu.be/gwfx6LDJnsE}{\url{youtu.be/gwfx6LDJnsE}}\end{center}
% Family of N=3 orbits shown for a/b=1.5 elliptic billiard. Two phenomena are shown:

- the 3 vertices of contact triangle of orbit's anticomplementary triangle lie on the billiard.
- the anticomplementary's circumbilliard is axis aligned with the billiard and its center is, as usual, its mittenpunkt, which is congruent with the orbit's Gergonne point X(7) (i.e., the gergonne is the orbit's mittenpunkt -- the origin -- reflected about the baricenter at twice the distance, which is where the Gergonne point lies, collinear with M and B).

Note: D (we call the "Darboux" point, X144) is the perspector between the anticomplementary and contact triangles (and the orbit's extouch triangle, extouch pts shown green on caustic). It is also the anticomplement (with respect to the baricenter) of the Gergonne point, X7. Notice D, M, B, and the Gergonne, are collinear.

https://dan-reznik.github.io/Elliptical-Billiards-Triangular-Orbits/
\subsection{Anticomplementary, Medial Triangles and the Intouch Triangle I}
\label{vid:xyHUwpvAj3g}
\noindent N=3, 3m13s (6/2019). 
\begin{center}\includegraphics[width=.5\textwidth]{pics/xyHUwpvAj3g.jpg} \\ 
\href{https://youtu.be/xyHUwpvAj3g}{\url{youtu.be/xyHUwpvAj3g}}\end{center}
% \input{descr/210_xyHUwpvAj3g}
\subsection{Locus of $X_{59}$ has 4 self-intersections}
\label{vid:pl_PqSuhlx0}
\noindent N=3, 3m13s (1/2020). 
\begin{center}\includegraphics[width=.5\textwidth]{pics/pl_PqSuhlx0.jpg} \\ 
\href{https://youtu.be/pl_PqSuhlx0}{\url{youtu.be/pl\_PqSuhlx0}}\end{center}
% Two elliptic billiards are shown: a/b=1.24 (left), and a/b=1.58005 (right). The locus of X(59) is shown [1] over the family of N=3 orbits. This locus has four self-intersections, two on the x and two on the y axis. It is at least a sextic curve (intersect a line parallel and near the y axis w/ curve and get 6 intersections).

When a/b is less (resp. more) than sqrt(2*sqrt(2)-1)) = 1.352..., the N=3 family only contains acute (resp. both acute and obtuse) triangles [2]. Therefore the left (resp. right) billiard only contains acute (resp. both acute and obtuse) triangles.

The right billiard is very special as when X(59) crosses one of the self-intersections, the orbit triangle is a perfect right triangle (X(4), shown, will be at an alternate vertex).

Open challenges: let one vertex P1 of the orbit be give by P1(t) = (a cos(t), b sin(t)).

1) Compute expressions for t (in terms of a,b) such that X(59) is at the lower self-intersections w/ the y-axis.
2)  Compute an expression for a/b such that when X(59) is at the lower self-intersection with the y-axis, the orbit is a right-triangle. (we know a/b~1.58). "t" for this position should be obtainable from (1).

[1] Clark Kimberling, "Encyclopedia of Triangle Centers", https://faculty.evansville.edu/ck6/encyclopedia/ETC.html
[2] Dan Reznik, Ronaldo Garcia and Jair Koiller, "New Properties of Triangular Orbits in Elliptic Billiards", 2020. In preparation.
\subsection{Locus of Bevan Point $X_{40}$ identical to billiard when a/b=golden ratio}
\label{vid:rg28gGr-Qeo}
\noindent N=3, 3m13s (1/2020). 
\begin{center}\includegraphics[width=.5\textwidth]{pics/rg28gGr-Qeo.jpg} \\ 
\href{https://youtu.be/rg28gGr-Qeo}{\url{youtu.be/rg28gGr-Qeo}}\end{center}
% \input{descr/221_rg28gGr-Qeo}
\subsection{Anticomplementary, Medial Triangles and the Intouch Triangle II}
\label{vid:e-mToZlkHtc}
\noindent N=3, 4m49s (1/2020). 
\begin{center}\includegraphics[width=.5\textwidth]{pics/e-mToZlkHtc.jpg} \\ 
\href{https://youtu.be/e-mToZlkHtc}{\url{youtu.be/e-mToZlkHtc}}\end{center}
% An a/b=1.5 Elliptic Billiard (EB). is shown (blue) as well as the 1d family of 3-periodic orbits (blue, filled gray). Also drawn are the orbit's Anticomplementary Triangle (ACT, blue) and Medial Triangle (red).

The ACT circubmilliard (CB, blue) is an ellipse (blue) centered on the ACT Mittenpunkt X7 (anticomplement of X9) inside which the ACT is a billiard orbit. The ACT is homothetic (double-scale and flipped) to the orbit, so its CB is twice the size of the EB, and their axes are parallel.

The ACT's Incircle (blue) passes through the ACT's Intouchpoints. Surprisingly, these lie dynamically on the EB. The ACT Incenter is X8 (anticomplement of X1 not shown). X9,X2,X7 are collinear.

The Medial Triangle (red) is formed by the sides midpoints. It is also homothetic to the orbit (half-scale and flipped) so its CB will be half the size of the EB and their axes will be parallel. The Medial CB is centered on X142, the complement of X9. One can regard the orbit as the Medial's ACT, therefore the latter's Incircle (green) will have Intouchpoints which lie on the Medial EB. X142, X2 and X9 are collinear.

Also shown is the line connecting X_i, i=7,142,2,9,144. Their consecutive distances are proportional to 3:1:2:6. X144 was included since it is the perspector of the ACT and its Intouch Triangle.
\subsection{Circular loci for $X_{140}$ and $X_{547}$ over 3-periodics in the Elliptic Billiard}
\label{vid:_umuHLl9cCU}
\noindent N=3, 5m34s (7/2020). 
\begin{center}\includegraphics[width=.5\textwidth]{pics/_umuHLl9cCU.jpg} \\ 
\href{https://youtu.be/_umuHLl9cCU}{\url{youtu.be/\_umuHLl9cCU}}\end{center}
% \input{descr/219__umuHLl9cCU}
\subsection{The locus of $X_{140}$ is a circle over 3-periodics in the Elliptic Billiard}
\label{vid:4g5G9eluxJo}
\noindent N=3, 5m34s (7/2020). 
\begin{center}\includegraphics[width=.5\textwidth]{pics/4g5G9eluxJo.jpg} \\ 
\href{https://youtu.be/4g5G9eluxJo}{\url{youtu.be/4g5G9eluxJo}}\end{center}
% Consider an elliptic billiard with semiaxes a,b and its family of 3-periodics. 

When a/b≈2.81652, the locus of X(140), the midpoint of X(3) and X(5) [1], is a perfect *circle* of radius ≈0.59282.

When a/b is less (resp. more) than that threshold, the locus if a wide (resp. tall) ellipse. 

No other triangle center reviewed so far has been found to produce a perfectly circular locus when a/b is *not* 1. Indeed, several triangle centers do describe circle or point loci when the billiard itself is a circle. So X(140) is very unusual and could be unique.

For more of our own research on the topic see [2].

[1] Clark Kimberling, Encyclopedia of Triangle Centers (ETC), https://faculty.evansville.edu/ck6/encyclopedia/ETC.html
[2] D. Reznik, R. Garcia, and J. Koiller, "N-Periodics in the Elliptic Billiard", https://dan-reznik.github.io/Elliptical-Billiards-Triangular-Orbits/videos.html
\subsection{I love loci: classic triangle centers as one vertex moves parallel to the base}
\label{vid:Y50RFjhvsAo}
\noindent N=3, 1m56s (12/2020). 
\begin{center}\includegraphics[width=.5\textwidth]{pics/Y50RFjhvsAo.jpg} \\ 
\href{https://youtu.be/Y50RFjhvsAo}{\url{youtu.be/Y50RFjhvsAo}}\end{center}
% Consider a family of triangles (blue) with two "base" vertices P1,P2 stationary a third one P3 which moves parallel to the base (dashed blue), at some distance y0.

The video shows the loci of the incenter X1, barycenter X2, circumcenter X3, orthocenter X4, and de Longchamps point X20. Over the above triangle family these execute an ellipse-like curve, a straight line, a vertical line, a down parabola, and an up parabola, respectively.

Notice the loci of X1,X2,X4,X20 all meet at a particular point.
\subsection{Triangle family with 2 fixed vertices whose incenter $X_{1}$ sweeps a circle}
\label{vid:MPKt7Q2DhJc}
\noindent N=3, 1m45s (12/2020). 
\begin{center}\includegraphics[width=.5\textwidth]{pics/MPKt7Q2DhJc.jpg} \\ 
\href{https://youtu.be/MPKt7Q2DhJc}{\url{youtu.be/MPKt7Q2DhJc}}\end{center}
% \input{descr/224_MPKt7Q2DhJc}
\subsection{Loci of Triangle Centers of Poncelet 3-Periodics IV: Outer Elllipse, Inner Non-Concentric Circle}
\label{vid:w7sZ5O8k4xU}
\noindent N=3, 2m25s (3/2021). 
\begin{center}\includegraphics[width=.5\textwidth]{pics/w7sZ5O8k4xU.jpg} \\ 
\href{https://youtu.be/w7sZ5O8k4xU}{\url{youtu.be/w7sZ5O8k4xU}}\end{center}
% \input{descr/229_w7sZ5O8k4xU}
\subsection{Loci of Triangle Centers of Poncelet 3-Periodics: 
I Generic Pair, Elliptic Loci}
\label{vid:p1medAei_As}
\noindent N=3, 2m25s (3/2021). 
\begin{center}\includegraphics[width=.5\textwidth]{pics/p1medAei_As.jpg} \\ 
\href{https://youtu.be/p1medAei_As}{\url{youtu.be/p1medAei\_As}}\end{center}
% Shown is the family of Poncelet 3-periodics (blue) interscribed between two non-concentric, unaligned ellipses, as well as the elliptic loci of Xk, k=2,3,4,5,20. A few remarks from [1]:

a) The condition for the locus of a triangle center to be an ellipse is that it is a fixed linear combination of X2 and X3, i.e., it will lie on the Euler Line (dashed gray).

b) the centers of the ellipses are collinear (magenta line). 

c) The loci of both X2 and X4 are always axis aligned with the outer ellipse. 

[1] M Helman, D Laurain, R Garcia, and D Reznik, "Invariant Center Power and Elliptic Loci of Poncelet Triangles", 2021. arXiv:2102.09438
\subsection{Loci of Triangles Centers of Poncelet 3-Periodics II: Pair with Circumcircle, Circular Loci}
\label{vid:HXgJQo2UT_8}
\noindent N=3, 2m25s (3/2021). 
\begin{center}\includegraphics[width=.5\textwidth]{pics/HXgJQo2UT_8.jpg} \\ 
\href{https://youtu.be/HXgJQo2UT_8}{\url{youtu.be/HXgJQo2UT\_8}}\end{center}
% Shown is the family of Poncelet 3-periodics (blue) inscribed in a circle and circumscribing a non-concentric ellipse, as as well as the circular loci of Xk, k=2,3,4,5,20. The left (resp. right) picture shows a choice of inner ellipse which contains (does not contain) the center of the outer circle. In the first (resp. second) case, all (resp. some) 3-periodics are acute, and the locus of X4 and X20 is interior (resp. both interior and exterior) to the outer circle).

A few remarks from [1].

a) The condition for the locus of a triangle center to be an ellipse is that it is a fixed linear combination of X2 and X3, i.e., it will lie on the Euler Line (dashed gray). In the present case (outer circle, inner ellipse), the loci degenerate to circles.

b) the centers of the ellipses are collinear (magenta line). 

c) If the caustic contains (resp. does not contain) the center of the outer circle, all (resp. some) 3-periodics are acute (resp. and some are obtuse).

[1] M Helman, D Laurain, R Garcia, and D Reznik, "Invariant Center Power and Elliptic Loci of Poncelet Triangles", 2021. arXiv:2102.09438
\subsection{Poncelet 3-Periodics in a Non-Concentric, Unaligned Ellipse Pair}
\label{vid:bjHpXVyXXVc}
\noindent N=3, 2m25s (3/2021). 
\begin{center}\includegraphics[width=.5\textwidth]{pics/bjHpXVyXXVc.jpg} \\ 
\href{https://youtu.be/bjHpXVyXXVc}{\url{youtu.be/bjHpXVyXXVc}}\end{center}
% \input{descr/232_bjHpXVyXXVc}
\subsection{Poncelet 3-Periodics in Generic Pair and Affine Image with Circumcircle (Blaschke Parametrization)}
\label{vid:6xSFBLWIkTM}
\noindent N=3, 2m25s (3/2021). 
\begin{center}\includegraphics[width=.5\textwidth]{pics/6xSFBLWIkTM.jpg} \\ 
\href{https://youtu.be/6xSFBLWIkTM}{\url{youtu.be/6xSFBLWIkTM}}\end{center}
% Shown (left) is a 1d family of 3-Periodics interscribed in a pair of ellipses as well as (right) their affine image such that the outer ellipse is sent to (the complex) circle. The vertices of the original family (Pi) are sent to a triple of complex numbers z1, z2, z3. A parametrization described in [1] is used by us to arrive at several results concerning loci of triangle centers [2].

[1] U. Daepp, P. Gorkin, A. Shaffer, and K. Voss, "Finding Ellipses: what Blaschke Products, Poncelet’s Theorem, and the Numerical Range Know about Each Other, MAA Press/AMS, 2019, isbn 9781470443832.

[2] M Helman, D Laurain, R Garcia, and D Reznik, "Invariant Center Power and Elliptic Loci of Poncelet Triangles", 2021. arXiv:2102.09438
\subsection{Loci of Triangles Centers of Poncelet 3-Periodics III: Concentric Tilted Ellipse Pair}
\label{vid:hpb7ZgKWjUY}
\noindent N=3, 2m25s (3/2021). 
\begin{center}\includegraphics[width=.5\textwidth]{pics/hpb7ZgKWjUY.jpg} \\ 
\href{https://youtu.be/hpb7ZgKWjUY}{\url{youtu.be/hpb7ZgKWjUY}}\end{center}
% \input{descr/234_hpb7ZgKWjUY}
\subsection{Elliptic Billiard 3-Periodics: triangle centers whose loci sweep 2 circles and 2 segments}
\label{vid:haFTsq5UyK4}
\noindent N=3, 2m25s (3/2021). 
\begin{center}\includegraphics[width=.5\textwidth]{pics/haFTsq5UyK4.jpg} \\ 
\href{https://youtu.be/haFTsq5UyK4}{\url{youtu.be/haFTsq5UyK4}}\end{center}
% A confocal pair of ellipses is shown which admits a 1d family of Poncelet 3-periodics (since the pair is confocal these are known as "elliptic billiard" trajectories).

The video shows an interesting property: over this family, there are 4 individual triangle centers at fixed linear combinations of X2 and X3 such that (i) two of them sweep perpendicular segments, and (ii) the other two sweep circles.

Details for their construction to appear in an upcoming (2021) publication w M. Helman, D. Laurain, R. Garcia, and D. Reznik.
\subsection{Gallery of Artful, Colorful Triangle Loci. Music by Tchaikovsky}
\label{vid:l-O5UT8tpuw}
\noindent N=3, 15m29s (5/2021). 
\begin{center}\includegraphics[width=.5\textwidth]{pics/l-O5UT8tpuw.jpg} \\ 
\href{https://youtu.be/l-O5UT8tpuw}{\url{youtu.be/l-O5UT8tpuw}}\end{center}
% Shown are some ~180 (colored) loci of triangle centers and vertices over certain (mostly Poncelet) triangle families.

Original gallery: https://docs.google.com/presentation/d/1pHPrM3wdiabtE1_gzp5ZUAzMZku1GB0-u0TNQBxEGhM/edit?usp=sharing

For information on the the geometry used to create these loci, see:

https://dan-reznik.github.io/ellipse-mounted-triangles/

The JS app was co-developed with Iverton Darlan in the 2nd half of 2020.
\subsection{Family of Poncelet Triangles between Concentric, Axis-Parallel Ellipses centered on $X_{1249}$}
\label{vid:QQSN_ndDJQk}
\noindent N=3, 2m25s (5/2021). 
\begin{center}\includegraphics[width=.5\textwidth]{pics/QQSN_ndDJQk.jpg} \\ 
\href{https://youtu.be/QQSN_ndDJQk}{\url{youtu.be/QQSN\_ndDJQk}}\end{center}
% \input{descr/226_QQSN_ndDJQk}
\subsection{Loci of Incenter and Excenters over Poncelet 3-Periodics in a Generic Ellipse Pair}
\label{vid:z7qDgJEgPVY}
\noindent N=3, 2m25s (6/2021). 
\begin{center}\includegraphics[width=.5\textwidth]{pics/z7qDgJEgPVY.jpg} \\ 
\href{https://youtu.be/z7qDgJEgPVY}{\url{youtu.be/z7qDgJEgPVY}}\end{center}
% A family of Poncelet 3-periodics (blue) is shown interscribed between two generic nested ellipses. Recall the incenter X1 is where the internal bisectors concur. The excentral triangle (solid green) is bounded by the external bisectors of the 3-periodic. Its vertices are known as the excenters. 

The video shows the loci of both the incenter and the excenters (solid green, and dashed green, respectively). Notice the former bounds a convex region while the latter doesn't.

\section{N>3 Periodics (26)}

\subsection{4-periodics and Monge's Orthoptic Circle II}
\label{vid:BTSNc_YN0lo}
\noindent N=4, 2m25s (5/2019). 
\begin{center}\includegraphics[width=.5\textwidth]{pics/BTSNc_YN0lo.jpg} \\ 
\href{https://youtu.be/BTSNc_YN0lo}{\url{youtu.be/BTSNc\_YN0lo}}\end{center}
% Shown is the family of non-intersecting 4-periodic orbits in the elliptic billiard (they are all parallelograms!), a/b=1.5, as well as its excentral quadrangle, formed by the intersection of the lines tangent to the ellipse at consecutive  orbit vertices (a perfect rectangle). Not shown is the locus of the rectangle vertices: a perfect circle, known as Monge's Orthoptic Circle. Similar to the mittenpunkt X(9) for triangular orbits, a middlespoint at the origin can be constructed by drawing a line from each of the exquadrangles' vertices (virtual "excenters") through the the median of the corresponding orbit edge. Nice!

More Info: https://dan-reznik.github.io/Elliptical-Billiards-Triangular-Orbits/
\subsection{4-periodics: Loci of Triangle Centers for Vertex Triad}
\label{vid:y2bnml8heGg}
\noindent N=4, 1m13s (5/2019). 
\begin{center}\includegraphics[width=.5\textwidth]{pics/y2bnml8heGg.jpg} \\ 
\href{https://youtu.be/y2bnml8heGg}{\url{youtu.be/y2bnml8heGg}}\end{center}
% \input{descr/237_y2bnml8heGg}
\subsection{5-periodics: Loci of Subtriangles (123 and 124)}
\label{vid:4lj9yQ-e_cE}
\noindent N=5, 1m13s (5/2019). 
\begin{center}\includegraphics[width=.5\textwidth]{pics/4lj9yQ-e_cE.jpg} \\ 
\href{https://youtu.be/4lj9yQ-e_cE}{\url{youtu.be/4lj9yQ-e\_cE}}\end{center}
% Shown side by side are the family of pentagonal orbits in an elliptic billiard (a/b=1.5). On the left you see loci for P1,P2,P3, of the orbit, and on the right you see loci for P1,P2,P4.
\subsection{5-periodics: locus of P1, P2, P3 triangle}
\label{vid:yQMOtAGdrqA}
\noindent N=5, 1m13s (5/2019). 
\begin{center}\includegraphics[width=.5\textwidth]{pics/yQMOtAGdrqA.jpg} \\ 
\href{https://youtu.be/yQMOtAGdrqA}{\url{youtu.be/yQMOtAGdrqA}}\end{center}
% \input{descr/239_yQMOtAGdrqA}
\subsection{5-periodics: locus of P1, P2, P4 triangle}
\label{vid:2MA1h-dMnw8}
\noindent N=5, 1m13s (5/2019). 
\begin{center}\includegraphics[width=.5\textwidth]{pics/2MA1h-dMnw8.jpg} \\ 
\href{https://youtu.be/2MA1h-dMnw8}{\url{youtu.be/2MA1h-dMnw8}}\end{center}
% \input{descr/240_2MA1h-dMnw8}
\subsection{Upright 5-periodic family}
\label{vid:RQE1s2siPSo}
\noindent N=5, 2m25s (5/2019). 
\begin{center}\includegraphics[width=.5\textwidth]{pics/RQE1s2siPSo.jpg} \\ 
\href{https://youtu.be/RQE1s2siPSo}{\url{youtu.be/RQE1s2siPSo}}\end{center}
% Shown in gray: elliptic billiard (a=1.5, b=1), and (dashed) the family of pentagonal orbits (N=5), with each vertex identified: ABCDE.

Shown in red: an "upright" version of ABCDE, where A is "pinned" to (0,1) and its bisector is fixed to the vertical. Vertices of this "normalized" pentagon are shown A', B', etc. The barycenter of this pentagon is labeled G. Also shown are the loci of the four "free" vertices of the normalized pentagon, as well as that of G.

More Info: https://dan-reznik.github.io/Elliptical-Billiards-Triangular-Orbits/
\subsection{6-periodic family}
\label{vid:YZfFGew4azI}
\noindent N=6, 1m12s (6/2019). 
\begin{center}\includegraphics[width=.5\textwidth]{pics/YZfFGew4azI.jpg} \\ 
\href{https://youtu.be/YZfFGew4azI}{\url{youtu.be/YZfFGew4azI}}\end{center}
% The family of hexagonal (N=6) orbits is shown in an elliptic billiard. The black dot drawn at the origin corresponds to the stationary position of the vertex, perimeter, and area centroids of all orbits. The triangle formed by vertices 1,3,5 is shown shaded, and the location (and locus) of its major triangular centers is also shown, revealing elliptical paths.

More Info: https://dan-reznik.github.io/Elliptical-Billiards-Triangular-Orbits/
\subsection{Enagramma Mysticum: loci of side intersections}
\label{vid:ECo1hTCVuDg}
\noindent N=9, 2m25s (6/2019). 
\begin{center}\includegraphics[width=.5\textwidth]{pics/ECo1hTCVuDg.jpg} \\ 
\href{https://youtu.be/ECo1hTCVuDg}{\url{youtu.be/ECo1hTCVuDg}}\end{center}
% The video shows the loci of intersections of side pairs in an an N=9 elliptic billiard orbit (a/b=1.5). These form three confocal elliptic loci as the orbit family is swept.

https://dan-reznik.github.io/Elliptical-Billiards-Triangular-Orbits/
\subsection{Octagramma Mysticum I}
\label{vid:mDomB-_GiNA}
\noindent N=8, 2m24s (6/2019). 
\begin{center}\includegraphics[width=.5\textwidth]{pics/mDomB-_GiNA.jpg} \\ 
\href{https://youtu.be/mDomB-_GiNA}{\url{youtu.be/mDomB-\_GiNA}}\end{center}
% \input{descr/246_mDomB-_GiNA}
\subsection{Octagramma Mysticum II}
\label{vid:xgdgx0erM58}
\noindent N=8, 2m24s (6/2019). 
\begin{center}\includegraphics[width=.5\textwidth]{pics/xgdgx0erM58.jpg} \\ 
\href{https://youtu.be/xgdgx0erM58}{\url{youtu.be/xgdgx0erM58}}\end{center}
% In 1871 Cayley published a paper about a very curious property of octagons inscribed in conics, known as "Octagramma Mysticum", also explained here: https://arxiv.org/pdf/1209.4795.pdf

Namely, if an octagon is inscribed in an ellipse, the intersection of lines passing thru every other side will all land on the same conic, e.g., an outer ellipse. 

The video shows what happens when the octagon is an N-periodic *orbit* of an elliptic billiard (a/b=1.5).

- We found the loci of the intersection points will be an ellipse *confocal* with the billiard.

- The video shows that in fact N=5,6,7 will also produce elliptic foci which are confocal.

- We also found that intersections of  P(i,i+1) with P(i+3,i+4) also yield confocal elliptic loci, larger than the former ones.

https://dan-reznik.github.io/Elliptical-Billiards-Triangular-Orbits/
\subsection{4-periodics and Monge's Orthoptic Circle I}
\label{vid:9fI3iM2jrmI}
\noindent N=4, 2m25s (7/2019). 
\begin{center}\includegraphics[width=.5\textwidth]{pics/9fI3iM2jrmI.jpg} \\ 
\href{https://youtu.be/9fI3iM2jrmI}{\url{youtu.be/9fI3iM2jrmI}}\end{center}
% Shown is an a/b=1.5 elliptic billiard with its family of N=4 (quadrangular orbits). These are known to be perfect parallelograms. Also known is the fact that their tangential (excentral) polygon is a rectangle and the locus of the latter's vertices is circular. This is known as Monge's Orthoptic Circle and/or Monge's Theorem, after French Mathematician Gaspar Monge (1746-1818).

More Info: https://dan-reznik.github.io/Elliptical-Billiards-Triangular-Orbits/
\subsection{Family of Orbits and Their Caustics}
\label{vid:Y3q35DObfZU}
\noindent N=3 to 6, 1m37s (9/2019). 
\begin{center}\includegraphics[width=.5\textwidth]{pics/Y3q35DObfZU.jpg} \\ 
\href{https://youtu.be/Y3q35DObfZU}{\url{youtu.be/Y3q35DObfZU}}\end{center}
% \input{descr/243_Y3q35DObfZU}
\subsection{Mittenpunkt-like Construction of a Stationary Point}
\label{vid:TV2p7fPlYfE}
\noindent N=4 to 7, 2m25s (9/2019). 
\begin{center}\includegraphics[width=.5\textwidth]{pics/TV2p7fPlYfE.jpg} \\ 
\href{https://youtu.be/TV2p7fPlYfE}{\url{youtu.be/TV2p7fPlYfE}}\end{center}
% \input{descr/245_TV2p7fPlYfE}
\subsection{Generalized Mittenpunkt and On-Caustic Extouchpoints}
\label{vid:Bpc-MrR2IMc}
\noindent N=4,5, 1m37s (9/2019). 
\begin{center}\includegraphics[width=.5\textwidth]{pics/Bpc-MrR2IMc.jpg} \\ 
\href{https://youtu.be/Bpc-MrR2IMc}{\url{youtu.be/Bpc-MrR2IMc}}\end{center}
% N=4 and N=5 orbit families are shown in two Elliptic Billiards with a/b=1.5. Their Tangential Polygon is shown (green). whose vertices are intersections of tangents to consecutive orbit's vertices.  Two phenomena are illustrated:

1) Generalized N=3 Mittenpunkt: (dashed red) lines drawn from each Tangential vertex through the corresponding sides' midpoints all concur at the Billiard Center

2) Generalized N=3 Extouchpoints: the feet of (dashed green) perpendiculars dropped from each tangential vertex to the corresponding side is congruent with the point of tangency of that side with the confocal caustic, i.e., the locus of these feet is the caustic.

More info: https://dan-reznik.github.io/Elliptical-Billiards-Triangular-Orbits/
\subsection{Ellipse-Inscribed Parallelogram: invariants of the Pedal Polygon with respect to boundary points I}
\label{vid:7eUQQgR-w3c}
\noindent N=4, 5m25s (6/2020). 
\begin{center}\includegraphics[width=.5\textwidth]{pics/7eUQQgR-w3c.jpg} \\ 
\href{https://youtu.be/7eUQQgR-w3c}{\url{youtu.be/7eUQQgR-w3c}}\end{center}
% \input{descr/252_7eUQQgR-w3c}
\subsection{Ellipse-Inscribed Parallelogram: invariants of the Pedal Polygon with respect to boundary points II}
\label{vid:sViggJv4xyQ}
\noindent N=4, 3m30s (6/2020). 
\begin{center}\includegraphics[width=.5\textwidth]{pics/sViggJv4xyQ.jpg} \\ 
\href{https://youtu.be/sViggJv4xyQ}{\url{youtu.be/sViggJv4xyQ}}\end{center}
% Let E be an ellipse (black) with a,b semiaxes, and P be a point on its boundary. Consider a parallelogram Q (red) inscribed in E [note 1]. Drop perpendiculars from P onto the sides of Q, let their feet define a "pedal polygon"  Q' [note 2], shown black and pink-filled in the video.

Observation: for a given Q, the area of Q' is invariant over all P on the boundary of E. This is cool! Could be obvious by the way. An elegant one-line proof has been kindly offered by A. Akopyan:

The area of quadrilateral is the product of diagonals by sin of angle between them. So for any fixed parallelogram and any point P it is constant

note 1: this is a 1-d poncelet family w a confocal caustic, i.e., the parallelograms are 4-periodic billiard trajectories within E
note 2: Alternate pairs of feet are collinear with P (since alternate sides of Q are parallel). Let these feet defined a "pedal polygon"  (black, pink-filled).
\subsection{Elliptic Billiard with Perpendicular Reflection Rule}
\label{vid:0Qqi7ubS9Bw}
\noindent N=n/a, 28m45s (6/2020). 
\begin{center}\includegraphics[width=.5\textwidth]{pics/0Qqi7ubS9Bw.jpg} \\ 
\href{https://youtu.be/0Qqi7ubS9Bw}{\url{youtu.be/0Qqi7ubS9Bw}}\end{center}
% Consider an elliptic billiard w/ the following reflection rule: at every bounce proceed perpendicular to the previous segment. These are known to be integrable [1].

In this video I manually stumble upon a few periodic trajectories which emerge from such a reflection rule, some displaying quite beautiful geometry.

[1] Daniel Genin, Boris Khesin and Serge Tabachnikov, "Geodesics on an ellipsoid in Minkowski space", 2007
\subsection{Invariant Area Ratios to Minimum-Area Steiner Pedal Polygons}
\label{vid:f0JwRlu7iaY}
\noindent N=5, 6m11s (7/2020). 
\begin{center}\includegraphics[width=.5\textwidth]{pics/f0JwRlu7iaY.jpg} \\ 
\href{https://youtu.be/f0JwRlu7iaY}{\url{youtu.be/f0JwRlu7iaY}}\end{center}
% Given a polygon P with vertices Pi, and internal angles θi, i=1,2,...,N, its Steiner Curvature Centroid K is the weighted average of its vertices with weights equal to sin(2θi). In 1825 eminent swiss mathematician Jakob Steiner not only discovered K (which he calls "Krümmungs Schwerpunkt") but also showed that the pedal polygon Q of P wrt to K has extremal area over all possible pedals. It is minimal (resp. maximal) when the sum of sin(2θi) is positive (resp. negative) [1]. 

In this video I am investigating said pedal polygons with computed over the family of N-periodics in the elliptic billiard (EB) which we shall call P. Let P' and P'' denote the tangent (aka outer) and inner polygons to the N-periodic. Let their areas be denoted A, A', A''. Let the curvature centroids be denoted K, K', K'', and the area of the pedal polygons with respect to the latter be Ak, Ak', Ak''.  The following observations have been experimentally made, for more details refer to [2]:

a) When N is even: K,K',K'' are stationary at the EB center.
b) When N is odd, the 3 centroids sweep each a numerically-perfect ellipse.

For both cases the following are suprisingly invartiant:

* A/Ak
* A'/Ak'
* A''/Ak''

Since A'/A and A/A'' are invariant, this also implies Ak'/Ak, Ak/Ak'', and Ak'/Ak'' are invariant.

References

[1] J. Steiner, "Über den Krümmungs-Schwerpunkt ebener Curven", 1838. [about the curvature centroids of plane curves].
[2] D. Reznik, R. Garcia, and J. Koiller, "Forty New Invariants of N-Periodics in the Elliptic Billiard", 2020. https://arxiv.org/abs/2004.12497
\subsection{Circumcircles of Focus with Consecutive Vertices Homothetic to Focus Antipedal}
\label{vid:kVxh5jfZb9Q}
\noindent N=5,6, 3m20s (10/2020). 
\begin{center}\includegraphics[width=.5\textwidth]{pics/kVxh5jfZb9Q.jpg} \\ 
\href{https://youtu.be/kVxh5jfZb9Q}{\url{youtu.be/kVxh5jfZb9Q}}\end{center}
% \input{descr/250_kVxh5jfZb9Q}
\subsection{Incenters of Focus Triads: Invariant Area Ratio to N-Periodic and Elliptic Locus}
\label{vid:ehnbRnCUmS0}
\noindent N=5, 2m53s (10/2020). 
\begin{center}\includegraphics[width=.5\textwidth]{pics/ehnbRnCUmS0.jpg} \\ 
\href{https://youtu.be/ehnbRnCUmS0}{\url{youtu.be/ehnbRnCUmS0}}\end{center}
% After an idea by Peter Roitman [2].

Consider the family of N-periodics (blue) in the Elliptic Billiard E (black) w semiaxes a,b and foci f1,f2.

Let P(i), i=1,2,...,N denote its vertices, in the video N=5, a/b=2. Let A denote its (variable) area.

Consider the polygon (dark green) whose vertices Q(i) are the incenters of triads [P(i),P(i+1),f1], i=1,...N. Let A' denote its (variable) area. This is reminiscent of the so-called "Japanese Theorem" which we studied in [1] for a bicentric Poncelet pair. The incircles of each triad are shown (dashed green) as well as the "focal" spokes (lines connecting f1 to the vertices).

The video shows the following two new properties:

1) Over the family, the ratio A/A' is invariant, in fact for any choice of N and a/b.

2) The locus of the Q(i) is an ellipse E' (solid light green) centered on O1 to the left of the billiard center. E' is non-concentric, though  axis-aligned with E (their major axis are on the same line).

3) Not shown. In a separate video [2] we show a similar construction where instead of the incenters we consider the polygon of circumcenters of the same triads. It turns out the ratio of sum of inradii divided by the sum of circumradii is invariant.

[1] A. Akopyan and D. Reznik, "Non-Concentric Circular Poncelet Pair: Invariant Sum of Japanese Theorem Inradii", https://youtu.be/BEvdUUolUXI
[2] P. Roitman and D. Reznik, "Circumcircles of Focus with Consecutive Vertices Homothetic to Focus Antipedal", https://youtu.be/kVxh5jfZb9Q
\subsection{An Invariant Based on Inradii and Circumradii of Subtriangles in the Elliptic Billiard}
\label{vid:ipOEfbxWsdk}
\noindent N=5, 2m52s (10/2020). 
\begin{center}\includegraphics[width=.5\textwidth]{pics/ipOEfbxWsdk.jpg} \\ 
\href{https://youtu.be/ipOEfbxWsdk}{\url{youtu.be/ipOEfbxWsdk}}\end{center}
% Consider an elliptic billiard (black) with semiaxes a,b and the family of N-periodics (blue). Let P(i), i=1,2,...N denote its vertices and f1,f2 its foci.  Also shown is the confocal caustic (brown).

Consider N subtriangles Ti=[P(i),P(i+1),f1] which together tesselate the N-periodic. Let Ri (resp. ri) denote the circumradius (resp. inradius) of Ti.

Experimental claim: the ratio of the sum(Ri) by the sum(ri) is invariant for all N.

Any help proving this is appreciated!
\subsection{Cremona-Inversive Polygon of Odd-N-Periodics in the Elliptic Billiard: Zero Signed Area}
\label{vid:GrDh0mmtHCE}
\noindent N=5, 1m46s (11/2020). 
\begin{center}\includegraphics[width=.5\textwidth]{pics/GrDh0mmtHCE.jpg} \\ 
\href{https://youtu.be/GrDh0mmtHCE}{\url{youtu.be/GrDh0mmtHCE}}\end{center}
% Shown is the family of 5-periodics (blue) interscribed in a confocal pair (elliptic billiard: black and brown ellipses).

Let the billiard major, minor axes define an xy reference system with (0,0) at the billiard center.

Define the Cremona-Inversive-Polygon (CIP, dashed blue) as having vertices at the images of the original ones unver the following Cremona transform [1]:

(x,y) ↦ (1/x,1/y)

The purple curve represents the image of the outer ellipse under this transformation. So the vertices of the CIP must lie on it.

Surprisingly, the signed area of this polygon is dynamically zero. Indeed, this is the case for any odd N (pending proof).

When N=3, the CIP has collinear vertices, i.e., it is a zero-area degenerate polygon, see: https://bit.ly/2JkYQzD

Interestingly, for the homothetic Poncelet family, the signed area of the CIP is zero for *all* N, except, for the N=4 case. See the N=3 case here: https://bit.ly/37dJS6z 

[1] "Cremona Transformation", Encyclopedia of Mathematics, https://encyclopediaofmath.org/wiki/Cremona_transformation
\subsection{Family of Poncelet 7-Periodics Interscribed Between Two Ellipses in General Position}
\label{vid:kzxf7ZgJ5Hw}
\noindent N=7, 2m25s (6/2021). 
\begin{center}\includegraphics[width=.5\textwidth]{pics/kzxf7ZgJ5Hw.jpg} \\ 
\href{https://youtu.be/kzxf7ZgJ5Hw}{\url{youtu.be/kzxf7ZgJ5Hw}}\end{center}
% \input{descr/257_kzxf7ZgJ5Hw}
\subsection{N-Periodics in the Elliptic Billiard: Invariant Sum of Cosines}
\label{vid:qP67bdqS3nQ}
\noindent N=3,4,5,6,7,8, 2m25s (7/2021). 
\begin{center}\includegraphics[width=.5\textwidth]{pics/qP67bdqS3nQ.jpg} \\ 
\href{https://youtu.be/qP67bdqS3nQ}{\url{youtu.be/qP67bdqS3nQ}}\end{center}
% Shown are 6 families of N-periodics in the elliptic billiard, for N=3,4,5,6,7,8. Without loss of generality, the outer ellipses have a/b=1.3. Invariant quantities include (i) perimeter L, (ii) Joachimsthal's constant J, (iii) and the sum of internal angle cosines, which is equal to (iv) JL-N.
\subsection{N-Periodics in the Homothetic Pair: Invariant Sum of Cotangents}
\label{vid:TzKsr-x1KFk}
\noindent N=3,4,5,6,7,8, 2m25s (7/2021). 
\begin{center}\includegraphics[width=.5\textwidth]{pics/TzKsr-x1KFk.jpg} \\ 
\href{https://youtu.be/TzKsr-x1KFk}{\url{youtu.be/TzKsr-x1KFk}}\end{center}
% \input{descr/260_TzKsr-x1KFk}
\subsection{Invariant Sums of Generalized Exradii (and their inverses) over various Poncelet families}
\label{vid:yiGIl-nXXj8}
\noindent N=3,4,5,6,7, 11m4s (9/2021). 
\begin{center}\includegraphics[width=.5\textwidth]{pics/yiGIl-nXXj8.jpg} \\ 
\href{https://youtu.be/yiGIl-nXXj8}{\url{youtu.be/yiGIl-nXXj8}}\end{center}
% Consider a Poncelet family of N-gons interscribed between an outer conic E and an inner conic E'. Let ri be the ith exradius of an N-gon, i=1,...,N.  The video shows that for all N:

a) sum(1/ri) is invariant if E' is a circle.

b) sum(ri) is also invariant if E' is a concentric circle, or the pair (E,E') is a bicentric pair. 

For N=3 (a) is obvious since sum(1/ri)=1/r, where r is the (fixed) inradius, and (b) is a corollary of relation [1, eqn. (4)]:

r1+r2+r3=4R+r

noting that the circumradius R is fixed [2, Thm. 1] over this family (the inradius r is constant by definition).

Note: toward the last quarter of the video I was trying to say that sum(1/ri) though invariant whenever the caustic is a circle, it is only equal to 1/r if N=3. 

[1] E. Weisstein, "Exradius", MathWorld, 2021. https://mathworld.wolfram.com/Exradius.html
[2]  Ronaldo Garcia and Dan Reznik. Family ties: Relating Poncelet 3-periodics by their properties. J. Croatian Soc. for Geom. & Gr. (KoG), to appear, 2021. arXiv:2012.11270

\section{Original 2011 (5)}

\subsection{3-Periodic trajectories I}
\label{vid:9zAr5-nm7mw}
\noindent N=3, 1m41s (10/2011). 
\begin{center}\includegraphics[width=.5\textwidth]{pics/9zAr5-nm7mw.jpg} \\ 
\href{https://youtu.be/9zAr5-nm7mw}{\url{youtu.be/9zAr5-nm7mw}}\end{center}
% \input{descr/261_9zAr5-nm7mw}
\subsection{3-Periodic trajectories II}
\label{vid:A7mPzrNJHkA}
\noindent N=3, 1m40s (10/2011). 
\begin{center}\includegraphics[width=.5\textwidth]{pics/A7mPzrNJHkA.jpg} \\ 
\href{https://youtu.be/A7mPzrNJHkA}{\url{youtu.be/A7mPzrNJHkA}}\end{center}
% \input{descr/262_A7mPzrNJHkA}
\subsection{3-Periodic trajectories III}
\label{vid:6yXA0dyWhFY}
\noindent N=3, 1m41s (10/2011). 
\begin{center}\includegraphics[width=.5\textwidth]{pics/6yXA0dyWhFY.jpg} \\ 
\href{https://youtu.be/6yXA0dyWhFY}{\url{youtu.be/6yXA0dyWhFY}}\end{center}
% A light ray (red arrow) injected in the vertical direction into a reflective ellipse from a varying point on its boundary. The injection angle with the vertical is varied from -90 to 90 degrees. A 200-reflection trajectory is shown, revealing certain well-known patterns, chiefly that all rays are tangent to a virtual confocal conic, known as the "caustic": an ellipse if the initial ray does not pass between the foci, and a hyperbola if it does [1].

For most injection angles, the trajectory is space-filling, however, for a dense set, it becomes N-periodic, i.e., polygonal with N sides. This happens when a certain quantity τ (the translation in the billiard map) divided by 2π is a rational number [1].

See [2] for a gallery of N-periodics obtained from this experiment and [3] for further experimental work. Also, we've uploaded 100s of videos on this channel on youtube.

[1] S. Tabachnikov, "Geometry and Billiards", 1991. http://www.personal.psu.edu/sot2/books/billiardsgeometry.pdf

[2] D. Reznik, "N-periodics in the elliptic billiard", 2011, http://www.personal.psu.edu/sot2/books/billiardsgeometry.pdf

[3] D. Reznik, R. Garcia, and J. Koiller, "Invariants of N-Perioics in the Elliptic Billiard", 2019.
\subsection{Locus of incenter is elliptic for family of 3-periodics}
\label{vid:BBsyM7RnswA}
\noindent N=3, PT25S (10/2011). 
\begin{center}\includegraphics[width=.5\textwidth]{pics/BBsyM7RnswA.jpg} \\ 
\href{https://youtu.be/BBsyM7RnswA}{\url{youtu.be/BBsyM7RnswA}}\end{center}
% An a/b=1.5 elliptic billiard is shown as well as its N=3 (triangular) family of orbits. The locus of the orbits' incenter (green dot) is shown as a dotted line. This locus has been proven to be perfectly elliptic [1]. For more info, please refer to our webpage [2].

[1] Olga Romaskevich, "On the incenters of triangular orbits on elliptic billiards", Enseign. Math,  60(3), pp. 247-255, 2014.

[2] https://dan-reznik.github.io/Elliptical-Billiards-Triangular-Orbits/
More info: https://dan-reznik.github.io/Elliptical-Billiards-Triangular-Orbits/
\subsection{Locus of the incircle touchpoints is a higher-order curve}
\label{vid:9xU6T7hQMzs}
\noindent N=3, PT25S (10/2011). 
\begin{center}\includegraphics[width=.5\textwidth]{pics/9xU6T7hQMzs.jpg} \\ 
\href{https://youtu.be/9xU6T7hQMzs}{\url{youtu.be/9xU6T7hQMzs}}\end{center}
% \input{descr/265_9xU6T7hQMzs}

\section{Orthic Phenomena (3)}

\subsection{Locus orthic triangle's incenter is a 4-arc ellipse}
\label{vid:3qJnwpFkUFQ}
\noindent N=3, 1m13s (5/2019). 
\begin{center}\includegraphics[width=.5\textwidth]{pics/3qJnwpFkUFQ.jpg} \\ 
\href{https://youtu.be/3qJnwpFkUFQ}{\url{youtu.be/3qJnwpFkUFQ}}\end{center}
% Family of 3-periodic orbits in elliptic billiard. For every orbit, compute its orthocenter (orange dot), orthic triangle (defined by the feet of the three altitudes), and its incenter (green dot). The locus of the latter is mesmerizing: made up of 4 exact elliptical arcs arrange as a square, where the north and south arcs coincide with the billiard and the left and right coincide with the locus of the orbit's orthocenter.

More info: https://dan-reznik.github.io/Elliptical-Billiards-Triangular-Orbits/
\subsection{Locus of orthocenter, orthic orthocenter, incenter, and orthic orthic's incenter}
\label{vid:HY577AZVi7I}
\noindent N=3, 1m37s (5/2019). 
\begin{center}\includegraphics[width=.5\textwidth]{pics/HY577AZVi7I.jpg} \\ 
\href{https://youtu.be/HY577AZVi7I}{\url{youtu.be/HY577AZVi7I}}\end{center}
% Consider the family of 3-periodics in an Elliptic Billiard (EB) with  aspect ratio a/b. 

The following is shown while a/b is varied from 1.1 to 3.5 in steps of 0.01:

a) dark gray (the billiard)
b) orange: locus of orbits' orthocenter
c) green: locus of orbits' orthic triangle's incenter. note that when this curve is detached from the orthocenter (orange), the orbit is obtuse
d) blue: the orthic's orthocenter
e) red: the orthic's orthic's incenter. note that when this curve is detached from [d], the orthic triangle is obtuse.

More info: https://dan-reznik.github.io/Elliptical-Billiards-Triangular-Orbits/
\subsection{Excentral of Orthic for Acute and Obtuse Triangles}
\label{vid:-bLuvICzmqM}
\noindent N=3, 1m49s (5/2019). 
\begin{center}\includegraphics[width=.5\textwidth]{pics/-bLuvICzmqM.jpg} \\ 
\href{https://youtu.be/-bLuvICzmqM}{\url{youtu.be/-bLuvICzmqM}}\end{center}
% T (blue) is a reference triangle w/ vertices ABC, where AB is on the x axis and C is made to move along a straight line above AB. Shown in red is T's orthic triangle Th (red), and the latter's excentral triangle Te (green). Note that the excentral triangle is always acute. Note also that if T is acute, the excentral of its orthic will be T itself. But when T is obtuse, (1) two of Th's vertices will be outside of T, and (2) Te will be the acute triangle ABH, where H (shown orange) is the orthocenter of T.

More info: https://dan-reznik.github.io/Elliptical-Billiards-Triangular-Orbits/

\section{Pascal Theorem (6)}

\subsection{Pascal's Theorem I: Testing if 6 points lie on the same conic}
\label{vid:OeN47WSnLkA}
\noindent N=n/a, 9m44s (12/2021). 
\begin{center}\includegraphics[width=.5\textwidth]{pics/OeN47WSnLkA.jpg} \\ 
\href{https://youtu.be/OeN47WSnLkA}{\url{youtu.be/OeN47WSnLkA}}\end{center}
% Pascal's theorem is a beautiful result from the 16-yr old Blaise Pascal. The video shows how to apply it to test if 6 points lie on the same conic.
\subsection{Pascal's Theorem II: finding a point on a 5-pt conic that lies along a specified direction}
\label{vid:6DdScxLWiIc}
\noindent N=n/a, 11m14s (12/2021). 
\begin{center}\includegraphics[width=.5\textwidth]{pics/6DdScxLWiIc.jpg} \\ 
\href{https://youtu.be/6DdScxLWiIc}{\url{youtu.be/6DdScxLWiIc}}\end{center}
% Given five points, without ever drawing the associated conic, the video illustrates a technique, based on Pascal's theorem, which allows for one to locate the intersection of an arbitrary line through one of the five points with the conic, thereby allowing one to sweep the conic continuously.
\subsection{Pascal's Theorem III: locating the center of a 5-pt conic via conjugate diameters}
\label{vid:t8B6jEJ4DbU}
\noindent N=n/a, 11m33s (12/2021). 
\begin{center}\includegraphics[width=.5\textwidth]{pics/t8B6jEJ4DbU.jpg} \\ 
\href{https://youtu.be/t8B6jEJ4DbU}{\url{youtu.be/t8B6jEJ4DbU}}\end{center}
% \input{descr/271_t8B6jEJ4DbU}
\subsection{Pascal's Theorem IV: locating the axes of a conic passing through 3 points and of a known center}
\label{vid:wnYPz_okdJc}
\noindent N=n/a, 11m31s (12/2021). 
\begin{center}\includegraphics[width=.5\textwidth]{pics/wnYPz_okdJc.jpg} \\ 
\href{https://youtu.be/wnYPz_okdJc}{\url{youtu.be/wnYPz\_okdJc}}\end{center}
% Video shows a straight-edge + ruler construction that locates the two perpendicular axes of a conic where only 3 points on it and its center are given. This is after [1].

[1] Paul Yiu, "Introduction to the Geometry of the Triangle", Florida Atlantic University lecture notes (2001), page 150-151. URL: math.fau.edu/Yiu/YIUIntroductionToTriangleGeometry121226.pdf
\subsection{Pascal's Theorem V: drawing the tangent to a 5-pt conic at a given point on the conic}
\label{vid:O7x513NKAuw}
\noindent N=n/a, 7m28s (12/2021). 
\begin{center}\includegraphics[width=.5\textwidth]{pics/O7x513NKAuw.jpg} \\ 
\href{https://youtu.be/O7x513NKAuw}{\url{youtu.be/O7x513NKAuw}}\end{center}
% The video shows an easy method to draw a tangent to any given point on a conic specified by five points.
\subsection{Pascal's Theorem VI: locating the vertices, co-vertices, and foci of a 5-pt ellipse}
\label{vid:QAHkVIVEDso}
\noindent N=n/a, 11m22s (12/2021). 
\begin{center}\includegraphics[width=.5\textwidth]{pics/QAHkVIVEDso.jpg} \\ 
\href{https://youtu.be/QAHkVIVEDso}{\url{youtu.be/QAHkVIVEDso}}\end{center}
% This video walks through a ruler-and-compass method which allows one to locate the vertices and foci of an ellipse specified by 5 points whose center and axes were previously located (see previous videos in the series). The method presented is based on [1].

[1] Paul Yiu, "Introduction to the Geometry of the Triangle", Florida Atlantic University lecture notes (2001), p 148. URL: math.fau.edu/Yiu/YIUIntroductionToTriangleGeometry121226.pdf.

\section{Pedal Invariants (10)}

\subsection{Concyclic Feet of Focal Pedals and Invariant Product of Sums of Lengths for odd N}
\label{vid:OT-xAdbOp8o}
\noindent N=5,6, 4m2s (4/2020). 
\begin{center}\includegraphics[width=.5\textwidth]{pics/OT-xAdbOp8o.jpg} \\ 
\href{https://youtu.be/OT-xAdbOp8o}{\url{youtu.be/OT-xAdbOp8o}}\end{center}
% \input{descr/279_OT-xAdbOp8o}
\subsection{Altitude Invariants to N-Periodics and their Tangential Polygons (N=3, 4)}
\label{vid:MvZhWbI6iB8}
\noindent N=3,4, 4m1s (4/2020). 
\begin{center}\includegraphics[width=.5\textwidth]{pics/MvZhWbI6iB8.jpg} \\ 
\href{https://youtu.be/MvZhWbI6iB8}{\url{youtu.be/MvZhWbI6iB8}}\end{center}
% \input{descr/275_MvZhWbI6iB8}
\subsection{Altitude Invariants to N-Periodics and their Tangential Polygons (N=5, 6)}
\label{vid:RNmHROZNGj8}
\noindent N=5,6, 4m1s (4/2020). 
\begin{center}\includegraphics[width=.5\textwidth]{pics/RNmHROZNGj8.jpg} \\ 
\href{https://youtu.be/RNmHROZNGj8}{\url{youtu.be/RNmHROZNGj8}}\end{center}
% \input{descr/276_RNmHROZNGj8}
\subsection{Invariant sum of squared altitudes from each focus to tangential polygon sides}
\label{vid:VUtBRzmbOYU}
\noindent N=3,4,5,6,7,8, 4m1s (4/2020). 
\begin{center}\includegraphics[width=.5\textwidth]{pics/VUtBRzmbOYU.jpg} \\ 
\href{https://youtu.be/VUtBRzmbOYU}{\url{youtu.be/VUtBRzmbOYU}}\end{center}
% Six a/b=1.5 elliptic billiards (EB, black) are shown, as well as their N-Periodic families (blue). Top row: N=3,4,5, bottom row: N=6,7,8. For each N-periodic the tangential polygon (green) is shown, call it P', whose sides are tangent to the EB at the N-periodic vertices.

From each focus f1 (or f2), drop N perpedinculars to the sides of P'.

Property 1: their feet all lie on a circle with radius a, the 
EB major semi-axis.

Property 2: Let the lengths of perpendiculars from f1 and f2 be denoted by e_1(i) and e_2(i), i=1 to N. It turns out that for all N greater than 2 (odd or even):

\sum_{i=1}^N{ e_1(i)^2 } = \sum_{i=1}^N{ e_2(i)^2 } is invariant over the family of N-periodics.

Soundtrack: Ana Vidovic, "Asturias" (Isaac Albeñiz)
\subsection{Sum of square altitudes from arbitrary point to N-periodic tangents is invariant}
\label{vid:btexFUUBpjQ}
\noindent N=5, 4m1s (4/2020). 
\begin{center}\includegraphics[width=.5\textwidth]{pics/btexFUUBpjQ.jpg} \\ 
\href{https://youtu.be/btexFUUBpjQ}{\url{youtu.be/btexFUUBpjQ}}\end{center}
% An a/b=1.5 elliptic billiard (EB) is shown (black) as well as its foci f1 and f2, and its family of N-periodics (blue), N=5 is used without loss of generality. Also shown is the "tangential polygon" (green), tangent to the EB at the N-periodic vertices. 

For each of the 6 EBs drawn, a point "m" is chosen at a different location on the plane. For each case, N perpendiculars are dropped from "m" to the sides of the tangential polygon, aka. altitudes.

The video illustrates that independent of the position of "m" the sum of the square lenghts of said N altitudes is invariant (though its invariant value depends on m's location, with the minimum occurring at the EB center, top left). Notice the invariance is also true when "m" is exterior to the EB.

Soundtrack: Joaquín Rodrigo, "Concierto para Aranjuez"
\subsection{Pedal Polygons for the N-Periodic and its Tangent Polygon: Area Ratio Invariances}
\label{vid:6F7Y3UKJzdk}
\noindent N=5,6, 4m1s (4/2020). 
\begin{center}\includegraphics[width=.5\textwidth]{pics/6F7Y3UKJzdk.jpg} \\ 
\href{https://youtu.be/6F7Y3UKJzdk}{\url{youtu.be/6F7Y3UKJzdk}}\end{center}
% Six copies of the same a/b=1.5 elliptic billiard (EB, black) and its confocal caustic (brown) are shown.  Let the N-periodics (blue) be denoted as P and the tangent polygon (green) P'.

Let A, A' denote the areas of P and P'. It has been shown A'/A (resp. A A') is constant for odd (resp. even) N [1,2].

Here we are interested in the areas of pedal polygons defined by the feet of perpendiculars (dashed blue, green) dropped from a point "m" onto either P or P'. Call the former P_m (transparent blue) and latter P'_m (transparent green). Let A_m and A'_m denote their areas, respectively. 

More specifically, the video illustrates the invariance of ratio q=A_m/A'_m when "m" is:

left: at the center of the EB
middle:  at the right-side focus
right: at a fixed location on the first quadrant. 

1) Top row (N=5). q is conserved only when "m " is at the center of the EB or at a focus. This extends to all odd N.
 
2) Bottom row (N=6): q is conserved for *any* position of "m". This is true for all even N.

Soundtrack: Villa-Lobos

References:

[1] Arseniy Akopyan, Richard Schwartz, Serge Tabachnikov, "Billiards in ellipses revisited", 2020. https://arxiv.org/abs/2001.02934

[2] Ana C. Chavez-Caliz, "More about areas and centers of Poncelet polygons", 2020. https://arxiv.org/abs/2004.05404

[3] D. Reznik, R. Garcia, J. Koiller, "Can the Elliptic Billiard Still Surprise Us?", 2020, Math. Intelligencer, https://rdcu.be/b2cg1
\subsection{Pedal polygons from each focus have invariant area product}
\label{vid:sw8pJFMV00w}
\noindent N=5, 4m1s (4/2020). 
\begin{center}\includegraphics[width=.5\textwidth]{pics/sw8pJFMV00w.jpg} \\ 
\href{https://youtu.be/sw8pJFMV00w}{\url{youtu.be/sw8pJFMV00w}}\end{center}
% \input{descr/283_sw8pJFMV00w}
\subsection{Exploring Amazing Invariants of N-Periodics and their Pedal Polygons}
\label{vid:2yXbOV7qf7k}
\noindent N=3n/a12, 20m1s (4/2020). 
\begin{center}\includegraphics[width=.5\textwidth]{pics/2yXbOV7qf7k.jpg} \\ 
\href{https://youtu.be/2yXbOV7qf7k}{\url{youtu.be/2yXbOV7qf7k}}\end{center}
% Narrated experimentation with the family of N-periodics in an Elliptic BIlliard.

When one considers its "pedal polygons" (defined by dropping perpendiculars from an arbitrary point "m" on the plane to the N-periodic or tangential polygons sides), many wonderful things happen, including invariant area ratios [2,3,4] and products and stationary center of mass [1].

Phenomena are governed by (i) the parity of N: odd, even, multiple of four, etc., (iii) which polygon (N-periodic, outer, inner), and (ii) the position of m on the plane. 

0:00 to 5:00 -- area ratio invariants of N-periodics, and their outer
and inner polygon, some proofs in [2,4]
5:00 to 11:00 -- [new] area ratio invariants to pedal polygons
11:00 to 18:00 -- [new] stationary centers of mass of pedal polygons. related [1]
18:00 to 20:00 -- [new] sum (and sometimes product) of squared altitudes is invariant

Here's a list of the 11 new invariants identified in the video:

1) Pedal polygon from point m to N-periodic

1.1) N=0 (mod 4):Π(mi) if m = 0,f1,f2
1.2) N=2 (mod 4): Π(mi) if m = f1,f2

1.3) odd N: Am.A if m = O
1.4) N=0 (mod 4): Am.A for any m
1.5) N=2 (mod 4): Am/A for any m

2) Pedal polygon from point m to tangential polygon to N-periodic

2.1) all N: Σ(mi')^2 for any m
2.2) N even: Π(mi')=1.0 for m=f1,f2
2.3) additionally N=2 (mod 4): Π(mi') for m=O

2.4) odd N: Am' A' if m=O
2.5) N=0 (mod 4): Am'/A' for any m. Note: w/ N=4, Am'/A'=2 for any a/b and m
2.6) N=2 (mod4): Am'.A' for any m

Recent, related bilbiography:

[1] R. Schwartz, S. Tabachnikov, "Centers of mass of Poncelet polygons, 200 years after", 2016. arXiv:1607.04766
[2] Arseniy Akopyan, Richard Schwartz, Serge Tabachnikov "Billiards in ellipses revisited", 2020. arXiv:2001.02934
[3] Misha Bialy, Serge Tabachnikov , "Dan Reznik's identities and more", 2020. arXiv:2001.08469
[4]  Ana C. Chavez-Caliz, "More about areas and centers of Poncelet polygons", 2020. arXiv:2004.05404
[5] D. Reznik, R. Garcia, and J. Koiller, "Can the Elliptic Billiard Still Surprise Us?", Mathematical Intelligencer, vol 42, 2019. https://rdcu.be/b2cg1
\subsection{Area Invariants of Pedal and Antipedal Polygons}
\label{vid:LN623VjeeFQ}
\noindent N=3, 9m18s (4/2020). 
\begin{center}\includegraphics[width=.5\textwidth]{pics/LN623VjeeFQ.jpg} \\ 
\href{https://youtu.be/LN623VjeeFQ}{\url{youtu.be/LN623VjeeFQ}}\end{center}
% \input{descr/277_LN623VjeeFQ}
\subsection{Centroid Stationarity (even N)}
\label{vid:j_GD_g8aIbg}
\noindent N=4,6, 3m21s (4/2020). 
\begin{center}\includegraphics[width=.5\textwidth]{pics/j_GD_g8aIbg.jpg} \\ 
\href{https://youtu.be/j_GD_g8aIbg}{\url{youtu.be/j\_GD\_g8aIbg}}\end{center}
% \input{descr/278_j_GD_g8aIbg}

\section{Pencil of Circles (6)}

\subsection{Poncelet in Circle Pencil I: the geometric Flamenco of 5-Periodics with 4 caustics}
\label{vid:L5A_S4VQLiw}
\noindent N=5, 4m52s (7/2021). 
\begin{center}\includegraphics[width=.5\textwidth]{pics/L5A_S4VQLiw.jpg} \\ 
\href{https://youtu.be/L5A_S4VQLiw}{\url{youtu.be/L5A\_S4VQLiw}}\end{center}
% \input{descr/286_L5A_S4VQLiw}
\subsection{Poncelet in Circle Pencil II: constant-perimeter polar image of 5-gons wrt to a limiting point (billiard)}
\label{vid:8dRap3ZWQjQ}
\noindent N=5, 4m52s (7/2021). 
\begin{center}\includegraphics[width=.5\textwidth]{pics/8dRap3ZWQjQ.jpg} \\ 
\href{https://youtu.be/8dRap3ZWQjQ}{\url{youtu.be/8dRap3ZWQjQ}}\end{center}
% \input{descr/287_8dRap3ZWQjQ}
\subsection{Poncelet in Circle Pencil II: The automatically-closing N=4 family (zero limit-polar perimeter)}
\label{vid:-8CVK18UkM8}
\noindent N=5, 4m52s (8/2021). 
\begin{center}\includegraphics[width=.5\textwidth]{pics/-8CVK18UkM8.jpg} \\ 
\href{https://youtu.be/-8CVK18UkM8}{\url{youtu.be/-8CVK18UkM8}}\end{center}
% \input{descr/288_-8CVK18UkM8}
\subsection{Poncelet in Circle Pencil III: The automatically-closing N=6 family (zero limit-polar perimeter)}
\label{vid:ZTn8oJZ5p3o}
\noindent N=6, 2m26s (8/2021). 
\begin{center}\includegraphics[width=.5\textwidth]{pics/ZTn8oJZ5p3o.jpg} \\ 
\href{https://youtu.be/ZTn8oJZ5p3o}{\url{youtu.be/ZTn8oJZ5p3o}}\end{center}
% \input{descr/289_ZTn8oJZ5p3o}
\subsection{Poncelet in Circle Pencil IV: Polar images of all closing permutations have same invariant perimeter}
\label{vid:KQyLs1NyYZ0}
\noindent N=5, 4m52s (8/2021). 
\begin{center}\includegraphics[width=.5\textwidth]{pics/KQyLs1NyYZ0.jpg} \\ 
\href{https://youtu.be/KQyLs1NyYZ0}{\url{youtu.be/KQyLs1NyYZ0}}\end{center}
% \input{descr/290_KQyLs1NyYZ0}
\subsection{Inverting a triangle with respect to a pencil of circles}
\label{vid:hmj5Mf7TV_s}
\noindent N=3, 18m36s (8/2021). 
\begin{center}\includegraphics[width=.5\textwidth]{pics/hmj5Mf7TV_s.jpg} \\ 
\href{https://youtu.be/hmj5Mf7TV_s}{\url{youtu.be/hmj5Mf7TV\_s}}\end{center}
% \input{descr/285_hmj5Mf7TV_s}

\section{Pencil of Confocals (6)}

\subsection{Tangents from a point on boundary to caustics}
\label{vid:mkhhd536_2w}
\noindent N=n/a, 4m48s (6/2019). 
\begin{center}\includegraphics[width=.5\textwidth]{pics/mkhhd536_2w.jpg} \\ 
\href{https://youtu.be/mkhhd536_2w}{\url{youtu.be/mkhhd536\_2w}}\end{center}
% This video shows a moving point on the boundary of an elliptic billiard (a/b=1.5) and two tangents to each caustic, N=3,4,...7.

More info: https://dan-reznik.github.io/Elliptical-Billiards-Triangular-Orbits/
\subsection{Tangents to caustics from billiard's vertex lie on a single circle}
\label{vid:NsZUyDJ6IOs}
\noindent N=n/a, PT53S (6/2019). 
\begin{center}\includegraphics[width=.5\textwidth]{pics/NsZUyDJ6IOs.jpg} \\ 
\href{https://youtu.be/NsZUyDJ6IOs}{\url{youtu.be/NsZUyDJ6IOs}}\end{center}
% \input{descr/296_NsZUyDJ6IOs}
\subsection{Loci of tangents to confocals: point traverses entire elliptic boundary}
\label{vid:EL4vgcJaktc}
\noindent N=n/a, 2m25s (6/2019). 
\begin{center}\includegraphics[width=.5\textwidth]{pics/EL4vgcJaktc.jpg} \\ 
\href{https://youtu.be/EL4vgcJaktc}{\url{youtu.be/EL4vgcJaktc}}\end{center}
% \input{descr/291_EL4vgcJaktc}
\subsection{Loci of tangents to confocals: point traverses neighborhood of right vertex}
\label{vid:J5CA9UJVflI}
\noindent N=n/a, 6m1s (6/2019). 
\begin{center}\includegraphics[width=.5\textwidth]{pics/J5CA9UJVflI.jpg} \\ 
\href{https://youtu.be/J5CA9UJVflI}{\url{youtu.be/J5CA9UJVflI}}\end{center}
% An ellipse E is shown (a/b=1.5) as well as its foci. A point P is shown moving continously in the neighborhood of E's right vertex V. Also shown is the locus of tangent points from P to the continuous family of ellipses confocal with E. For each such confocal, two tangents are defined, giving rise to the two branches (blue, yellow) of the locus.  In the region traversed by P the locus undergoes a phase transition from a complex curve (when P is below V) to perfectly circular (when P is at V)  to a mirrorerd complex curve (when P is above V).

https://dan-reznik.github.io/Elliptical-Billiards-Triangular-Orbits/
\subsection{Locus of tangents from ellipse: -45, 45 degrees starting points}
\label{vid:lXhnBksS74E}
\noindent N=n/a, 1m21s (6/2019). 
\begin{center}\includegraphics[width=.5\textwidth]{pics/lXhnBksS74E.jpg} \\ 
\href{https://youtu.be/lXhnBksS74E}{\url{youtu.be/lXhnBksS74E}}\end{center}
% \input{descr/293_lXhnBksS74E}
\subsection{Locus of tangents from ellipse: 5, 95, -45, 45 degrees starting points}
\label{vid:Ac0iej_TaEc}
\noindent N=n/a, 1m21s (6/2019). 
\begin{center}\includegraphics[width=.5\textwidth]{pics/Ac0iej_TaEc.jpg} \\ 
\href{https://youtu.be/Ac0iej_TaEc}{\url{youtu.be/Ac0iej\_TaEc}}\end{center}
% Left Column: Top (resp. bottom) graph: locus of tangents to confocals to external ellipse from boundary points located at 0° and 90°, (resp. 5° and 95°), shown in red (resp. blue).

Right Column: Top (resp. bottom) graph: locus of tangents to confocals to external ellipse from boundary points located at 45° and 135° (resp. 45° and - 45°), shown in red (resp. blue).

Bottom graph: boun n is an ellipse E with b=1 and a varying from 1 to 3. Also shown are two points on E's boundary at locations:

https://dan-reznik.github.io/Elliptical-Billiards-Triangular-Orbits/

\section{Poncelet Family (25)}

\subsection{Poncelet Triangle Inscribed in Ellipse and Circumscribed in Circle}
\label{vid:I1BFOXN-EUw}
\noindent N=3, 3m13s (9/2019). 
\begin{center}\includegraphics[width=.5\textwidth]{pics/I1BFOXN-EUw.jpg} \\ 
\href{https://youtu.be/I1BFOXN-EUw}{\url{youtu.be/I1BFOXN-EUw}}\end{center}
% \input{descr/299_I1BFOXN-EUw}
\subsection{Pencil of N=3 Poncelet Ellipse Pairs: Loci of Triangular Centers}
\label{vid:B5dRXT8Xerw}
\noindent N=3, 8m1s (9/2019). 
\begin{center}\includegraphics[width=.5\textwidth]{pics/B5dRXT8Xerw.jpg} \\ 
\href{https://youtu.be/B5dRXT8Xerw}{\url{youtu.be/B5dRXT8Xerw}}\end{center}
% \input{descr/297_B5dRXT8Xerw}
\subsection{Poncelet Family of Triangles over the Family of N=3 Caustics}
\label{vid:53pCKKd_5qI}
\noindent N=3, 4m27s (9/2019). 
\begin{center}\includegraphics[width=.5\textwidth]{pics/53pCKKd_5qI.jpg} \\ 
\href{https://youtu.be/53pCKKd_5qI}{\url{youtu.be/53pCKKd\_5qI}}\end{center}
% Consider a fixed a=2, b=1 ellipse (black), and the family of N=3 poncelet caustics (billiard or non), gray, (a0,b0). The Cayley condition states that a0/a + b0/b = 1. The animation sweeps b0 from 0.05 to 0.95. For each such caustic, a particular N=3 polygon is shown as well as the loci formed by a few triangular centers.
\subsection{Three Geometers Walk into a Bar: the 3-periodic Poncelet-Steiner family has invariant Brocard angle.}
\label{vid:2fvGd8wioZY}
\noindent N=3, 7m39s (7/2020). 
\begin{center}\includegraphics[width=.5\textwidth]{pics/2fvGd8wioZY.jpg} \\ 
\href{https://youtu.be/2fvGd8wioZY}{\url{youtu.be/2fvGd8wioZY}}\end{center}
% \input{descr/317_2fvGd8wioZY}
\subsection{5-Periodic Poncelet Families and their Pedal Polygons with Respect to their Curvature Centroids}
\label{vid:RP18B827l5I}
\noindent N=5, 2m24s (7/2020). 
\begin{center}\includegraphics[width=.5\textwidth]{pics/RP18B827l5I.jpg} \\ 
\href{https://youtu.be/RP18B827l5I}{\url{youtu.be/RP18B827l5I}}\end{center}
% \input{descr/303_RP18B827l5I}
\subsection{An N=3 Poncelet family (outer circle, inner ellipse) equivalent to Poristic Excentrals}
\label{vid:wUu2iMesv3U}
\noindent N=3, 4m1s (7/2020). 
\begin{center}\includegraphics[width=.5\textwidth]{pics/wUu2iMesv3U.jpg} \\ 
\href{https://youtu.be/wUu2iMesv3U}{\url{youtu.be/wUu2iMesv3U}}\end{center}
% Family of Poncelet 3-Periodics between two concentric conics: an external circle and an internal ellipse. Indeed, these are closely related to the family of Poristic Triangles [1,2,3]. Several remarkable invariants are observed:

- Orthic Inradius (the X1 of the orthic is the same as X4 of the 3-periodics, since these turn out to be always acute).
- Orthic Circumradius
- |X5-X3|
- Axes of the MacBeath Inellipse (shown olive green). Rotates ridgly about X5 (its foci are X3 and X4).
- product of cosines (hold for all N, the orthics conserve the *sum* of cosines). Note: there was a bug in the reporting of this quantity which appears non-constant on the video. I will soon post a video where the bug is fixed confriming the invariance of the product of cosines.
- sum of squared sidelenghts

Relation to the Poristics:

Consider a reference frame centered on the orthic's circumcienter (X5 of the 3-periodic), with one axis oriented toward the orthic incenter (X4 of the 3-periodic). With respect to this reference frame: 

- the orthic triangles (orange in the video) are none other than Chapple's Poristic Family [1].
- the Poncelet 3-periodics (blue on video) are are the excentrals to the Poristic Family.
- the MacBeath inellipse becomes stationary. It is the caustic of the poristic family of excentrals. The outer conic is a circle equal to our original one.
- the original inner ellipse (black) becomes a rigidly rotating inconic centered in the excentrals' circumcenter. See [4]

We have studied these in [3].

[1] Chapple's Porism, Wikipedia. https://en.wikipedia.org/wiki/Poncelet%27s_closure_theorem
[2] B. Odehnal, "Poristic Loci of Triangle Centers", 2011, https://www.geometrie.tuwien.ac.at/odehnal/pltc.pdf
[3] R. Garcia and D. Reznik, "Related by Similarity:  Poristic Triangles and 3-Periodics in the Elliptic Billiard", 2020. https://arxiv.org/abs/2004.13509
[4] D. Reznik, "Poristic Family: X1-Ctr Circumconic & X40-Centered (Excentral) Inconic: Identical Invariant Axes", YouTube. https://youtu.be/PGdQY7f626Y
\subsection{Poncelet Family: Amazing Circular Locus of $X_{3}$ and the Steiner's Curvature Centroid}
\label{vid:601OfxuSDGc}
\noindent N=5, 3m28s (7/2020). 
\begin{center}\includegraphics[width=.5\textwidth]{pics/601OfxuSDGc.jpg} \\ 
\href{https://youtu.be/601OfxuSDGc}{\url{youtu.be/601OfxuSDGc}}\end{center}
% Left: 3-periodic Poncelet family (blue) between an external ellipse (black) and an intrerior, concentric circle (purple). By definition, this family has constant inradius. Surprisingly, it also conserves circumradius R, sum of cosines, product of half sines, and X3 moves along a circle centered on the common center O.

Right: external ellipse (black) and concentric internal circle (purple) suitable for 5-periodic Poncelet (blue) family. Suprisingly, this family *also* conserves the sum of cosines, and the product of half sines. Since we no longer have X3, let's use its polygonal "proxy", Jakob Steiner's Curvature Centroid K (Krümmungs Schwerpunkt): like X3, it is the aveage of the vertices where each is weighted by the sine of the double angle. Surprise: K also moves along a circle centered at O.
\subsection{Between an Ellipse and a Concentric Circle: Poncelet 3-Periodics Identical to Poristic Triangles.}
\label{vid:ML_AZoX736w}
\noindent N=3, 12m8s (7/2020). 
\begin{center}\includegraphics[width=.5\textwidth]{pics/ML_AZoX736w.jpg} \\ 
\href{https://youtu.be/ML_AZoX736w}{\url{youtu.be/ML\_AZoX736w}}\end{center}
% \input{descr/306_ML_AZoX736w}
\subsection{Between a Circle and a Concentric Ellipse: Poncelet 3-Periodics Identical to Poristic Excentrals.}
\label{vid:xM1SAZO9bDc}
\noindent N=3, 3m37s (8/2020). 
\begin{center}\includegraphics[width=.5\textwidth]{pics/xM1SAZO9bDc.jpg} \\ 
\href{https://youtu.be/xM1SAZO9bDc}{\url{youtu.be/xM1SAZO9bDc}}\end{center}
% Left: 1d family of 3-periodic Poncelet triangles (blue) inscribed in an cicle and circumscribing an inner, concentric ellipse. It turns out (i) the locus of the family's 9-point circle center X5 and orthocenter X4 are concentric circles, (ii) the orthic (red) has fixed inradius and circumradius, and (iii) it conserves the product of cosines and sum of squared sidelengths. The family's Macbeath inconic (green, centered on X5 and w foci on X4 and X3), is rigidly rotating,

Right: It turns out said family of orthics is identical (up to rotation about X3) to the family of Poristic triangles, i.e., 3-periodic Poncelet triangles of fixed inradius and circumradius [1]. 

Since Poristics have been shown to be the image of 3-periodics in the elliptic Billiard under a varying affine transform [2], all invariants valid for the 3-periodic orbits in  the billiard (see [3]) will be valid for the orthic (e.g., sum of cosines). Likewise, those valid for the excentrals (product of cosines) will be valid for the original 3-periodics (between circle and ellipse).

The original inner ellipse is identical, up to rotation, to the X40-centered inellipse to the poristics, which is rigidly rotating [2] and has axes equal to R-d and R+d, where d=|X1-X3| of the poristics, or |X4-X5| of the original 3-periodics. The latter's MacBeath (green) appears here as the stationary caustic to the excentrals (also shown green).

Music: Djavan, Samurai

References:

[1] W. Gallatly, "The modern geometry of the triangle", Francis Hodgson, 1914.
[2] R. Garcia and D. Reznik, "Related by Similiarity: Poristic Triangles and 3-Periodics in the Elliptic Billiard", April 2020, https://arxiv.org/abs/2004.13509
[3] D. Reznik, R. Garcia and J. Koiller, "Forty New Invariants of N-Periodics in the Elliptic Billiard", https://arxiv.org/abs/2004.12497
\subsection{Jean-Victor Poncelet \& Jakob Steiner walk into a Bierhaus + discover many invariants}
\label{vid:30cuWWaZv7A}
\noindent N=3,4,5,6, 4m50s (8/2020). 
\begin{center}\includegraphics[width=.5\textwidth]{pics/30cuWWaZv7A.jpg} \\ 
\href{https://youtu.be/30cuWWaZv7A}{\url{youtu.be/30cuWWaZv7A}}\end{center}
% Take a pair of concentric, axis-aligned ellipses which are homothetic to one another. Consider their 1d family of Poncelet Polygons. The video showcases a few suprising facts for N=3,4,5,6, namely:

1) for all N, the following are invariant over the 1d Poncelet family:

1.1) the area of the Poncelet polygons. (system is affine image of two-circle, regular poly case)
1.2.) the sum of sidelengths squared (proved, stems from cyclic trigonometric groups)
1.3) the sum of cotangents of angles (needs proof!)

2) Let K denote the Steiner's Curvature Centroid (Krümmungs-Schwerpunkt) of the Poncelet polygons. This is the weighted average of vertices where weights are the sum of double angles of the vertices.

2.1) for even N (resp. odd N), K is stationary at O (resp. moves along an ellipse, needs proof).
2.2) for odd N, the area of the pedal polygon of the N-periodic with respect to K is invariant! (needs proof)

Note: Steiner showed in 1825 that the pedal polygon w/ respect to K has extremal area.

Note that for N=3, K is the circumcenter X3. The pedal to a triangle wrt to X3 is the medial, whose area is 1/4 that of the reference (the latter's area is constant).
\subsection{3-Periodics in a Concentric Homothetic Poncelet Pair: Circular Loci of four Triangle Centers}
\label{vid:ZwTfwaJJitE}
\noindent N=3, 4m10s (8/2020). 
\begin{center}\includegraphics[width=.5\textwidth]{pics/ZwTfwaJJitE.jpg} \\ 
\href{https://youtu.be/ZwTfwaJJitE}{\url{youtu.be/ZwTfwaJJitE}}\end{center}
% \input{descr/300_ZwTfwaJJitE}
\subsection{3-Periodics in a Homothetic-Rotated Poncelet Pair: stationary orthocenter and loci of $X_{107}$ and $X_{122}$}
\label{vid:fpd_Zot5cKk}
\noindent N=3, 4m58s (8/2020). 
\begin{center}\includegraphics[width=.5\textwidth]{pics/fpd_Zot5cKk.jpg} \\ 
\href{https://youtu.be/fpd_Zot5cKk}{\url{youtu.be/fpd\_Zot5cKk}}\end{center}
% Consider a concentric pair of ellipses where the outer has semiaxes (a,b), and the inner (a0,b0), such that b0=(a^2 b)/(a^2+b^2), a0=b0*(b/a). This pair admits a 3-periodic Poncelet family w/ the remarkable property that X4 is stationary at the common center. 

Additionally, the locus of X107 (resp. X122) is identical to the outer (resp. inner) ellipse.
\subsection{5- and 7-Periodics on a Homothetic-Rotated Poncelet Pair: All Altitudes Meet at the Center}
\label{vid:gNHiZvBhKF8}
\noindent N=5,7, 2m13s (8/2020). 
\begin{center}\includegraphics[width=.5\textwidth]{pics/gNHiZvBhKF8.jpg} \\ 
\href{https://youtu.be/gNHiZvBhKF8}{\url{youtu.be/gNHiZvBhKF8}}\end{center}
% \input{descr/302_gNHiZvBhKF8}
\subsection{N-Periodics on a Homothetic-Rotated Poncelet Pair: All Altitudes Meet at the Center}
\label{vid:ttKjzWeG5B8}
\noindent N=3,4,5,6,7, 2m18s (8/2020). 
\begin{center}\includegraphics[width=.5\textwidth]{pics/ttKjzWeG5B8.jpg} \\ 
\href{https://youtu.be/ttKjzWeG5B8}{\url{youtu.be/ttKjzWeG5B8}}\end{center}
% \input{descr/312_ttKjzWeG5B8}
\subsection{Concentric Poncelet Pair w Incircle: Ratio of Sidelength Product to Perimeter is Invariant for odd N}
\label{vid:7Jg2nRkkUhQ}
\noindent N=3,5, 2m4s (8/2020). 
\begin{center}\includegraphics[width=.5\textwidth]{pics/7Jg2nRkkUhQ.jpg} \\ 
\href{https://youtu.be/7Jg2nRkkUhQ}{\url{youtu.be/7Jg2nRkkUhQ}}\end{center}
% \input{descr/308_7Jg2nRkkUhQ}
\subsection{Concentric Poncelet Pair w Circumcircle: Locus of Pseudo-Orthocenter is Circle (odd N) + Invariants}
\label{vid:3f6YBohQCFg}
\noindent N=3,5, 3m1s (8/2020). 
\begin{center}\includegraphics[width=.5\textwidth]{pics/3f6YBohQCFg.jpg} \\ 
\href{https://youtu.be/3f6YBohQCFg}{\url{youtu.be/3f6YBohQCFg}}\end{center}
% \input{descr/307_3f6YBohQCFg}
\subsection{Poncelet Invariants:
circular + point loci of the pseudo-circumcenter and pseudo-orthocenter, N=5, 6}
\label{vid:ZfQEDujbirQ}
\noindent N=5,6, 3m9s (8/2020). 
\begin{center}\includegraphics[width=.5\textwidth]{pics/ZfQEDujbirQ.jpg} \\ 
\href{https://youtu.be/ZfQEDujbirQ}{\url{youtu.be/ZfQEDujbirQ}}\end{center}
% \input{descr/316_ZfQEDujbirQ}
\subsection{Poncelet 3-Periodic Invariants (Outer Circle, Inner Concentric Ellipse) of the Nine-Point Center II}
\label{vid:8xlYaQfQCTw}
\noindent N=3, 2m51s (8/2020). 
\begin{center}\includegraphics[width=.5\textwidth]{pics/8xlYaQfQCTw.jpg} \\ 
\href{https://youtu.be/8xlYaQfQCTw}{\url{youtu.be/8xlYaQfQCTw}}\end{center}
% \input{descr/314_8xlYaQfQCTw}
\subsection{Family of 3-Periodics in Five Poncelet Pairs}
\label{vid:8hkeksAsx0E}
\noindent N=1,2,3,4,9, 1m13s (9/2020). 
\begin{center}\includegraphics[width=.5\textwidth]{pics/8hkeksAsx0E.jpg} \\ 
\href{https://youtu.be/8hkeksAsx0E}{\url{youtu.be/8hkeksAsx0E}}\end{center}
% The video allows us to observe the family of 3-periodics in five Poncelet ellipse pairs, four of them which are concentric and axis-aligned, to be sure:

0. Confocal. Here 3-periodics are billiard trajectories (the ellipse normal is the bisector at every vertex). The Mittenpunkt X9 is stationary as the system's center. and the ratio of inradius to circumradius is conserved, therefore the sum of cosines is as well.

I. Incircle. The family has stationary incircle and X1-centered circumellipse. Suprisingly, this system conserves circumradius (not shown). Since inradius is conserved by definition, so is its ratio to circumradius and therefor it shares with system 0 invariant sum of cosines.

II. Inellipse. This family has stationary circumcircle and X3-centered inellipse. Its invariants include the sum of squared sidelengths and the product of cosines.

III. Homothetic. The inner ellipse is a half-sized copy of the outer one, i.e., the family's Steiner Circum- and Inellipse are stationary, with their center X2 (Barycenter) stationary. This system conserves area, sum of squared sidelengths, and as a corollary, the Brocard angle ω (equiv. to the sum of cotangents).

IV. Dual. The inner ellipse is homothetic to the outer ellipse rotated 90 degrees. The family's orthocenter is stationary at the common center. No other invariants have yet been identified for this system.

V. Poristic: the only non-concentric system in the set, this system is in fact an image of system I under a rigid (variable) rotation about X1, and therefore shares all metric invariants with it (r/R, sum of cosines, etc.)
\subsection{Isodynamic Pedals and Isogonic Antipedals: Equilaterals with Constant Area in the Homothetic Pair}
\label{vid:7qoxAaG8sbk}
\noindent N=3, 3m13s (9/2020). 
\begin{center}\includegraphics[width=.5\textwidth]{pics/7qoxAaG8sbk.jpg} \\ 
\href{https://youtu.be/7qoxAaG8sbk}{\url{youtu.be/7qoxAaG8sbk}}\end{center}
% \input{descr/310_7qoxAaG8sbk}
\subsection{New Invariants of Poncelet N-Periodics in the Homothetic Pair}
\label{vid:2PdsC3CcqaE}
\noindent N=5, 5m1s (10/2020). 
\begin{center}\includegraphics[width=.5\textwidth]{pics/2PdsC3CcqaE.jpg} \\ 
\href{https://youtu.be/2PdsC3CcqaE}{\url{youtu.be/2PdsC3CcqaE}}\end{center}
% Consider the 1d family of Poncelet N-periodics inscribed in an external ellipse E=(a,b) and circumscribed about an internal ellipse E'=(a',b') which is concentric and homothetic to E. Without loss of generality, the video depicts (a,b)=(2,1) and N=5.

The following are invariants recently detected for this family.

a) area
b) sum of sidelengths squared
c) sum of cotangents

note: sum of the squares and cubes of cotangents are also constant [1].

The following are new ones:

d) sum of distances from focus to the vertices (elementary for even N)
e) sum of square distances from any point P0 on the plane (internal or external to E)

[1] S. Galkin, Private Communication, Aug-Sept. 2020.
\subsection{Concentric Ellipse-Circle Poncelet 3-Periodics: invariant inradius, circumradius and cosine sum}
\label{vid:eIxb1so6ORo}
\noindent N=3, PT52S (1/2021). 
\begin{center}\includegraphics[width=.5\textwidth]{pics/eIxb1so6ORo.jpg} \\ 
\href{https://youtu.be/eIxb1so6ORo}{\url{youtu.be/eIxb1so6ORo}}\end{center}
% Consider the family of Poncelet 3-periodics interscribed between an outer ellipse w semi-axes (a,b) and an inner unit circle.

Recall Cayley's condition for the existence of 3-periodics in a pair of concentric, axis aligned ellipses with semi-axes (a,b) and (a',b'): a'/a+b'/b=1. Since in this case a'=b'=1, then the following must hold:

 that 1/a + 1/b = 1

The video shows said family and the fact that its circumradius R is constant. Since the inradius r by definition is as well, the sum of cosines = 1+r/R is also invariant.

By definition, the incenter X1 is stationary at the common center. The locus of the circumcenter X3 is a concentric circle, i.e., the distance |X3-X1| is invariant.

This can be observed with our interactive app https://bit.ly/3oF1ujm
\subsection{Power Circles of Poncelet 3-Periodics have Invariant Total Area}
\label{vid:_psLvzlWTvQ}
\noindent N=3, 5m14s (1/2021). 
\begin{center}\includegraphics[width=.5\textwidth]{pics/_psLvzlWTvQ.jpg} \\ 
\href{https://youtu.be/_psLvzlWTvQ}{\url{youtu.be/\_psLvzlWTvQ}}\end{center}
% \input{descr/320__psLvzlWTvQ}
\subsection{Limiting Points of Poncelet 3-Periodic Circle Pairs: Loci, Properties, Invariants}
\label{vid:bHTLS2XzkIQ}
\noindent N=3, 17m20s (2/2021). 
\begin{center}\includegraphics[width=.5\textwidth]{pics/bHTLS2XzkIQ.jpg} \\ 
\href{https://youtu.be/bHTLS2XzkIQ}{\url{youtu.be/bHTLS2XzkIQ}}\end{center}
% \input{descr/319_bHTLS2XzkIQ}
\subsection{Six Families of Poncelet 3-Periodics in Concentric, Axis-Aligned Ellipses}
\label{vid:14TQ5WlZxUw}
\noindent N=3, 2m25s (5/2021). 
\begin{center}\includegraphics[width=.5\textwidth]{pics/14TQ5WlZxUw.jpg} \\ 
\href{https://youtu.be/14TQ5WlZxUw}{\url{youtu.be/14TQ5WlZxUw}}\end{center}
% Poncelet 3-periodic families shown in 6 different concentric, axis-parallel (CAP) pairs:

a) top: confocal, with incircle, with circumcircle
b) bottom: homothetic, dual, and confocal-excentral.

\section{Poncelet Plectra (3)}

\subsection{Poncelet Guitar Picks I: Cosine Space of Confocal and Incircle 3-Periodic Families}
\label{vid:uwdW95HI-q8}
\noindent N=3, 1m18s (2/2021). 
\begin{center}\includegraphics[width=.5\textwidth]{pics/uwdW95HI-q8.jpg} \\ 
\href{https://youtu.be/uwdW95HI-q8}{\url{youtu.be/uwdW95HI-q8}}\end{center}
% \input{descr/322_uwdW95HI-q8}
\subsection{Poncelet 3-Periodics with Circumcircle and Affine Excentral Image: Identical Invariant Cosine Product}
\label{vid:PMqoH4oGt10}
\noindent N=3, 2m25s (3/2021). 
\begin{center}\includegraphics[width=.5\textwidth]{pics/PMqoH4oGt10.jpg} \\ 
\href{https://youtu.be/PMqoH4oGt10}{\url{youtu.be/PMqoH4oGt10}}\end{center}
% Consider the of excentral triangles (orange) to 3-periodics in the elliptic billiard (not shown). The former is inscribed in an ellipse (since the locus of billiard excenters is an ellipse). Consider its affine image (light blue) such that the outer ellipse is sent to a circle and the inner one (elliptic billiard) is sent to a concentric ellipse. 

It turns out both families have invariant and identical product of cosines.
\subsection{Poncelet 3-Periodics with Incircle and Affine Confocal Image: Identical Invariant Sum of Cosines}
\label{vid:CKVoQvErjj4}
\noindent N=3, 2m25s (3/2021). 
\begin{center}\includegraphics[width=.5\textwidth]{pics/CKVoQvErjj4.jpg} \\ 
\href{https://youtu.be/CKVoQvErjj4}{\url{youtu.be/CKVoQvErjj4}}\end{center}
% \input{descr/324_CKVoQvErjj4}

\section{Poncelet Propellers (5)}

\subsection{Poncelet Propellers I: Ellipse Pair with Incircle, Invariant Total Area of Excentral Circumellipses}
\label{vid:tHUDfx9o0Wg}
\noindent N=3, 7m8s (12/2020). 
\begin{center}\includegraphics[width=.5\textwidth]{pics/tHUDfx9o0Wg.jpg} \\ 
\href{https://youtu.be/tHUDfx9o0Wg}{\url{youtu.be/tHUDfx9o0Wg}}\end{center}
% Consider the 3-periodic Poncelet family(blue) inscribed in an external ellipse (black) and circumscribed about an internal concentric circle (brown). The excentral triangle (dashed green) is also shown, its vertices are known as the "excenters". The following properties are illustrated:

a) the circumradius of the family is constant, shown before in [1].
b) the sum of areas of the three circumellipses (contain the vertices) centered on the excenters is invariant, though each is variable.
c) This result has been generalized to 3 other concentric Poncelet families [2], namely, for the constant total area property to hold, the 3 circumellipses need to be centered on the vertices of the anticevian triangle wrt to the common center.

[1] D. Reznik, "Between an Ellipse and a Concentric Circle: Poncelet 3-Periodics Identical to Poristic Triangles", YouTube, July 2020. https://youtu.be/ML_AZoX736w

[2] D. Reznik, "Constant-Area Poncelet Propellers", 2020. https://youtu.be/crXxPJ93ZDk
\subsection{Poncelet Propellers II: invariant total area of anticevian-circumellipse blades}
\label{vid:crXxPJ93ZDk}
\noindent N=3, 16m55s (12/2020). 
\begin{center}\includegraphics[width=.5\textwidth]{pics/crXxPJ93ZDk.jpg} \\ 
\href{https://youtu.be/crXxPJ93ZDk}{\url{youtu.be/crXxPJ93ZDk}}\end{center}
% This is a continuation of [1]. Consider a concentric pair of ellipses and its associated 3-periodic poncelet family. Consider the 3 circumellipses E1,E2,R3 of T centered on the vertices of T', the anticevian [2] of T with respect to the common center. The video shows (experimentally) that the sum of areas of the three said circumellipses is constant for the four pairs shown (confocal, with-incircle, with-circumcircle, homothetic).

[1] https://youtu.be/tHUDfx9o0Wg
[2] https://mathworld.wolfram.com/AnticevianTriangle.html
\subsection{Poncelet Propellers III: family of incircle 3-periodics and one excentral circumellipse}
\label{vid:JUCmAMsfdkI}
\noindent N=3, 1m26s (1/2021). 
\begin{center}\includegraphics[width=.5\textwidth]{pics/JUCmAMsfdkI.jpg} \\ 
\href{https://youtu.be/JUCmAMsfdkI}{\url{youtu.be/JUCmAMsfdkI}}\end{center}
% The family of 3-periodics (blue) is shown inscribed in an outer ellipse and circumscribed about a concentric unit circle. Also shown is the excentral triangle (green), which is also the anticevian triangle with respect to incenter, for this family stationary at the common center. Also shown is a circumellipse C1 (orange) centered on one of the excenters (a vertex of the excentral triangle). Over the family, C1's aspect ratio and area are variable.
\subsection{Poncelet Propellers IV: 3 excentral circumellipses and invariant total area}
\label{vid:ub4wAv8Hgb0}
\noindent N=3, 1m56s (1/2021). 
\begin{center}\includegraphics[width=.5\textwidth]{pics/ub4wAv8Hgb0.jpg} \\ 
\href{https://youtu.be/ub4wAv8Hgb0}{\url{youtu.be/ub4wAv8Hgb0}}\end{center}
% \input{descr/328_ub4wAv8Hgb0}
\subsection{Poncelet Propellers V: Total Area Remains Invariant for Non-Axis-Aligned Concentric Ellipse Pair}
\label{vid:FJXMpUcslaA}
\noindent N=3, 1m58s (1/2021). 
\begin{center}\includegraphics[width=.5\textwidth]{pics/FJXMpUcslaA.jpg} \\ 
\href{https://youtu.be/FJXMpUcslaA}{\url{youtu.be/FJXMpUcslaA}}\end{center}
% The family of Poncelet 3-periodics (blue) is shown interscribed between two concentric, non-axis aligned ellipses. Three circumellipses are shown centered on the vertices of the anticevian triangle (green) with respect to the common center. The video shows that even in this configuration, the sum total of circumellipse areas is invariant.

\section{Poristic (13)}

\subsection{Chapple's Porism from (1746) and Weaver (1927) and Odehnal (2011) Invariants}
\label{vid:DS4ryndDK6Q}
\noindent N=3, PT8M (3/2020). 
\begin{center}\includegraphics[width=.5\textwidth]{pics/DS4ryndDK6Q.jpg} \\ 
\href{https://youtu.be/DS4ryndDK6Q}{\url{youtu.be/DS4ryndDK6Q}}\end{center}
% This demonstrates William Chapple's (1718-1781) Porism, aka. a poristic system triangles [4]. In a 1746 essay in The Gentleman' s Magazine Chapple writes [1]: when two circles are the incircle and circumcircle of a triangle, then there is an infinite family of triangles for which they are the incircle (shown green) and circumcircle (shown purple).

Note this significantly predates Poncelet' s own 1822 work in this area.

We use R=1 and d=0.5, and r=0.375. 

Also we illustrate (i) an invariant reported in [2], that the antiorthic axis (blue, known as L1) is stationary, and another one (ii) reported in [3], that the locus of the excenters is a circle (shown orange).

Note that the point X(1155) sits still at the intersection of the X1X3 with L1.

Also note that the ratio of areas between excentral (green) and reference (blue) triangle is invariant, as this is a corollary to r/R being constant. In fact Aexc/A=2R/r.

Notice the circumcircle (purple) is the 9-point circle of the excentral, which passes thru the vertices of the excentral medial triangle (shown dashed pink).

Sountrack: Mozart Concerto for Piano no 25 in C major- Allegretto

References:

[1] William Chapple, Surveyor, https://en.wikipedia.org/wiki/William_Chapple_(surveyor)
[2] J. H. Weaver, "Invariants of a poristic system of triangles", Bull. Amer. Math. Soc., 33:2, 1927. https://projecteuclid.org/download/pdf_1/euclid.bams/1183492031
[3] B. Odehnal, "Poristic Loci of Triangle Centers, Journal of Geometry and Graphics", 15(1), 2011. https://www.geometrie.tuwien.ac.at/odehnal/pltc.pdf
[4] W. Gallatly, "The Modern Geometry of the Triangle", F. Hodgson, 1914.
\subsection{Circumbilliard of the Poristic Triangle Family: Invariant Aspect Ratio}
\label{vid:yEu2aPiJwQo}
\noindent N=3, 4m1s (4/2020). 
\begin{center}\includegraphics[width=.5\textwidth]{pics/yEu2aPiJwQo.jpg} \\ 
\href{https://youtu.be/yEu2aPiJwQo}{\url{youtu.be/yEu2aPiJwQo}}\end{center}
% The Poristic Triangle Family [1,2] is a set of triangles (blue) with a fixed Incircle (green) and Circumcircle (purple). Let the Circumcenter X3 be on the origin and the Incenter X1 be "d" units above it on the y axis.  Weaver [3] proved their Antiorthic Axis [6] is stationar0 (blue). Its (stationary) intersection with the X1X3 axis is X1155.

Odehnal showed [4] the vertices of the Excentral Triangle (green) sweep a circular locus (orange), centered on X40 and with radius 2R, where R is the circumradius (R=1 on the video). He also showed [4, pp 17] the locus of the Mittenpunkt X9 is a circle (red) centered on M9 = [0, d(2R−r)/(4R+r)] and with radius ρ9 = 2 R(R−2r) /(4R+r). 

This video shows two interesting new phenomena:

a) the Circumbilliard [8] of the poristic family has fixed aspect ratio. This stems from the fact that the family has fixed r/R [7].

b) Also, the Caustic to the Excentral Triangles (dashed green) is centered on X3 and has foci X40 and X1, i.e., it is the MacBeath Inconic of the Excentral Triangle [5].

Sountrack: Liszt - Liebestraum No. 3 (Love Dream)

References:

[1] William Chapple, Surveyor, https://en.wikipedia.org/wiki/William_Chapple_(surveyor)
[2] W. Gallatly, "The Modern Geometry of the Triangle", F. Hodgson, 1914.
[3] J. H. Weaver, "Invariants of a poristic system of triangles", Bull. Amer. Math. Soc., 33:2, 1927. https://projecteuclid.org/download/pdf_1/euclid.bams/1183492031
[4] Boris Odehnal, "Poristic Loci of Triangle Centers, Journal of Geometry and Graphics", 15(1), 2011. https://www.geometrie.tuwien.ac.at/odehnal/pltc.pdf
[5] MacBeath Inconic, https://mathworld.wolfram.com/MacBeathInconic.html
[6] Antiorthic Axis, https://mathworld.wolfram.com/AntiorthicAxis.html
[7] Dan Reznik, Ronaldo Garcia, and Jair Koiller, "Can the Elliptic Billiard Still Surprise Us?", Math, Intelligencer, 42, 2019, https://rdcu.be/b2cg1
[8] Dan Reznik and Ronaldo Garcia, "Circuminvariants of 3-periodics in the Elliptic Billiard, 2020. arXiv: https://arxiv.org/abs/2004.02680
\subsection{Poristic Triangle Family and the Amazing Invariant Excentral $X_{3}$-Centered Inconic}
\label{vid:0VHBjdHXbJc}
\noindent N=3, 4m1s (4/2020). 
\begin{center}\includegraphics[width=.5\textwidth]{pics/0VHBjdHXbJc.jpg} \\ 
\href{https://youtu.be/0VHBjdHXbJc}{\url{youtu.be/0VHBjdHXbJc}}\end{center}
% \input{descr/336_0VHBjdHXbJc}
\subsection{Simson Lines from $X_{100}$ and Excentral Medials are Parallel to L($X_{1}$, $X_{3}$).}
\label{vid:DfzPrZ0SRRc}
\noindent N=3, 4m1s (4/2020). 
\begin{center}\includegraphics[width=.5\textwidth]{pics/DfzPrZ0SRRc.jpg} \\ 
\href{https://youtu.be/DfzPrZ0SRRc}{\url{youtu.be/DfzPrZ0SRRc}}\end{center}
% \input{descr/340_DfzPrZ0SRRc}
\subsection{$X_{1}$-Centered Circumconic \& $X_{40}$-Centered (Excentral) Inconic: Identical Invariant Axes I}
\label{vid:hz0qEyVVvaI}
\noindent N=3, 4m1s (4/2020). 
\begin{center}\includegraphics[width=.5\textwidth]{pics/hz0qEyVVvaI.jpg} \\ 
\href{https://youtu.be/hz0qEyVVvaI}{\url{youtu.be/hz0qEyVVvaI}}\end{center}
% Two instances of the Poristic Triangle family [2,3] discovered by William Chapple in 1746 [1]. This family (blue) has a fixed incircle (green) and fixed circumcircle (purple), centered on the Incenter X1, and Circumcenter X3, respectively. Chapple, and later Euler, discovered that:

d=|X1X3|=Sqrt[R(R-2r)].

Where r,R represent the radius and circumradius. Note this implies: d no smaller than 2 [1].

For both the left (resp. right) simulation, R=1, and  r=0.3627 (resp. r=0.4494).

Also shown are the Excentral Triangles (green) and the locus (orange) of their vertices: a circle centered on X40 and of radius 2R, discovered by Boris Odehnal [4].

The main phenomenon portrayed refers to the two ellipses shown:

1) green: C1 = The X1-centered circumconic to the poristic triangles. Its semiaxes are *invariant* over the family at R+d, and R-d where d=Sqrt[R(R-2r)], i.e., C1 is a fixed ellipse rotating around X1.

2) red: I3' = The X40-centered Inconic to the Excentral Triangles (i.e., centered on the latter's circumcenter). Its semiaxes are *also* invariant and equal to R-d and R+d. I.e., this is a fixed ellipse rotating around X40.

Notice both meet the circumcircle at X100. 

The main result: C1 is a 90-degree rotated copy of I3'.

More info in [5,6].

Soundtrack: Beethoven, "Pathétique", 3rd movement.

[1] William Chapple, Surveryor: https://en.wikipedia.org/wiki/William_Chapple_(surveyor)
[2] W. Gallatly, "The Modern Geometry of the Triangle", F. Hodgson, 1914.
[3] J. H. Weaver, "Invariants of a poristic system of triangles", Bull. Amer. Math. Soc., 33:2, 1927. https://projecteuclid.org/download/pdf_1/euclid.bams/1183492031
[4] Boris Odehnal, "Poristic Loci of Triangle Centers, Journal of Geometry and Graphics", 15(1), 2011. http://www.heldermann-verlag.de/jgg/jgg15/j15h1odeh.pdf
[5] D. Reznik and R. Garcia, "Circuminvariants of 3-Periodics in the Elliptic Billiard", arXiv: https://arxiv.org/abs/2004.02680
[6] Ronaldo Garcia and Dan Reznik, "Related by Similiarity: Poristic Triangles and 3-Periodics in the Elliptic Billiard", 2020, arXiv: https://arxiv.org/abs/2004.13509
\subsection{$X_{1}$-Centered Circumconic \& $X_{40}$-Centered (Excentral) Inconic: Identical Invariant Axes II}
\label{vid:PGdQY7f626Y}
\noindent N=3, 4m1s (4/2020). 
\begin{center}\includegraphics[width=.5\textwidth]{pics/PGdQY7f626Y.jpg} \\ 
\href{https://youtu.be/PGdQY7f626Y}{\url{youtu.be/PGdQY7f626Y}}\end{center}
% The Poristic Triangle family is shown (blue) with fixed incircle (green) and circumcircle (purple). Let r,R represent the radius and circumradius. In this case, r/R=0.36266. Also shown are the Excentral Triangles (green) and the locus (orange) of its vertices: this is a circle centered on X40 and of radius 2R.

Two derived ellipses are shown:

1) green: C1 = The X1-centered circumconic to the poristic triangles. Its semiaxes are *invariant* over the family at R+d, and R-d, where d=Sqrt[R(R-2r)], i.e., C1 is a fixed ellipse rotating around X1.

2) red: I3' = The X40-centered Inconic to the Excentral Triangles (i.e., centered on the latter's circumcenter). Its semiaxes are *also* invariant and equal to R-d and R+d. I.e., this is a fixed ellipse rotating around X40.

The main result: C1 is a 90-degree rotated copy of I3'.

Soundtrack: Beethoven, "Pathétique", 3rd movement.
\subsection{Aspect Ratios of $X_{10}$- and Excentral $X_{5}$-Centered Circumconics are Invariant \& Equal}
\label{vid:-4AAUSFxvmo}
\noindent N=3, 4m1s (4/2020). 
\begin{center}\includegraphics[width=.5\textwidth]{pics/-4AAUSFxvmo.jpg} \\ 
\href{https://youtu.be/-4AAUSFxvmo}{\url{youtu.be/-4AAUSFxvmo}}\end{center}
% The family of poristic triangles [1] (blue) is shown with a fixed circumcircle (purple) and incircle (green), centered on the circumcenter X3 and incenter X1, and with radii R and r, respectively. Here r/R=0.36266.

Also shown is the excentral triangle (green). Its vertices (excenters) describe a circular locus (orange) of radius 2R, i.e., the circumcircle of the excentral triangles is stationary and centered on its X3, or the reference X40 [2].

Also shown is (i) E10, the X10 (spieker point) centered circumconic (pink) to the poristic triangles, (ii) E5', the circumconic to the excentrals (light blue) centered on their X5 (X3 of the reference). Interestingly, not only do both conserve the ratio of their axis' lengths (aspect ratio), this ratio is identical!

Other properties

 - the axes of E10 and E5' are parallel. these are also parallel to E9, the X9-centered circumellipse (aka, the Circumbilliard [3]).
- E10 contains X100.
- Other conics containing X100 and with parallel axes: E1, the X1-centered circumconic, and I3' and I5', the X3- and X5-centered inconics to the Excentral Triangle [3].

Sountrack: Chopin, "Polonaise"

References:

[1] W. Gallatly, "The Modern Geometry of the Triangle",  F. Hodgson, London 1914
[2] B. Odehnal, "Poristic Loci of Triangle Centers", 2011. http://www.heldermann-verlag.de/jgg/jgg15/j15h1odeh.pdf
[3] D. Reznik and R. Garcia, "The Circumbilliard: any Triangle has an Elliptic Billiard", 2020, in preparation.
\subsection{Loci of center and foci of the Circumbilliard to the Poristic Family are circles.}
\label{vid:LGgh11LMGGY}
\noindent N=3, 4m1s (4/2020). 
\begin{center}\includegraphics[width=.5\textwidth]{pics/LGgh11LMGGY.jpg} \\ 
\href{https://youtu.be/LGgh11LMGGY}{\url{youtu.be/LGgh11LMGGY}}\end{center}
% The poristic triangle family (blue) is shown with fixed incircle (green) and circumcircle (purple). A circumellipse, called the "Circumbilliard", is shown (black) for which the poristic triangle a 3-periodic billiard trajectory. Also shown is the locus of X9 (red), the center of the CB, known to be a circle [1]. Also circular is the locus of the CB's foci (cyan), recently discovered [2].

Soundtrack: Villa-Lobos, Choros No 1.

[1] B. Odehnal, "Poristic Loci of Triangle Centers", 2011, http://www.heldermann-verlag.de/jgg/jgg15/j15h1odeh.pdf
[2] R. Garcia and D. Reznik, "Invariants of the Poristic Family and their Relation to Elliptic Billiards", 2020, in preparation.
\subsection{Invariant aspect ratios for the Circumbilliard and Excentral $X_{6}$-Ctr Circumconic}
\label{vid:Fy4T-dmu-8s}
\noindent N=3, 4m1s (4/2020). 
\begin{center}\includegraphics[width=.5\textwidth]{pics/Fy4T-dmu-8s.jpg} \\ 
\href{https://youtu.be/Fy4T-dmu-8s}{\url{youtu.be/Fy4T-dmu-8s}}\end{center}
% In honor of eminent mathematician John H. Conway, lover of triangle geometry [1], deceased today, Sunday April 12, 2020.

The poristic triangle family (blue) [2] has a fixed incircle (green) and circumcircle (purple). These are centered on X(1) and X(3). Also shown is the excentral triangles (green). Their locus (orange) is a circle of radius 2R centered on X(40) [3].

The circumbilliard (black) E9 to the poristic triangles is a circumellipse which renders the latter a 3-periodic billiard orbit. It is dynamically centered on X(9) [4]. Also shown (light blue) is the circumconic E6' to the Excentrals centered on their Symmedian X(6) [X(9) of poristics]. The axes of E9 and E6 are congruent, and both circumconics maintain constant aspect ratio, though their axes are variable.

Note: the locus of the foci of E9 is a perfect circle [5]. However, the locus of the foci of E6' is non-elliptic (probably a quartic, to be determined). 

Sountrack: Pachelbel Canon in D.

References:

[1] J. H. Conway, "Conway Triangle Notation", https://en.wikipedia.org/wiki/Conway_triangle_notation
[2] W. Gallatly, "Modern Geometry of the Triangle", F. Hodgson, 1914.
[3] B. Odehnal, "Poristic Loci of Triangle Centers", 2011. http://www.heldermann-verlag.de/jgg/jgg15/j15h1odeh.pdf
[4] D. Reznik and R. Garcia, "The Circumbilliard: any triangle can have its own Elliptic Billiard", 2020. In preparation.
[5] D. Reznik and R. Garcia, "Poristic Family: center and foci of the Circumbilliard have circular loci.", YouTube, 2020. https://youtu.be/LGgh11LMGGY
\subsection{Side-by-Side View of Poristic and 3-Periodic Families}
\label{vid:NvjrX6XKSFw}
\noindent N=3, 4m1s (4/2020). 
\begin{center}\includegraphics[width=.5\textwidth]{pics/NvjrX6XKSFw.jpg} \\ 
\href{https://youtu.be/NvjrX6XKSFw}{\url{youtu.be/NvjrX6XKSFw}}\end{center}
% \input{descr/339_NvjrX6XKSFw}
\subsection{Feuerbach and Excentral Jerabek Hyperbolas to Poristic Family have invariant focal length ratio}
\label{vid:bn1tq6NU_y0}
\noindent N=3, 4m1s (4/2020). 
\begin{center}\includegraphics[width=.5\textwidth]{pics/bn1tq6NU_y0.jpg} \\ 
\href{https://youtu.be/bn1tq6NU_y0}{\url{youtu.be/bn1tq6NU\_y0}}\end{center}
% The Poristic Triangle family [2,3] was discovered by William Chapple in 1746 [1]. This family (blue) has a fixed incircle (green) and fixed circumcircle (purple) whose centers are the Incenter X1, and Circumcenter X3, respectively. Chapple, and later Euler, discovered that:

d=|X1X3|=Sqrt[R(R-2r)].

Where r,R represent the radius and circumradius, with R/r greate than 2.

Also shown are the Excentral Triangles (green) and the locus (orange) of their vertices: a circle centered on X40 and of radius 2R, discovered by Boris Odehnal [4].

The video superposed to the family two circumhyperbolas:

a) the Feuerbach Hyperbola F (light blue) of the poristics. This is a rectangular hyperbola centered on X11 [5]. Let λ denote its half focal length.

b) the Jerabek Hyperbola of the Excentrals. Also a rectangular hyperbola centered on their X125, i.e., X100 [6]. Let λ' denote its half focal length.

On the top area of the video one notices the ratio λ'/λ remains invariant over the poristic family and equal to Sqrt[2R/r], a result proven in [7].

Also shown (black) is the "circumbilliard"  to the poristics, i.e., a circumellipse inside which a triangle is a 3-periodic orbit [8]. Its axes are a9, b9. Notice the ratio a9/b9 is invariant over the porisitic family. Remarkably, the asymptotes of F and J' are parallel to its axes.

[1] William Chapple, Surveryor: https://en.wikipedia.org/wiki/William_Chapple_(surveyor)
[2] W. Gallatly, "The Modern Geometry of the Triangle", F. Hodgson, 1914.
[3] J. H. Weaver, "Invariants of a poristic system of triangles", Bull. Amer. Math. Soc., 33:2, 1927. https://www.ams.org/journals/bull/1927-33-02/S0002-9904-1927-04367-1/S0002-9904-1927-04367-1.pdf
[4] Boris Odehnal, "Poristic Loci of Triangle Centers, Journal of Geometry and Graphics", 15(1), 2011. 
[5] Feuerbach Hyperbola, https://mathworld.wolfram.com/FeuerbachHyperbola.html
[6] Jerabek Hyperbola, https://mathworld.wolfram.com/JerabekHyperbola.html
[7] Ronaldo Garcia and Dan Reznik, "Related by Similarity: the Poristic and 3-Periodic Triangle Families", April, 2020, arXiv.
[8] Dan Reznik and Ronaldo Garcia,  "The Circumbilliard: Any Triangle can be a 3-Periodic", April, 2020. https://arxiv.org/abs/2004.06776
\subsection{Reference \& Excentral Simson Lines have fixed points and are Orthogonal II}
\label{vid:B06SvYfNByE}
\noindent N=3, 10m42s (6/2020). 
\begin{center}\includegraphics[width=.5\textwidth]{pics/B06SvYfNByE.jpg} \\ 
\href{https://youtu.be/B06SvYfNByE}{\url{youtu.be/B06SvYfNByE}}\end{center}
% \input{descr/338_B06SvYfNByE}
\subsection{Reference \& Excentral Simson Lines have fixed points and are Orthogonal I}
\label{vid:M9NIRnfGtGc}
\noindent N=3, 12m2s (6/2020). 
\begin{center}\includegraphics[width=.5\textwidth]{pics/M9NIRnfGtGc.jpg} \\ 
\href{https://youtu.be/M9NIRnfGtGc}{\url{youtu.be/M9NIRnfGtGc}}\end{center}
% \input{descr/337_M9NIRnfGtGc}

\section{Quasi-Porisms (6)}

\subsection{Multiple-caustic Poncelet I: a curious N=6 porism}
\label{vid:jZQvbzD_DFw}
\noindent N=6, 2m13s (10/2021). 
\begin{center}\includegraphics[width=.5\textwidth]{pics/jZQvbzD_DFw.jpg} \\ 
\href{https://youtu.be/jZQvbzD_DFw}{\url{youtu.be/jZQvbzD\_DFw}}\end{center}
% The video shows a circle-inscribed N=6 Poncelet transverse which alternates between two concentric, elliptic caustics (identical  modulo 90 degree rotation). These have been chosen so as to perfectly close and N=6 Poncelet polygon for some starting P1 on the outer circle. Surprisingly, as P1 is slid, closure is nearly maintained, i.e., the distance between first and last vertex is variable but remains within a very small fraction (10^-4 to 10^-5) of the circumradius.
\subsection{Multiple-caustic Poncelet II: an N=14 example}
\label{vid:wLpLwcmBLms}
\noindent N=6,14, 1m20s (10/2021). 
\begin{center}\includegraphics[width=.5\textwidth]{pics/wLpLwcmBLms.jpg} \\ 
\href{https://youtu.be/wLpLwcmBLms}{\url{youtu.be/wLpLwcmBLms}}\end{center}
% \input{descr/344_wLpLwcmBLms}
\subsection{Multiple-Caustic Poncelet III: an N=10 example on variant of the bicentric family}
\label{vid:iN9RreO6ONA}
\noindent N=10, 3m27s (10/2021). 
\begin{center}\includegraphics[width=.5\textwidth]{pics/iN9RreO6ONA.jpg} \\ 
\href{https://youtu.be/iN9RreO6ONA}{\url{youtu.be/iN9RreO6ONA}}\end{center}
% The video illustrates a two-caustic pseudo-Poncelet porism involving an outer circle C and two internal caustics D1, D2, where D2 is a copy of D1 rigidly rotated about the center of C by 45 degrees. Their radii have been tuned so that the Poncelet transverse closes within 10 iterations. Surprisingly, we get a porism.

Note: closure is maintained within a fraction of 10^-4 to 10^-5 of the circumradius.
\subsection{Multiple-Caustic Poncelet IV: cases when a pseudo-porism is maintained and cases when it fails}
\label{vid:XkmyB2uY5XE}
\noindent N=4,6,8, 9m28s (10/2021). 
\begin{center}\includegraphics[width=.5\textwidth]{pics/XkmyB2uY5XE.jpg} \\ 
\href{https://youtu.be/XkmyB2uY5XE}{\url{youtu.be/XkmyB2uY5XE}}\end{center}
% \input{descr/346_XkmyB2uY5XE}
\subsection{Multiple-Caustic Poncelet V: a non-poristic and a quasi-poristic N=4, what is going on?}
\label{vid:gqzKEBVEoiE}
\noindent N=4, 4m9s (10/2021). 
\begin{center}\includegraphics[width=.5\textwidth]{pics/gqzKEBVEoiE.jpg} \\ 
\href{https://youtu.be/gqzKEBVEoiE}{\url{youtu.be/gqzKEBVEoiE}}\end{center}
% \input{descr/347_gqzKEBVEoiE}
\subsection{Multiple-Caustic Poncelet VI: more quasi-porisms with two circular caustics and round-robin tangents}
\label{vid:yypsjCNSYj0}
\noindent N=4,6,8,10,128, 11m5s (11/2021). 
\begin{center}\includegraphics[width=.5\textwidth]{pics/yypsjCNSYj0.jpg} \\ 
\href{https://youtu.be/yypsjCNSYj0}{\url{youtu.be/yypsjCNSYj0}}\end{center}
% Video explores N=4,6,8,10,18 quasi-porisms of a 2-caustic poncelet transverse onto two circular caustics used in round-robin fashion.

\section{Rotating Equilaterals (12)}

\subsection{Rotating Equilaterals I: cevian triangles with respect to a fixed point on circumcircle}
\label{vid:_QM6yNQ9heU}
\noindent N=3, 14m36s (10/2021). 
\begin{center}\includegraphics[width=.5\textwidth]{pics/_QM6yNQ9heU.jpg} \\ 
\href{https://youtu.be/_QM6yNQ9heU}{\url{youtu.be/\_QM6yNQ9heU}}\end{center}
% \input{descr/349__QM6yNQ9heU}
\subsection{Rotating Equilaterals II: loci of cevian triangle centers post-affine transformation}
\label{vid:f01RPhQS0mQ}
\noindent N=3, 10m34s (10/2021). 
\begin{center}\includegraphics[width=.5\textwidth]{pics/f01RPhQS0mQ.jpg} \\ 
\href{https://youtu.be/f01RPhQS0mQ}{\url{youtu.be/f01RPhQS0mQ}}\end{center}
% Mimi and I explore the loci of triangle centers of cevians of rotating equilaterals with respect to a fixed point on the circumcircle, pre- and post-affine transformation, i.e., over Poncelet homothetics, in which case the circumcircle is the Steiner ellipse.
\subsection{Rotating Equilaterals III: loci of cevian triangle centers w.r.t. a fixed point on the incircle}
\label{vid:1Xtb0H1S8Z4}
\noindent N=3, 12m51s (10/2021). 
\begin{center}\includegraphics[width=.5\textwidth]{pics/1Xtb0H1S8Z4.jpg} \\ 
\href{https://youtu.be/1Xtb0H1S8Z4}{\url{youtu.be/1Xtb0H1S8Z4}}\end{center}
% We explore the curious loci of cevian triangle centers of a family of rotating equilaterals with respect to a point P on the latter's incirlce.
\subsection{Rotating Equilaterals IV: loci of cevian triangle centers w.r.t. any point on the plane}
\label{vid:z_hndz19lH4}
\noindent N=3, 18m21s (10/2021). 
\begin{center}\includegraphics[width=.5\textwidth]{pics/z_hndz19lH4.jpg} \\ 
\href{https://youtu.be/z_hndz19lH4}{\url{youtu.be/z\_hndz19lH4}}\end{center}
% We explore the curious loci of cevian triangle centers of a family of rotating equilaterals with respect to a point P anywhere on the plane.
\subsection{Rotating Equilaterals V: Eversions of the Anticevian triangle w.r.t. a point P on the circumcircle}
\label{vid:aGjKEfAv2V8}
\noindent N=3, 3m34s (10/2021). 
\begin{center}\includegraphics[width=.5\textwidth]{pics/aGjKEfAv2V8.jpg} \\ 
\href{https://youtu.be/aGjKEfAv2V8}{\url{youtu.be/aGjKEfAv2V8}}\end{center}
% \input{descr/354_aGjKEfAv2V8}
\subsection{Rotating Equilaterals IX: Loci of $X_{6}$, $X_{13}$, $X_{14}$, $X_{15}$, $X_{16}$ of antipedal triangles}
\label{vid:RF6Mm65qQgI}
\noindent N=3, 11m18s (10/2021). 
\begin{center}\includegraphics[width=.5\textwidth]{pics/RF6Mm65qQgI.jpg} \\ 
\href{https://youtu.be/RF6Mm65qQgI}{\url{youtu.be/RF6Mm65qQgI}}\end{center}
% \input{descr/353_RF6Mm65qQgI}
\subsection{Rotating Equilaterals VI: The nature of eversions of the anticevian triangle w.r.t. a point P}
\label{vid:hLKLq8eUmsY}
\noindent N=3, 3m46s (10/2021). 
\begin{center}\includegraphics[width=.5\textwidth]{pics/hLKLq8eUmsY.jpg} \\ 
\href{https://youtu.be/hLKLq8eUmsY}{\url{youtu.be/hLKLq8eUmsY}}\end{center}
% The video shows that for any triangle the P-anticevian has one vertex on the line of infinity when P is on a midline of the reference triangle.

NOTE: whenever I say "antipedal" I mean "anticevian"
\subsection{Rotating Equilaterals VII: Conic locus of $X_{2}$ of antipedal triangles}
\label{vid:8NBh0H2Vps0}
\noindent N=3, 10m7s (10/2021). 
\begin{center}\includegraphics[width=.5\textwidth]{pics/8NBh0H2Vps0.jpg} \\ 
\href{https://youtu.be/8NBh0H2Vps0}{\url{youtu.be/8NBh0H2Vps0}}\end{center}
% We investigate the locus of the barycenter of antipedal triangles over a rotating family of equilaterals with respect to a fixed point P. We show this locus is an ellipse/parabola/hyperbola if P is interior, on, exterior to the incircle of the medial triangle (also known as the Spieker circle).
\subsection{Rotating Equilaterals VIII: Loci of $X_{3}$, $X_{4}$, $X_{5}$ of antipedal triangles}
\label{vid:wiknzClcm4s}
\noindent N=3, 14m57s (10/2021). 
\begin{center}\includegraphics[width=.5\textwidth]{pics/wiknzClcm4s.jpg} \\ 
\href{https://youtu.be/wiknzClcm4s}{\url{youtu.be/wiknzClcm4s}}\end{center}
% We investigate the locus of the circumcenter X3, orthocenter X4, and 9-pt-center X5 of antipedal triangles over a rotating family of equilaterals with respect to a fixed point P. We show all are conics and the conic type depends on the position of P chiefly with respect to the incircle and the Spieker circle of the family.
\subsection{Rotating Equilaterals X: Two incredible pedal and antipedal loci over Poncelet homothetics}
\label{vid:aRadZgO4n2Q}
\noindent N=3, 7m53s (10/2021). 
\begin{center}\includegraphics[width=.5\textwidth]{pics/aRadZgO4n2Q.jpg} \\ 
\href{https://youtu.be/aRadZgO4n2Q}{\url{youtu.be/aRadZgO4n2Q}}\end{center}
% The video describes two (yet unproven) phenomena regarding the loci of X2 and X3 of pedals and antipedals of Poncelet triangles in the homothetic family with respect to a point P on the outer conic E (Steiner ellipse). Let its aspect ratio be a/b. Namely:

a) the locus of X3 of pedals is a straight line (that of X2 looks like a cardioid)
b) the locus of X2 of antipedals is a stationary point (that of X3 is an axis-aligned ellipse)
c) the locus of (b) over all P is an ellipse E2 concentric and axis-aligned with E, and with aspect ratio is b/a.

d) for a given Poncelet triangle, over all P on E, the locus of the vertices of antipedals are 3 distinct ellipses Z1,Z2,Z3, all with aspect ratio b/a. These and E2 all meet at a point. 
e) over Poncelet, the sum of the areas of the Zi is invariant!

Cheers!
\subsection{Rotating Equilaterals XI: Loci of triangle centers of pedal triangles}
\label{vid:S2OZLny2Hfo}
\noindent N=3, 20m44s (10/2021). 
\begin{center}\includegraphics[width=.5\textwidth]{pics/S2OZLny2Hfo.jpg} \\ 
\href{https://youtu.be/S2OZLny2Hfo}{\url{youtu.be/S2OZLny2Hfo}}\end{center}
% The video explores a rich set of loci swept by triangle centers of pedal triangles over equilaterals (or affine transformations thereof) wrt point P on one of the main conics or on some general position.
\subsection{Rotating Equilaterals XII: Loci of triangle centers of antipedal triangles}
\label{vid:m7iqvZFL-o8}
\noindent N=3, 29m36s (10/2021). 
\begin{center}\includegraphics[width=.5\textwidth]{pics/m7iqvZFL-o8.jpg} \\ 
\href{https://youtu.be/m7iqvZFL-o8}{\url{youtu.be/m7iqvZFL-o8}}\end{center}
% This is the 12th and final video on this experimental series. Here we explore myriad phenomena associated with P-antipedal triangles to a family of rotating equilaterals, including loci of their vertices and their triangle centers. P can be anywhere on the plane or on one of the main circles associated with the reference family. Said phenomena are also analyzed upon affine transformations of the base family.

\section{Self-Intersected (16)}

\subsection{Self-intersecting 5-periodics (pentagram)}
\label{vid:ECe4DptduJY}
\noindent N=5, 2m25s (6/2019). 
\begin{center}\includegraphics[width=.5\textwidth]{pics/ECe4DptduJY.jpg} \\ 
\href{https://youtu.be/ECe4DptduJY}{\url{youtu.be/ECe4DptduJY}}\end{center}
% \input{descr/361_ECe4DptduJY}
\subsection{Self-intersecting 4-periodics (bowtie and tangential polygon) II}
\label{vid:cCYxN7ueGV4}
\noindent N=4, 2m25s (6/2019). 
\begin{center}\includegraphics[width=.5\textwidth]{pics/cCYxN7ueGV4.jpg} \\ 
\href{https://youtu.be/cCYxN7ueGV4}{\url{youtu.be/cCYxN7ueGV4}}\end{center}
% An elliptic billiard with a/b=1.5 is shown. The family of N=4 self-intersecting "bowtie" orbits (blue) is shown for all valid starting positions on the top boundary. Also shown are two branches of the confocal hyperbolic caustics (green) to which every orbit is tangent. A dotted inverted V is shown depicting the position in which the N=4 orbit contains two coinciding branches (double orbit).

Observation 1: it turns out if a/b is smaller than sqrt(2) no self-intersecting N=4 are possible. 

Observation 2: this orbit conserves the sum of its internal cosines (in the case of N=4 non-self-intersecting this value is trivially zero).

More info: https://dan-reznik.github.io/Elliptical-Billiards-Triangular-Orbits/
\subsection{Self-intersecting 5-periodics (pentagram): Locus of Internal Intersections}
\label{vid:ZaqvmK22pBM}
\noindent N=5, 3m12s (7/2019). 
\begin{center}\includegraphics[width=.5\textwidth]{pics/ZaqvmK22pBM.jpg} \\ 
\href{https://youtu.be/ZaqvmK22pBM}{\url{youtu.be/ZaqvmK22pBM}}\end{center}
% An a/b=1.15 elliptic billiard is shown (black) as well as its family of N=5 self-intersecting (pentagram) orbits (blue). Also shown are the latter's 5 points of self-intersection (forming a pentagon), their elliptic locus (purple), and the caustic (green).

Notice the pentagon of self intersections *is* a billiard orbit of its the locus of its vertices, which is an ellipse.

More Info:  https://dan-reznik.github.io/Elliptical-Billiards-Triangular-Orbits/
\subsection{Self-intersecting 4-periodics (bowtie and tangential polygon) I}
\label{vid:C8W2e6ftfOw}
\noindent N=4, 4m27s (1/2020). 
\begin{center}\includegraphics[width=.5\textwidth]{pics/C8W2e6ftfOw.jpg} \\ 
\href{https://youtu.be/C8W2e6ftfOw}{\url{youtu.be/C8W2e6ftfOw}}\end{center}
% \input{descr/371_C8W2e6ftfOw}
\subsection{Elliptic Billiard: Self-Intersected 6-Periodics (type I)}
\label{vid:fOD85MNrmdQ}
\noindent N=6, 1m31s (11/2020). 
\begin{center}\includegraphics[width=.5\textwidth]{pics/fOD85MNrmdQ.jpg} \\ 
\href{https://youtu.be/fOD85MNrmdQ}{\url{youtu.be/fOD85MNrmdQ}}\end{center}
% This is joint work w/ R. Garcia.

Family of self-intersected 6-periodics (blue) in the elliptic billiard of type I tangent to a hyperbolic caustic (green). A doubled-up configuration (red) is shown.

Type II 6-periodics are shown in https://youtu.be/gQ-FbSq7wWY
\subsection{Elliptic Billiard: Self-Intersected 6-Periodics (type II)}
\label{vid:gQ-FbSq7wWY}
\noindent N=6, 3m23s (11/2020). 
\begin{center}\includegraphics[width=.5\textwidth]{pics/gQ-FbSq7wWY.jpg} \\ 
\href{https://youtu.be/gQ-FbSq7wWY}{\url{youtu.be/gQ-FbSq7wWY}}\end{center}
% This is joint work with R. Garcia.

Family of self-intersected 6-periodics (blue) in the elliptic billiard of type II tangent to a hyperbolic caustic (green). A doubled-up configuration (red) is shown.

6-Periodics of Type I are shown in https://youtu.be/fOD85MNrmdQ
\subsection{Self-Intersected 6-Periodics in the Elliptic Billiard: Invariant Perimeter Focus-Inversive Polygon}
\label{vid:7lXwjXj-8YY}
\noindent N=6, 1m57s (11/2020). 
\begin{center}\includegraphics[width=.5\textwidth]{pics/7lXwjXj-8YY.jpg} \\ 
\href{https://youtu.be/7lXwjXj-8YY}{\url{youtu.be/7lXwjXj-8YY}}\end{center}
% Shown are Elliptic Billiard families of two types of self-intersected 6-periodics (blue); these are tangent to hyperbolic confocal caustics (green). Note that both families conserve their perimeter and sum of cosines.

Also shown are their inversive polygons (pink), whose vertices are inversions of the original ones with respect to a unit-circle centered on the left focus. Concerning the inversive polygon:

1) In both cases the perimeter is invariant.

2) Only the left case conserves the sum of cosines. We think this could have something to do with the fact that both 6-periodic and the inversive have zero signed area.
\subsection{Elliptic Billiard 8-Periodics: Null sum of double cosines of outer polygon}
\label{vid:GEmV_U4eRIE}
\noindent N=8, 1m43s (11/2020). 
\begin{center}\includegraphics[width=.5\textwidth]{pics/GEmV_U4eRIE.jpg} \\ 
\href{https://youtu.be/GEmV_U4eRIE}{\url{youtu.be/GEmV\_U4eRIE}}\end{center}
% \input{descr/363_GEmV_U4eRIE}
\subsection{Family of Self-Intersecting 4-Periodics in the Elliptic Billiard: Inversive Polygon is a Segment}
\label{vid:207Ta31Pl9I}
\noindent N=4, 1m36s (11/2020). 
\begin{center}\includegraphics[width=.5\textwidth]{pics/207Ta31Pl9I.jpg} \\ 
\href{https://youtu.be/207Ta31Pl9I}{\url{youtu.be/207Ta31Pl9I}}\end{center}
% \input{descr/369_207Ta31Pl9I}
\subsection{Type I Self-Intersected 8-Periodics in the Elliptic Billiard and the Inversive Polygon}
\label{vid:5Lt9atsZhRs}
\noindent N=8, 2m54s (11/2020). 
\begin{center}\includegraphics[width=.5\textwidth]{pics/5Lt9atsZhRs.jpg} \\ 
\href{https://youtu.be/5Lt9atsZhRs}{\url{youtu.be/5Lt9atsZhRs}}\end{center}
% Joint work w Ronaldo Garcia.

Show is a the family of type I self-intersected 8-periodics (blue) in the elliptic billiard (black) (for type II see [1]). All trajectory segments are tangent to a confocal hyperbolic caustic (brown).

Also shown is the constant-perimeter inversive polygon (pink), where the inversion is wrt unit circle centered on left focus. 

[1] Type II Self-Intersected N-Periodics, https://youtu.be/93xpGnDxyi0
\subsection{Type II Self-Intersected 8-Periodics in the Elliptic Billiard + Outer \& Inversive Polygons}
\label{vid:93xpGnDxyi0}
\noindent N=8, 3m26s (11/2020). 
\begin{center}\includegraphics[width=.5\textwidth]{pics/93xpGnDxyi0.jpg} \\ 
\href{https://youtu.be/93xpGnDxyi0}{\url{youtu.be/93xpGnDxyi0}}\end{center}
% \input{descr/375_93xpGnDxyi0}
\subsection{Elliptic Billiard Self-Intersected 7-Periodics, a/b=2: Invariant Perimeter Focus-Inversive Polygons}
\label{vid:gf_aHyvbqOY}
\noindent N=7, 1m45s (11/2020). 
\begin{center}\includegraphics[width=.5\textwidth]{pics/gf_aHyvbqOY.jpg} \\ 
\href{https://youtu.be/gf_aHyvbqOY}{\url{youtu.be/gf\_aHyvbqOY}}\end{center}
% \input{descr/364_gf_aHyvbqOY}
\subsection{Family of self-intersected N=8 with turning number 2 in the Elliptic Billiiard}
\label{vid:JwD_w5ecPYs}
\noindent N=8, 1m31s (11/2020). 
\begin{center}\includegraphics[width=.5\textwidth]{pics/JwD_w5ecPYs.jpg} \\ 
\href{https://youtu.be/JwD_w5ecPYs}{\url{youtu.be/JwD\_w5ecPYs}}\end{center}
% \input{descr/368_JwD_w5ecPYs}
\subsection{The two types of self-intersected 7-periodics in the Elliptic Billiard}
\label{vid:yzBG8rgPUP4}
\noindent N=7, 1m44s (11/2020). 
\begin{center}\includegraphics[width=.5\textwidth]{pics/yzBG8rgPUP4.jpg} \\ 
\href{https://youtu.be/yzBG8rgPUP4}{\url{youtu.be/yzBG8rgPUP4}}\end{center}
% \input{descr/373_yzBG8rgPUP4}
\subsection{Elliptic Billiard: Vertices of Self-Intersected 4-Periodics \& Outer Polygon are concyclic with foci}
\label{vid:4g-JBshX10U}
\noindent N=4, 2m19s (11/2020). 
\begin{center}\includegraphics[width=.5\textwidth]{pics/4g-JBshX10U.jpg} \\ 
\href{https://youtu.be/4g-JBshX10U}{\url{youtu.be/4g-JBshX10U}}\end{center}
% \input{descr/367_4g-JBshX10U}
\subsection{Self-Intersected Four Periodics in the Elliptic Billiard: Chockfull of Properties}
\label{vid:GZCrek7RTpQ}
\noindent N=4, 9m13s (1/2021). 
\begin{center}\includegraphics[width=.5\textwidth]{pics/GZCrek7RTpQ.jpg} \\ 
\href{https://youtu.be/GZCrek7RTpQ}{\url{youtu.be/GZCrek7RTpQ}}\end{center}
% The video shows the property-rich family of self-intersected 4-periodics in the elliptic billiard. A few of the properties covered:

a) constant perimeter
b) vertices concyclic with foci. this implies the inversive polygon with respect to a focus degenerates to a segment;
b.1) [not shown]: power of the center wrt to circumcircle is constant.
c) side midpoints are collinear on horizontal line. their locus is an 8-shaped curve
d) the vertices of the "excentral polygon" are also concyclic with the foci.
d.1) the power of the center with respect to the excentral circumcircle is also constant and equal to b.1.

\section{Stationary Circles (5)}

\subsection{Cosine Circle of Excentral Triangle is Stationary}
\label{vid:ACinCf-D_Ok}
\noindent N=3, 6m25s (7/2019). 
\begin{center}\includegraphics[width=.5\textwidth]{pics/ACinCf-D_Ok.jpg} \\ 
\href{https://youtu.be/ACinCf-D_Ok}{\url{youtu.be/ACinCf-D\_Ok}}\end{center}
% \input{descr/378_ACinCf-D_Ok}
\subsection{Intersections of Excentral Triangle and its Reflection is a Circle}
\label{vid:hCQIT6_XhaQ}
\noindent N=3, 6m25s (7/2019). 
\begin{center}\includegraphics[width=.5\textwidth]{pics/hCQIT6_XhaQ.jpg} \\ 
\href{https://youtu.be/hCQIT6_XhaQ}{\url{youtu.be/hCQIT6\_XhaQ}}\end{center}
% \input{descr/379_hCQIT6_XhaQ}
\subsection{Locus of Intersection of Anti-Tangents is Stationary Circle}
\label{vid:CrOSI8d8qDc}
\noindent N=3, 2m1s (7/2019). 
\begin{center}\includegraphics[width=.5\textwidth]{pics/CrOSI8d8qDc.jpg} \\ 
\href{https://youtu.be/CrOSI8d8qDc}{\url{youtu.be/CrOSI8d8qDc}}\end{center}
% \input{descr/380_CrOSI8d8qDc}
\subsection{5-periodics and a stationary circle}
\label{vid:dINE4aH1cvk}
\noindent N=5, 2m25s (7/2019). 
\begin{center}\includegraphics[width=.5\textwidth]{pics/dINE4aH1cvk.jpg} \\ 
\href{https://youtu.be/dINE4aH1cvk}{\url{youtu.be/dINE4aH1cvk}}\end{center}
% \input{descr/377_dINE4aH1cvk}
\subsection{Stationary circles for N=3 to 8}
\label{vid:EFeINGIDFrg}
\noindent N=3 to 8, 2m25s (7/2019). 
\begin{center}\includegraphics[width=.5\textwidth]{pics/EFeINGIDFrg.jpg} \\ 
\href{https://youtu.be/EFeINGIDFrg}{\url{youtu.be/EFeINGIDFrg}}\end{center}
% \input{descr/381_EFeINGIDFrg}

\section{Steiner's Hat (9)}

\subsection{Equal sum of distances from each focus to vertices of antipedal polygon}
\label{vid:UzBt4R1jGYE}
\noindent N=3,4,5,6, 5m2s (5/2020). 
\begin{center}\includegraphics[width=.5\textwidth]{pics/UzBt4R1jGYE.jpg} \\ 
\href{https://youtu.be/UzBt4R1jGYE}{\url{youtu.be/UzBt4R1jGYE}}\end{center}
% Consider the family of N-periodics in the Elliptic Billiard. Let Pi, i=1,...,N be its vertices. Its antipedal polygon w/ respect to a point M has sides which pass through the Pi and are perpendicular to the lines Pi-M.

Let the antipedal vertices with respect to f1 be called Q_{1,i}, and with respect to f2 be called Q_{2,i}.

This (narrated) video shows that the sum of distances from f1 to the the antipedal vertices wrt f1 is the same as the sum of distances from f2 to the antipedal vertices wrt to f2. This is invariant k_{603} in [1].

[1] D. Reznik, R. Garcia, and J. Koiller, "Forty new invariants in the Elliptic Billiard", April 2020. ArXiv: https://arxiv.org/abs/2004.12497
\subsection{Pedal Polygons to N-Periodics with respect to a Focus: Concyclic Vertices and Circular Caustic}
\label{vid:7TE3a5vEWuU}
\noindent N=3,4,5,6, 3m50s (5/2020). 
\begin{center}\includegraphics[width=.5\textwidth]{pics/7TE3a5vEWuU.jpg} \\ 
\href{https://youtu.be/7TE3a5vEWuU}{\url{youtu.be/7TE3a5vEWuU}}\end{center}
% We've been interested in properties of N-periodics in the Elliptic Billiard (EB) [1,2]. 

Let P_i denote the vertices of an N-periodic, i=1,...,N. Consider a point M. Let the feet Q_i of perpendiculars dropped from M onto sides (P_i,P_{i+1}) be the vertices of the so-called "Pedal Polygon with respect to M".

If M is a focus of the EB, then over the family of N-periodics two remarkable things happen:

a) the Q_i move along a circle (shown dashed) concentric with the EB and of radius equal to the major axis of the confocal caustic to the N-periodics (not shown).

b) the caustic to the family of pedal polygons is _also_ a circle (shown solid) whose center lies on the major axis of the EB.

Therefore the family of pedal polygons with respect to a focus of the EB is Ponceletian, with outer and inner conics being the [a] and [b] non-concentric circles.

An explanation for the phenomenon has been kindly extended [3]: the polar image of an ellipse with respect to a circle with center at one focus is also circle.  So, the N-periodic between confocal ellipses is a dual to the Poncelet trajectory between two circles.

The video shows said pedal families for N=3,4,5,6.

Soundtrack: Alexander Borodin, "Polovetsian Dances"

References:

[1] Dan Reznik, Ronaldo Garcia, Jair Koiller, "Can the Elliptic Billiard Still Suprise Us?", Math. Intelligencer, Vol 42, Dec 2019, https://rdcu.be/b2cg1

[2] Dan Reznik, Ronaldo Garcia, Jair Koiller, "Forty New Invariants of N-Periodics in the Elliptic Billiard", April 2020, https://arxiv.org/abs/2004.12497

[3] Arseniy Akoypan, Private Communication, April, 2020.
\subsection{The Envelope of Ellipse Antipedals is a Constant-Area Deltoid I}
\label{vid:wetmchfY5jI}
\noindent N=n/a, 2m25s (5/2020). 
\begin{center}\includegraphics[width=.5\textwidth]{pics/wetmchfY5jI.jpg} \\ 
\href{https://youtu.be/wetmchfY5jI}{\url{youtu.be/wetmchfY5jI}}\end{center}
% Take an ellipse E with semiaxes a,b. Let P(t) = [a cos(t), b sin(t)] be a point on E, t in [0,2pi]. Let M be a fixed point on M's boundary.

The envelope of lines L(t) passing through P(t) and perpendicular to P(t)-M is a deltoid (3-cusps) and its area is the same for all M on E [1].

The video shows said envelopes for a/b=1.125, Sqrt[2], and the golden ratio (left, mid, right). For each case the dynamically invariant area A is shown at the top.

Also shown is the center of area of the deltoid, which appears to move along an elliptic path (dashed red). For comparison, a 90-degree rotated copy of the ellipse is also shown (dashed black). Numerically the former is always exterior to the latter.

Notice when a/b=Sqrt[2], the focal length is unity and one of the deltoid cusps passes through the center of E.

[1] Ronaldo Garcia, Private Communication, May, 2020.
\subsection{The Envelope of Ellipse Antipedals is a Constant-Area Deltoid II}
\label{vid:_SuGYRS43EU}
\noindent N=n/a, 2m25s (5/2020). 
\begin{center}\includegraphics[width=.5\textwidth]{pics/_SuGYRS43EU.jpg} \\ 
\href{https://youtu.be/_SuGYRS43EU}{\url{youtu.be/\_SuGYRS43EU}}\end{center}
% Take an ellipse E with semiaxes a,b. Let P(t) = [a cos(t), b sin(t)] be a point on E, t in [0,2pi]. Let M be a fixed point on M's boundary.

The envelope of lines L(t) passing through P(t) and perpendicular to P(t)-M is a deltoid and its area is the same for all M on E [1].

The video shows said deltoidal envelop for a=1.5, b=1. The dynamically-computed area A is shown at the top.

Also shown is the center of area of the deltoid, which appears to move along an elliptic path (dashed red). For comparison, a 90-degree rotated copy of the ellipse is also shown (dashed black). Numerically the former is always exterior to the latter, no matter what a/b is.

Note: when a/b=Sqrt[2], one of its cusps passes through the center of E (show on a sister video [2]).

[1] Ronaldo Garcia, Private Communication, May, 2020.

[2] https://youtu.be/wetmchfY5jI
\subsection{A narrated tour of the Garcia Deltoid: Surprising Invariants and Properties}
\label{vid:LxADeM1-WHw}
\noindent N=n/a, 17m33s (5/2020). 
\begin{center}\includegraphics[width=.5\textwidth]{pics/LxADeM1-WHw.jpg} \\ 
\href{https://youtu.be/LxADeM1-WHw}{\url{youtu.be/LxADeM1-WHw}}\end{center}
% We explore a slew of interesting properties and invariants of a special envelope of lines discovered by Prof. Ronaldo Garcia [1] which we have baptized the "Steiner's Hat", as it is related to Steiner's Dekltoid [4].

Take an ellipse E with semiaxes (a,b). Let M be a fixed point on E, and P(t) = [a cos(t), b sin(t)], t in [0,2pi] be a parametric point also on E.

The family of lines L(t) passing through P(t) and perpendicular to P(t)-M envelops, also known as the negative pedal curve or orthocaustic to the ellipse, is a 3-cuspid deltoid. Surpisingly its area is invariant over all M [1].

The 3 preimages of the deltoid cusp's on E form a triangle with barycenter stationary at E's center, and whose area is also invariant. Therefore E is its Steiner Circumellipse. The caustic to this family is its Steiner Inellipse, i.e., of axes (a/2,b/2).

However, Steiner's hat is different from Steiner's! :^)

Note: during the video I will occasionally and mistakenly call the ellipse a "billiard". Just disregard that. This geometry shown here is unrelated to Elliptic Billiards.

References:

[1] Prof. Ronaldo Garcia (UFG), Private Communication, May, 2020.
[2] https://youtu.be/wetmchfY5jI
[3] https://youtu.be/_SuGYRS43EU
[4] Steiner Deltoid, https://mathworld.wolfram.com/SteinerDeltoid.html
\subsection{Properties of Osculating Circles to the Ellipse at the 3 Cusp Pre-Images}
\label{vid:NwXc-Vfjs98}
\noindent N=n/a, 9m46s (5/2020). 
\begin{center}\includegraphics[width=.5\textwidth]{pics/NwXc-Vfjs98.jpg} \\ 
\href{https://youtu.be/NwXc-Vfjs98}{\url{youtu.be/NwXc-Vfjs98}}\end{center}
% This is a continuation of joint with Prof Ronaldo Garcia and Hellmuth Stachel, see [2]. Let E be an ellipse, and let M be a point on it about which we will calculate a (deltoidal) Orthocaustic, aka Negative Pedal Curve [1]. Let P_i' the 3 cusps of said envelope and P_i their preimages on E [2]. 

Here we cover properties associated with the Osculating Circles centered at K_i and tangent to the Ellipse at the P_i.

Note: The center of an osculating circle is known to lie on the Evolute to the curve [3], and for the Ellipse, this is an astroid of known parametrization [4]. 

We have found:

1) The three osculating circles tangent to E at the P_i intersect at M. 
2) The K_i form a triangle T whose area is invariant over all M. 
3) The P_i's lie on the 3 osculating circles. Corollary: the lines P_i K_i are diameters of the osculating cirlces.
4) Said diameters concur at M.
5) Let T' be the triangle defined by the deltoid cusps P_i. Let A' be its area, and A be the area of T. A'/A=4, irrespective of E's dimensions.

[1] Orthocaustic, aka., negative pedal curve: https://mathworld.wolfram.com/NegativePedalCurve.html
[2] Steiner's Hat, https://youtu.be/LxADeM1-WHw
[3] Osculating Circle, https://mathworld.wolfram.com/OsculatingCircle.html
[4] Ellipse Evolute, https://mathworld.wolfram.com/EllipseEvolute.html
\subsection{Locus of Cusps and Deltoid Center of Area}
\label{vid:rZht21KFXk4}
\noindent N=n/a, 4m1s (6/2020). 
\begin{center}\includegraphics[width=.5\textwidth]{pics/rZht21KFXk4.jpg} \\ 
\href{https://youtu.be/rZht21KFXk4}{\url{youtu.be/rZht21KFXk4}}\end{center}
% \input{descr/385_rZht21KFXk4}
\subsection{Concyclic pre-images, osculating circles, and 3 area-invariant triangles}
\label{vid:fwyr6LXFS1c}
\noindent N=n/a, 4m45s (6/2020). 
\begin{center}\includegraphics[width=.5\textwidth]{pics/fwyr6LXFS1c.jpg} \\ 
\href{https://youtu.be/fwyr6LXFS1c}{\url{youtu.be/fwyr6LXFS1c}}\end{center}
% \input{descr/383_fwyr6LXFS1c}
\subsection{Rotated Negative Pedal Curve of Ellipse is Area-Invariant}
\label{vid:DgADxkqlKSw}
\noindent N=n/a, 12m46s (6/2020). 
\begin{center}\includegraphics[width=.5\textwidth]{pics/DgADxkqlKSw.jpg} \\ 
\href{https://youtu.be/DgADxkqlKSw}{\url{youtu.be/DgADxkqlKSw}}\end{center}
% We demonstrate further properties of Steiner's Hat Δ: this the negative pedal curve NPC [1] of the ellipse E with respect to a fixed point M on its boundary. We've seen before that the area A of Δ is invariant over M.

This time we consider a "rotated" NPC: the envelope of lines passing through a point P(t) on the boundary of E, rotated θ degrees about P(t) with respect to the perpendicular to [P(t)-M].

1) Over all M, these are also constant-area deltoids Δ*(θ) with three cusps. 

2) Δ*(0) is the "original" Stainer's Hat. Let A(θ) denote the area of Δ*(θ). It turns out A(θ)/A = cos^2(θ). Δ*(90)=M, a point, and A(90)=0.

3) The pre-images of the cusp of Δ*(θ) coincide with Pi, the pre-images of the cusps of Δ. Let Pi' denote the cusps of Δ. Recall PiPi' concurred at C2, the center of area of Δ.

4) Let Pi* denote the cusps of Δ*(θ). Lines PiPi* concur at C2*, Δ*(θ)'s center of area.

5) We saw before M,C2,Pi were concyclic. It turns out C2* lies on this same circle. As one varies θ from 0 to 90, C2* moves along a circular arc on this circle from C2 to M.

Note: like Δ=Δ*(90), Δ*(θ) is likely an affine image of the Steiner Hypocyclid.

[1] Mathworld, "Ellipse negative pedal curve", https://mathworld.wolfram.com/EllipseNegativePedalCurve.html 
[2] Mathworld, "Steiner Deltoid", https://mathworld.wolfram.com/SteinerDeltoid.html

\section{Subtris (5)}

\subsection{Triangulation-Independent Kimberling Centers of Mass (S. Tabachikov \& E. Tsukerman, 2015)}
\label{vid:aAqGkNsFKaM}
\noindent N=5, 10m37s (3/2021). 
\begin{center}\includegraphics[width=.5\textwidth]{pics/aAqGkNsFKaM.jpg} \\ 
\href{https://youtu.be/aAqGkNsFKaM}{\url{youtu.be/aAqGkNsFKaM}}\end{center}
% In [1],  a curious phenomenon, akin to "Archimedes' Principle", is described:

given an N-gon and some triangulation of its interior, the "circumcenter of mass" (X3*) is defined as the weighted average of circumcenters X3 of individual triangles in the triangulation, weighed by their areas.

It turns out X3* is independent of the particular triangulation picked. For an interactive visualization of this phenomenon, see the video by E. Tsukerman [2].

The video demonstrates a result in Section 4 of [1], namely: the X(k)-of-mass will be independent of triangulation if X(k) is on the Euler line of a triangle and is a fixed linear combination of X2 and X3. Amongst the first 1000 on ETC [3] the following triangle centers have this property:

2 (barycenter), 3 (circumcenter), 4 (orthocenter), 5 (nine-point center), 20 (de Longchamps), 140, 376, 381, 382, 546, 547, 548, 549, 550, 631, 632.

[1] S. Tabachnikov and E. Tsukerman, "Remarks on the Circumcenter of
Mass", Arnold Mathematical Journal volume 1, pp. 101–112, 2015.

[2] E. Tsukerman, "Circumcenter of Mass", YouTube, 2015. https://youtu.be/V8aZbMdxwIU

[3] C. Kimberling, "Encyclopedia of Triangle Centers", 2021. https://faculty.evansville.edu/ck6/encyclopedia/ETC.html
\subsection{Sub-Orthocenter Triangle I: Same Area as Reference}
\label{vid:SZxeu5YIWpQ}
\noindent N=3, 7m9s (3/2021). 
\begin{center}\includegraphics[width=.5\textwidth]{pics/SZxeu5YIWpQ.jpg} \\ 
\href{https://youtu.be/SZxeu5YIWpQ}{\url{youtu.be/SZxeu5YIWpQ}}\end{center}
% This is probably a classical result, but interesting nonetheless.

Take a triangle T=ABC and a point M on its plane. Define 3 subtriangles by joining M consecutive vertices:

Ta = MBC, Tb = MCA, Tc = MAB

Let T4={X4a, X4b, X4c} be the triangle of the orthocenters of Ta,Tb,Tc.

Claim: the area of T4 is equal to that of T irrespective of M.
\subsection{Sub-Orthocenter Triangle II: Invariant Perimeter over Circumcircle}
\label{vid:GXwuDV0fdoU}
\noindent N=3, 9m59s (3/2021). 
\begin{center}\includegraphics[width=.5\textwidth]{pics/GXwuDV0fdoU.jpg} \\ 
\href{https://youtu.be/GXwuDV0fdoU}{\url{youtu.be/GXwuDV0fdoU}}\end{center}
% Correction: the narration starts with "in the next video". I meant "in the previous video" [1]. 

Take a genetric triangle T=ABC, and a point M which defines the following 3 subtriangles Ti={MBC, MCA, MAB}. Let T' denote the triangle whose vertices are the orthocenters of the Ti.

We saw in a previous video [1] that Area(T')=Area(T) irrespective of M. Here we report on two other phenomena:

a) if M = orthocenter X4 of T, then T'=T (areas are automatically equal).
b) if M is anywhere on the circumcircle of T, then T' is a reflection of T, i.e., their perimeters are the same.

[1] D. Reznik, "Triangle of Sub-Orthocenters: Same Area as Reference!", YouTube, March, 2021. https://youtu.be/SZxeu5YIWpQ
\subsection{Dynamics of the ``Circumcenter Map'': Periodicity, Stability, Converging, and Diverging Zones}
\label{vid:y6F8SmA67pw}
\noindent N=3, 30m58s (3/2021). 
\begin{center}\includegraphics[width=.5\textwidth]{pics/y6F8SmA67pw.jpg} \\ 
\href{https://youtu.be/y6F8SmA67pw}{\url{youtu.be/y6F8SmA67pw}}\end{center}
% \input{descr/391_y6F8SmA67pw}
\subsection{GPU-based Visualization of Convergence and Divergence Zones of the Circumcenter Map}
\label{vid:PaOPmRraQxQ}
\noindent N=3, 1m27s (4/2021). 
\begin{center}\includegraphics[width=.5\textwidth]{pics/PaOPmRraQxQ.jpg} \\ 
\href{https://youtu.be/PaOPmRraQxQ}{\url{youtu.be/PaOPmRraQxQ}}\end{center}
% GPU-based interactive viz of regions of convergence and divergence of the circumcenter map [1]. 

In Mathematica computing these can take several minutes and/or hrs and even days.

In contrast, a modern GPU can compute dozens of frames per second, allowing for total interactivity: regions get distorted as one drags around vertices of the starting polygon (N=3,4,5...).

The regions can also be colored by area expansion or contraction and/or rotation after N applications of the map. Lines of neutral scaling or rotation can be visualized as well as their intersections.

GPU wizardry: Nicholas McDonald [2]

 [1] D. Reznik and R. Garcia, "Dynamics of the Circumcenter Map", Wolfram Community, April, 2021. https://community.wolfram.com/groups/-/m/t/2234577

[2] Nicholas' Blog, https://weigert.vsos.ethz.ch/about/

\section{Swans (3)}

\subsection{Motion of $X_{88}$ with respect to collinear $X_{100}$ and $X_{1}$}
\label{vid:DaoNJRcf-0E}
\noindent N=3, 4m49s (12/2019). 
\begin{center}\includegraphics[width=.5\textwidth]{pics/DaoNJRcf-0E.jpg} \\ 
\href{https://youtu.be/DaoNJRcf-0E}{\url{youtu.be/DaoNJRcf-0E}}\end{center}
% Shown are four elliptic billiards with a/b of 1.25, 1.395, 1.485, and 2. For each of these the family of N=3 (triangular) orbits is shown. For each billiard the orbit's incenter X1, the anticomplement of the Feuerbach X100, and X88 (the isogonal complement of X44 [1]) are shown. A few known properties of X88 for any triangle [1]:

- X88 and X100 lie on X9-centered circumconic (the fixed billiard in this case, as X9 is stationary for the 3-periodic family).
- X88 is collinear with X100 and X1.

New results [2]:

- At a/b less than 1.485 the movement of X88 is monotonically opposed to that of the orbit's vertices. At a/b greater than that threshold, the movement becomes non-monotonic. The velocity changes sign twice when X88 is in the vicinity of either billiard horizontal vertex.

Let T=ABC be the orbit triangle with sides a ≤  b ≤ c

- X88 will coincide with B iff b = (a+c)/2. This is in general a scalene triangle. In this case, X1 is the midpoint between X100 and X88. the letter ρ above each billiard denotes the |X1-X100|/|X1-X88| ratio. So when X88 is on an orbit vertex, ρ=1.
- The only right-triangle for which X88 coincides with B is a:b:c = 3:4:5
- A 3:4:5 triangle has an X9-centered circumconic with semiaxis ratio of 1.3924. Equivalently, this is the aspect ratio of the only billiard which admits a 3:4:5 triangle.

[1] Clark Kimberling, "Encyclopedia of Triangle Centers", https://faculty.evansville.edu/ck6/encyclopedia/ETC.html

[2] Dan Reznik, Ronaldo Garcia, and Jair Koiller, "Loci of Triangular Centers in an Elliptic Billiard", 2020. https://arxiv.org/abs/2001.08041
\subsection{Dance of the Swans: $X_{88}$ and $X_{162}$ (part II)}
\label{vid:uv_lZSmX9Pk}
\noindent N=3, 6m1s (3/2020). 
\begin{center}\includegraphics[width=.5\textwidth]{pics/uv_lZSmX9Pk.jpg} \\ 
\href{https://youtu.be/uv_lZSmX9Pk}{\url{youtu.be/uv\_lZSmX9Pk}}\end{center}
% An a/b=2.5 elliptic billiard is shown (black) as well as its family of 3-periodics (blue). As these rotate, triangle centers X(88) and X(162) execute a remarkable dance on the boundary of the Billiard. Tough both cover the entire Billiard, neither move monotonically, nor do they ever cross each other.

Note: a slightly improved version pf this video is available here: https://youtu.be/ljGTtA1x-Sk

Soundtrack: Pyotr Ilyich Tchaikovsky, "Swan Lake"

for oriented (but slower) swans see also: https://youtu.be/ljGTtA1x-Sk
\subsection{Dance of the Swans: $X_{88}$ and $X_{162}$ (part I)}
\label{vid:ljGTtA1x-Sk}
\noindent N=3, 8m1s (3/2020). 
\begin{center}\includegraphics[width=.5\textwidth]{pics/ljGTtA1x-Sk.jpg} \\ 
\href{https://youtu.be/ljGTtA1x-Sk}{\url{youtu.be/ljGTtA1x-Sk}}\end{center}
% The trees are in their autumn beauty,
The woodland paths are dry,
Under the October twilight the water
Mirrors a still sky;
Upon the brimming water among the stones
Are nine-and-fifty swans.
[...]
But now they drift on the still water,
Mysterious, beautiful;
Among what rushes will they build,
By what lake's edge or pool
Delight men's eyes when I awake some day
To find they have flown away? --W. B. Yeats

The video shows the family of Poncelet 3-periodics (blue) in an elliptic billiard (black). Triangle centers X(88) and X(162), imagined as swooning swans, execute a remarkable ``dance´´ on the boundary of an elliptic ``pond´´. Though both swans swim along the entire Billiard, neither moves monotonically, nor do they ever cross each other. For details see [1]. For a live animation see https://bit.ly/3f6M9Wh

[1] D. Reznik,  R. Garcia. J, Koiller, "The Ballet of Triangle Centers
on the Elliptic Billiard", Journal for Geometry and Graphics, Volume 24 (2020), No. 1, 79–101. ​https://www.heldermann-verlag.de/jgg/jgg24/j24h1rezn.pdf

Soundtrack: Pyotr Ilyich Tchaikovsky, "Swan Lake"

older (faster) video: https://youtu.be/uv_lZSmX9Pk

\section{Tangential Polygon (4)}

\subsection{Locus of Vertices of the Excentral Polygon}
\label{vid:kaYWlBTpUPw}
\noindent N=3 to 6, 2m25s (6/2019). 
\begin{center}\includegraphics[width=.5\textwidth]{pics/kaYWlBTpUPw.jpg} \\ 
\href{https://youtu.be/kaYWlBTpUPw}{\url{youtu.be/kaYWlBTpUPw}}\end{center}
% The famiy of N=3,4,5,6 orbits are shown in separate elliptic billiards (all with a/b=1.5). Also shown are their excentral polygons, whose vertices are intersections of tangent lines at consecutive orbit vertices. We noticed these trace out elliptic loci non-confocal with the billiard (green dots are foci of the loci). For N=3, the locus is an upright ellipse, being horizontal for all others. As N grows, the loci foci approach the billiard's.

https://dan-reznik.github.io/Elliptical-Billiards-Triangular-Orbits/
\subsection{5-periodics and feet of excenters}
\label{vid:PRkhrUNTXd8}
\noindent N=5, 1m13s (6/2019). 
\begin{center}\includegraphics[width=.5\textwidth]{pics/PRkhrUNTXd8.jpg} \\ 
\href{https://youtu.be/PRkhrUNTXd8}{\url{youtu.be/PRkhrUNTXd8}}\end{center}
% An Elliptic Billiard (a/b=1.5) is shown (black) as well as the family of non-intersecting pentagonal (N=5) orbits (blue). For each orbit the Tangential Polygon is shown (green). Its vertices intersections of tangents at consecutive orbit's vertices. The locus of these vertices is the dashed green a non-confocal ellipse (its foci are the green dots on the x axis). 

The feet of perpendiculars dropped from each tangential vertex to the corresponding side is congruent with the point of tangency of that side with the confocal caustic, i.e., the locus of these feet is the caustic.

Also drawn is the (red) polygon connecting the feet of the aforementioned perpendiculars. This polygon is a projective dual to the original orbit.

https://dan-reznik.github.io/Elliptical-Billiards-Triangular-Orbits/
\subsection{Locus of meetpoints of Excentral-to-Orbit Perpendiculars I}
\label{vid:JTasf8JKoH0}
\noindent N=3,4,5, 1m12s (6/2019). 
\begin{center}\includegraphics[width=.5\textwidth]{pics/JTasf8JKoH0.jpg} \\ 
\href{https://youtu.be/JTasf8JKoH0}{\url{youtu.be/JTasf8JKoH0}}\end{center}
% An elliptic billiard (a/b=1.5) is shown (black). Also shown (blue) is the family of N=4 non-intersecting (quadrangular) orbits. These orbits are known to have (i) constant perimeter, (ii) be parallelograms, and (iii) have vertices of the tangential polygon (green) which move along a circle (dashed green). The confocal caustic to the orbits is shown as the green ellipse.

The feet of perpendiculars dropped from each vertex of the tangential polygon to the corresponding side on the orbit coincide with the points of orbit-caustic tangency (green points). Interconnecting these produces another parallelogram (red).

For each vertex of the tangential polygon, draw a segment passing through the aforementioned foot. These meet at four points (shown red), which connected also form a parallelogram (shown red), and whose locus is the 4-petal upright rose.

https://dan-reznik.github.io/Elliptical-Billiards-Triangular-Orbits/
\subsection{Locus of meetpoints of Excentral-to-Orbit Perpendiculars II}
\label{vid:ugRFxo0l2OI}
\noindent N=3,4,5, 1m12s (6/2019). 
\begin{center}\includegraphics[width=.5\textwidth]{pics/ugRFxo0l2OI.jpg} \\ 
\href{https://youtu.be/ugRFxo0l2OI}{\url{youtu.be/ugRFxo0l2OI}}\end{center}
% The famly of non-intersecting N=5 (pentagonal) orbits (blue) for an a/b=1.5 elliptical billiard (black) is shown. For each orbit its excentral polygon (green) is drawn, as well as the elliptic locus of its vertices (dashed green). The latter's foci are the green dots on the x axis. Also dhown is the confocal caustic (green ellipse) to the orbits. Consider dropping a perpendicular from each excentral vertex to its corresponding orbit side. Each will be congruent with the touchpoints of that side with the caustic, and their locus will be caustic. The red points are intersections of said perpendiculars, taken two consecutive ones at a time. For N=5 they form a 5-gon which alternates between convex and concave. Any of its vertices will trace out the 4-petal flower drawn.

https://dan-reznik.github.io/Elliptical-Billiards-Triangular-Orbits/

\section{Tris Enveloping Ellipses (4)}

\subsection{Equilaterals Enveloping an Ellipse I -- Loci of Vertices and Incenter}
\label{vid:LBs3VxbMxPc}
\noindent N=3, 2m26s (7/2021). 
\begin{center}\includegraphics[width=.5\textwidth]{pics/LBs3VxbMxPc.jpg} \\ 
\href{https://youtu.be/LBs3VxbMxPc}{\url{youtu.be/LBs3VxbMxPc}}\end{center}
% Left (resp. right): family right isosceles triangles P1P2P3 is shown circumscribing an a/b=2 (resp. 3) ellipse. P1 rides on Monge's orthoptic circle (the 90-degree isoptic) [1], whose radius is Sqrt[a^2+b^2]. Also shown is the locus of P2 and P3 (dashed blue) and that of the incenter (X1).

[1] https://en.wikipedia.org/wiki/Orthoptic_(geometry)
\subsection{90-deg Isosceles Enveloping an Ellipse -- Non-Conic Loci of Vertices and Centroid}
\label{vid:-WaEYct_x7U}
\noindent N=3, 2m26s (7/2021). 
\begin{center}\includegraphics[width=.5\textwidth]{pics/-WaEYct_x7U.jpg} \\ 
\href{https://youtu.be/-WaEYct_x7U}{\url{youtu.be/-WaEYct\_x7U}}\end{center}
% The video shows a family of equilateral triangles P1P2P3 whose sides are dynamically tangent to an ellipse with a/b=2.5 (left) and a/b=3.5 (right). Also shown is the locus of the vertices (the outer component of a quartic -- the 60-degree isoptic curve, see [1]) and the non-conic locus of the centroid G. Anyone care to derive what this curve is?

[1] https://math.stackexchange.com/questions/2990274/find-the-locus-of-the-vertices-of-equilateral-triangle-circumscribing-the-ellips
\subsection{Equilaterals Enveloping an Ellipse II -- Non-Conic Centroid Locus \& Astroidal Envelope of Bisectors}
\label{vid:sZka-yj8IR4}
\noindent N=3, 2m26s (7/2021). 
\begin{center}\includegraphics[width=.5\textwidth]{pics/sZka-yj8IR4.jpg} \\ 
\href{https://youtu.be/sZka-yj8IR4}{\url{youtu.be/sZka-yj8IR4}}\end{center}
% \input{descr/405_sZka-yj8IR4}
\subsection{Family of 90-60-30 Right Triangles Enveloping an Ellipse -- Loci of Vertices and Centers $X_{1}$-$X_{20}$}
\label{vid:7cLPQkVvVQM}
\noindent N=3, 2m26s (7/2021). 
\begin{center}\includegraphics[width=.5\textwidth]{pics/7cLPQkVvVQM.jpg} \\ 
\href{https://youtu.be/7cLPQkVvVQM}{\url{youtu.be/7cLPQkVvVQM}}\end{center}
% The 1d family of 90-60-30 right triangles (blue) enveloping and a/b=3.5 ellipse (black) is shown. The locus of each of the vertices is shown blue and corresponds to the 90-, 60-, 30-isoptics of the ellipse [1], i.e., the outer component of certain quartics. Recall the 90-degree isoptic, i.e., the ellipse "orthoptic" is a circle known as Monge's circle. Also shown are the loci of triangle centers in sequence, from X(1) to X(20).

[1] https://en.wikipedia.org/wiki/Orthoptic_(geometry)

\section{Unrolled 3-Periodics (4)}

\subsection{Fixed central billiard}
\label{vid:v7CDrOTFDzo}
\noindent N=3, 3m13s (1/2020). 
\begin{center}\includegraphics[width=.5\textwidth]{pics/v7CDrOTFDzo.jpg} \\ 
\href{https://youtu.be/v7CDrOTFDzo}{\url{youtu.be/v7CDrOTFDzo}}\end{center}
% \input{descr/407_v7CDrOTFDzo}
\subsection{Pin P1 and n1}
\label{vid:20fx69L_gnU}
\noindent N=3, 4m1s (1/2020). 
\begin{center}\includegraphics[width=.5\textwidth]{pics/20fx69L_gnU.jpg} \\ 
\href{https://youtu.be/20fx69L_gnU}{\url{youtu.be/20fx69L\_gnU}}\end{center}
% An a/b=1.5 elliptic billiard is shown as well as its family of 3-periodic trajectories. The billiard is reflected at every bounce point so that the entire trajectory is shown as a (constant lenght L) straight line P1 to P1''. Here the starting vertex P1 is pinned to the origin as is the vector normal at that point, which is "pinned" at the vertical direction. This makes the locus of the endpoint P1'' be an arc of a circle. Loci are drawn for a few intermediate points.
\subsection{Pin P1 at origin}
\label{vid:cPDPb7RmXR4}
\noindent N=3, 4m1s (1/2020). 
\begin{center}\includegraphics[width=.5\textwidth]{pics/cPDPb7RmXR4.jpg} \\ 
\href{https://youtu.be/cPDPb7RmXR4}{\url{youtu.be/cPDPb7RmXR4}}\end{center}
% An a/b=1.5 elliptic billiard is shown as well as its family of 3-periodic trajectories. The billiard is reflected at every bounce point so that the entire trajectory is shown as a (constant lenght L) straight line P1 to P1''. Here the starting vertex P1 is pinned to the origin. Since L is constant, the locus of the endpoint P1'' is a circle. Loci are drawn for intermediate trajectory vertices.
\subsection{Pin P1 at origin and P1'' vertically above it}
\label{vid:uh45MBlOORE}
\noindent N=3, 4m1s (1/2020). 
\begin{center}\includegraphics[width=.5\textwidth]{pics/uh45MBlOORE.jpg} \\ 
\href{https://youtu.be/uh45MBlOORE}{\url{youtu.be/uh45MBlOORE}}\end{center}
% An a/b=1.5 elliptic billiard is shown as well as its family of 3-periodic trajectories. The billiard is reflected at every bounce point so that the entire trajectory is shown as a (constant lenght L) straight line P1 to P1''. Here the starting vertex P1 is pinned to the origin and P1'' vertically above it. Since L is constant, P1'' is fixed. Loci are drawn for intermediate trajectory vertices.

Curious property: despite all asymmetries, the locus of the circumcenter of the P3P1'P2'' triangle (filled in light yellow) is a horizontal segment (purple).

Proof by Mark Helman, Jan 4, 2020: let A, B, C, D are arbitrary points on a line. Let FE,EG pass through B,C, respectively. Circumcenter of EFG will fall on the bisector of segment AD.


\bibliographystyle{alpha}
\bibliography{refs,refs_pub,refs_acc,refs_sub}

\end{document}