\documentclass[12pt]{article}

\title{Videos of Experiments (2019--2021)}

\author{Dan S. Reznik}

\usepackage{enumitem}
\usepackage{url}
\usepackage[breaklinks=true]{hyperref}
%\usepackage[dvipsnames]{xcolor}
\hypersetup{
    pdftoolbar=true,        % show Acrobat’s toolbar?
    pdfmenubar=true,        % show Acrobat’s menu?
    pdffitwindow=false,     % window fit to page when
    pdfstartview={FitH},    % fits the width of 
    colorlinks=true,       % false: boxed links
    linkcolor=red, 
    citecolor=blue,   
    filecolor=black, 
    urlcolor=red
}

\begin{document}

\maketitle

\tableofcontents\section{Affine Images (4)}

\begin{enumerate}[resume]
\item \textit{Affine images of billiard N-periodics Ia: N=3 quartet}, 2021. youtu.be/\href{https://youtu.be/EJzqsELkPN4}{\nolinkurl{EJzqsELkPN4}}
\item \textit{Affine images of billiard N-periodics Ib: N=3 trio}, 2021. youtu.be/\href{https://youtu.be/HjBZdrR3Azs}{\nolinkurl{HjBZdrR3Azs}}
\item \textit{Affine images of billiard N-periodics IIa: N=5 quartet \& trio}, 2021. youtu.be/\href{https://youtu.be/VQ4fB_s33HE}{\nolinkurl{VQ4fB\_s33HE}}
\item \textit{Affine images of billiard N-periodics IIb: N=5 role reversal}, 2021. youtu.be/\href{https://youtu.be/7KtTtXXJlEI}{\nolinkurl{7KtTtXXJlEI}}
\end{enumerate}

\section{Anticevian (2)}

\begin{enumerate}[resume]
\item \textit{The Anticevian Polygon I: Basic Phenomena}, 2021. youtu.be/\href{https://youtu.be/FKvDfamTy-Y}{\nolinkurl{FKvDfamTy-Y}}
\item \textit{The Anticevian Polygon II: Cyclical Projectivities}, 2021. youtu.be/\href{https://youtu.be/BLHBlOWrtjs}{\nolinkurl{BLHBlOWrtjs}}
\end{enumerate}

\section{Area Invariants (3)}

\begin{enumerate}[resume]
\item \textit{Amazing Ellipse Pedal and Contrapedal Curves: area invariance for all pedal points on a circle!}, 2020. youtu.be/\href{https://youtu.be/UUnvj7VIYso}{\nolinkurl{UUnvj7VIYso}}
\item \textit{Regular Polygons: the Signed Area of the Antipedal Polygon Vanishes along a Circle?}, 2020. youtu.be/\href{https://youtu.be/9PZ6_bHz2UE}{\nolinkurl{9PZ6\_bHz2UE}}
\item \textit{Steiner's Krümmungs-Schwerpunkt implies Area-Invariant Interpolated Pedal Curve over Circles}, 2020. youtu.be/\href{https://youtu.be/gR8Axe823_M}{\nolinkurl{gR8Axe823\_M}}
\end{enumerate}

\section{Bicentric Family (9)}

\begin{enumerate}[resume]
\item \textit{Bicentric Family I: Properties of Pedals, Polars, \& Inversions of N-Periodics in Elliptic Billiard}, 2021. youtu.be/\href{https://youtu.be/jhXDKRFLpVk}{\nolinkurl{jhXDKRFLpVk}}
\item \textit{Bicentric Family II: invariant perimeter limiting point pedals}, 2021. youtu.be/\href{https://youtu.be/A7F3szW7rUE}{\nolinkurl{A7F3szW7rUE}}
\item \textit{Bicentric Family III: Equiperimeter "Limiting" Pedal Polygons to the Bicentric Family}, 2021. youtu.be/\href{https://youtu.be/6TmaezNFrOs}{\nolinkurl{6TmaezNFrOs}}
\item \textit{Bicentric Family IV: Constant-Perimeter Limiting Point Pedals in the N=4 Case}, 2021. youtu.be/\href{https://youtu.be/fZe6elRTfeA}{\nolinkurl{fZe6elRTfeA}}
\item \textit{Zero-Area N=4 Bicentric Pedals}, 2021. youtu.be/\href{https://youtu.be/hwx1i-W6yLQ}{\nolinkurl{hwx1i-W6yLQ}}
\item \textit{The Jacobi-Poncelet Bicentric family: 3 derived constant perimeter families}, 2021. youtu.be/\href{https://youtu.be/8m21fCz8eX4}{\nolinkurl{8m21fCz8eX4}}
\item \textit{Non-Concentric Circular Poncelet Pair: Invariant Sum of Japanese Theorem Inradii (A. Akopyan)}, 2020. youtu.be/\href{https://youtu.be/BEvdUUolUXI}{\nolinkurl{BEvdUUolUXI}}
\item \textit{Peripheral triangles: circular locus of incenters and invariant sum of inradii}, 2021. youtu.be/\href{https://youtu.be/HnqqaqDf2mo}{\nolinkurl{HnqqaqDf2mo}}
\item \textit{Peripheral triangles: Constant Inradius Sum, Japanese-Style Triangles, Incenter Circular Loci}, 2021. youtu.be/\href{https://youtu.be/TGwlfBUtKrs}{\nolinkurl{TGwlfBUtKrs}}
\end{enumerate}

\section{Brocard (12)}

\begin{enumerate}[resume]
\item \textit{Poncelet 3-Periodics of Homothetic Pair: Elliptic Loci of Brocard Pts + Vertices of 1st Brocard Tri}, 2020. youtu.be/\href{https://youtu.be/13i3JGY-fK4}{\nolinkurl{13i3JGY-fK4}}
\item \textit{It takes 2 to tango: Brocard-Poncelet Porism, stationary Brocard Points and invariant Brocard Angle}, 2020. youtu.be/\href{https://youtu.be/JANPPLET0so}{\nolinkurl{JANPPLET0so}}
\item \textit{Joined at the hip: Brocard Porism, Steiner Ellipses, and the Homothetic Poncelet Pair}, 2020. youtu.be/\href{https://youtu.be/h3GZz7pcJp0}{\nolinkurl{h3GZz7pcJp0}}
\item \textit{The Poncelet Homothetic Pair contains an Aspect-Ratio Invariant Brocard Inellipse}, 2020. youtu.be/\href{https://youtu.be/DIm2qTxGWXE}{\nolinkurl{DIm2qTxGWXE}}
\item \textit{Brocard Porism: Locus of 1st, 2nd, 5th, and 7th Brocard Triangles' Vertices are Circles}, 2020. youtu.be/\href{https://youtu.be/_bK-BCQv24A}{\nolinkurl{\_bK-BCQv24A}}
\item \textit{Russian-Doll nesting of Brocard porisms courtesy of the second Brocard triangle}, 2020. youtu.be/\href{https://youtu.be/T7c4CDHIk7s}{\nolinkurl{T7c4CDHIk7s}}
\item \textit{Rusian-doll nesting of Brocard porisms: concyclic sequence of Brocard points and the Beltrami points}, 2020. youtu.be/\href{https://youtu.be/Z3YlEbCFbnA}{\nolinkurl{Z3YlEbCFbnA}}
\item \textit{Brocard Porism: equilateral Isodynamic Pedals have invariant area ratio + circular centroidal locus}, 2020. youtu.be/\href{https://youtu.be/s4DF-iZZO8Y}{\nolinkurl{s4DF-iZZO8Y}}
\item \textit{Continuous Family of Brocard Porisms with Stationary Isodynamic Points X15 and X16}, 2020. youtu.be/\href{https://youtu.be/jY_8zxBljuk}{\nolinkurl{jY\_8zxBljuk}}
\item \textit{The Family of Second Brocard Triangles in the Brocard Porism}, 2020. youtu.be/\href{https://youtu.be/Wgwh4-neJp4}{\nolinkurl{Wgwh4-neJp4}}
\item \textit{Brocard Porism: Family of Second Brocard Triangles is a second Brocard Porism}, 2020. youtu.be/\href{https://youtu.be/MprJtB4UW9s}{\nolinkurl{MprJtB4UW9s}}
\item \textit{Invariants of the Generalized Brocard Porism}, 2021. youtu.be/\href{https://youtu.be/UYI_lBubKXA}{\nolinkurl{UYI\_lBubKXA}}
\end{enumerate}

\section{Cayley-Poncelet (3)}

\begin{enumerate}[resume]
\item \textit{Cayley-Poncelet Phenomena I: Finding an Ellipse Pair in General Position which admits 3-Periodics}, 2021. youtu.be/\href{https://youtu.be/virCpDtEvJU}{\nolinkurl{virCpDtEvJU}}
\item \textit{Cayley-Poncelet Phenomena II: Invariant Power of Center wrt Circumrcircle and Euler's Circle}, 2021. youtu.be/\href{https://youtu.be/4xsm_hQU-dE}{\nolinkurl{4xsm\_hQU-dE}}
\item \textit{Area Invariants of Poncelet N-Periodics and their Polar Polygons}, 2021. youtu.be/\href{https://youtu.be/2TgmJ-YHydQ}{\nolinkurl{2TgmJ-YHydQ}}
\end{enumerate}

\section{Convex Combinations (3)}

\begin{enumerate}[resume]
\item \textit{Barycenter with Median, and Incenter with Intouchpoint}, 2019. youtu.be/\href{https://youtu.be/3Gr3Nh5-jHs}{\nolinkurl{3Gr3Nh5-jHs}}
\item \textit{Orthocenter with one altitude foot, and Circumcenter with median}, 2019. youtu.be/\href{https://youtu.be/HZFjkWD_CnE}{\nolinkurl{HZFjkWD\_CnE}}
\item \textit{Excenter and its corresponding Extouch point}, 2019. youtu.be/\href{https://youtu.be/OD8Ah0hf8yQ}{\nolinkurl{OD8Ah0hf8yQ}}
\end{enumerate}

\section{Cross-Ratio Invariants (6)}

\begin{enumerate}[resume]
\item \textit{Poncelet Cross-Ratio Invariants I: exploring N=5,6,7,8}, 2021. youtu.be/\href{https://youtu.be/vgHoLM5pg6o}{\nolinkurl{vgHoLM5pg6o}}
\item \textit{Poncelet Cross-Ratio Invariants II: 3 diagonals in the N=5 case}, 2021. youtu.be/\href{https://youtu.be/XqpYVrXbK5s}{\nolinkurl{XqpYVrXbK5s}}
\item \textit{Poncelet Cross-Ratio Invariants III: 3 diagonals in the N=5 self-intersected case}, 2021. youtu.be/\href{https://youtu.be/4bd0YhQZMPM}{\nolinkurl{4bd0YhQZMPM}}
\item \textit{Poncelet Cross-Ratio Invariants IV: N=6 and N=8 and unit cross-ratio products}, 2021. youtu.be/\href{https://youtu.be/i6krQu5Ls1E}{\nolinkurl{i6krQu5Ls1E}}
\item \textit{Poncelet Cross-Ratio Invariants V: Bicentric Pentagons, Pentagrams and their Projective Images}, 2021. youtu.be/\href{https://youtu.be/b-WIfLej_yY}{\nolinkurl{b-WIfLej\_yY}}
\item \textit{Unit Product of Cross-Ratios in a Hexagon Independent of Vertices}, 2021. youtu.be/\href{https://youtu.be/heiqfRTQ2Mc}{\nolinkurl{heiqfRTQ2Mc}}
\end{enumerate}

\section{Early Results (8)}

\begin{enumerate}[resume]
\item \textit{Elliptic Loci of X(1) to X(5) and Euler Line}, 2019. youtu.be/\href{https://youtu.be/sMcNzcYaqtg}{\nolinkurl{sMcNzcYaqtg}}
\item \textit{Loci of Vertices of Medial, Intouch and Feuerbach Triangles is not elliptic}, 2019. youtu.be/\href{https://youtu.be/OGvCQbYqJyI}{\nolinkurl{OGvCQbYqJyI}}
\item \textit{Mittenpunkt is stationary at center of billiard}, 2019. youtu.be/\href{https://youtu.be/tMrBqfRBYik}{\nolinkurl{tMrBqfRBYik}}
\item \textit{Feuerbach Point Sweeps Billiard and its Anti-Complement and Extouch Points sweep caustic}, 2019. youtu.be/\href{https://youtu.be/TXdg7tUl8lc}{\nolinkurl{TXdg7tUl8lc}}
\item \textit{Conservation of Sum and Product of Cosines}, 2019. youtu.be/\href{https://youtu.be/P8ykpE_ZbZ8}{\nolinkurl{P8ykpE\_ZbZ8}}
\item \textit{Stationary Circle for N=5}, 2019. youtu.be/\href{https://youtu.be/dINE4aH1cvk}{\nolinkurl{dINE4aH1cvk}}
\item \textit{Stationary Circles}, 2019. youtu.be/\href{https://youtu.be/EFeINGIDFrg}{\nolinkurl{EFeINGIDFrg}}
\item \textit{Generalization of the Stationary Mittenpunkt and Caustic-Sweeping Extouchpoints}, 2019. youtu.be/\href{https://youtu.be/Bpc-MrR2IMc}{\nolinkurl{Bpc-MrR2IMc}}
\end{enumerate}

\section{Ellipse-Inscribed Triangles (8)}

\begin{enumerate}[resume]
\item \textit{Loci of Ellipse-Inscribed Triangles I: Basic Phenomena}, 2020. youtu.be/\href{https://youtu.be/zjiNgfndBWg}{\nolinkurl{zjiNgfndBWg}}
\item \textit{Loci of Ellipse-Inscribed Triangles II: X(ρ) slides merrily along the Euler line}, 2020. youtu.be/\href{https://youtu.be/w5KuN_0rQBQ}{\nolinkurl{w5KuN\_0rQBQ}}
\item \textit{Loci of Ellipse-Inscribed Triangles III: family of V1V2 parallels causes rigid locus translation}, 2020. youtu.be/\href{https://youtu.be/zFOeENDJRho}{\nolinkurl{zFOeENDJRho}}
\item \textit{Loci of Ellipse-Inscribed Triangles IV: Multiple Loci Over Parallel V1V2}, 2020. youtu.be/\href{https://youtu.be/TpBjKlkFjkg}{\nolinkurl{TpBjKlkFjkg}}
\item \textit{Loci of Ellipse-Inscribed Triangles V: Circular Loci if V1V2 Horizontal or Vertical for Certain ρ}, 2020. youtu.be/\href{https://youtu.be/nLeKvxcicNY}{\nolinkurl{nLeKvxcicNY}}
\item \textit{Ellipse-Inscribed Triangles VI: Envelope of X4 Loci is Area-Invariant and Cousin of Pascal's Limaçon}, 2020. youtu.be/\href{https://youtu.be/sPQrz7ddRfA}{\nolinkurl{sPQrz7ddRfA}}
\item \textit{Ellipse-Mounted Triangles: Elliptic locus of the Orthocenter X(4) and suprising area invariance!}, 2020. youtu.be/\href{https://youtu.be/Fo-tNRcA-CQ}{\nolinkurl{Fo-tNRcA-CQ}}
\item \textit{Circle-Mounted Triangles: Surprising Loci of the Brocard Points}, 2020. youtu.be/\href{https://youtu.be/Ms8jC9yOKU4}{\nolinkurl{Ms8jC9yOKU4}}
\end{enumerate}

\section{Ellipse Echoes (1)}

\begin{enumerate}[resume]
\item \textit{Ellipse Echoes I: Circular wavefronts released into circular and elliptic cavities}, 2021. youtu.be/\href{https://youtu.be/LQnNLMhH9EE}{\nolinkurl{LQnNLMhH9EE}}
\end{enumerate}

\section{Envelopes (9)}

\begin{enumerate}[resume]
\item \textit{Envelope of Antiorthic and Gergonne Lines}, 2020. youtu.be/\href{https://youtu.be/Q7l6_Z4IyEI}{\nolinkurl{Q7l6\_Z4IyEI}}
\item \textit{Evolute of Elliptic Billiard and Envelope of X(1)-X(5)}, 2020. youtu.be/\href{https://youtu.be/eBStp-7X5yE}{\nolinkurl{eBStp-7X5yE}}
\item \textit{Envelope of 3-Periodic Vertex with Triangle Center}, 2020. youtu.be/\href{https://youtu.be/bRY61RdxCkM}{\nolinkurl{bRY61RdxCkM}}
\item \textit{Evolute Triangles of P1(t) with X(i)}, 2020. youtu.be/\href{https://youtu.be/DhqDdMAlBZM}{\nolinkurl{DhqDdMAlBZM}}
\item \textit{Elliptic Envelope of P1(t) with P1(t+pi/2)}, 2020. youtu.be/\href{https://youtu.be/8a4JoddyEyc}{\nolinkurl{8a4JoddyEyc}}
\item \textit{Envelope of 3-Periodic P1 and reflected P2 is Elliptic}, 2020. youtu.be/\href{https://youtu.be/GJgiUulX1aU}{\nolinkurl{GJgiUulX1aU}}
\item \textit{The Bat-Envelope of X(48) and X(37143)}, 2020. youtu.be/\href{https://youtu.be/Kr93eFZnB_U}{\nolinkurl{Kr93eFZnB\_U}}
\item \textit{Envelopes of Sides of Derived Triangles}, 2020. youtu.be/\href{https://youtu.be/SJrgWtdX8xU}{\nolinkurl{SJrgWtdX8xU}}
\item \textit{Envelope of Simson Lines from X100 and X99 to two N=3 Poncelet Families}, 2020. youtu.be/\href{https://youtu.be/79veSHrElb4}{\nolinkurl{79veSHrElb4}}
\end{enumerate}

\section{Frégier (3)}

\begin{enumerate}[resume]
\item \textit{Frégier Phenomena I: Area-Invariant Envelope of Chords}, 2021. youtu.be/\href{https://youtu.be/UCCG5AT8dh8}{\nolinkurl{UCCG5AT8dh8}}
\item \textit{Frégier Phenomena II: Envelope of Chords over M}, 2021. youtu.be/\href{https://youtu.be/mJ6opTPFAO4}{\nolinkurl{mJ6opTPFAO4}}
\item \textit{Frégier Phenomena III: Circular Envelopes and a New Invariant for a Poncelet Family}, 2021. youtu.be/\href{https://youtu.be/AzNXeBU2NTI}{\nolinkurl{AzNXeBU2NTI}}
\end{enumerate}

\section{Harmonic Polygons (5)}

\begin{enumerate}[resume]
\item \textit{Inversive-Area Iso-Contours of Harmonic Polygons (obtained as Polar Images of the Homothetic family)}, 2021. youtu.be/\href{https://youtu.be/erV4NILgWrE}{\nolinkurl{erV4NILgWrE}}
\item \textit{Inversive Image of Pivoting Harmonic Family Part I: Stationary Inversive Symmedian/Lemoine Point}, 2021. youtu.be/\href{https://youtu.be/jn3GD9SEkdU}{\nolinkurl{jn3GD9SEkdU}}
\item \textit{Inversive Image of Pivoting Harmonic Family Part II: Rigid Rotations about Harmonic Limiting Points}, 2021. youtu.be/\href{https://youtu.be/9GwfELl-tHk}{\nolinkurl{9GwfELl-tHk}}
\item \textit{Harmonic Polygons, Part III: Isocurves of Inversive Brocard Angle are Circles of the Schoute Pencil}, 2021. youtu.be/\href{https://youtu.be/VFqSvpuU0wg}{\nolinkurl{VFqSvpuU0wg}}
\item \textit{Harmonic Polygons, Part IV: Four Harmonic Inversive Images of Steiner's Porism}, 2021. youtu.be/\href{https://youtu.be/PGaUZHOvQIg}{\nolinkurl{PGaUZHOvQIg}}
\end{enumerate}

\section{Homothetic Pair (5)}

\begin{enumerate}[resume]
\item \textit{Homothetic Poncelet Pair: Invariant-Area Evolute Polygons}, 2020. youtu.be/\href{https://youtu.be/JCj0q7_hlA8}{\nolinkurl{JCj0q7\_hlA8}}
\item \textit{Homothetic Poncelet Pair: Zero-Area Evolute Polygons}, 2020. youtu.be/\href{https://youtu.be/3nvXYFoI5Wg}{\nolinkurl{3nvXYFoI5Wg}}
\item \textit{Homothetic Poncelet Pair: Invariant-Area N=5 Evolute Polygons and the Ellipse Evolute}, 2020. youtu.be/\href{https://youtu.be/ChsfLzKrb4o}{\nolinkurl{ChsfLzKrb4o}}
\item \textit{Zero-Area N=3 Homothetic Evolute Polygons}, 2020. youtu.be/\href{https://youtu.be/f80QaYs5_J4}{\nolinkurl{f80QaYs5\_J4}}
\item \textit{Zero area N=3 evolute polygons are segments which intersect at X(76)}, 2021. youtu.be/\href{https://youtu.be/OFA_j25R8ks}{\nolinkurl{OFA\_j25R8ks}}
\end{enumerate}

\section{Hyperbolic Billiard (2)}

\begin{enumerate}[resume]
\item \textit{3-Periodics in a Hyperbolic Billiard}, 2020. youtu.be/\href{https://youtu.be/kaUdX0eGDec}{\nolinkurl{kaUdX0eGDec}}
\item \textit{A triangle, its elliptic billiard, and three associated hyperbolic billiards}, 2020. youtu.be/\href{https://youtu.be/mkdv4401-zY}{\nolinkurl{mkdv4401-zY}}
\end{enumerate}

\section{IMPA 33o CBM (4)}

\begin{enumerate}[resume]
\item \textit{Fenômenos Euclidianos em Famílias Ponceletianas, Aula 01 -- Apresentação}, 2021. youtu.be/\href{https://youtu.be/84sG5siQWwg}{\nolinkurl{84sG5siQWwg}}
\item \textit{Fenômenos Euclidianos em Famílias Ponceletianas, Aula 03 -- Revisão Geometria dos Triângulos}, 2021. youtu.be/\href{https://youtu.be/AKmvd1uCZ9o}{\nolinkurl{AKmvd1uCZ9o}}
\item \textit{Fenômenos Euclidianos em Famílias Ponceletianas, Aula 04 -- Invariantes Bilhar Elíptico N=3}, 2021. youtu.be/\href{https://youtu.be/ti1rkBK62ck}{\nolinkurl{ti1rkBK62ck}}
\item \textit{Fenômenos Euclidianos em Famílias Ponceletianas, Aula 08 -- Famílias Concêntricas}, 2021. youtu.be/\href{https://youtu.be/fMC4nmQ6OQE}{\nolinkurl{fMC4nmQ6OQE}}
\end{enumerate}

\section{In- and Circum-parabolas (10)}

\begin{enumerate}[resume]
\item \textit{Surprising Loci of Inparabolas over Poncelet triangles inscribed in a circle: Part I}, 2021. youtu.be/\href{https://youtu.be/HTe3Wlqctq4}{\nolinkurl{HTe3Wlqctq4}}
\item \textit{Surprising Loci of Inparabolas over Poncelet triangles inscribed in a circle: Part II}, 2021. youtu.be/\href{https://youtu.be/qicI7zl7ICM}{\nolinkurl{qicI7zl7ICM}}
\item \textit{Family of tangential triangles to N=3 bicentrics: locus of vertices and of X4}, 2021. youtu.be/\href{https://youtu.be/9thwJcfUBmM}{\nolinkurl{9thwJcfUBmM}}
\item \textit{The amazing family of polar triangles derived from a parabola-inscribed Poncelet family}, 2021. youtu.be/\href{https://youtu.be/L2UpEHFQ6CY}{\nolinkurl{L2UpEHFQ6CY}}
\item \textit{Poncelet "homothetic" triangle family and its family of circumparabolas}, 2021. youtu.be/\href{https://youtu.be/DKd7kjnVVTc}{\nolinkurl{DKd7kjnVVTc}}
\item \textit{Focus Locus Hocus Pocus: Circumparabolas of the Bicentric Family}, 2021. youtu.be/\href{https://youtu.be/_7gv3Pqed6M}{\nolinkurl{\_7gv3Pqed6M}}
\item \textit{More Circumparabola Loci over Poncelet Triangle: the case of the Polar triangle}, 2021. youtu.be/\href{https://youtu.be/tTTIs_zxU3U}{\nolinkurl{tTTIs\_zxU3U}}
\item \textit{Loci of circumparabolas of an equilateral triangle and associated family of polar triangles}, 2021. youtu.be/\href{https://youtu.be/51jSJmaZ8Vk}{\nolinkurl{51jSJmaZ8Vk}}
\item \textit{Loci of a family of parabola-inscribed equilaterals and their polar triangles}, 2021. youtu.be/\href{https://youtu.be/5nnPWQGp1tE}{\nolinkurl{5nnPWQGp1tE}}
\item \textit{General Circle-Inscribed Poncelet Triangles and Loci of the Inparabola Vertex}, 2021. youtu.be/\href{https://youtu.be/Y_Mh4zPtwWM}{\nolinkurl{Y\_Mh4zPtwWM}}
\end{enumerate}

\section{Inconics \& Circumconics (17)}

\begin{enumerate}[resume]
\item \textit{The X(1)- and X(2)-centered circumellipses}, 2019. youtu.be/\href{https://youtu.be/AQ2AITmMs-g}{\nolinkurl{AQ2AITmMs-g}}
\item \textit{Locus of Intersection of X(1)- and X(2)-centered circumellipses}, 2019. youtu.be/\href{https://youtu.be/PTkpvvsjqNc}{\nolinkurl{PTkpvvsjqNc}}
\item \textit{The X(100)-centered Excentral Jerabek Hyperbola}, 2019. youtu.be/\href{https://youtu.be/uS0V1YjmEyY}{\nolinkurl{uS0V1YjmEyY}}
\item \textit{The Feuerbach and Excentral Hyperbolas}, 2019. youtu.be/\href{https://youtu.be/T5vXNsRcHZg}{\nolinkurl{T5vXNsRcHZg}}
\item \textit{The Jerabek Hyperbola and Circumbilliard of the Excentral Triangle}, 2019. youtu.be/\href{https://youtu.be/7Q1TCbW2jFM}{\nolinkurl{7Q1TCbW2jFM}}
\item \textit{Peter Moses' Points on the X(9)-centered circumellipse}, 2019. youtu.be/\href{https://youtu.be/JdcJt5PExsw}{\nolinkurl{JdcJt5PExsw}}
\item \textit{Invariants of the X(1)-centered circumellipse}, 2019. youtu.be/\href{https://youtu.be/82gYh_3hIe4}{\nolinkurl{82gYh\_3hIe4}}
\item \textit{Invariants of the Steiner Circum and Inconics}, 2019. youtu.be/\href{https://youtu.be/YQpX1eZ6O0I}{\nolinkurl{YQpX1eZ6O0I}}
\item \textit{The Yff Parabola, Contact Triangle \& Loci of Vertex and Axis Foot}, 2020. youtu.be/\href{https://youtu.be/BaQmt3hHVtw}{\nolinkurl{BaQmt3hHVtw}}
\item \textit{Every triangle has a unique Circumbilliard}, 2019. youtu.be/\href{https://youtu.be/vSCnorIJ2X8}{\nolinkurl{vSCnorIJ2X8}}
\item \textit{Circumbilliard of anticomplementary triangle}, 2019. youtu.be/\href{https://youtu.be/18RyUdh8qLk}{\nolinkurl{18RyUdh8qLk}}
\item \textit{Orthic Circumbilliard \& Locus of Its Mittenpunkt}, 2020. youtu.be/\href{https://youtu.be/5KL8st2vIb0}{\nolinkurl{5KL8st2vIb0}}
\item \textit{Circumbilliards of Triangles Derived from 3-Periodics}, 2020. youtu.be/\href{https://youtu.be/Og7xLgkrLqw}{\nolinkurl{Og7xLgkrLqw}}
\item \textit{Feuerbach and Excentral Jerabek Circumhyperbolas: Invariant Focal Length Ratio}, 2020. youtu.be/\href{https://youtu.be/ewioM6-nCpY}{\nolinkurl{ewioM6-nCpY}}
\item \textit{Excentral MacBeath Inconic: Invariant Aspect Ratio}, 2020. youtu.be/\href{https://youtu.be/IxrIkW5tj20}{\nolinkurl{IxrIkW5tj20}}
\item \textit{X(3)-Centered Excentral Inconic: Invariant Aspect Ratio}, 2020. youtu.be/\href{https://youtu.be/ojxzOS1Sjjo}{\nolinkurl{ojxzOS1Sjjo}}
\item \textit{Excentral X(3)-Centered \& MacBeath Inconics: Invariant Aspect Ratio}, 2020. youtu.be/\href{https://youtu.be/CHbrZvx1I8w}{\nolinkurl{CHbrZvx1I8w}}
\end{enumerate}

\section{Inversions of Pivoting Triangles (8)}

\begin{enumerate}[resume]
\item \textit{Isocurves of distance between triangle center in reference triangle vs in inversive (or polar) image}, 2021. youtu.be/\href{https://youtu.be/SvG8ogSzZh4}{\nolinkurl{SvG8ogSzZh4}}
\item \textit{Circular isocurves of distance between Symmedian of a reference triangle and its inversive image}, 2021. youtu.be/\href{https://youtu.be/uoAocIk8LYI}{\nolinkurl{uoAocIk8LYI}}
\item \textit{Stationary Symmedian Point of the Inversive Image of an X(6)-Pivoting Triangle}, 2021. youtu.be/\href{https://youtu.be/GLRZSbzzP1U}{\nolinkurl{GLRZSbzzP1U}}
\item \textit{Inversive Image of Pivoting Triangle, Part I: Stationary Symmedian Point X(6) of Inversive}, 2021. youtu.be/\href{https://youtu.be/mI4BcoUnb6U}{\nolinkurl{mI4BcoUnb6U}}
\item \textit{Inversive Image of Pivoting Triangle, Part II: Conic Loci of X(3) and X(6) of Inversive}, 2021. youtu.be/\href{https://youtu.be/qVCjQfXkK_k}{\nolinkurl{qVCjQfXkK\_k}}
\item \textit{Inversive Image of Pivoting Triangle, Part III: Piecewise Point-Circular Loci of X'(15) and X'(16)}, 2021. youtu.be/\href{https://youtu.be/iFEHMSELf7U}{\nolinkurl{iFEHMSELf7U}}
\item \textit{Inversive Image of Pivoting Triangle, Part IV: Piecewise Point-Elliptic Loci of X'(61) and X'(62)}, 2021. youtu.be/\href{https://youtu.be/HPp5I_kCf0g}{\nolinkurl{HPp5I\_kCf0g}}
\item \textit{Inversive Image of Pivoting Triangle, Part V: Loci of the Isodynamic Midpoint X'(187)}, 2021. youtu.be/\href{https://youtu.be/qZFd0SwJ8xg}{\nolinkurl{qZFd0SwJ8xg}}
\end{enumerate}

\section{Inversive Poncelet (20)}

\begin{enumerate}[resume]
\item \textit{Circles Galore I: Loci of Focus-Inversive 3-Periodics in the Elliptic Billiard (11 notable centers)}, 2020. youtu.be/\href{https://youtu.be/tKB-50zW8F4}{\nolinkurl{tKB-50zW8F4}}
\item \textit{Circles Galore II: 29 Loci of Focus-Inversive 3-Periodics in the Elliptic Billiard}, 2020. youtu.be/\href{https://youtu.be/srjm23nQbMc}{\nolinkurl{srjm23nQbMc}}
\item \textit{Circles Galore III: Loci of Focus-Inversive 3-Periodics in the Elliptic Billiard (9 notable centers)}, 2020. youtu.be/\href{https://youtu.be/OAD2hpCRgCI}{\nolinkurl{OAD2hpCRgCI}}
\item \textit{Elliptic Billiard N-Periodics: invariant sum of inverse focal distances \& inversive Pascal Limaçon}, 2020. youtu.be/\href{https://youtu.be/FmWq1YiAs5o}{\nolinkurl{FmWq1YiAs5o}}
\item \textit{Inversive Elliptic Billiard N-Periodics are Circular Arcs Interscribed between two Pascal Limaçons}, 2020. youtu.be/\href{https://youtu.be/bFsehskizls}{\nolinkurl{bFsehskizls}}
\item \textit{Inversive Invariants of Elliptic Billiard N-Periodics Nestled within Pascal's Limaçon}, 2020. youtu.be/\href{https://youtu.be/wkstGKq5jOo}{\nolinkurl{wkstGKq5jOo}}
\item \textit{Focus-Inversive N=3 Family in the Elliptic Billiard: Pascal Limaçon-Inscribed Billiard Triangles!}, 2020. youtu.be/\href{https://youtu.be/Y-j5eXqKGQE}{\nolinkurl{Y-j5eXqKGQE}}
\item \textit{Invariants of Inversive, Polar, and Dual Polygons derived from N-Periodics in the Elliptic Billiard}, 2020. youtu.be/\href{https://youtu.be/qyAHOW32NXY}{\nolinkurl{qyAHOW32NXY}}
\item \textit{Invariant Inversive Perimeter (all N) and Area Product (odd N)}, 2020. youtu.be/\href{https://youtu.be/bTkbdEPNUOY}{\nolinkurl{bTkbdEPNUOY}}
\item \textit{Focus-Inversive Polygons' Equi-Area Pedal Polygons (wrt foci)}, 2020. youtu.be/\href{https://youtu.be/0L2uMk2xyKk}{\nolinkurl{0L2uMk2xyKk}}
\item \textit{Invariant Area Ratio Between Focus-Inversive Polygons for all N}, 2020. youtu.be/\href{https://youtu.be/eG4UCgMkKl8}{\nolinkurl{eG4UCgMkKl8}}
\item \textit{Centers of Inversive Arcs area a Bicentric Poncelet Family w/ Invariants}, 2020. youtu.be/\href{https://youtu.be/mXkk_4RYrnU}{\nolinkurl{mXkk\_4RYrnU}}
\item \textit{Invariant Inversive perimeter and N=6 a/b=2 Null Antipedal Area}, 2020. youtu.be/\href{https://youtu.be/fOAES-CzjNI}{\nolinkurl{fOAES-CzjNI}}
\item \textit{Loci of Invariant Inversive perimeter and N=6 a/b=2 Null Antipedal Area}, 2020. youtu.be/\href{https://youtu.be/HMhZW_kWLGw}{\nolinkurl{HMhZW\_kWLGw}}
\item \textit{Elliptic Billiard Focus-Inversive N-periodics: Loci of Vertex, Perimeter, Area Centroids are Circles}, 2020. youtu.be/\href{https://youtu.be/hjzW84ZZApA}{\nolinkurl{hjzW84ZZApA}}
\item \textit{Self-Intersected 5-periodics in the Elliptic Billiard: Loci of Focus-Inversive Centroids are Circles}, 2020. youtu.be/\href{https://youtu.be/7bzID9SVwqM}{\nolinkurl{7bzID9SVwqM}}
\item \textit{Inversive Symmedian I: 4 Families with Stationary Symmedian Points}, 2021. youtu.be/\href{https://youtu.be/F0YuAx2Dy6M}{\nolinkurl{F0YuAx2Dy6M}}
\item \textit{Inversive Symmedian II: Stationary Inversive X6 in Excentral Family}, 2021. youtu.be/\href{https://youtu.be/vHUsFzvnRu4}{\nolinkurl{vHUsFzvnRu4}}
\item \textit{Inversive Symmedian III: Stationary Cosine Circle of Inversive Triangles}, 2021. youtu.be/\href{https://youtu.be/wwX_QfkjVi0}{\nolinkurl{wwX\_QfkjVi0}}
\item \textit{A Rose in the Elliptic Billiard: the Constant-Perimeter, Focus-Inversive Family}, 2021. youtu.be/\href{https://youtu.be/wkstGKq5jOo}{\nolinkurl{wkstGKq5jOo}}
\end{enumerate}

\section{Isodynamic Points (4)}

\begin{enumerate}[resume]
\item \textit{Isodynamic points I: construction via Apollonius' circles}, 2021. youtu.be/\href{https://youtu.be/L0pr8PuPavk}{\nolinkurl{L0pr8PuPavk}}
\item \textit{Isodynamic points II: centers of Apollonius' circles are collinear \& reciprocity wrt circumcircle}, 2021. youtu.be/\href{https://youtu.be/Bqd7mHHDRnU}{\nolinkurl{Bqd7mHHDRnU}}
\item \textit{Isodynamic points III: constructing X15 with barycentric coordinates}, 2021. youtu.be/\href{https://youtu.be/uUSKWHrctOQ}{\nolinkurl{uUSKWHrctOQ}}
\item \textit{Isodynamic Points IV: flank triangles between hexagons erected on a triangle's sides have common X16}, 2021. youtu.be/\href{https://youtu.be/e3MkijszDEA}{\nolinkurl{e3MkijszDEA}}
\end{enumerate}

\section{Isogonal and Isotomic (2)}

\begin{enumerate}[resume]
\item \textit{Antiorthic Axis and 5 points on the Billiard}, 2019. youtu.be/\href{https://youtu.be/vyHZ8fwyiE8}{\nolinkurl{vyHZ8fwyiE8}}
\item \textit{Isotomic and Isogonal Conjugates of Billiard with respect to the 3-periodic family}, 2019. youtu.be/\href{https://youtu.be/C0fIMK6fuAU}{\nolinkurl{C0fIMK6fuAU}}
\end{enumerate}

\section{Isogonals over Poncelet (2)}

\begin{enumerate}[resume]
\item \textit{Extending A. Skutin's result I: locus of isogonal conjugate of fixed point over Poncelet triangles}, 2021. youtu.be/\href{https://youtu.be/x0953ASLkuk}{\nolinkurl{x0953ASLkuk}}
\item \textit{Extending A. Skutin's result II: conic frontiers of isogonal loci \& confocal envelope of line loci}, 2021. youtu.be/\href{https://youtu.be/o9iHWbk4aPc}{\nolinkurl{o9iHWbk4aPc}}
\end{enumerate}

\section{Locus App (5)}

\begin{enumerate}[resume]
\item \textit{Loci of Ellipse-Inscribed Triangles: Part 01 - Intro to the App}, 2020. youtu.be/\href{https://youtu.be/o63QTcpDqNA}{\nolinkurl{o63QTcpDqNA}}
\item \textit{Loci of Ellipse-Inscribed Triangles: Part 02 - The Homothetic Family}, 2020. youtu.be/\href{https://youtu.be/1i4ys4TFw48}{\nolinkurl{1i4ys4TFw48}}
\item \textit{Loci of Ellipse-Inscribed Triangles: Part 03 - Derived Triangles}, 2020. youtu.be/\href{https://youtu.be/DSeZXMnuirA}{\nolinkurl{DSeZXMnuirA}}
\item \textit{Loci of Ellipse-Inscribed Triangles: Part 04 - Locus Type}, 2020. youtu.be/\href{https://youtu.be/QIcx89W6J_k}{\nolinkurl{QIcx89W6J\_k}}
\item \textit{Segment Locus of the Center of a Certain Apollonius' Circle}, 2021. youtu.be/\href{https://youtu.be/Ovypr11bNqU}{\nolinkurl{Ovypr11bNqU}}
\end{enumerate}

\section{Misc (23)}

\begin{enumerate}[resume]
\item \textit{Horizontal-Vertical Billiard in a Rhombus and Parallelogram: are there N-Periodics?}, 2020. youtu.be/\href{https://youtu.be/XKCYL-8hEVA}{\nolinkurl{XKCYL-8hEVA}}
\item \textit{Elliptic Billiards in Brazil}, 2019. youtu.be/\href{https://youtu.be/PHitZFbps8M}{\nolinkurl{PHitZFbps8M}}
\item \textit{Loci of Outer Napoleon Equilateral Construction}, 2019. youtu.be/\href{https://youtu.be/70-E-NZrNCQ}{\nolinkurl{70-E-NZrNCQ}}
\item \textit{Conservation of Sum and Product of Cosines}, 2019. youtu.be/\href{https://youtu.be/P8ykpE_ZbZ8}{\nolinkurl{P8ykpE\_ZbZ8}}
\item \textit{The Miquel Point of the Extouch and Excentral Triangles}, 2019. youtu.be/\href{https://youtu.be/CKaV_AKZc1U}{\nolinkurl{CKaV\_AKZc1U}}
\item \textit{An invariant in the parabolic pair associated with the N=3 family}, 2020. youtu.be/\href{https://youtu.be/VpDrCPG6th0}{\nolinkurl{VpDrCPG6th0}}
\item \textit{Non-monotonic X(88) and the X(1)-X(100) envelope}, 2020. youtu.be/\href{https://youtu.be/nJLp--JjDZU}{\nolinkurl{nJLp--JjDZU}}
\item \textit{The Thomson Cubic of 3-periodics}, 2020. youtu.be/\href{https://youtu.be/uNHIZXgZDOs}{\nolinkurl{uNHIZXgZDOs}}
\item \textit{Locus and elliptic envelope of excircle tangents' hexagon (side touchpoints)}, 2020. youtu.be/\href{https://youtu.be/4XMTSvZtTJo}{\nolinkurl{4XMTSvZtTJo}}
\item \textit{Six-Point Conic passes through Sideline Tangents to Excircles}, 2020. youtu.be/\href{https://youtu.be/kVtTR-aINX4}{\nolinkurl{kVtTR-aINX4}}
\item \textit{Reuleaux Triangle: Properties of Negative Pedal Curve, and Exploring its Billiard Trajectories}, 2020. youtu.be/\href{https://youtu.be/aGtDyVjlrvM}{\nolinkurl{aGtDyVjlrvM}}
\item \textit{Pascal's Limaçon as Envelope of Circles}, 2020. youtu.be/\href{https://youtu.be/495A_ZjgcyE}{\nolinkurl{495A\_ZjgcyE}}
\item \textit{Elliptic Billiard 3-Periodics: Invariants of the Focal Hyperbola}, 2020. youtu.be/\href{https://youtu.be/_ydXSm-QplM}{\nolinkurl{\_ydXSm-QplM}}
\item \textit{Ellipse Maximal Distance Chords I: locus of the farthest midpoint can be a 4-leaf clover}, 2021. youtu.be/\href{https://youtu.be/kJUkidbrn3g}{\nolinkurl{kJUkidbrn3g}}
\item \textit{Ellipse Maximal Distance Chords II: the Astroidal Evolute}, 2021. youtu.be/\href{https://youtu.be/YvoyN46biq8}{\nolinkurl{YvoyN46biq8}}
\item \textit{Circumellipscevian Triangle: area ratio invariant in N=3 Concentric Poncelet}, 2021. youtu.be/\href{https://youtu.be/4Q1uouMQzXU}{\nolinkurl{4Q1uouMQzXU}}
\item \textit{Excentral-Tangential Family of Elliptic Billiard 3-Periodics: invariant perimeter and stationary X7}, 2021. youtu.be/\href{https://youtu.be/Pqer9GfADqc}{\nolinkurl{Pqer9GfADqc}}
\item \textit{Apresentação do Curso 33o CBM IMPA-2021
"Invariantes Ponceletianas: um Passeio Experimental"}, 2021. youtu.be/\href{https://youtu.be/UakZhTIQVro}{\nolinkurl{UakZhTIQVro}}
\item \textit{Interpolation of pedal and contrapedal curves}, 2021. youtu.be/\href{https://youtu.be/0SW_tBUeNKg}{\nolinkurl{0SW\_tBUeNKg}}
\item \textit{Ellipse-inscribed triangles w/ perimeter-trisecting vertices. What is the locus of the centroid?}, 2021. youtu.be/\href{https://youtu.be/bGAE4IgBPN4}{\nolinkurl{bGAE4IgBPN4}}
\item \textit{Cramer-Castillon \& Poncelet: surprising Brocard and Lemoine harmonies}, 2021. youtu.be/\href{https://youtu.be/Aw4U2Ah1Byc}{\nolinkurl{Aw4U2Ah1Byc}}
\item \textit{Sixty Pascal Lines over Certain N=6 Poncelet Families}, 2021. youtu.be/\href{https://youtu.be/qB-tGP6dQkk}{\nolinkurl{qB-tGP6dQkk}}
\item \textit{Prof. Pamfilos' Construction of 6 points on a parabola}, 2021. youtu.be/\href{https://youtu.be/ElxjUgHmKNU}{\nolinkurl{ElxjUgHmKNU}}
\end{enumerate}

\section{Multiple Caustics (16)}

\begin{enumerate}[resume]
\item \textit{Poncelet Triangle Families w/ Multiple Caustics (derived from the Poristic and Confocal Families)}, 2021. youtu.be/\href{https://youtu.be/8HXgkuY-nFQ}{\nolinkurl{8HXgkuY-nFQ}}
\item \textit{Half N=4 Poncelet Triangles: elliptic locus of barycenter over 4 different families}, 2021. youtu.be/\href{https://youtu.be/6yNod1LFVrY}{\nolinkurl{6yNod1LFVrY}}
\item \textit{Poncelet Triangles with Smoothly-Varying Circular Caustic: the amazing loci of X(11) and X(59)}, 2021. youtu.be/\href{https://youtu.be/2VqgB6KvP2g}{\nolinkurl{2VqgB6KvP2g}}
\item \textit{Bicentrics w/ Two Caustics I: circular locus of X(1)}, 2021. youtu.be/\href{https://youtu.be/OM7uilfdGgk}{\nolinkurl{OM7uilfdGgk}}
\item \textit{Bicentrics w/ Two Caustics II: circular Loci of X(1), X(40), X(165)}, 2021. youtu.be/\href{https://youtu.be/qJGhf798E-s}{\nolinkurl{qJGhf798E-s}}
\item \textit{Bicentrics w/ Two Caustics III: sextic loci of X(2), X(4), X(5)}, 2021. youtu.be/\href{https://youtu.be/6Fqp6Z1Q-0A}{\nolinkurl{6Fqp6Z1Q-0A}}
\item \textit{Bicentrics w/ Two Caustics IV: loci of X(1) (circle) and X(2) (sextic), varying caustic}, 2021. youtu.be/\href{https://youtu.be/3dnsWPlAmxE}{\nolinkurl{3dnsWPlAmxE}}
\item \textit{Bicentrics w/ Two Caustics V: 1 circle, 2 non-conic loci of the excenters (also: X1,X40 circular)}, 2021. youtu.be/\href{https://youtu.be/qdqIuT-Qk6k}{\nolinkurl{qdqIuT-Qk6k}}
\item \textit{Confocals w/ Two Caustics I: locus of the incenter X1 for 4 different caustics}, 2021. youtu.be/\href{https://youtu.be/C14TL430UBc}{\nolinkurl{C14TL430UBc}}
\item \textit{Bicentrics w/ Three Caustics I: Non-Conic Loci of X1,X2,X6,X8,X10}, 2021. youtu.be/\href{https://youtu.be/cvB0A7LmlZc}{\nolinkurl{cvB0A7LmlZc}}
\item \textit{Bicentrics w/ Three Caustics II -- Non-conic loci of excenters and  incenter}, 2021. youtu.be/\href{https://youtu.be/A_-U2VvM5kY}{\nolinkurl{A\_-U2VvM5kY}}
\item \textit{Bicentrics w/ Three Caustics III: loci of incenter \& barycenter over 4 possible triangle choices}, 2021. youtu.be/\href{https://youtu.be/E1Rcu38MePQ}{\nolinkurl{E1Rcu38MePQ}}
\item \textit{Bicentrics w/ Three Caustics IV: loci of symmedian \& Nagel points over 4 triangle choices}, 2021. youtu.be/\href{https://youtu.be/u1_uANWDNr8}{\nolinkurl{u1\_uANWDNr8}}
\item \textit{Confocals w/ Two Caustics II: locus of the incenter X1 over confocal caustic sweep}, 2021. youtu.be/\href{https://youtu.be/kCY6KHFDV2M}{\nolinkurl{kCY6KHFDV2M}}
\item \textit{Confocals w/ Two Caustics III: loci of excenters: two elliptic, one non-conic}, 2021. youtu.be/\href{https://youtu.be/AisIrfn4IGg}{\nolinkurl{AisIrfn4IGg}}
\item \textit{Confocals w/ Two Caustics IV: two notable loci of the excenters (N=4,6 caustics)}, 2021. youtu.be/\href{https://youtu.be/wB9bVkY9rqU}{\nolinkurl{wB9bVkY9rqU}}
\end{enumerate}

\section{N-Point Porisms (3)}

\begin{enumerate}[resume]
\item \textit{N-Point Porism I: three-point quasi-porisms}, 2021. youtu.be/\href{https://youtu.be/ui0YmSqR-vI}{\nolinkurl{ui0YmSqR-vI}}
\item \textit{N-Point Porism II: 3-point chord iteration: porism iff one point lies on mystery ellipse}, 2021. youtu.be/\href{https://youtu.be/hvTYkkvhePQ}{\nolinkurl{hvTYkkvhePQ}}
\item \textit{N-Point Porism III: 3-point chord iteration \& elliptic loci of third point for various N}, 2021. youtu.be/\href{https://youtu.be/a6zoq1YNPrw}{\nolinkurl{a6zoq1YNPrw}}
\end{enumerate}

\section{N=3 Loci (25)}

\begin{enumerate}[resume]
\item \textit{Locus of Feuerbach point, its anticomplement and three extouchpoints}, 2019. youtu.be/\href{https://youtu.be/TXdg7tUl8lc}{\nolinkurl{TXdg7tUl8lc}}
\item \textit{Locus of Excentral and Anticomplementary Triangles and Objects}, 2019. youtu.be/\href{https://youtu.be/50dyxWJhfN4}{\nolinkurl{50dyxWJhfN4}}
\item \textit{3-Periodics and Derived Triangles}, 2019. youtu.be/\href{https://youtu.be/xyroRTEVNDc}{\nolinkurl{xyroRTEVNDc}}
\item \textit{Elliptic Loci of X(1) to X(5) and Euler Line}, 2019. youtu.be/\href{https://youtu.be/sMcNzcYaqtg}{\nolinkurl{sMcNzcYaqtg}}
\item \textit{Locus of several triangular centers is elliptic}, 2019. youtu.be/\href{https://youtu.be/f84W2aVnMpU}{\nolinkurl{f84W2aVnMpU}}
\item \textit{Locus of vertices of Feuerbach Triangle is non-elliptic}, 2019. youtu.be/\href{https://youtu.be/YPz0_xbit2I}{\nolinkurl{YPz0\_xbit2I}}
\item \textit{Non-Elliptic loci of vertices of Medial, Intouch and Feuerbach triangles}, 2019. youtu.be/\href{https://youtu.be/OGvCQbYqJyI}{\nolinkurl{OGvCQbYqJyI}}
\item \textit{Locus of Bevan Point X(40) is similar to billiard}, 2019. youtu.be/\href{https://youtu.be/NwPioKleiyU}{\nolinkurl{NwPioKleiyU}}
\item \textit{Locus of Bevan Point X(40) identical to billiard when a/b=golden ratio}, 2020. youtu.be/\href{https://youtu.be/rg28gGr-Qeo}{\nolinkurl{rg28gGr-Qeo}}
\item \textit{Locus of X(59) has 4 self-intersections}, 2020. youtu.be/\href{https://youtu.be/pl_PqSuhlx0}{\nolinkurl{pl\_PqSuhlx0}}
\item \textit{Anticomplementary triangle intouchpoints}, 2019. youtu.be/\href{https://youtu.be/NzGKU75-Fuo}{\nolinkurl{NzGKU75-Fuo}}
\item \textit{Anticomplementary, Medial Triangles and the Intouch Triangle}, 2019. youtu.be/\href{https://youtu.be/xyHUwpvAj3g}{\nolinkurl{xyHUwpvAj3g}}
\item \textit{The locus of X(140) is a circle over 3-periodics in the Elliptic Billiard}, 2020. youtu.be/\href{https://youtu.be/4g5G9eluxJo}{\nolinkurl{4g5G9eluxJo}}
\item \textit{Circular loci for X(140) and X(547) over 3-periodics in the Elliptic Billiard}, 2020. youtu.be/\href{https://youtu.be/_umuHLl9cCU}{\nolinkurl{\_umuHLl9cCU}}
\item \textit{I love loci! Classic triangle centers as one vertex moves parallel to the base}, 2020. youtu.be/\href{https://youtu.be/Y50RFjhvsAo}{\nolinkurl{Y50RFjhvsAo}}
\item \textit{Triangle family with 2 fixed vertices whose incenter X1 sweeps a circle}, 2020. youtu.be/\href{https://youtu.be/MPKt7Q2DhJc}{\nolinkurl{MPKt7Q2DhJc}}
\item \textit{Gallery of Artful, Colorful Triangle Loci. Music by Tchaikovsky}, 2021. youtu.be/\href{https://youtu.be/l-O5UT8tpuw}{\nolinkurl{l-O5UT8tpuw}}
\item \textit{Family of Poncelet Triangles between Concentric, Axis-Parallel Ellipses centered on X(1249)}, 2021. youtu.be/\href{https://youtu.be/QQSN_ndDJQk}{\nolinkurl{QQSN\_ndDJQk}}
\item \textit{Loci of Triangle Centers of Poncelet 3-Periodics: 
I Generic Pair, Elliptic Loci}, 2021. youtu.be/\href{https://youtu.be/p1medAei_As}{\nolinkurl{p1medAei\_As}}
\item \textit{Loci of Triangles Centers of Poncelet 3-Periodics II: Pair with Circumcircle, Circular Loci}, 2021. youtu.be/\href{https://youtu.be/HXgJQo2UT_8}{\nolinkurl{HXgJQo2UT\_8}}
\item \textit{Loci of Triangle Centers of Poncelet 3-Periodics IV: Outer Elllipse, Inner Non-Concentric Circle}, 2021. youtu.be/\href{https://youtu.be/w7sZ5O8k4xU}{\nolinkurl{w7sZ5O8k4xU}}
\item \textit{Poncelet 3-Periodics in a Non-Concentric, Unaligned Ellipse Pair}, 2021. youtu.be/\href{https://youtu.be/bjHpXVyXXVc}{\nolinkurl{bjHpXVyXXVc}}
\item \textit{Poncelet 3-Periodics in Generic Pair and Affine Image with Circumcircle (Blaschke Parametrization)}, 2021. youtu.be/\href{https://youtu.be/6xSFBLWIkTM}{\nolinkurl{6xSFBLWIkTM}}
\item \textit{Elliptic Billiard 3-Periodics: triangle centers whose loci sweep 2 circles and 2 segments}, 2021. youtu.be/\href{https://youtu.be/haFTsq5UyK4}{\nolinkurl{haFTsq5UyK4}}
\item \textit{Loci of Incenter and Excenters over Poncelet 3-Periodics in a Generic Ellipse Pair}, 2021. youtu.be/\href{https://youtu.be/z7qDgJEgPVY}{\nolinkurl{z7qDgJEgPVY}}
\end{enumerate}

\section{N>3 Periodics (23)}

\begin{enumerate}[resume]
\item \textit{4-periodics and Monge's Orthoptic Circle}, 2019. youtu.be/\href{https://youtu.be/9fI3iM2jrmI}{\nolinkurl{9fI3iM2jrmI}}
\item \textit{4-periodics: Loci of Triangle Centers for Vertex Triad}, 2019. youtu.be/\href{https://youtu.be/y2bnml8heGg}{\nolinkurl{y2bnml8heGg}}
\item \textit{5-periodics: locus of P1,P2,P3 triangle}, 2019. youtu.be/\href{https://youtu.be/yQMOtAGdrqA}{\nolinkurl{yQMOtAGdrqA}}
\item \textit{5-periodics: locus of P1,P2,P4 triangle}, 2019. youtu.be/\href{https://youtu.be/2MA1h-dMnw8}{\nolinkurl{2MA1h-dMnw8}}
\item \textit{5-periodics: Loci of Subtriangles (123 and 124)}, 2019. youtu.be/\href{https://youtu.be/4lj9yQ-e_cE}{\nolinkurl{4lj9yQ-e\_cE}}
\item \textit{Upright 5-periodic family}, 2019. youtu.be/\href{https://youtu.be/RQE1s2siPSo}{\nolinkurl{RQE1s2siPSo}}
\item \textit{6-periodic family}, 2019. youtu.be/\href{https://youtu.be/YZfFGew4azI}{\nolinkurl{YZfFGew4azI}}
\item \textit{Octagramma Mysticum}, 2019. youtu.be/\href{https://youtu.be/mDomB-_GiNA}{\nolinkurl{mDomB-\_GiNA}}
\item \textit{Enagramma Mysticum: loci of side intersections}, 2019. youtu.be/\href{https://youtu.be/ECo1hTCVuDg}{\nolinkurl{ECo1hTCVuDg}}
\item \textit{Mittenpunkt-like Construction of a Stationary Point}, 2019. youtu.be/\href{https://youtu.be/TV2p7fPlYfE}{\nolinkurl{TV2p7fPlYfE}}
\item \textit{Generalized Mittenpunkt and On-Caustic Extouchpoints}, 2019. youtu.be/\href{https://youtu.be/Bpc-MrR2IMc}{\nolinkurl{Bpc-MrR2IMc}}
\item \textit{Family of Orbits and Their Caustics}, 2019. youtu.be/\href{https://youtu.be/Y3q35DObfZU}{\nolinkurl{Y3q35DObfZU}}
\item \textit{Ellipse-Inscribed Parallelogram: invariants of the Pedal Polygon w/ respect to boundary points}, 2020. youtu.be/\href{https://youtu.be/7eUQQgR-w3c}{\nolinkurl{7eUQQgR-w3c}}
\item \textit{Elliptic Billiard with Perpendicular Reflection Rule}, 2020. youtu.be/\href{https://youtu.be/0Qqi7ubS9Bw}{\nolinkurl{0Qqi7ubS9Bw}}
\item \textit{Invariant Area Ratios to Minimum-Area Steiner Pedal Polygons}, 2020. youtu.be/\href{https://youtu.be/f0JwRlu7iaY}{\nolinkurl{f0JwRlu7iaY}}
\item \textit{Circumcircles of Focus with Consecutive Vertices Homothetic to Focus Antipedal}, 2020. youtu.be/\href{https://youtu.be/kVxh5jfZb9Q}{\nolinkurl{kVxh5jfZb9Q}}
\item \textit{Incenters of Focus Triads: Invariant Area Ratio to N-Periodic and Elliptic Locus}, 2020. youtu.be/\href{https://youtu.be/ehnbRnCUmS0}{\nolinkurl{ehnbRnCUmS0}}
\item \textit{An Invariant Based on Inradii and Circumradii of Subtriangles in the Elliptic Billiard}, 2020. youtu.be/\href{https://youtu.be/ipOEfbxWsdk}{\nolinkurl{ipOEfbxWsdk}}
\item \textit{Cremona-Inversive Polygon of Odd-N-Periodics in the Elliptic Billiard: Zero Signed Area}, 2020. youtu.be/\href{https://youtu.be/GrDh0mmtHCE}{\nolinkurl{GrDh0mmtHCE}}
\item \textit{Family of Poncelet 7-Periodics Interscribed Between Two Ellipses in General Position}, 2021. youtu.be/\href{https://youtu.be/kzxf7ZgJ5Hw}{\nolinkurl{kzxf7ZgJ5Hw}}
\item \textit{N-Periodics in the Elliptic Billiard: Invariant Sum of Cosines}, 2021. youtu.be/\href{https://youtu.be/qP67bdqS3nQ}{\nolinkurl{qP67bdqS3nQ}}
\item \textit{N-Periodics in the Homothetic Pair: Invariant Sum of Cotangents}, 2021. youtu.be/\href{https://youtu.be/TzKsr-x1KFk}{\nolinkurl{TzKsr-x1KFk}}
\item \textit{Invariant Sums of Generalized Exradii (and their inverses) over various Poncelet families}, 2021. youtu.be/\href{https://youtu.be/yiGIl-nXXj8}{\nolinkurl{yiGIl-nXXj8}}
\end{enumerate}

\section{Original 2011 (3)}

\begin{enumerate}[resume]
\item \textit{3-Periodic trajectories}, 2011. youtu.be/\href{https://youtu.be/9zAr5-nm7mw}{\nolinkurl{9zAr5-nm7mw}}
\item \textit{Locus of incenter is elliptic for family of 3-periodics}, 2011. youtu.be/\href{https://youtu.be/BBsyM7RnswA}{\nolinkurl{BBsyM7RnswA}}
\item \textit{Locus of the incircle touchpoints is a higher-order curve}, 2011. youtu.be/\href{https://youtu.be/9xU6T7hQMzs}{\nolinkurl{9xU6T7hQMzs}}
\end{enumerate}

\section{Orthic Phenomena (3)}

\begin{enumerate}[resume]
\item \textit{Locus orthic triangle's incenter is a 4-arc ellipse}, 2019. youtu.be/\href{https://youtu.be/3qJnwpFkUFQ}{\nolinkurl{3qJnwpFkUFQ}}
\item \textit{Locus of orthocenter, orthic orthocenter, incenter, and orthic orthic's incenter}, 2019. youtu.be/\href{https://youtu.be/HY577AZVi7I}{\nolinkurl{HY577AZVi7I}}
\item \textit{Excentral of Orthic for Acute and Obtuse Triangles}, 2019. youtu.be/\href{https://youtu.be/-bLuvICzmqM}{\nolinkurl{-bLuvICzmqM}}
\end{enumerate}

\section{Pedal Invariants (10)}

\begin{enumerate}[resume]
\item \textit{Concyclic Feet of Focal Pedals and Invariant Product of Sums of Lengths for odd N}, 2020. youtu.be/\href{https://youtu.be/OT-xAdbOp8o}{\nolinkurl{OT-xAdbOp8o}}
\item \textit{Invariant sum of squared altitudes from each focus to tangential polygon sides}, 2020. youtu.be/\href{https://youtu.be/VUtBRzmbOYU}{\nolinkurl{VUtBRzmbOYU}}
\item \textit{Altitude Invariants to N-Periodics and their Tangential Polygons (N=3,4)}, 2020. youtu.be/\href{https://youtu.be/MvZhWbI6iB8}{\nolinkurl{MvZhWbI6iB8}}
\item \textit{Altitude Invariants to N-Periodics and their Tangential Polygons (N=5,6)}, 2020. youtu.be/\href{https://youtu.be/ZMHLmWXeKrM}{\nolinkurl{ZMHLmWXeKrM}}
\item \textit{Sum of square altitudes from arbitrary point to N-periodic tangents is invariant}, 2020. youtu.be/\href{https://youtu.be/RNmHROZNGj8}{\nolinkurl{RNmHROZNGj8}}
\item \textit{Pedal polygons from each focus have invariant area product}, 2020. youtu.be/\href{https://youtu.be/sw8pJFMV00w}{\nolinkurl{sw8pJFMV00w}}
\item \textit{Pedal Polygons for the N-Periodic and its Tangent Polygon: Area Ratio Invariances}, 2020. youtu.be/\href{https://youtu.be/6F7Y3UKJzdk}{\nolinkurl{6F7Y3UKJzdk}}
\item \textit{Exploring Amazing Invariants of N-Periodics and their Pedal Polygons}, 2020. youtu.be/\href{https://youtu.be/2yXbOV7qf7k}{\nolinkurl{2yXbOV7qf7k}}
\item \textit{Area Invariants of Pedal and Antipedal Polygons}, 2020. youtu.be/\href{https://youtu.be/LN623VjeeFQ}{\nolinkurl{LN623VjeeFQ}}
\item \textit{Centroid Stationarity (even N)}, 2020. youtu.be/\href{https://youtu.be/j_GD_g8aIbg}{\nolinkurl{j\_GD\_g8aIbg}}
\end{enumerate}

\section{Pencil of Circles (6)}

\begin{enumerate}[resume]
\item \textit{Poncelet in Circle Pencil I: the geometric Flamenco of 5-Periodics with 4 caustics}, 2021. youtu.be/\href{https://youtu.be/L5A_S4VQLiw}{\nolinkurl{L5A\_S4VQLiw}}
\item \textit{Poncelet in Circle Pencil II: constant-perimeter polar image of 5-gons wrt to a limiting point (billiard)}, 2021. youtu.be/\href{https://youtu.be/8dRap3ZWQjQ}{\nolinkurl{8dRap3ZWQjQ}}
\item \textit{Poncelet in Circle Pencil II: The automatically-closing N=4 family (zero limit-polar perimeter)}, 2021. youtu.be/\href{https://youtu.be/-8CVK18UkM8}{\nolinkurl{-8CVK18UkM8}}
\item \textit{Poncelet in Circle Pencil III: The automatically-closing N=6 family (zero limit-polar perimeter)}, 2021. youtu.be/\href{https://youtu.be/ZTn8oJZ5p3o}{\nolinkurl{ZTn8oJZ5p3o}}
\item \textit{Poncelet in Circle Pencil IV: Polar images of all closing permutations have same invariant perimeter}, 2021. youtu.be/\href{https://youtu.be/KQyLs1NyYZ0}{\nolinkurl{KQyLs1NyYZ0}}
\item \textit{Inverting a triangle with respect to a pencil of circles}, 2021. youtu.be/\href{https://youtu.be/hmj5Mf7TV_s}{\nolinkurl{hmj5Mf7TV\_s}}
\end{enumerate}

\section{Pencil of Confocals (6)}

\begin{enumerate}[resume]
\item \textit{Tangents from a point on boundary to caustics}, 2019. youtu.be/\href{https://youtu.be/mkhhd536_2w}{\nolinkurl{mkhhd536\_2w}}
\item \textit{Tangents to caustics from billiard's vertex lie on a single circle}, 2019. youtu.be/\href{https://youtu.be/NsZUyDJ6IOs}{\nolinkurl{NsZUyDJ6IOs}}
\item \textit{Loci of tangents to confocals: point traverses entire elliptic boundary}, 2019. youtu.be/\href{https://youtu.be/EL4vgcJaktc}{\nolinkurl{EL4vgcJaktc}}
\item \textit{Loci of tangents to confocals: point traverses neighborhood of right vertex}, 2019. youtu.be/\href{https://youtu.be/J5CA9UJVflI}{\nolinkurl{J5CA9UJVflI}}
\item \textit{Locus of tangents from ellipse: -45,45 degrees starting points}, 2019. youtu.be/\href{https://youtu.be/lXhnBksS74E}{\nolinkurl{lXhnBksS74E}}
\item \textit{Locus of tangents from ellipse: 5,95,-45,45 degrees starting points}, 2019. youtu.be/\href{https://youtu.be/Ac0iej_TaEc}{\nolinkurl{Ac0iej\_TaEc}}
\end{enumerate}

\section{Poncelet Family (25)}

\begin{enumerate}[resume]
\item \textit{Family of 3-Periodics in Five Poncelet Pairs}, 2020. youtu.be/\href{https://youtu.be/8hkeksAsx0E}{\nolinkurl{8hkeksAsx0E}}
\item \textit{Isodynamic Pedals and Isogonic Antipedals: Equilaterals with Constant Area in the Homothetic Pair}, 2020. youtu.be/\href{https://youtu.be/7qoxAaG8sbk}{\nolinkurl{7qoxAaG8sbk}}
\item \textit{Poncelet Triangle Inscribed in Ellipse and Circumscribed in Circle}, 2019. youtu.be/\href{https://youtu.be/I1BFOXN-EUw}{\nolinkurl{I1BFOXN-EUw}}
\item \textit{Poncelet Family of Triangles over the Family of N=3 Caustics}, 2019. youtu.be/\href{https://youtu.be/53pCKKd_5qI}{\nolinkurl{53pCKKd\_5qI}}
\item \textit{Pencil of N=3 Poncelet Ellipse Pairs: Loci of Triangular Centers}, 2019. youtu.be/\href{https://youtu.be/B5dRXT8Xerw}{\nolinkurl{B5dRXT8Xerw}}
\item \textit{Three Geometers Walk into a Bar: the 3-periodic Poncelet-Steiner family has invariant Brocard angle.}, 2020. youtu.be/\href{https://youtu.be/2fvGd8wioZY}{\nolinkurl{2fvGd8wioZY}}
\item \textit{Poncelet Family: Amazing Circular Locus of X3 and the Steiner's Curvature Centroid}, 2020. youtu.be/\href{https://youtu.be/601OfxuSDGc}{\nolinkurl{601OfxuSDGc}}
\item \textit{5-Periodic Poncelet Families and their Pedal Polygons with Respect to their Curvature Centroids}, 2020. youtu.be/\href{https://youtu.be/RP18B827l5I}{\nolinkurl{RP18B827l5I}}
\item \textit{An N=3 Poncelet family (outer circle, inner ellipse) equivalent to Poristic Excentrals}, 2020. youtu.be/\href{https://youtu.be/wUu2iMesv3U}{\nolinkurl{wUu2iMesv3U}}
\item \textit{Between an Ellipse and a Concentric Circle: Poncelet 3-Periodics Identical to Poristic Triangles.}, 2020. youtu.be/\href{https://youtu.be/ML_AZoX736w}{\nolinkurl{ML\_AZoX736w}}
\item \textit{Between a Circle and a Concentric Ellipse: Poncelet 3-Periodics Identical to Poristic Excentrals.}, 2020. youtu.be/\href{https://youtu.be/xM1SAZO9bDc}{\nolinkurl{xM1SAZO9bDc}}
\item \textit{Jean-Victor Poncelet \& Jakob Steiner walk into a Bierhaus + discover many invariants! Prost! Santé!}, 2020. youtu.be/\href{https://youtu.be/30cuWWaZv7A}{\nolinkurl{30cuWWaZv7A}}
\item \textit{3-Periodics in a Concentric Homothetic Poncelet Pair: Circular Loci of four Triangle Centers}, 2020. youtu.be/\href{https://youtu.be/ZwTfwaJJitE}{\nolinkurl{ZwTfwaJJitE}}
\item \textit{3-Periodics in a Homothetic-Rotated Poncelet Pair: stationary orthocenter and loci of X107 and X122}, 2020. youtu.be/\href{https://youtu.be/fpd_Zot5cKk}{\nolinkurl{fpd\_Zot5cKk}}
\item \textit{N-Periodics on a Homothetic-Rotated Poncelet Pair: All Altitudes Meet at the Center!}, 2020. youtu.be/\href{https://youtu.be/ttKjzWeG5B8}{\nolinkurl{ttKjzWeG5B8}}
\item \textit{5- and 7-Periodics on a Homothetic-Rotated Poncelet Pair: All Altitudes Meet at the Center}, 2020. youtu.be/\href{https://youtu.be/gNHiZvBhKF8}{\nolinkurl{gNHiZvBhKF8}}
\item \textit{Concentric Poncelet Pair w Incircle: Ratio of Sidelength Product to Perimeter is Invariant for odd N}, 2020. youtu.be/\href{https://youtu.be/7Jg2nRkkUhQ}{\nolinkurl{7Jg2nRkkUhQ}}
\item \textit{Concentric Poncelet Pair w Circumcircle: Locus of Pseudo-Orthocenter is Circle (odd N) + Invariants}, 2020. youtu.be/\href{https://youtu.be/3f6YBohQCFg}{\nolinkurl{3f6YBohQCFg}}
\item \textit{Poncelet Invariants:
circular + point loci of the pseudo-circumcenter and pseudo-orthocenter, N=5,6}, 2020. youtu.be/\href{https://youtu.be/ZfQEDujbirQ}{\nolinkurl{ZfQEDujbirQ}}
\item \textit{Poncelet 3-Periodic Invariants (Outer Circle, Inner Concentric Ellipse) of the Nine-Point Center II}, 2020. youtu.be/\href{https://youtu.be/8xlYaQfQCTw}{\nolinkurl{8xlYaQfQCTw}}
\item \textit{New Invariants of Poncelet N-Periodics in the Homothetic Pair}, 2020. youtu.be/\href{https://youtu.be/2PdsC3CcqaE}{\nolinkurl{2PdsC3CcqaE}}
\item \textit{Concentric Ellipse-Circle Poncelet 3-Periodics: invariant inradius, circumradius and cosine sum}, 2021. youtu.be/\href{https://youtu.be/eIxb1so6ORo}{\nolinkurl{eIxb1so6ORo}}
\item \textit{Power Circles of Poncelet 3-Periodics have Invariant Total Area}, 2021. youtu.be/\href{https://youtu.be/_psLvzlWTvQ}{\nolinkurl{\_psLvzlWTvQ}}
\item \textit{Limiting Points of Poncelet 3-Periodic Circle Pairs: Loci, Properties, Invariants}, 2021. youtu.be/\href{https://youtu.be/bHTLS2XzkIQ}{\nolinkurl{bHTLS2XzkIQ}}
\item \textit{Six Families of Poncelet 3-Periodics in Concentric, Axis-Aligned Ellipses}, 2021. youtu.be/\href{https://youtu.be/14TQ5WlZxUw}{\nolinkurl{14TQ5WlZxUw}}
\end{enumerate}

\section{Poncelet Guitar Picks (1)}

\begin{enumerate}[resume]
\item \textit{Poncelet Guitar Picks I: Cosine Space of Confocal and Incircle 3-Periodic Families}, 2021. youtu.be/\href{https://youtu.be/uwdW95HI-q8}{\nolinkurl{uwdW95HI-q8}}
\end{enumerate}

\section{Poncelet Loci (1)}

\begin{enumerate}[resume]
\item \textit{Loci of Triangles Centers of Poncelet 3-Periodics III: Concentric Tilted Ellipse Pair}, 2021. youtu.be/\href{https://youtu.be/hpb7ZgKWjUY}{\nolinkurl{hpb7ZgKWjUY}}
\end{enumerate}

\section{Poncelet Plectra (2)}

\begin{enumerate}[resume]
\item \textit{Poncelet 3-Periodics with Incircle and Affine Confocal Image: Identical Invariant Sum of Cosines}, 2021. youtu.be/\href{https://youtu.be/CKVoQvErjj4}{\nolinkurl{CKVoQvErjj4}}
\item \textit{Poncelet 3-Periodics w/ Circumcircle and Affine Excentral Image: Identical Invariant Cosine Product}, 2021. youtu.be/\href{https://youtu.be/PMqoH4oGt10}{\nolinkurl{PMqoH4oGt10}}
\end{enumerate}

\section{Poncelet Propellers (5)}

\begin{enumerate}[resume]
\item \textit{Poncelet Propellers I: Ellipse Pair with Incircle, Invariant Total Area of Excentral Circumellipses}, 2020. youtu.be/\href{https://youtu.be/tHUDfx9o0Wg}{\nolinkurl{tHUDfx9o0Wg}}
\item \textit{Poncelet Propellers II: invariant total area of anticevian-circumellipse blades}, 2020. youtu.be/\href{https://youtu.be/crXxPJ93ZDk}{\nolinkurl{crXxPJ93ZDk}}
\item \textit{Poncelet Propellers III: family of incircle 3-periodics and one excentral circumellipse}, 2021. youtu.be/\href{https://youtu.be/JUCmAMsfdkI}{\nolinkurl{JUCmAMsfdkI}}
\item \textit{Poncelet Propellers IV: 3 excentral circumellipses and invariant total area}, 2021. youtu.be/\href{https://youtu.be/ub4wAv8Hgb0}{\nolinkurl{ub4wAv8Hgb0}}
\item \textit{Poncelet Propellers V: Total Area Remains Invariant for Non-Axis-Aligned Concentric Ellipse Pair}, 2021. youtu.be/\href{https://youtu.be/FJXMpUcslaA}{\nolinkurl{FJXMpUcslaA}}
\end{enumerate}

\section{Poristic (11)}

\begin{enumerate}[resume]
\item \textit{Chapple's Porism from (1746) and Weaver (1927) and Odehnal (2011) Invariants}, 2020. youtu.be/\href{https://youtu.be/DS4ryndDK6Q}{\nolinkurl{DS4ryndDK6Q}}
\item \textit{Circumbilliard of the Poristic Triangle Family: Invariant Aspect Ratio}, 2020. youtu.be/\href{https://youtu.be/yEu2aPiJwQo}{\nolinkurl{yEu2aPiJwQo}}
\item \textit{Poristic Triangle Family and the Amazing Invariant Excentral X3-Centered Inconic}, 2020. youtu.be/\href{https://youtu.be/0VHBjdHXbJc}{\nolinkurl{0VHBjdHXbJc}}
\item \textit{Simson Lines from X100 and Excentral Medials are Parallel to L(X1,X3).}, 2020. youtu.be/\href{https://youtu.be/DfzPrZ0SRRc}{\nolinkurl{DfzPrZ0SRRc}}
\item \textit{X1-Centered Circumconic \& X40-Centered (Excentral) Inconic: Identical Invariant Axes}, 2020. youtu.be/\href{https://youtu.be/hz0qEyVVvaI}{\nolinkurl{hz0qEyVVvaI}}
\item \textit{Loci of center and foci of the Circumbilliard to the Poristic Family are circles.}, 2020. youtu.be/\href{https://youtu.be/LGgh11LMGGY}{\nolinkurl{LGgh11LMGGY}}
\item \textit{Aspect Ratios of X10- and Excentral X5-Centered Circumconics are Invariant \& Equal}, 2020. youtu.be/\href{https://youtu.be/-4AAUSFxvmo}{\nolinkurl{-4AAUSFxvmo}}
\item \textit{Invariant aspect ratios for the Circumbilliard and Excentral X6-Ctr Circumconic}, 2020. youtu.be/\href{https://youtu.be/Fy4T-dmu-8s}{\nolinkurl{Fy4T-dmu-8s}}
\item \textit{Side-by-Side View of Poristic and 3-Periodic Families}, 2020. youtu.be/\href{https://youtu.be/NvjrX6XKSFw}{\nolinkurl{NvjrX6XKSFw}}
\item \textit{Feuerbach and Excentral Jerabek Hyperbolas to Poristic Family have invariant focal length ratio}, 2020. youtu.be/\href{https://youtu.be/bn1tq6NU_y0}{\nolinkurl{bn1tq6NU\_y0}}
\item \textit{Reference \& Excentral Simson Lines have fixed points and are Orthogonal!}, 2020. youtu.be/\href{https://youtu.be/M9NIRnfGtGc}{\nolinkurl{M9NIRnfGtGc}}
\end{enumerate}

\section{Quasi-Porisms (6)}

\begin{enumerate}[resume]
\item \textit{Multiple-caustic Poncelet I: a curious N=6 porism}, 2021. youtu.be/\href{https://youtu.be/jZQvbzD_DFw}{\nolinkurl{jZQvbzD\_DFw}}
\item \textit{Multiple-caustic Poncelet II: an N=14 example}, 2021. youtu.be/\href{https://youtu.be/wLpLwcmBLms}{\nolinkurl{wLpLwcmBLms}}
\item \textit{Multiple-Caustic Poncelet III: an N=10 example on variant of the bicentric family}, 2021. youtu.be/\href{https://youtu.be/iN9RreO6ONA}{\nolinkurl{iN9RreO6ONA}}
\item \textit{Multiple-Caustic Poncelet IV: cases when a pseudo-porism is maintained and cases when it fails}, 2021. youtu.be/\href{https://youtu.be/XkmyB2uY5XE}{\nolinkurl{XkmyB2uY5XE}}
\item \textit{Multiple-Caustic Poncelet V: a non-poristic and a quasi-poristic N=4, what is going on?}, 2021. youtu.be/\href{https://youtu.be/gqzKEBVEoiE}{\nolinkurl{gqzKEBVEoiE}}
\item \textit{Multiple-Caustic Poncelet VI: more quasi-porisms with two circular caustics and round-robin tangents}, 2021. youtu.be/\href{https://youtu.be/yypsjCNSYj0}{\nolinkurl{yypsjCNSYj0}}
\end{enumerate}

\section{Rotating Equilaterals (12)}

\begin{enumerate}[resume]
\item \textit{Rotating Equilaterals I: cevian triangles with respect to a fixed point on circumcircle}, 2021. youtu.be/\href{https://youtu.be/_QM6yNQ9heU}{\nolinkurl{\_QM6yNQ9heU}}
\item \textit{Rotating Equilaterals II: loci of cevian triangle centers post-affine transformation}, 2021. youtu.be/\href{https://youtu.be/f01RPhQS0mQ}{\nolinkurl{f01RPhQS0mQ}}
\item \textit{Rotating Equilaterals III: loci of cevian triangle centers w.r.t. a fixed point on the incircle}, 2021. youtu.be/\href{https://youtu.be/1Xtb0H1S8Z4}{\nolinkurl{1Xtb0H1S8Z4}}
\item \textit{Rotating Equilaterals IV: loci of cevian triangle centers w.r.t. any point on the plane}, 2021. youtu.be/\href{https://youtu.be/z_hndz19lH4}{\nolinkurl{z\_hndz19lH4}}
\item \textit{Rotating Equilaterals V: Eversions of the Anticevian triangle w.r.t. a point P on the circumcircle}, 2021. youtu.be/\href{https://youtu.be/aGjKEfAv2V8}{\nolinkurl{aGjKEfAv2V8}}
\item \textit{Rotating Equilaterals VI: The nature of eversions of the anticevian triangle w.r.t. a point P}, 2021. youtu.be/\href{https://youtu.be/hLKLq8eUmsY}{\nolinkurl{hLKLq8eUmsY}}
\item \textit{Rotating Equilaterals VII: Conic locus of X2 of antipedal triangles}, 2021. youtu.be/\href{https://youtu.be/8NBh0H2Vps0}{\nolinkurl{8NBh0H2Vps0}}
\item \textit{Rotating Equilaterals VIII: Loci of X3,X4,X5 of antipedal triangles}, 2021. youtu.be/\href{https://youtu.be/wiknzClcm4s}{\nolinkurl{wiknzClcm4s}}
\item \textit{Rotating Equilaterals IX: Loci of X6,X13,X14,X15,X16 of antipedal triangles}, 2021. youtu.be/\href{https://youtu.be/RF6Mm65qQgI}{\nolinkurl{RF6Mm65qQgI}}
\item \textit{Rotating Equilaterals X: Two incredible pedal and antipedal loci over Poncelet homothetics}, 2021. youtu.be/\href{https://youtu.be/aRadZgO4n2Q}{\nolinkurl{aRadZgO4n2Q}}
\item \textit{Rotating Equilaterals XI: Loci of triangle centers of pedal triangles}, 2021. youtu.be/\href{https://youtu.be/S2OZLny2Hfo}{\nolinkurl{S2OZLny2Hfo}}
\item \textit{Rotating Equilaterals XII: Loci of triangle centers of antipedal triangles}, 2021. youtu.be/\href{https://youtu.be/m7iqvZFL-o8}{\nolinkurl{m7iqvZFL-o8}}
\end{enumerate}

\section{Self-Intersected (15)}

\begin{enumerate}[resume]
\item \textit{Self-intersecting 4-periodics (bowtie and tangential polygon)}, 2020. youtu.be/\href{https://youtu.be/C8W2e6ftfOw}{\nolinkurl{C8W2e6ftfOw}}
\item \textit{Self-intersecting 5-periodics (pentagram)}, 2019. youtu.be/\href{https://youtu.be/ECe4DptduJY}{\nolinkurl{ECe4DptduJY}}
\item \textit{Self-intersecting 5-periodics (pentagram): Locus of Internal Intersections}, 2019. youtu.be/\href{https://youtu.be/ZaqvmK22pBM}{\nolinkurl{ZaqvmK22pBM}}
\item \textit{Elliptic Billiard: Self-Intersected 6-Periodics (type I)}, 2020. youtu.be/\href{https://youtu.be/fOD85MNrmdQ}{\nolinkurl{fOD85MNrmdQ}}
\item \textit{Elliptic Billiard: Self-Intersected 6-Periodics (type II)}, 2020. youtu.be/\href{https://youtu.be/gQ-FbSq7wWY}{\nolinkurl{gQ-FbSq7wWY}}
\item \textit{Self-Intersected 6-Periodics in the Elliptic Billiard: Invariant Perimeter Focus-Inversive Polygon}, 2020. youtu.be/\href{https://youtu.be/7lXwjXj-8YY}{\nolinkurl{7lXwjXj-8YY}}
\item \textit{Elliptic Billiard 8-Periodics: Null sum of double cosines of outer polygon}, 2020. youtu.be/\href{https://youtu.be/GEmV_U4eRIE}{\nolinkurl{GEmV\_U4eRIE}}
\item \textit{Family of Self-Intersecting 4-Periodics in the Elliptic Billiard: Inversive Polygon is a Segment}, 2020. youtu.be/\href{https://youtu.be/207Ta31Pl9I}{\nolinkurl{207Ta31Pl9I}}
\item \textit{Type II Self-Intersected 8-Periodics in the Elliptic Billiard + Outer \& Inversive Polygons}, 2020. youtu.be/\href{https://youtu.be/93xpGnDxyi0}{\nolinkurl{93xpGnDxyi0}}
\item \textit{Type I Self-Intersected 8-Periodics in the Elliptic Billiard and the Inversive Polygon}, 2020. youtu.be/\href{https://youtu.be/5Lt9atsZhRs}{\nolinkurl{5Lt9atsZhRs}}
\item \textit{Elliptic Billiard Self-Intersected 7-Periodics, a/b=2: Invariant Perimeter Focus-Inversive Polygons}, 2020. youtu.be/\href{https://youtu.be/gf_aHyvbqOY}{\nolinkurl{gf\_aHyvbqOY}}
\item \textit{Family of self-intersected N=8 w/ turning number 2 in the Elliptic Billiiard}, 2020. youtu.be/\href{https://youtu.be/JwD_w5ecPYs}{\nolinkurl{JwD\_w5ecPYs}}
\item \textit{The two types of self-intersected 7-periodics in the Elliptic Billiard}, 2020. youtu.be/\href{https://youtu.be/yzBG8rgPUP4}{\nolinkurl{yzBG8rgPUP4}}
\item \textit{Elliptic Billiard: Vertices of Self-Intersected 4-Periodics \& Outer Polygon are concyclic w/ foci}, 2020. youtu.be/\href{https://youtu.be/4g-JBshX10U}{\nolinkurl{4g-JBshX10U}}
\item \textit{Self-Intersected Four Periodics in the Elliptic Billiard: Chockfull of Properties}, 2021. youtu.be/\href{https://youtu.be/GZCrek7RTpQ}{\nolinkurl{GZCrek7RTpQ}}
\end{enumerate}

\section{Stationary Circles (5)}

\begin{enumerate}[resume]
\item \textit{Cosine Circle of Excentral Triangle is Stationary}, 2019. youtu.be/\href{https://youtu.be/ACinCf-D_Ok}{\nolinkurl{ACinCf-D\_Ok}}
\item \textit{Locus of Intersection of Anti-Tangents is Stationary Circle}, 2019. youtu.be/\href{https://youtu.be/CrOSI8d8qDc}{\nolinkurl{CrOSI8d8qDc}}
\item \textit{Intersections of Excentral Triangle and its Reflection is a Circle}, 2019. youtu.be/\href{https://youtu.be/hCQIT6_XhaQ}{\nolinkurl{hCQIT6\_XhaQ}}
\item \textit{5-periodics and a stationary circle}, 2019. youtu.be/\href{https://youtu.be/dINE4aH1cvk}{\nolinkurl{dINE4aH1cvk}}
\item \textit{Stationary circles for N=3 to 8}, 2019. youtu.be/\href{https://youtu.be/EFeINGIDFrg}{\nolinkurl{EFeINGIDFrg}}
\end{enumerate}

\section{Steiner's Hat (8)}

\begin{enumerate}[resume]
\item \textit{Equal sum of distances from each focus to vertices of antipedal polygon}, 2020. youtu.be/\href{https://youtu.be/UzBt4R1jGYE}{\nolinkurl{UzBt4R1jGYE}}
\item \textit{Pedal Polygons to N-Periodics with respect to a Focus: Concyclic Vertices and Circular Caustic}, 2020. youtu.be/\href{https://youtu.be/7TE3a5vEWuU}{\nolinkurl{7TE3a5vEWuU}}
\item \textit{The Envelope of Ellipse Antipedals is a Constant-Area Deltoid}, 2020. youtu.be/\href{https://youtu.be/wetmchfY5jI}{\nolinkurl{wetmchfY5jI}}
\item \textit{A narrated tour of the Garcia Deltoid: Surprising Invariants and Properties}, 2020. youtu.be/\href{https://youtu.be/LxADeM1-WHw}{\nolinkurl{LxADeM1-WHw}}
\item \textit{Properties of Osculating Circles to the Ellipse at the 3 Cusp Pre-Images}, 2020. youtu.be/\href{https://youtu.be/NwXc-Vfjs98}{\nolinkurl{NwXc-Vfjs98}}
\item \textit{Locus of Cusps and Deltoid Center of Area}, 2020. youtu.be/\href{https://youtu.be/rZht21KFXk4}{\nolinkurl{rZht21KFXk4}}
\item \textit{Concyclic pre-images, osculating circles, and 3 area-invariant triangles}, 2020. youtu.be/\href{https://youtu.be/fwyr6LXFS1c}{\nolinkurl{fwyr6LXFS1c}}
\item \textit{Rotated Negative Pedal Curve of Ellipse is Area-Invariant}, 2020. youtu.be/\href{https://youtu.be/DgADxkqlKSw}{\nolinkurl{DgADxkqlKSw}}
\end{enumerate}

\section{Subtris (5)}

\begin{enumerate}[resume]
\item \textit{Triangulation-Independent Kimberling Centers of Mass (S. Tabachikov \& E. Tsukerman, 2015)}, 2021. youtu.be/\href{https://youtu.be/aAqGkNsFKaM}{\nolinkurl{aAqGkNsFKaM}}
\item \textit{Sub-Orthocenter Triangle I: Same Area as Reference!}, 2021. youtu.be/\href{https://youtu.be/SZxeu5YIWpQ}{\nolinkurl{SZxeu5YIWpQ}}
\item \textit{Sub-Orthocenter Triangle II: Invariant Perimeter over Circumcircle}, 2021. youtu.be/\href{https://youtu.be/GXwuDV0fdoU}{\nolinkurl{GXwuDV0fdoU}}
\item \textit{Dynamics of the "Circumcenter Map": Periodicity, Stability, Converging, and Diverging Zones}, 2021. youtu.be/\href{https://youtu.be/y6F8SmA67pw}{\nolinkurl{y6F8SmA67pw}}
\item \textit{GPU-based Visualization of Convergence and Divergence Zones of the Circumcenter Map}, 2021. youtu.be/\href{https://youtu.be/PaOPmRraQxQ}{\nolinkurl{PaOPmRraQxQ}}
\end{enumerate}

\section{Swans (2)}

\begin{enumerate}[resume]
\item \textit{Motion of X(88) with respect to collinear X(100) and X(1)}, 2020. youtu.be/\href{https://youtu.be/DaoNJRcf-0E}{\nolinkurl{DaoNJRcf-0E}}
\item \textit{Dance of the Swans: X(88) and X(162)}, 2020. youtu.be/\href{https://youtu.be/ljGTtA1x-Sk}{\nolinkurl{ljGTtA1x-Sk}}
\end{enumerate}

\section{Tangential Polygon (3)}

\begin{enumerate}[resume]
\item \textit{Locus of Vertices of the Excentral Polygon}, 2019. youtu.be/\href{https://youtu.be/kaYWlBTpUPw}{\nolinkurl{kaYWlBTpUPw}}
\item \textit{5-periodics and feet of excenters}, 2019. youtu.be/\href{https://youtu.be/PRkhrUNTXd8}{\nolinkurl{PRkhrUNTXd8}}
\item \textit{Locus of meetpoints of Excentral-to-Orbit Perpendiculars}, 2019. youtu.be/\href{https://youtu.be/NwPioKleiyU}{\nolinkurl{NwPioKleiyU}}
\end{enumerate}

\section{Tris Enveloping Ellipses (4)}

\begin{enumerate}[resume]
\item \textit{90-deg Isosceles Enveloping an Ellipse -- Non-Conic Loci of Vertices and Centroid}, 2021. youtu.be/\href{https://youtu.be/-WaEYct_x7U}{\nolinkurl{-WaEYct\_x7U}}
\item \textit{Equilaterals Enveloping an Ellipse I -- Loci of Vertices and Incenter}, 2021. youtu.be/\href{https://youtu.be/LBs3VxbMxPc}{\nolinkurl{LBs3VxbMxPc}}
\item \textit{Equilaterals Enveloping an Ellipse II -- Non-Conic Centroid Locus \& Astroidal Envelope of Bisectors}, 2021. youtu.be/\href{https://youtu.be/sZka-yj8IR4}{\nolinkurl{sZka-yj8IR4}}
\item \textit{Family of 90-60-30 Right Triangles Enveloping an Ellipse -- Loci of Vertices and Centers X(1)-X(20)}, 2021. youtu.be/\href{https://youtu.be/7cLPQkVvVQM}{\nolinkurl{7cLPQkVvVQM}}
\end{enumerate}

\section{Unrolled 3-Periodics (4)}

\begin{enumerate}[resume]
\item \textit{Fixed central billiard}, 2020. youtu.be/\href{https://youtu.be/v7CDrOTFDzo}{\nolinkurl{v7CDrOTFDzo}}
\item \textit{Pin P1 and n1}, 2020. youtu.be/\href{https://youtu.be/20fx69L_gnU}{\nolinkurl{20fx69L\_gnU}}
\item \textit{Pin P1 at origin}, 2020. youtu.be/\href{https://youtu.be/cPDPb7RmXR4}{\nolinkurl{cPDPb7RmXR4}}
\item \textit{Pin P1 at origin and P1'' vertically above it}, 2020. youtu.be/\href{https://youtu.be/uh45MBlOORE}{\nolinkurl{uh45MBlOORE}}
\end{enumerate}
\end{document}